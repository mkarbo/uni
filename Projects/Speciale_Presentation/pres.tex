\documentclass[9pt,notheorems,xcolor=pdftex,dvipsnames,table]{beamer}
\usepackage[utf8]{inputenc}
%\usefonttheme{serif}
\usetheme{Madrid}
\definecolor{light-gray}{gray}{0.95}
\setbeamercolor{background canvas}{bg=light-gray}
\linespread{1.2}
\usepackage[absolute,overlay]{textpos}
\usepackage{graphicx}
\usepackage{soul}
\usepackage{ulem}
%\setbeamertemplate{sidebar right}{}
\setbeamertemplate{navigation symbols}{}
\makeatletter
\setbeamertemplate{footline}
{
  \leavevmode%
  \hbox{%
  \begin{beamercolorbox}[wd=.333333\paperwidth,ht=2.25ex,dp=1ex,center]{author in head/foot}%
    \usebeamerfont{author in head/foot}\insertauthor
  \end{beamercolorbox}%
   \begin{beamercolorbox}[wd=.333333\paperwidth,ht=2.25ex,dp=1ex,center]{author in head/foot}%
    \usebeamerfont{author in head/foot}
  \end{beamercolorbox}%
  \begin{beamercolorbox}[wd=.333333\paperwidth,ht=2.25ex,dp=1ex,right]{date in head/foot}%
    \usebeamerfont{date in head/foot}\insertshortdate{}\hspace*{2em}
    \insertframenumber{} / \inserttotalframenumber\hspace*{2ex} 
  \end{beamercolorbox}}%
  \vskip0pt%
}
\makeatother
\newcommand{\dd}{\partial}

%\setbeamertemplate{footline}{%
%\hfill\usebeamertemplate***{navigation symbols}
%\hspace{1cm}\insertframenumber{}/\inserttotalframenumber}

\usepackage{calligra}

\usepackage{tikz}
\usepackage{tikz-cd}

\newcommand{\lv}{\lVert}
\newcommand{\rv}{\rVert}

\DeclareMathOperator{\supp}{supp}
\DeclareMathOperator{\Ext}{Ext}
\DeclareMathOperator{\Aut}{Aut}
\DeclareMathOperator{\Ran}{Ran}
\DeclareMathOperator{\Prob}{Prob}
\DeclareMathOperator{\conv}{conv}
\DeclareMathOperator{\AR}{AR}
\DeclareMathOperator{\Homeo}{Homeo}
\newcommand{\acts}[1]{\stackrel{#1}{\curvearrowright}}
\newcommand{\cc}{C_c}
\renewcommand{\emph}[1]{{\textit{#1}}}
\renewcommand{\{}{\left\lbrace}
\renewcommand{\}}{\right\rbrace}
\newcommand{\C}{\mathbb{C}}
\newcommand{\R}{\mathbb{R}}
\newcommand{\Z}{\mathbb{Z}}
\newcommand{\A}{\mathcal{A}}
\newcommand{\Q}{\mathbb{Q}}
\renewcommand{\H}{\mathcal{H}}
\newcommand{\N}{\mathbb{N}}
\newcommand{\Gal}{\mathrm{Gal}}
\newcommand{\IN}{\mathrm{in}_{\leq}}
\newcommand{\lex}{\leq_{\mathrm{lex}}}
\newcommand{\lcm}{\mathrm{lcm}}
\newcommand{\wdots}[2]{ #1, \ldots ,#2 }
\renewcommand{\P}{\{ \wdots{p_1}{p_m} \}}
\newcommand{\K}{K[ \wdots{x_1}{x_n} ]}
\newcommand{\I}{\langle \wdots{p_1}{p_m}  \rangle}


\newtheorem{theorem}{Sætning}
\newtheorem{definition}{Definition}
\newtheorem{example}{Example}


\usepackage{array}
\usepackage{amsthm,amsmath,amssymb}
\usepackage{multirow}
\usepackage{multicol}
\usepackage{url}

\title{Simplicty and uniqueness of trace of group $C^*$-algebras}
\subtitle{Bachelor thesis defense}
\author{Malthe Munk Karbo}
\institute{Advisor: Mikael Rørdam}
\date{February 3, 2017}


\newcommand{\divides}{\bigm|}
\newcommand{\ndivides}{%
  \mathrel{\mkern.5mu % small adjustment
    % superimpose \nmid to \big|
    \ooalign{\hidewidth$\big|$\hidewidth\cr$\nmid$\cr}%
  }%
}

\AtBeginSection[]
{
  \begin{frame}
    \frametitle{Table of Contents}
    \tableofcontents[currentsection]
  \end{frame}
}


\begin{document}
\frame{\titlepage}

\begin{frame}
\frametitle{Table of Contents}
\tableofcontents
\end{frame}

\section{Introduction, history and motivation}
\begin{frame}[t]
\frametitle{What the hell is my thesis about? Crossed products and dynamics}
\visible<2->{The main subject of this thesis is a theory describing the ideal structure of something called \emph{crossed products} associated to a certain type of dynamical systems.}

\mbox{}

\visible<3->{A natural question is: \textbf{What is the all the fuzz about?}} \visible<4->{The crossed product construction associates to a $C^*$-dynamical system $(A,G,\alpha)$ a $C^*$-algebra $A \rtimes_\alpha G$.}\visible<5->{ This construction is \underline{important}:}
\begin{itemize}
	\item<6-> It provides many important and interesting examples of new $C^*$-algebras.
	\item<7-> It encodes quite a bit of information about both $G$ and $A$ and the action $G \acts \alpha A$.
	\item<8-> The construction served as a way of answering many questions in the history of operator algebras.
	\item<9-> Ties together with classic theory of dynamics.
	\item<10-> Ties together with many areas in $C^*$-theory, including the group $C^*$-algebra construction.
\end{itemize}
\visible<11->{\textbf{Okay, but why do we care about the ideal structure of this construction?}}\visible<12->{ Many answers, most notably we want to do \underline{classification.}}

\visible<13->{And the most important question:}\visible<14->{ \textbf{Real world application?}}\visible<15->{\textbf{ (money?)}} \visible<16->{Probably not}
\end{frame}
\begin{frame}[t]
\frametitle{What the hell is my thesis about? Crossed products and dynamics}
	\visible<2->{To a $C^*$-dynamical system $(A,G,\alpha)$, our goal is to answer the following question:}
	\begin{center}
		\visible<3->{\emph{What properties $P$ describes the ideal structure of $A \rtimes_\alpha G$?}}
	\end{center}
	\visible<4->{During this talk, the hope is that we will provide some answers to this question.}
	
	\mbox{}

	\visible<5->{In particular, we shall cover following cases:}
	\begin{itemize}
		\item<6-> $A$ is abelian - this corresponds to a topological dynamical system.
		\item<7-> $G$ is abelian - by means of group spectra.
		\item<8-> $A = \C$ - this corresponds to the group $C^*$-algebra constructon.
	\end{itemize}

	\visible<9->{However, answering this question be quite hard, especially when $G$ is non-discrete!}
\end{frame}
\section{Some notation and preliminaries}
\begin{frame}[t]

\end{frame}
\end{document}
