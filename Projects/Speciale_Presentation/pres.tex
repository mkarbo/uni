\documentclass[9pt,notheorems,xcolor=pdftex,dvipsnames,table]{beamer}
\usepackage[utf8]{inputenc}
%\usefonttheme{serif}
\usetheme{Madrid}
\definecolor{light-gray}{gray}{0.95}
\setbeamercolor{background canvas}{bg=light-gray}
\linespread{1.2}
\usepackage[absolute,overlay]{textpos}
\usepackage{graphicx}
\usepackage{soul}
\usepackage{ulem}
%\setbeamertemplate{sidebar right}{}
\setbeamertemplate{navigation symbols}{}
\makeatletter
\setbeamertemplate{footline}
{
  \leavevmode%
  \hbox{%
  \begin{beamercolorbox}[wd=.333333\paperwidth,ht=2.25ex,dp=1ex,center]{author in head/foot}%
    \usebeamerfont{author in head/foot}\insertauthor
  \end{beamercolorbox}%
   \begin{beamercolorbox}[wd=.333333\paperwidth,ht=2.25ex,dp=1ex,center]{author in head/foot}%
    \usebeamerfont{author in head/foot}
  \end{beamercolorbox}%
  \begin{beamercolorbox}[wd=.333333\paperwidth,ht=2.25ex,dp=1ex,right]{date in head/foot}%
    \usebeamerfont{date in head/foot}\insertshortdate{}\hspace*{2em}
    \insertframenumber{} / \inserttotalframenumber\hspace*{2ex} 
  \end{beamercolorbox}}%
  \vskip0pt%
}
\makeatother
\newcommand{\dd}{\partial}

%\setbeamertemplate{footline}{%
%\hfill\usebeamertemplate***{navigation symbols}
%\hspace{1cm}\insertframenumber{}/\inserttotalframenumber}

\usepackage{calligra}

\usepackage{tikz}
\usepackage{tikz-cd}

\newcommand{\lv}{\lVert}
\newcommand{\rv}{\rVert}
\renewcommand{\d}{\mathrm{d}}
\DeclareMathOperator{\supp}{supp}
\DeclareMathOperator{\Ext}{Ext}
\DeclareMathOperator{\Aut}{Aut}
\DeclareMathOperator{\Ran}{Ran}
\DeclareMathOperator{\Prob}{Prob}
\DeclareMathOperator{\conv}{conv}
\DeclareMathOperator{\AR}{AR}
\DeclareMathOperator{\Homeo}{Homeo}
\newcommand{\acts}[1]{\stackrel{#1}{\curvearrowright}}
\newcommand{\cc}{C_c}
\renewcommand{\emph}[1]{{\textit{#1}}}
\renewcommand{\{}{\left\lbrace}
\renewcommand{\}}{\right\rbrace}
\newcommand{\C}{\mathbb{C}}
\newcommand{\R}{\mathbb{R}}
\newcommand{\Z}{\mathbb{Z}}
\newcommand{\A}{\mathcal{A}}
\newcommand{\Q}{\mathbb{Q}}
\renewcommand{\H}{\mathcal{H}}
\newcommand{\N}{\mathbb{N}}
\newcommand{\Gal}{\mathrm{Gal}}
\newcommand{\IN}{\mathrm{in}_{\leq}}
\newcommand{\lex}{\leq_{\mathrm{lex}}}
\newcommand{\lcm}{\mathrm{lcm}}
\newcommand{\wdots}[2]{ #1, \ldots ,#2 }
\renewcommand{\P}{\{ \wdots{p_1}{p_m} \}}
\newcommand{\K}{K[ \wdots{x_1}{x_n} ]}
\newcommand{\I}{\langle \wdots{p_1}{p_m}  \rangle}

\renewcommand{\hat}[1]{\widehat{#1}}

\newtheorem{theorem}{Sætning}
\newtheorem{definition}{Definition}
\newtheorem{example}{Example}


\usepackage{array}
\usepackage{amsthm,amsmath,amssymb}
\usepackage{multirow}
\usepackage{multicol}
\usepackage{url}

\title{Simplicty of Crossed Products of $C^*$-algebras}
\subtitle{Masters thesis defense}
\author{Malthe Munk Karbo}
\institute{Advisor: Søren Eilers}
\date{February 15, 2019}


\newcommand{\divides}{\bigm|}
\newcommand{\ndivides}{%
  \mathrel{\mkern.5mu % small adjustment
    % superimpose \nmid to \big|
    \ooalign{\hidewidth$\big|$\hidewidth\cr$\nmid$\cr}%
  }%
}

\AtBeginSection[]
{
  \begin{frame}
    \frametitle{Table of Contents}
    \tableofcontents[currentsection]
  \end{frame}
}


\begin{document}
\frame{\titlepage}

\begin{frame}
\frametitle{Table of Contents}
\tableofcontents
\end{frame}

\section{Introduction, history and motivation}
\begin{frame}[t]
\frametitle{What the hell is my thesis about? Crossed products and dynamics}
\visible<2->{The main subject of this thesis is a theory describing the ideal structure of something called \emph{crossed products} associated to a certain type of dynamical systems.}

\mbox{}

\visible<3->{A natural question is: \textbf{What is the all the fuzz about?}} \visible<4->{The crossed product construction associates to a $C^*$-dynamical system $(A,G,\alpha)$ a $C^*$-algebra $A \rtimes_\alpha G$.}\visible<5->{ This construction is \underline{important}:}
\begin{itemize}
	\item<6-> It provides many important and interesting examples of new $C^*$-algebras.
	\item<7-> It encodes quite a bit of information about both $G$ and $A$ and the action $G \acts \alpha A$.
	\item<8-> The construction served as a way of answering many questions in the history of operator algebras.
	\item<9-> Ties together with classic theory of dynamics.
	\item<10-> Ties together with many areas in $C^*$-theory, including the group $C^*$-algebra construction.
\end{itemize}
\visible<11->{\textbf{Okay, but why do we care about the ideal structure of this construction?}}\visible<12->{ Many answers, most notably we want to do \underline{classification.}}

\visible<13->{And the most important question:}\visible<14->{ \textbf{Real world application?}}\visible<15->{\textbf{ (money?)}} \visible<16->{Probably not}
\end{frame}
\begin{frame}[t]
\frametitle{What the hell is my thesis about? Crossed products and dynamics}
	\visible<2->{To a $C^*$-dynamical system $(A,G,\alpha)$, our goal is to answer the following question:}
	\begin{center}
		\visible<3->{\emph{Question 1: What properties $P$ describes the ideal structure of $A \rtimes_\alpha G$?}}
	\end{center}
	\visible<4->{During this talk, the hope is that we will provide some answers to this question.} \visible<5->{\textbf{Answering question 1 in the above form is \underline{hard}}}. 
	
	\mbox{}

	\visible<5->{Instead, we break it into special cases}\visible<6->{ Some cases discussed today will be:}
	\begin{itemize}
		\item<7-> Which properties $P$ answers Question 1 when $A$ is abelian? 
		\item<8-> Which properties $P$ answers Question 1 when $G$ is abelian?
		\item<9-> Which properties $P$ answers Question 1 when $A = \C$?
		\item<10-> Which properties $P$ answers Question 1 when $G$ is discrete?
	\end{itemize}
	\visible<11->{And of course, we have overlaps between the above cases.}
\end{frame}
\section{Some notation and preliminaries}
\begin{frame}
\frametitle{Disclaimer}
\visible<2->{This will be \underline{very}} \visible<3->{educational}\visible<2->{ for all of you.} \visible<4->{But \textit{maybe} not that much fun for some of you.}
\end{frame}
\begin{frame}[t]
\frametitle{A short primer on $C^*$-dynamical systems.}
\visible<2->{Lots of prerequisite theory needed to understand todays talk.}\visible<3->{ \textbf{Not} possible to cover all.}
\visible<4->{
\begin{block}{Definitin}
	An abstract $C^*$-algebra $A$ is a Banach algebra $A$ equipped with an involution such that 
	\begin{align*}
		\lv a \rv ^2 = \lv a^*a\rv,
	\end{align*}
	for all $a \in A$.
\end{block}
}
\visible<5->{Notable examples include}
\begin{itemize}
	\item<6-> The algebra $C_0(X)$ of continuous functions vanishing at infinity on a locally compact Hausdorff space $X$.
	\item<7-> The algebra $\mathbb{B}(\H)$ of bounded linear operators on a Hilbert space $\H$.
	\item<8-> The group $C^*$-algebras $C_r^*(G)$ and $C^*(G)$ of a locally compact Hausdorff group $G$.
\end{itemize}
\visible<9->{All of these $C^*$-algebras play a crucial role in the theory of crossed products.}
\end{frame}
\begin{frame}[t]
\frametitle{A short primer on $C^*$-dynamical systems.}
\visible<1->{
\begin{block}{Definitin}
	An abstract \emph{$C^*$-algebra} $A$ is an involutive Banach algebra $A$ satisfying 
	\begin{align*}
		\lv a \rv ^2 = \lv a^*a\rv,
	\end{align*}
	for all $a \in A$.
\end{block}
}
\visible<2->{
\begin{block}{Definition}
	A \emph{representation} of a $C^*$-algebra $A$ is a pair $\left( \pi, \H \right)$ consisting of a bounded $*$-homomorphism $\pi \colon A \to \mathbb{B}(\H)$.
\end{block}
}
\visible<3->{
	\begin{block}{The Gelfand-Neimark-Segal result}
	The Gelfand-Neimark-Segal result is an important result stating that all $C^*$-algebras $A$ are faithfully represented on a Hilbert space.
\end{block}
}
\end{frame}
\begin{frame}[t]
	\frametitle{A short primer on topological group theory}
	\visible<2->{
		\begin{block}{Definition}
			A topological group $G$ is topological space equipped with a group structure such that the maps $G \times G \ni (s,t) \mapsto st$ and $G \ni t \mapsto t^{-1} \in G$ are continuous for all $s,t \in G$.\visible<3->{ By $G$ we will \textbf{always} denote a topological group $G$ with a locally compact Hausdorff topology.}
		\end{block}
	}
	\visible<4->{Important cases include $\Z$, $\R$ and $S^1$.}
	\visible<5->{Topological group are important and arises naturally in many fields of mathematics.}\visible<6->{ We will need results from the theory of \textbf{abstract harmonic analysis}, \textbf{representation theory} and \textbf{Fourier analysis}}
	\visible<7->{
		\begin{block}{Definition}
			A unitary representation of a group $G$ is a pair $(u,\H)$ consisting of a group homomorphism $u \colon G \to \mathcal{U}(\H)$ such that for all $\xi \in \H$ the map $t \mapsto u_t \xi$ is continuous (= $u$ is strongly continuous).
		\end{block}
	}
\end{frame}
\begin{frame}[t]
	\frametitle{A short primer on topological group theory}
	\visible<1->{
		\begin{block}{Definition}
			A topological group $G$ is topological space equipped with a group structure such that the maps $G \times G \ni (s,t) \mapsto st$ and $G \ni t \mapsto t^{-1} \in G$ are continuous for all $s,t \in G$.\visible<1->{ By $G$ we will \textbf{always} denote a topological group $G$ with a locally compact Hausdorff topology.}
		\end{block}
	}
	\visible<1->{
		\begin{block}{Definition}
			A unitary representation of a group $G$ is a pair $(u,\H)$ consisting of a group homomorphism $u \colon G \to \mathcal{U}(\H)$ such that for all $\xi \in \H$ the map $t \mapsto u_t \xi$ is continuous (= $u$ is strongly continuous).
		\end{block}
	}
	\visible<2->{An important example is the \emph{left-regular representation} of $G$:}
\visible<3->{
	\begin{block}{Definition}
		The left-regular representation of $G$ is the tuple $(\lambda, L^2(G))$, where
		\begin{align*}
			\lambda_t \xi(s) = \xi(t^{-1}s), 
		\end{align*}
		for $\xi \in L^2(G)$ and $s,t \in G$.
	\end{block}
}
\end{frame}
\begin{frame}[t]
	\frametitle{The perhaps shortest primer on $C^*$-dynamical systems}
	\visible<2->{
		\begin{block}{Definition}
			A \emph{$C^*$-dynamical system} is a triple $(A,G,\alpha)$ consisting of
			\begin{itemize}
				\item<3-> A $C^*$-algebra $A$,
				\item<4-> A topological group $G$ and
				\item<5-> a strongly continuous group homomorphism $\alpha \colon G \to \mathrm{Aut}(A)$ ($\forall a \in A : G \ni  t \mapsto \alpha_t(a) \in A \text{ is continuous}$).
			\end{itemize}
		\end{block}
	}
	\visible<7->{
		\begin{block}{Definition}
			A covariant representation of $(A,G,\alpha)$ is a triple $(\pi, u , \H)$ consisting of representations of $A$ and $G$ as bounded operators on a common Hilbert space $\H$ satisfying the relation
			\begin{align*}
				u_t \pi(a) u_t^* = \pi(\alpha_t(a)) \text{ for all } a \in A \text{ and } t \in G.
			\end{align*}
		\end{block}
	}
	\visible<8->{When $G$ is discrete and $A$ is unital, it is easy to associate to such a triple a \textbf{nice} new $C^*$-algebra:}
\end{frame}
\begin{frame}[t]
	\frametitle{Cooking with $C^*$-algebras}
	\visible<2->{
		\begin{block}{Discrete crossed product cookbook recipe}
\begin{enumerate}
	\item<3-> Given $(A,G,\alpha)$ with $G$ discrete and $A$ unital, consider the $*$-algebra 
			\begin{align*}
				C_c(G,A) = \{ \sum_{t \in F} a_t \delta_t  \mid F \subseteq G \text{ finite} \}.
			\end{align*}
	\item<4-> Pick a covariant representation $(\pi,u,\H)$ of $(A,G,\alpha)$.
	\item<5-> Define a representation $\pi \rtimes u$ of $C_c(G,A)$ as operators on $\H$ by
		\begin{align*}
			\pi \rtimes u \left( \sum a_t \delta_t \right) := \sum \pi(a_t) u_t.
		\end{align*}
	\item<6-> Take the closure of $\pi \rtimes u(C_c(G,A))$ to obtain a $C^*$-algebra encoding some information about the dynamical system $(A,G,\alpha)$.
\end{enumerate}
		\end{block}
	}
	\visible<7->{This is the building stone for constructing crossed products.} \visible<8->{Fails when $G$ is non-discrete.}
\end{frame}
\section{The general crossed product construction}
\begin{frame}
	\frametitle{An interlude - how do we proceed?}
	\visible<2->{As previously discussed, we have a goal:}\visible<3->{ To describe the ideal structure of general crossed products by means of certain properties.}
	
	\visible<4->{To reach this goal today (we hope), we lay out the following plan for today:}
	\visible<5->{
\begin{enumerate}
	\item\mbox{} \visible<6->{Cover the construction of the crossed product in the general case.}
	\item\mbox{} \visible<7->{Proceed with describing the ideal structure.}
	\item\mbox{} \visible<8->{Conclusion.}
	\item<5-8>
	\item<5-8>
	\item<5-8>
	\item<5-8>
	\item<5-8>
	\item<5-8>
	\item<5-8>
	\item<5-8>
	\item<5-8>
	\item<5-8>
\end{enumerate}
	}
	\visible<9->{}
\end{frame}
\begin{frame}[t]
	\frametitle{The issue with non-discrete groups}
	\begin{itemize}
		\item<2-> The goal: Given $(A,G,\alpha)$ and a covariant representation $(\pi,u,\H)$, we wish to construct a represention $\pi \rtimes u$ of $C_c(G,A)$ on $\H$.
		\item<3-> The model: given $f \in C_c(G,A)$ we want
		\begin{align*}
			\pi \rtimes u(f) = \int_G \pi(f(s) u_s \d s \in \mathbb{B}(\H)
		\end{align*}
	\item<4-> The issue?\visible<5->{ Is the map $s \mapsto \pi(f(s)) u_s$ continuous? Is this integral well-defined?}
	\end{itemize}
	\visible<6->{The solution? Multiplier algebra $M(A)$ of $A$:}
\end{frame}
\begin{frame}[t]
	\frametitle{The issue with non-discrete groups}
	The solution? The multiplier algebra $M(A)$ of $A$: \visible<2->{We equip it with a topology; the \emph{strict topology} (denoted by $M_s(A)$).}
	\visible<3->{
		\begin{block}{Theorem}
			Let $A$ be a $C^*$-algebra. Then there is a unique linear map $\int \colon C_c(G,M_s(A)) \to M(A)$ such that for all non-degenerate representations $\pi \colon A \to \mathbb{B}(H)$, $\xi, \eta \in \H$ and $f \in C_c(G,M_s(A))$ we have
			\begin{align*}
				\langle \overline \pi\left( \int(f) \right) \xi , \eta \rangle = \int_G \langle \overline \pi(f(s)) \xi , \eta \rangle \d s.
			\end{align*}
			And we write $\int(f) := \int_G f(s) \d s$.
		\end{block}
}
\end{frame}
\begin{frame}[t]
	\frametitle{The issue with non-discrete groups}
	\begin{itemize}
		\item<1-> The goal: Given $(A,G,\alpha)$ and a covariant representation $(\pi,u,\H)$, we wish to construct a represention $\pi \rtimes u$ of $C_c(G,A)$ on $\H$.
		\item<1-> The model: given $f \in C_c(G,A)$ we want
		\begin{align*}
			\pi \rtimes u(f) = \int_G \pi(f(s) u_s \d s \in \mathbb{B}(\H)
		\end{align*}
	\item<1-> The issue?\visible<1->{ Is the map $s \mapsto \pi(f(s)) u_s$ continuous? Is this integral well-defined?}
	\end{itemize}
	\visible<1->{The solution? Multiplier algebra $M(A)$ of $A$:}\visible<2->{
		\begin{align*}
			G \ni s \mapsto \pi(f(s)) u_s \in \mathbb{B}(\H)
		\end{align*}
		is strictly continuous for all $f \in C_c(G,A)$.
	}\visible<3->{ We set
	\begin{align*}
		\pi \rtimes u(f) := \int(\pi(f)u) = \int_G \pi(f(s)) u_s \d s.
	\end{align*}	
}
\end{frame}
\begin{frame}[t]
	\frametitle{The crossed product construction}
	\visible<2->{
		\begin{block}{Definition}
			The \emph{full crossed product} of $(A,G,\alpha)$ is the $C^*$-algebra $A \rtimes_\alpha G$ obtained by taking the closure of $C_c(G,A)$ with respect to the norm
			\begin{align*}
				\lv f \rv_u = \sup\{\lv \pi \rtimes u (f)\rv  \mid (\pi,u,\H) \text{ is a (non-degenerate) covariant representation}\}
			\end{align*}
		\end{block}
	}
	\visible<3->{
		\begin{block}{Theorem}
			There is a faithful homomorphism $A \hookrightarrow M(A \rtimes_\alpha G)$ and an injective strictly continuous group homomorphism $U \colon G \to \mathcal{U}(M(A\rtimes_\alpha G))$ such that for $a \in A$, $s,r \in G$ and $f \in C_c(G,A)$
			\begin{align*}
				af(s) = a f(s), \quad U_r f(s) = \alpha_r(f(r^{-1}s)) \text{  and  } \alpha_r(a) = U_r a U_r^*.
			\end{align*}
		\end{block}
	}
	\visible<4->{
		\begin{block}{Theorem}
			The diagram commutes for all non-degenerate covariant representations $(\pi,u,\H)$.
		\end{block}
	}
	\visible<5->{Makes up for lack of copies inside $A \rtimes_\alpha G$.}
\end{frame}
\begin{frame}[t]
	\frametitle{The reduced crossed product construction}
	\visible<2->{The previous construction is the 'large' crossed product. We also have a small one.}
	\visible<3->{
		\begin{block}{Definition}
			Let $(\pi,\H)$ be a faithful representation of $A$. Then we may define a representation $(\tilde \pi, L^2(G, \H)$ by $\tilde \pi(a) \xi(s) = \pi(\alpha_{r^{-1}}(a)) \xi(r)$ for $\xi \in L^2(G,\H)$ and $r \in G$.
			
			\visible<4->{ Abusing notation, $\lambda$ will also denote the representation of $G$ on $L^2(G,H)$ given by 
				\begin{align*}
				\lambda_s \xi(r) = \xi(s^{-1}r).
			\end{align*}}
		\end{block}
	}
	\visible<5->{
		\begin{block}{Lemma}
			The triple $(\tilde \pi, \lambda, L^2(G,\H))$ is a covariant representation. 
		\end{block}
	}
\end{frame}
\begin{frame}[t]
	\frametitle{The reduced crossed product construction}
	\visible<1->{The previous construction is the 'large' crossed product. We also have a small one.}
	\visible<1->{
		\begin{block}{Lemma}
			The triple $(\tilde \pi, \lambda, L^2(G,\H))$ is a covariant representation.
		\end{block}
	}
	\visible<2->{
	\begin{block}{Theorem}	
	If $\tilde \pi$ is faithful (non-degenerate) then $\tilde \pi \rtimes \lambda$ is faithful (non-degenerate).
	\end{block}
}
	\visible<3->{
		\begin{block}{Definition}
			The reduced crossed product is the $C^*$-algebra $A \rtimes_{\alpha,r}G$ obtained by taking the closure of $\tilde \pi \rtimes \lambda(C_c(G,A))$ in $\mathbb{B}(L^2(G,\H))$ where $\pi$ is any faithful non-degenerate representation of $A$ on $\H$. 
\end{block}
	}
	\visible<4->{$A \rtimes_{\alpha ,r }G$ is much more concretely defined than $A \rtimes_\alpha G$, and thus easier to work with.}
	\visible<5->{
		\begin{block}{Theorem}
			If $G$ is amenable (=admits left-invariant mean), then $A \rtimes_\alpha G \cong A \rtimes_{\alpha,r}G$	
		\end{block}
	}
\end{frame}
\section{Simplicity}
\begin{frame}[t]
	\visible<2->{Techniques for determining the ideal structure of an abstract $C^*$-algebra can be hard.}\visible<3->{ Further assumptions can help, but might not hold generally}
	\visible<4->{
		\begin{block}{Result for general simplicity}
			Suppose that $A$ is a \underline{unital} $C^*$-algebra equipped with a faithful tracial state $\tau$. If to each $a \in A$ it holds that
			\begin{align*}
				\overline{\mathrm{conv}\{ u a u^* \mid u \text{ unitary} \}} \cap \C1 \neq \emptyset,
			\end{align*}
		then $A$ is simple
	\end{block}
	}
	\visible<5->{Works for group $C^*$-algebras of discrete groups, but $A \rtimes_\alpha G$ is not unital for non-discrete groups}

	\mbox{}

	\visible<6->{Need a less general approach to answer Question 1.}
\end{frame}

\section{The case of $G$ abelian}
\begin{frame}[t]
	\frametitle{The case of $G$ abelian}
	\visible<2->{The goal of today was to find properties $P$ answering our initial question.}

	\visible<3->{When $G$ is abelian, we have a rich theory drawing on the strengths of abstract harmonic analysis whenever $G \acts \alpha A$:}
	\begin{itemize}
		\item<4-> $\rightsquigarrow \hat G \acts{ \hat{\alpha}} A \rtimes_\alpha G$, $\hat \alpha_\chi(f)(s) = \chi(s)f(s)$.
		\item<5-> $\rightsquigarrow$ a duality theory called Takai duality;
		\begin{align*}
			(A \rtimes_\alpha G \rtimes_{\hat{\alpha}} \hat G , G,  \hat{ \hat \alpha}) \cong (A \otimes \mathbb{K}(L^2(G)),  G, \alpha \otimes \mathrm{Ad} \rho)
		\end{align*}
	\end{itemize}
	\visible<6->{
		\begin{block}{Definition}
			The Arveson spectrum of $\alpha$ is the set
			\begin{align*}
				\mathrm{Sp}(\alpha) := \{ \chi \in \hat G \ \mid \ \alpha_f= 0 \text{ implies } \hat f(\chi) = 0 \text{ for all } f \in L^1(G) \}
			\end{align*}
\end{block}
	}
\end{frame}

\begin{frame}[t]
	\frametitle{The case of $G$ abelian}
	\begin{itemize}
		\item	$ G \acts \alpha A\rightsquigarrow \hat G \acts{ \hat{\alpha}} A \rtimes_\alpha G$, $\hat \alpha_\chi(f)(s) = \chi(s)f(s)$.
	\end{itemize}
		\begin{block}{Definition}
			The Arveson spectrum of $\alpha$ is the set
			\begin{align*}
				\mathrm{Sp}(\alpha) := \{ \chi \in \hat G \ \mid \ \alpha_f= 0 \text{ implies } \hat f(\chi) = 0 \text{ for all } f \in L^1(G) \}
			\end{align*}
\end{block}
\visible<2->{Here $\alpha _f$ is the weak$^*$-integral of $f$ in $A$}
\visible<3->{
	\begin{block}{Definition}
	The Connes spectrum $\hat G(\alpha)$ of the action $\hat G$ is the set
	\begin{align*}
		\bigcap_{D \in H^\alpha(A)}\mathrm{Sp}(\alpha |_D)
	\end{align*}
\end{block}
}
\end{frame}
\begin{frame}[t]
	\frametitle{The case of $G$ abelian}
\visible<1->{
	\begin{block}{Definition}
	The Connes spectrum $\hat G(\alpha)$ of the action $\hat G$ is the set
	\begin{align*}
		\bigcap_{D \in H^\alpha(A)}\mathrm{Sp}(\alpha |_D)
	\end{align*}
\end{block}
}
\visible<2->{
Turns out that the Connes spectrum encodes quite a bit of information about the dynamical system, the action and the crossed product.
}
\visible<3->{
	\begin{block}{Theorem}
		Suppose that $G$ is discrete and abelian. Then the following are equivalent;
		\begin{enumerate}
			\item $A \rtimes_\alpha G$ is simple (!),
			\item a. $A$ is $G$-simple and b. $\hat G(\alpha) = \hat G$.
		\end{enumerate}
	\end{block}
}
\visible<4->{Was the first result in a series of Olesen and Pedersen about ideals in $A \rtimes_\alpha G$.}\visible<5->{ The spectra helped connecting various theories and techniques together.}
\end{frame}

\begin{frame}[t]
	\frametitle{The case of $G$ abelian}
\visible<1->{
Turns out that the Connes spectrum encodes quite a bit of information about the dynamical system, the action and the crossed product.
}
\visible<1->{
	\begin{block}{Theorem}
		Suppose that $G$ is discrete and abelian. Then the following are equivalent;
		\begin{enumerate}
			\item $A \rtimes_\alpha G$ is simple (!),
			\item a. $A$ is $G$-simple and b. $\hat G(\alpha) = \hat G$.
		\end{enumerate}
	\end{block}
}
\visible<2->{
	\begin{block}{Theorem}	
	The following are equivalent
	\begin{enumerate}
		\item $\hat G(\alpha) = \hat G$.
		\item For each $\sigma \in \hat G$ and each non-zero ideal $I \subseteq A \rtimes_\alpha G$ we have $I \cap \hat \alpha_\sigma(I) \neq 0$.
		\item For each non-zero ideal $I \subseteq A \rtimes_\alpha G$ there is a non-zero $\hat G$-invariant ideal $I_0 \subseteq I$.
		\item $A$ detects ideals in $A \rtimes_\alpha G$.
	\end{enumerate}
	\end{block}

}
\end{frame}

\section{The case of $A$ abelian and $G$ discrete}
\begin{frame}[t]
	\frametitle{The case of $A=C_0(X)$ abelian and $G$ discrete}
	\visible<2->{Some years after the results of Olesen and Pedersen, a new result emerged;}\visible<3->{ the Archbold-Spielberg result.}
	\visible<4->{
		\begin{block}{}
			There is a one-to-one correspondance
			\begin{align*}
				\{\text{Topological dynamical systems}\} &\leftrightarrow \{C^*\text{-dynamical abelian systems} \}\\
				\{ (X,G)\} &\leftrightarrow \{ (C_0(X),G,\alpha) \}
			\end{align*}
\end{block}
	}
	\visible<5->{
		\begin{definition}
			The action $\alpha$ is topologically free if the set
			\begin{align*}
				G_x := \{ t \in G \mid t.x = x\}
			\end{align*}
			has dense complement for each $x \in X$
		\end{definition}
	}
\end{frame}

\begin{frame}[t]
	\frametitle{The case of $A=C_0(X)$ abelian and $G$ discrete}
	\visible<1->{
		\begin{definition}
			The action $\alpha$ is topologically free if the set
			\begin{align*}
				G_x := \{ t \in G \mid t.x = x\}
			\end{align*}
			has dense complement for each $x \in X$
		\end{definition}
	}
	\visible<2->{
		\begin{block}{Archbold-Spielberg for commutative $C^*$-algebra}
			The algebra $C_0(X) \rtimes_\alpha G$ is simple if and only if $\alpha $ is topologically free, minimal and $A \rtimes_\alpha G \cong A \rtimes_{\alpha,r}G$ via a regular representation $\tilde \pi \rtimes \lambda$.	
		\end{block}
	}
\end{frame}
\section{An application of the theory}
\begin{frame}[t]
	\frametitle{A recent application ($G$ discrete)}
	\visible<2->{
	Recently (= within the last 5 years), the field studying $C^*$-simplicity of groups has seen tremendous development.} \visible<3->{Why?}\visible<4->{ A series of result connecting results about
	}
	\begin{itemize}
		\item<5-> Group $C^*$-algebras,
		\item<6-> Crossed products,
		\item<7-> Boundary actions and
		\item<8-> Topological freeness.
	\end{itemize}
	\visible<9->{What happened?}
	\begin{itemize}
		\item<10-> Kennedy and Kalantar paper was released
		\item<11-> Gave new angle on old problem; How do we describe simplicity of $C_r^*(G)$ by algebraic properties of $G$?
		\item<12->Answered question about implications; \begin{center}
$C_r^*(G)$ simple $\implies$ $C_r^*(G)$ unique trace $\iff$ trivial amenable radical of $G$.
\end{center}
	\end{itemize}
\end{frame}

\begin{frame}[t]
	\frametitle{A recent application ($G$ discrete)}
	\begin{itemize}
		\item<1->Answered question about implications; \begin{center}
$C_r^*(G)$ simple $\implies$ $C_r^*(G)$ unique trace $\iff$ trivial amenable radical of $G$.
\end{center}
	\end{itemize}
	\visible<2->{
		\begin{block}{Kennedy, Kalantar, Breulliard and Ozawa}
	The following are equivalent:	
\begin{enumerate}
		\item $G$ is $C^*$-simple.
		\item $C(X) \rtimes_r G$ is simple for all $G$-boundaries $X$.
		\item $C(\dd_F G) \rtimes_r G$ is simple.
		\item The action $G \acts {} \dd_F G$ is free.
		\item The action $G \acts {} \dd_F G$ is topologically free.
		\item There is some $G$-boundary $X$ such that the action is topologically free.
		\item Whenever $X$ is a compact minimal $G$-space, amenability of $G_x$ for some $x \in X$ implies topological freeness of the action on $X$.
		\item If $G \acts \alpha A$ with $A$ unital, then $A \rtimes_{\alpha,r} G$ is simple whenever $A$ is $G$-simple.
	\end{enumerate}
\end{block}
	}
\end{frame}
\begin{frame}
Thanks for listening.
\end{frame}



\begin{frame}
	\visible<10->{\mbox{}}
\end{frame}



\begin{frame}
	\visible<100->{\mbox{}}
\end{frame}



\begin{frame}
	
\end{frame}


\begin{frame}
	
\end{frame}


\begin{frame}
	
\end{frame}


\begin{frame}
	
\end{frame}


\begin{frame}
	
\end{frame}


\begin{frame}
	
\end{frame}


\begin{frame}
	
\end{frame}


\begin{frame}
	
\end{frame}


\begin{frame}
	
\end{frame}


\begin{frame}
	
\end{frame}


\begin{frame}
	
\end{frame}


\begin{frame}
	
\end{frame}


\begin{frame}
	
\end{frame}


\begin{frame}
	
\end{frame}


\begin{frame}
	
\end{frame}


\begin{frame}
	
\end{frame}


\begin{frame}
	
\end{frame}


\begin{frame}
	
\end{frame}


\begin{frame}
	
\end{frame}


\begin{frame}
	
\end{frame}


\begin{frame}
	
\end{frame}


\begin{frame}
	
\end{frame}


\begin{frame}
	
\end{frame}


\begin{frame}
	
\end{frame}


\begin{frame}
	
\end{frame}


\begin{frame}
	
\end{frame}


\begin{frame}
	
\end{frame}


\begin{frame}
	
\end{frame}


\begin{frame}
	
\end{frame}


\begin{frame}
	
\end{frame}


\begin{frame}
	
\end{frame}


\begin{frame}
	
\end{frame}


\begin{frame}
	
\end{frame}


\begin{frame}
	
\end{frame}


\begin{frame}
	
\end{frame}


\begin{frame}
	
\end{frame}


\begin{frame}
	
\end{frame}


\begin{frame}
	
\end{frame}


\begin{frame}
	
\end{frame}


\begin{frame}
	
\end{frame}


\begin{frame}
	
\end{frame}


\begin{frame}
	
\end{frame}


\begin{frame}
	
\end{frame}


\begin{frame}
	
\end{frame}


\begin{frame}
	
\end{frame}


\begin{frame}
	
\end{frame}


\begin{frame}
	
\end{frame}


\begin{frame}
	
\end{frame}


\begin{frame}
	
\end{frame}


\begin{frame}
	
\end{frame}


\begin{frame}
	
\end{frame}


\begin{frame}
	
\end{frame}


\begin{frame}
	
\end{frame}


\begin{frame}
	
\end{frame}


\begin{frame}
	
\end{frame}


\begin{frame}
	
\end{frame}


\begin{frame}
	
\end{frame}


\begin{frame}
	
\end{frame}


\begin{frame}
	
\end{frame}


\begin{frame}
	
\end{frame}


\begin{frame}
	
\end{frame}


\begin{frame}
	
\end{frame}


\begin{frame}
	
\end{frame}


\begin{frame}
	
\end{frame}


\begin{frame}
	
\end{frame}


\begin{frame}
	
\end{frame}


\begin{frame}
	
\end{frame}


\begin{frame}
	
\end{frame}


\begin{frame}
	
\end{frame}


\begin{frame}
	
\end{frame}


\begin{frame}
	
\end{frame}


\begin{frame}
	
\end{frame}


\begin{frame}
	
\end{frame}


\begin{frame}
	
\end{frame}


\begin{frame}
	
\end{frame}


\begin{frame}
	
\end{frame}


\begin{frame}
	
\end{frame}


\begin{frame}
	
\end{frame}


\begin{frame}
	
\end{frame}


\begin{frame}
	
\end{frame}


\begin{frame}
	
\end{frame}


\begin{frame}
	
\end{frame}


\begin{frame}
	
\end{frame}


\begin{frame}
	
\end{frame}


\begin{frame}
	
\end{frame}


\begin{frame}
	
\end{frame}


\begin{frame}
	
\end{frame}


\begin{frame}
	
\end{frame}


\begin{frame}
	
\end{frame}


\begin{frame}
	
\end{frame}


\begin{frame}
	
\end{frame}


\begin{frame}
	
\end{frame}


\begin{frame}
	
\end{frame}


\begin{frame}
	
\end{frame}


\begin{frame}
	
\end{frame}


\begin{frame}
	
\end{frame}


\begin{frame}
	
\end{frame}


\begin{frame}
	
\end{frame}


\begin{frame}
	
\end{frame}


\begin{frame}
	
\end{frame}


\begin{frame}
	
\end{frame}

\begin{frame}
	\visible<10->{\mbox{}}
\end{frame}



\begin{frame}
	\visible<10->{\mbox{}}
\end{frame}



\begin{frame}
	\visible<10->{\mbox{}}
\end{frame}



\begin{frame}
	\visible<10->{\mbox{}}
\end{frame}



\begin{frame}
	\visible<10->{\mbox{}}
\end{frame}



\begin{frame}
	\visible<10->{\mbox{}}
\end{frame}



\begin{frame}
	\visible<10->{\mbox{}}
\end{frame}



\begin{frame}
	\visible<10->{\mbox{}}
\end{frame}



\begin{frame}
	\visible<10->{\mbox{}}
\end{frame}



\begin{frame}
	\visible<10->{\mbox{}}
\end{frame}



\begin{frame}
	\visible<10->{\mbox{}}
\end{frame}



\begin{frame}
	\visible<10->{\mbox{}}
\end{frame}



\begin{frame}
	\visible<10->{\mbox{}}
\end{frame}



\begin{frame}
	\visible<10->{\mbox{}}
\end{frame}



\begin{frame}
	\visible<10->{\mbox{}}
\end{frame}



\begin{frame}
	\visible<10->{\mbox{}}
\end{frame}



\begin{frame}
	\visible<10->{\mbox{}}
\end{frame}



\begin{frame}
	\visible<10->{\mbox{}}
\end{frame}



\begin{frame}
	\visible<10->{\mbox{}}
\end{frame}



\begin{frame}
	\visible<10->{\mbox{}}
\end{frame}



\begin{frame}
	\visible<10->{\mbox{}}
\end{frame}



\begin{frame}
	\visible<10->{\mbox{}}
\end{frame}



\begin{frame}
	\visible<10->{\mbox{}}
\end{frame}



\begin{frame}
	\visible<10->{\mbox{}}
\end{frame}



\begin{frame}
	\visible<10->{\mbox{}}
\end{frame}



\begin{frame}
	\visible<10->{\mbox{}}
\end{frame}



\begin{frame}
	\visible<10->{\mbox{}}
\end{frame}



\begin{frame}
	\visible<10->{\mbox{}}
\end{frame}



\begin{frame}
	\visible<10->{\mbox{}}
\end{frame}



\begin{frame}
	\visible<10->{\mbox{}}
\end{frame}



\begin{frame}
	\visible<10->{\mbox{}}
\end{frame}



\begin{frame}
	\visible<10->{\mbox{}}
\end{frame}



\begin{frame}
	\visible<10->{\mbox{}}
\end{frame}



\begin{frame}
	\visible<10->{\mbox{}}
\end{frame}



\begin{frame}
	\visible<10->{\mbox{}}
\end{frame}



\begin{frame}
	\visible<10->{\mbox{}}
\end{frame}



\begin{frame}
	\visible<10->{\mbox{}}
\end{frame}



\begin{frame}
	\visible<10->{\mbox{}}
\end{frame}



\begin{frame}
	\visible<10->{\mbox{}}
\end{frame}



\begin{frame}
	\visible<10->{\mbox{}}
\end{frame}



\begin{frame}
	\visible<10->{\mbox{}}
\end{frame}



\begin{frame}
	\visible<10->{\mbox{}}
\end{frame}



\begin{frame}
	\visible<10->{\mbox{}}
\end{frame}



\begin{frame}
	\visible<10->{\mbox{}}
\end{frame}



\begin{frame}
	\visible<10->{\mbox{}}
\end{frame}



\begin{frame}
	\visible<10->{\mbox{}}
\end{frame}



\begin{frame}
	\visible<10->{\mbox{}}
\end{frame}



\begin{frame}
	\visible<10->{\mbox{}}
\end{frame}



\begin{frame}
	\visible<10->{\mbox{}}
\end{frame}



\begin{frame}
	\visible<10->{\mbox{}}
\end{frame}



\begin{frame}
	\visible<10->{\mbox{}}
\end{frame}



\begin{frame}
	\visible<10->{\mbox{}}
\end{frame}



\begin{frame}
	\visible<10->{\mbox{}}
\end{frame}



\begin{frame}
	\visible<10->{\mbox{}}
\end{frame}



\begin{frame}
	\visible<10->{\mbox{}}
\end{frame}



\begin{frame}
	\visible<10->{\mbox{}}
\end{frame}



\begin{frame}
	\visible<10->{\mbox{}}
\end{frame}



\begin{frame}
	\visible<10->{\mbox{}}
\end{frame}



\begin{frame}
	\visible<10->{\mbox{}}
\end{frame}



\begin{frame}
	\visible<10->{\mbox{}}
\end{frame}



\begin{frame}
	\visible<10->{\mbox{}}
\end{frame}



\begin{frame}
	\visible<10->{\mbox{}}
\end{frame}



\begin{frame}
	\visible<10->{\mbox{}}
\end{frame}



\begin{frame}
	\visible<10->{\mbox{}}
\end{frame}



\begin{frame}
	\visible<10->{\mbox{}}
\end{frame}



\begin{frame}
	\visible<10->{\mbox{}}
\end{frame}



\begin{frame}
	\visible<10->{\mbox{}}
\end{frame}



\begin{frame}
	\visible<10->{\mbox{}}
\end{frame}



\begin{frame}
	\visible<10->{\mbox{}}
\end{frame}



\begin{frame}
	\visible<10->{\mbox{}}
\end{frame}



\begin{frame}
	\visible<10->{\mbox{}}
\end{frame}



\begin{frame}
	\visible<10->{\mbox{}}
\end{frame}



\begin{frame}
	\visible<10->{\mbox{}}
\end{frame}



\begin{frame}
	\visible<10->{\mbox{}}
\end{frame}



\begin{frame}
	\visible<10->{\mbox{}}
\end{frame}



\begin{frame}
	\visible<10->{\mbox{}}
\end{frame}



\begin{frame}
	\visible<10->{\mbox{}}
\end{frame}



\begin{frame}
	\visible<10->{\mbox{}}
\end{frame}



\begin{frame}
	\visible<10->{\mbox{}}
\end{frame}



\end{document}
