\documentclass[9pt,notheorems,xcolor=pdftex,dvipsnames,table]{beamer}
\usepackage[utf8]{inputenc}
%\usefonttheme{serif}
\usetheme{Madrid}
\definecolor{light-gray}{gray}{0.95}
\setbeamercolor{background canvas}{bg=light-gray}
\linespread{1.2}
\usepackage[absolute,overlay]{textpos}
\usepackage{graphicx}
\usepackage{soul}
\usepackage{ulem}
%\setbeamertemplate{sidebar right}{}
\setbeamertemplate{navigation symbols}{}
\makeatletter
\setbeamertemplate{footline}
{
  \leavevmode%
  \hbox{%
  \begin{beamercolorbox}[wd=.333333\paperwidth,ht=2.25ex,dp=1ex,center]{author in head/foot}%
    \usebeamerfont{author in head/foot}\insertauthor
  \end{beamercolorbox}%
   \begin{beamercolorbox}[wd=.333333\paperwidth,ht=2.25ex,dp=1ex,center]{author in head/foot}%
    \usebeamerfont{author in head/foot}
  \end{beamercolorbox}%
  \begin{beamercolorbox}[wd=.333333\paperwidth,ht=2.25ex,dp=1ex,right]{date in head/foot}%
    \usebeamerfont{date in head/foot}\insertshortdate{}\hspace*{2em}
    \insertframenumber{} / \inserttotalframenumber\hspace*{2ex} 
  \end{beamercolorbox}}%
  \vskip0pt%
}
\makeatother
\newcommand{\dd}{\partial}

%\setbeamertemplate{footline}{%
%\hfill\usebeamertemplate***{navigation symbols}
%\hspace{1cm}\insertframenumber{}/\inserttotalframenumber}

\usepackage{calligra}

\usepackage{tikz}
\usepackage{tikz-cd}

\newcommand{\lv}{\lVert}
\newcommand{\rv}{\rVert}

\DeclareMathOperator{\supp}{supp}
\DeclareMathOperator{\Ext}{Ext}
\DeclareMathOperator{\Aut}{Aut}
\DeclareMathOperator{\Ran}{Ran}
\DeclareMathOperator{\Prob}{Prob}
\DeclareMathOperator{\conv}{conv}
\DeclareMathOperator{\AR}{AR}
\DeclareMathOperator{\Homeo}{Homeo}
\newcommand{\acts}[1]{\stackrel{#1}{\curvearrowright}}
\newcommand{\cc}{C_c}
\renewcommand{\emph}[1]{{\textit{#1}}}
\renewcommand{\{}{\left\lbrace}
\renewcommand{\}}{\right\rbrace}
\newcommand{\C}{\mathbb{C}}
\newcommand{\R}{\mathbb{R}}
\newcommand{\Z}{\mathbb{Z}}
\newcommand{\A}{\mathcal{A}}
\newcommand{\Q}{\mathbb{Q}}
\renewcommand{\H}{\mathcal{H}}
\newcommand{\N}{\mathbb{N}}
\newcommand{\Gal}{\mathrm{Gal}}
\newcommand{\IN}{\mathrm{in}_{\leq}}
\newcommand{\lex}{\leq_{\mathrm{lex}}}
\newcommand{\lcm}{\mathrm{lcm}}
\newcommand{\wdots}[2]{ #1, \ldots ,#2 }
\renewcommand{\P}{\{ \wdots{p_1}{p_m} \}}
\newcommand{\K}{K[ \wdots{x_1}{x_n} ]}
\newcommand{\I}{\langle \wdots{p_1}{p_m}  \rangle}

\renewcommand{\hat}[1]{\widehat{#1}}

\newtheorem{theorem}{Sætning}
\newtheorem{definition}{Definition}
\newtheorem{example}{Example}


\usepackage{array}
\usepackage{amsthm,amsmath,amssymb}
\usepackage{multirow}
\usepackage{multicol}
\usepackage{url}

\title{Simplicty and uniqueness of trace of group $C^*$-algebras}
\subtitle{Bachelor thesis defense}
\author{Malthe Munk Karbo}
\institute{Advisor: Mikael Rørdam}
\date{February 3, 2017}


\newcommand{\divides}{\bigm|}
\newcommand{\ndivides}{%
  \mathrel{\mkern.5mu % small adjustment
    % superimpose \nmid to \big|
    \ooalign{\hidewidth$\big|$\hidewidth\cr$\nmid$\cr}%
  }%
}

\AtBeginSection[]
{
  \begin{frame}
    \frametitle{Table of Contents}
    \tableofcontents[currentsection]
  \end{frame}
}


\begin{document}
\frame{\titlepage}

\begin{frame}
\frametitle{Table of Contents}
\tableofcontents
\end{frame}

\section{Introduction, history and motivation}
\begin{frame}[t]
\frametitle{What the hell is my thesis about? Crossed products and dynamics}
\visible<2->{The main subject of this thesis is a theory describing the ideal structure of something called \emph{crossed products} associated to a certain type of dynamical systems.}

\mbox{}

\visible<3->{A natural question is: \textbf{What is the all the fuzz about?}} \visible<4->{The crossed product construction associates to a $C^*$-dynamical system $(A,G,\alpha)$ a $C^*$-algebra $A \rtimes_\alpha G$.}\visible<5->{ This construction is \underline{important}:}
\begin{itemize}
	\item<6-> It provides many important and interesting examples of new $C^*$-algebras.
	\item<7-> It encodes quite a bit of information about both $G$ and $A$ and the action $G \acts \alpha A$.
	\item<8-> The construction served as a way of answering many questions in the history of operator algebras.
	\item<9-> Ties together with classic theory of dynamics.
	\item<10-> Ties together with many areas in $C^*$-theory, including the group $C^*$-algebra construction.
\end{itemize}
\visible<11->{\textbf{Okay, but why do we care about the ideal structure of this construction?}}\visible<12->{ Many answers, most notably we want to do \underline{classification.}}

\visible<13->{And the most important question:}\visible<14->{ \textbf{Real world application?}}\visible<15->{\textbf{ (money?)}} \visible<16->{Probably not}
\end{frame}
\begin{frame}[t]
\frametitle{What the hell is my thesis about? Crossed products and dynamics}
	\visible<2->{To a $C^*$-dynamical system $(A,G,\alpha)$, our goal is to answer the following question:}
	\begin{center}
		\visible<3->{\emph{Question 1: What properties $P$ describes the ideal structure of $A \rtimes_\alpha G$?}}
	\end{center}
	\visible<4->{During this talk, the hope is that we will provide some answers to this question.} \visible<5->{\textbf{Answering question 1 in the above form is \underline{hard}}}. 
	
	\mbox{}

	\visible<5->{Instead, we break it into special cases}\visible<6->{ Some cases discussed today will be:}
	\begin{itemize}
		\item<7-> Which properties $P$ answers Question 1 when $A$ is abelian? 
		\item<8-> Which properties $P$ answers Question 1 when $G$ is abelian?
		\item<9-> Which properties $P$ answers Question 1 when $A = \C$?
		\item<10-> Which properties $P$ answers Question 1 when $G$ is discrete?
	\end{itemize}
	\visible<11->{And of course, we have overlaps between the above cases.}
\end{frame}
\section{Some notation and preliminaries}
\begin{frame}
\frametitle{Disclaimer}
	\visible<2->{The theory discussed today is quite heavy and dull for the uninitiated.}\visible<3->{ Sorry.}
\end{frame}
\begin{frame}[t]
\frametitle{A short primer on $C^*$-dynamical systems.}
\visible<2->{Lots of prerequisite theory needed to understand todays talk.}\visible<3->{ \textbf{Not} possible to cover all.}
\visible<4->{
\begin{block}{Definitin}
	An abstract $C^*$-algebra $A$ is a Banach algebra $A$ equipped with an involution such that 
	\begin{align*}
		\lv a \rv ^2 = \lv a^*a\rv,
	\end{align*}
	for all $a \in A$.
\end{block}
}
\visible<5->{Notable examples include}
\begin{itemize}
	\item<6-> The algebra $C_0(X)$ of continuous functions vanishing at infinity on a locally compact Hausdorff space $X$.
	\item<7-> The algebra $\mathbb{B}(\H)$ of bounded linear operators on a Hilbert space $\H$.
	\item<8-> The group $C^*$-algebras $C_r^*(G)$ and $C^*(G)$ of a locally compact Hausdorff group $G$.
\end{itemize}
\visible<9->{All of these $C^*$-algebras play a crucial role in the theory of crossed products.}
\end{frame}
\begin{frame}[t]
\frametitle{A short primer on $C^*$-dynamical systems.}
\visible<1->{
\begin{block}{Definitin}
	An abstract \emph{$C^*$-algebra} $A$ is an involutive Banach algebra $A$ satisfying 
	\begin{align*}
		\lv a \rv ^2 = \lv a^*a\rv,
	\end{align*}
	for all $a \in A$.
\end{block}
}
\visible<2->{
\begin{block}{Definition}
	A \emph{representation} of a $C^*$-algebra $A$ is a pair $\left( \pi, \H \right)$ consisting of a bounded $*$-homomorphism $\pi \colon A \to \mathbb{B}(\H)$.
\end{block}
}
\visible<3->{
	\begin{block}{The Gelfand-Neimark-Segal result}
	The Gelfand-Neimark-Segal result is an important result stating that all $C^*$-algebras $A$ are faithfully represented on a Hilbert space.
\end{block}
}
\end{frame}
\begin{frame}[t]
	\frametitle{A short primer on topological group theory}
	\visible<2->{
		\begin{block}{Definition}
			A topological group $G$ is topological space equipped with a group structure such that the maps $G \times G \ni (s,t) \mapsto st$ and $G \ni t \mapsto t^{-1} \in G$ are continuous for all $s,t \in G$.\visible<3->{ By $G$ we will \textbf{always} denote a topological group $G$ with a locally compact Hausdorff topology.}
		\end{block}
	}
	\visible<4->{Important cases include $\Z$, $\R$ and $S^1$.}
	\visible<5->{Topological group are important and arises naturally in many fields of mathematics.}\visible<6->{ We will need results from the theory of \textbf{abstract harmonic analysis}, \textbf{representation theory} and \textbf{Fourier analysis}}
	\visible<7->{
		\begin{block}{Definition}
			A unitary representation of a group $G$ is a pair $(u,\H)$ consisting of a group homomorphism $u \colon G \to \mathcal{U}(\H)$ such that for all $\xi \in \H$ the map $t \mapsto u_t \xi$ is continuous (= $u$ is strongly continuous).
		\end{block}
	}
\end{frame}
\begin{frame}[t]
	\frametitle{A short primer on topological group theory}
	\visible<1->{
		\begin{block}{Definition}
			A topological group $G$ is topological space equipped with a group structure such that the maps $G \times G \ni (s,t) \mapsto st$ and $G \ni t \mapsto t^{-1} \in G$ are continuous for all $s,t \in G$.\visible<1->{ By $G$ we will \textbf{always} denote a topological group $G$ with a locally compact Hausdorff topology.}
		\end{block}
	}
	\visible<1->{
		\begin{block}{Definition}
			A unitary representation of a group $G$ is a pair $(u,\H)$ consisting of a group homomorphism $u \colon G \to \mathcal{U}(\H)$ such that for all $\xi \in \H$ the map $t \mapsto u_t \xi$ is continuous (= $u$ is strongly continuous).
		\end{block}
	}
	\visible<2->{An important example is the \emph{left-regular representation} of $G$:}
\visible<3->{
	\begin{block}{Definition}
		The left-regular representation of $G$ is the tuple $(\lambda, L^2(G))$, where
		\begin{align*}
			\lambda_t \xi(s) = \xi(t^{-1}s), 
		\end{align*}
		for $\xi \in L^2(G)$ and $s,t \in G$.
	\end{block}
}
\end{frame}
\begin{frame}[t]
	\frametitle{The perhaps shortest primer on $C^*$-dynamical systems}
	\visible<2->{
		\begin{block}{Definition}
			A \emph{$C^*$-dynamical system} is a triple $(A,G,\alpha)$ consisting of
			\begin{itemize}
				\item<3-> A $C^*$-algebra $A$,
				\item<4-> A topological group $G$ and
				\item<5-> a strongly continuous group homomorphism $\alpha \colon G \to \mathrm{Aut}(A)$ ($\forall a \in A : G \ni  t \mapsto \alpha_t(a) \in A \text{ is continuous}$).
			\end{itemize}
		\end{block}
	}
	\visible<7->{
		\begin{block}{Definition}
			A covariant representation of $(A,G,\alpha)$ is a triple $(\pi, u , \H)$ consisting of representations of $A$ and $G$ as bounded operators on a common Hilbert space $\H$ satisfying the relation
			\begin{align*}
				u_t \pi(a) u_t^* = \pi(\alpha_t(a)) \text{ for all } a \in A \text{ and } t \in G.
			\end{align*}
		\end{block}
	}
	\visible<8->{When $G$ is discrete and $A$ is unital, it is easy to associate to such a triple a \textbf{nice} new $C^*$-algebra:}
\end{frame}
\begin{frame}[t]
	\frametitle{The perhaps shortest primer on $C^*$-dynamical systems and crossed products}
	\visible<1->{When $G$ is discrete and $A$ is unital, it is easy to associate to such a triple a \textbf{nice} new $C^*$-algebra:}
	\visible<2->{
		\begin{block}{Discrete crossed product cookbook recipe}
\begin{enumerate}
	\item<2-> Given $(A,G,\alpha)$ with $G$ discrete and $A$ unital, consider the $*$-algebra 
			\begin{align*}
				C_c(G,A) = \{ \sum_{t \in F} a_t \delta_t  \mid F \subseteq G \text{ finite} \}.
			\end{align*}
	\item<3-> Pick a covariant representation $(\pi,u,\H)$ of $(A,G,\alpha)$ (they exist!)
	\item<4-> Define a representation $\pi \rtimes u$ of $C_c(G,A)$ by
		\begin{align*}
			\pi \rtimes u \left( \sum a_t \delta_t \right) := \sum \pi(a_t) u_t	
		\end{align*}
	\item<5-> Take the closure of $\pi \rtimes u(C_c(G,A))$
\end{enumerate}
		\end{block}
	}
	\visible<6->{This is the building stone for constructing crossed products.} \visible<7->{Fails when $G$ is non-discrete.}
\end{frame}
\section{The general crossed product construction}
\end{document}
