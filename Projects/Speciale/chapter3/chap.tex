\chapter{The Group Spectra And Applications}
As previously mentioned, we will now begin describing the theory, both old and new, describing the ideal structure of crossed products $A \rtimes_\alpha G$ for topological groups $G$. We will begin by introducing some new structures and concepts associated to this theory.

Since we will describe theory which only describes dynamical systems $(A,G,\alpha)$ where $G$ is abelian and also theory which describes the general case, we introduce the notation that a triple $(A, G, \alpha)$ is a $C^*$-dynamic a-system if $G \acts \alpha A$ and $G$ is abelian. 

\sssection{The Arveson and Connes spectra}
For the time being, we will unless otherwise stated, assume that $G$ is abelian with dual group $\hat G$ as in the previous chapter. Recall that whenever we have a $C^*$-dynamical a-system $(A, G, \alpha)$, we also obtain an association of $L^1(G)$ as $A$-valued integral-weights via $ A \ni x \mapsto \alpha_f(x) \in A$ for $f \in L^1(G)$. In fact, for each bounded Radon measure $\mu \in M(G)$ we obtain an $A$-valued weight $\alpha_\mu$ satisfying
\begin{align*}
	\varphi(\alpha_f(a)) = \int_G \varphi(\alpha_t(x)) \d \mu(t), \text{ for all } a \in A \text{ and } \varphi \in A^*.
\end{align*}
We say that $\alpha_f$ and $\alpha_\mu$ are the \myemph{weak$^*$-integrals} of $f$ and $\mu$, respectively. Fore a more detailed exposition of vector-valued integration, see e.g., \cite[Appendix 3, Lemma 7.4.4]{pedersenalgauto} or \cite[Appendix A.3]{folland2016fourier}. For a group $G$ we define the subset $\cc^1(G)$ of $\cc(G)$ by
\begin{align*}
	\cc^1(G):= \{ f \in \cc(G) \mid \hat f \in \cc(\hat G)\},
\end{align*}
which is a dense ideal of $L^1(G)$, see \cite[Theorem 4.60]{folland2016fourier}.
\begin{definition}
	Let $(A, G, \alpha)$ be a $C^*$-dynamical system. For open sets $U \subseteq \hat G$, we define the \myemph{spectral $R$-subspace} $R^\alpha(U)$ to be the subspace of $A$ defined by
	\begin{align*}
		R^\alpha(U) := \overline{ \Span \left\{ \alpha_f(a) \mid a \in A, f \in \cc^1(G) \text{ with } \supp \hat f \subseteq U \right\} }^{w},
	\end{align*}
	and similarly, for each closed subset $K \subseteq \hat G$ we define the \myemph{spectral $M$-subspace} $M^\alpha(K)$ to be the subspace of $A$ defined by
	\begin{align*}
		M^\alpha(K) := \overline{\Span \left\{ x \in A \mid \alpha_f(x) = 0 \text{ for all } f \in \cc^1(G) \text{ with } \supp \hat f \subseteq K^c \right\}},
	\end{align*}
	i.e., as the annihilator of $R^{\alpha'}(K^c)$, where $\alpha'_f(\varphi)$ is the element of $A^*$ such that $\alpha'_f(\varphi)(x)=\varphi(\alpha_f(x))$ for all $x \in A$.
\end{definition}
By \cite[Theorem 8.1.4]{pedersenalgauto}, it follows that the $R$- and $M$-spaces are $G$-invariant subsets, and
\begin{align*}
	\bigcap_i M^\alpha(K_i) = M^\alpha(\bigcap_I K_i),
\end{align*}
for all collections of subsets $(K_i)$ of $\hat G$. This allows us to give the following definition:
\begin{definition}
	The \myemph{Arveson Spectrum} of an action $G \acts \alpha A$ is defined to be the smallest closed subset $K$ of $\hat G$ such that $M^\alpha(K) = A$, and we denote it by $\Sp \alpha$.
\end{definition}
In his dissertation (see \cite{connesclassification}), Alain Connes proved a series of equivalent characterizations of $\Sp \alpha$, which we will need. We include it without proof and instead refer the reader to \cite[Proposition 8.1.9 and 8.1.8]{pedersenalgauto} for the proof.
\begin{proposition}
	Let $(A, G, \alpha)$ be a $C^*$-dynamical a-system. Then for $\sigma \in \hat G$ the following are equivalent:
	\begin{enumerate}
		\item $\sigma \in \Sp \alpha$,
		\item For all neighborhoods $U$ of $\sigma$: $R^\alpha(U) \neq 0$,
		\item There is a net of unit vectors $(a_i) \subseteq A$ such that $\lv\alpha_t(x_i) - (t,\sigma)x_i\rv \to 0$ uniformly on compacta of $G$,
		\item For every finite Radon measure $\mu$ on $G$: $|\hat \mu(\sigma)| \leq \lv \alpha_\mu\rv$,
		\item For every $f \in L^1(G)$: $|\hat f (\sigma)| \leq \lv \alpha_f\rv$,
		\item If $f \in L^1(G)$, $\alpha_f=0$, then $\hat f(\sigma)=0$ and
		\item It holds that $M^{\alpha}(\{\sigma\}) := \{x \in A \mid \hat \alpha_t(x) = (t,\sigma)x \text{ for all } t \in G\} \neq 0$.
	\end{enumerate}
\end{proposition}
One very common use of the above is to take the last condition as the definition:
\begin{align*}
		\Sp \alpha := \left\{ \chi \in \hat G \mid \hat f(\chi) = 0 \text{ whenever } f \in L^1(G) \text{ satisfies } \alpha_f(x) = 0 \text{ for all } x \in A  \right\}.
	\end{align*}
We will frequently switch between the above equivalent conditions.

The main motivation (in our case) for introducing the Arveson spectrum is to define the Connes Spectrum. For this, we need to introduce the concept of hereditary sub-$C^*$-algebras:
\begin{definition}
	A $C^*$-subalgebra $B\subseteq A$ is \myemph{hereditary} if for all positive $x \in A_+$ we have $x \in B_+$ whenever $y \in B_+$ with $0 \leq x \leq y$. We define $\mathscr{H}(A)$ to be the set of non-zero hereditary $C^*$-subalgebras of $A$. 
	
	If furthermore we assume that $(A, G, \alpha)$ is a $C^*$-dynamical system, we defin $\mathscr{H}^\alpha(A)$ to be the set
	\begin{align*}
		\mathscr{H}^\alpha (A) := \{ D \in \mathscr{H}(A) \mid \alpha_g(D)=D \text{ for all } g \in G\}.
	\end{align*}
\end{definition}
\begin{remark}
	There is a bijection between the set of closed left-ideals of a $C^*$-algebra $A$ and the set of hereditary $C^*$-subalgebras of $A$ which maps a closed left-ideal $L$ to $L^* \cap L$, see e.g. \cite[II.5.3.2]{blackadar}.
\end{remark}
This bijection will be useful soon, since we may define the Connes spectrum of an action $G \acts \alpha A$ now:
\begin{definition}
	The \myemph{Connes Spectrum} of an action $G \acts \alpha A$ is the subset $\Gamma(\alpha)$ of $\hat G$ given by
	\begin{align*}
		\hat G ( \alpha) := \bigcap_{D \in \mathscr{H}^\alpha (A)} \Sp {\alpha|_D}.
	\end{align*}
\end{definition}
\sssection{The Connes Spectrum and Simplicity}
As we shall see, the Connes spectrum is very important in the theory of classification of crossed products, for instance it is an invariant under an equivalence relation called exterior equivalence (yet to be defined) and the Connes spectrum of the dual system detects ideals in $A$:
\begin{proposition}
	Let $(A, G, \alpha)$ be a $C^*$-dynamical a-system. Then $t \in G(\hat \alpha)$ if and only if $I \cap \alpha_t(I) \neq 0$ for all non-zero ideals of $A$.
	\label{connesideal1}
\end{proposition}
\begin{remark}
	The proof of this, due to Olesen and Pedersen, relies on the theory of $W^*$-dynamical systems, which quite similar to that of $C^*$-dynamical systems. We will use some properties without proof, as it will take us too much off course to develop properly, and instead refer the reader to relevant material on the subject whenever needed.	
\begin{definition}
	A \myemph{$W^*$-system} is a triple $(\mathscr{M},G,\beta)$ where $\mathscr{M}$ is a von-Neumann Algebra, $G$ a locally compact Hausdorff group and $\beta \colon G \to \mathrm{Aut}(\mathscr{M})$ is group homomorphism which is continuous with respect to the pointwise weak topology on $\mathrm{Aut}(\mathscr{M})$.
\end{definition}
	 We refer the curious reader to \cite[Chapter 7.4 and 7.10]{pedersenalgauto} for a thorough exposition of the subjects.
\end{remark}
\begin{proof}
	We will be considering $A \rtimes_\alpha G$ as a $G$-product, so that $A \subseteq M(A \rtimes\alpha G)$ is the set of elements satisfying Landstad's conditions \todo{Cref Landstad}. Assume that $I \cap \alpha_t(I) = 0$ for $0 \neq I \in \mathcal{I}(A)$. \todo{Find out how this works} Let $0 \neq x,y \in I_+$ be such that $yx=y$ (e.g., by considering $(x-\varepsilon)$ for some $ \lv x \rv >\varepsilon > 0$, which will belong to $I \subseteq A^1$ and suitable increasing function $f_\varepsilon$). Let $K$ be a compact neighborhood of $0 \in G$ such that $ \lv \alpha_s(x)-x	\rv < 1$ for $s \in K$. Then $1-(\alpha_s(x)-x)$ is invertible with inverse $\sum_{n=0}^\infty (\alpha_s(x)-x)^n$, see ee.g. \cite[Proposition 2.1]{zhu}. Then
	\begin{align*}
		\alpha_s(y)=\alpha_s(y)x \sum_{n=0}^\infty (\alpha_s(x)-x)^n \in I,
	\end{align*}
	since $\alpha_s(y)-\alpha_s(yx)=0$, implying that 
	\begin{align*}
	y a \alpha_{t-s}(y) = \alpha_{-s}(\alpha_s(y)\alpha_s(a)\alpha_t(y)) \in \subseteq \alpha_{-s}(I A \alpha_t(I)) = 0,
	\end{align*}
	for all $a \in A$ and $s \in K$. Every $y$ in $A$ satisfies $\hat\alpha(y) = y$ by Landstad's condition. Hence $B:= C^*(\{y z y \mid z \in A \rtimes_\alpha G\}) \subseteq A \rtimes_\alpha G$ is a $\hat \alpha$-invariant hereditary subalgebra, i.e., $B \in \mathscr{H}^{\hat \alpha}(A \rtimes_\alpha G)$. We claim that $t \not \in \Sp {\hat \alpha |_B}$: Pick any $g \in L^1(G)$ with continuous $\hat g$ such that $\mathrm{supp} \hat g \subseteq t-K$. We show that $\hat \alpha_g |_B = 0$ so that $t \not\in \Sp {\hat \alpha|_B}$, by density of $\hat {L^1(\hat G)}$ in $C_0(G)$; let $b \in B$ be of the form
	\begin{align*}
		b = y \lambda_f^* a \lambda_f y,
	\end{align*}
	for some $f \in L^1(G) \cap L^2(G)$ and $ a \in A \rtimes_\alpha G$, so that $b$ is $\hat \alpha$-integrable by \todo{add ref}. Then, by the Fourier Inversion formula, we have 
	\begin{align*}
		\hat \alpha_g(b) &= \int_{\hat G} y \hat \alpha_\sigma(\lambda_f^* a \lambda_f) y g(\sigma) \d \sigma \\
		&= \int_{\hat G} \int_G y \hat \alpha_\sigma(\lambda_f^* a \lambda_f) \overline{(s,\sigma)} y \hat g(s) \d s \d \sigma\\
		&= \int_{\hat G} \int_G y \hat \alpha_\sigma(\lambda_f^* a \lambda_f) \lambda_{-s} \lambda_s \overline{(s,\sigma)} y \hat g(s) \d s \d \sigma\\
		&= \int_{\hat G} \int_G y \hat \alpha_{\sigma}(\lambda_f^* a \lambda_f \lambda_{-s}) \lambda_s y \hat g(s) \d s \d \sigma\\
		&= \int_{G} y \left( \int_{\hat G} \hat \alpha_{\sigma}(\lambda_f^* a \lambda_f \lambda_{-s}\right) \lambda_s \d \sigma y \hat g(s) \d s\\
		&= \int_G y I(\lambda_f^* a \lambda_f \lambda_{-s}) \lambda_s y \hat g(s) \d s\\
		&= \int_G y I(\lambda_f^* a \lambda_f \lambda_{-s}) \alpha_s(y) \lambda_s \hat g(s)\d s = 0,
	\end{align*}
	where we used the identity of \cref{eq:olpe1lemma2.8} and covariance of $(\alpha_s,\lambda_s)$. The last equality follows from $I(\lambda_f^* a \lambda_f \lambda_{-s}) \in A$ by \cref{olpe1lemma2.8} and the fact that $y a' \alpha_s(y)= 0 $ for all $a' \in A$ and $s \in \mathrm{supp} \hat g$ by the above. We know that the set of elements of the form $y \lambda_f^* a \lambda_f y$ spans a dense subset of $B$, hence $\hat \alpha_g | B \neq 0$. Since $g$ can be chosen such that $\hat g (t) \neq 0$, we conclude that $t \not \in \Sp {\hat \alpha|_B}$, hence not in $G(\hat \alpha)$.

	For the converse, assume $t \not \in G(\hat \alpha)$, and let $U$ be a neighborhood of $t$, $B \in \mathscr{H}^{\hat \alpha}(A \rtimes_\alpha G)$  witness this, i.e., such that $\hat \alpha _g | B = 0$ whenever $g \in L^1(\hat G)$ with $\mathrm{supp} \hat g \subseteq U $. Choose a neighborhood $U' \subseteq U$ and a neighborhood $0 \in U_0$ such that $U' + U_0-U_0 \subseteq U$, doable choosing $U_0$ arbitrarily small. Given $f_1,f_2 \in L^1(G)$ with support contained in $U_0$, then for all $g \in L^1(\hat G)$ such that $\mathrm{supp} \hat g \subseteq U'$ we have
\begin{align*}
	\hat {((s,\cdot)g)}(t) = \int_{\hat G} (t+s, \sigma) g(\sigma) \d \sigma = \hat g(s+t)  \neq 0 \iff t \in \mathrm{supp} \hat g - s \subseteq U'+U_0-U_0 \subseteq U
\end{align*}
for all $s \in U_0-U_0$, so $sg := (s, \cdot) g$ will satisfy $\hat \alpha_g = 0$ on $B$. Hence, if $y \in B$ we have
\begin{align*}
	\hat \alpha_g(\lambda_{f_1}^* y \lambda_{f_2}) &=  \int_{G \times G \times \hat G} \lambda_{-r} \hat \alpha_\sigma \lambda_s (s-r,\sigma) g(\sigma) \overline{f_1}(r) f_2(s) \d s \d r \d \sigma\\
	&=  \int_{G \times G \times \hat G} \alpha_{-r}(\hat \alpha_\sigma(y)) \lambda_{-r} \lambda_{s+r}(s,\sigma) g(\sigma) \overline{f_1}(r) f_2(s+r) \d s \d r \d \sigma\\
	&= \int_{G} \int_G \alpha_{-r}(\hat \alpha_{sg}(y)) \lambda_s \overline{f_1}(r) f_2(s+r) \d s \d r\\
	&= 0,
\end{align*}
since $r \in \mathrm{supp} f_1$ and $s+r \in \mathrm{supp} f_2$ implies $s \in U_0-U_0$ so $\hat \alpha_{sg}(y) = 0$. Since $B$ is a hereditary $C^*$-subalgebra, there is a left ideal of $A \rtimes_\alpha G$, which necessarily must be $\hat G$ invariant by invariance of $B$, usch that $B = L^* \cap L$. Let $f \in L^1(G) \cap L^2(G)$, such that $f \neq 0 \in L^1(G)$ and with $\mathrm{supp} f \subseteq U_0$. Define
\begin{align*}
	L_0 :=	\overline{\Span {\left\{\bigcup_{\sigma \in \hat G} L \lambda_{\sigma f}\right\}}},
\end{align*}
so that $L_0$ is a closed $\hat G$-invariant left ideal of $A \rtimes_\alpha G$ giving rise to a hereditary $C^*$-subalgebra given by $B_0:= L_0^* \cap L_0 \in \mathscr H^{\hat \alpha}(A \rtimes_{\alpha}G)$. By the above computations, we see that $\hat \alpha_g(\lambda_{\sigma_1 f}^* y_1^* y_2 \lambda_{\sigma_2 f}) = 0$ for all $\sigma_i \in \hat G$ and $f_i \in L^1(G)$ whenever $g \in L^1(G)$ with $\mathrm{supp} \hat g \subseteq U'$. In particular, this means that $t \not\in \Sp {\hat \alpha|_{B_0}}$.

Let $\mathscr{M}_0$ denote the strong operator closure of $B_0$ in $\mathbb{B}(L^2(G,H))$ (recall that $G$ is abelian, so $A \rtimes_\alpha G \subseteq \mathbb{B}(L^2(G,H))$), and let $\beta$ denote the extension of $\hat \alpha$ making $(\mathscr{M}_0, \hat G, \beta)$ into a $W^*$-dynamical system, i.e., $\beta = \hat \alpha|_{\mathscr{M}_0}$. We define $R^{\beta}(\Omega)$ to be $\sigma$-weak closure of $R^{\hat \alpha}(\Omega)$ for open sets $\Omega \subseteq \hat G$. In particular, we have $\Sp {\hat \alpha|_{B_0}}= \Sp {\beta|_{\mathscr{M}_0}}$ , see e.g. \cite[Proposition 8.8.9]{pedersenalgauto}, so that $t \not \in \Sp {\beta|_{\mathscr{M}_0}}$.

Since $B_0$ is the closure of a set of integrable positive elements (see \cref{olpe1lemma2.7}), let $0 \neq y \in B_0$ be a positive and integrable element and let $x_0 := I(y_0)$. Then $x_0 \in \mathscr{M}_0$, for let $g_i \in \cc(\hat G)$ be a net of positive functions such that $g_i(\sigma)\to 1$ for all $\sigma \in \hat G$. Then, if $x_i := \hat \alpha_{g_i}(y_0) = \int_{\hat G} \hat \alpha_{\sigma}(y_0) g_i(\sigma) \d \sigma $, we have a net $(x_i) \subseteq B_0$ which increases strongly to $x_0$. Let $a \in A$. Then $x_0 x \lambda_t x_0 = \lim_{i-\text{strong}} x_i x \lambda_t x_i \in \mathscr{M}_0$. By \cref{olpe1lemma2.8}, $x_0 \in A$, and hence for all $\sigma \in \hat G$:
\begin{align*}
	\beta_{\sigma}(x_0 x \lambda_t x_0) = \hat \alpha_\sigma( x_0 x \lambda_t x_0) = x_0 x \hat \alpha_\sigma(\lambda_t) x_0 =  (t,\sigma) x_0 x \lambda_t x_0,
\end{align*}
by Landstad's condition and the properties of a $G$-product. However, the only element of $\mathscr{M}_0 \subseteq A \rtimes_\alpha G$ which satisfies $\hat \alpha_\sigma (a)=(t,\sigma) a$ is $0$, since $t \not \in \Sp {\hat \alpha |_{\mathscr{M}_0}}$.

In particular,
\begin{align*}
	x_0 x \alpha_t(x_0) = (x_0 x \lambda_t x_0) \lambda_{-t} = 0,
\end{align*}
for all $x \in A$. If $I = \overline{(x_0)} \subseteq A$ is the smalles ideal containing $x_0$, then $0 \neq I \subseteq A$ with $I \cap \alpha_t(I)=I\cdot\alpha_t(I)=0$.
\end{proof}
We now define an equivalence on the of $(A,G)$-$C^*$-dynamical systems, i.e., the class of $C^*$-dynamical systems $(A, G, \alpha)$ for fixed $(A,G)$. 
\begin{definition}
	We say that the $C^*$-dynamical systems $(A,G,\alpha)$ and $(A, G, \alpha')$ are \myemph{exterior equivalent} if there is a $C^*$-dynamical system $(M_2(A), G, \lambda)$ such that
	\begin{align*}
		\lambda_t (x \oplus y) = \alpha_t(x) \oplus \alpha'_t(y),
	\end{align*}
	for all $x,y \in A$ and $t \in G$.
\end{definition}
\begin{lemma}
	A necessary and sufficient condition for $(A, G , \alpha)$ and $(A, G , \alpha')$ being exterior equivalent is that there exists a strongly continuous map $G \to \mathcal{U} M(A)$ such that
	\begin{enumerate}
		\item $u_{s+t}=u_s\alpha_s(u_t)$ for all $s,t \in G$ and\\
		\item $\alpha_t' = (\mathrm{Ad} u_t ) \alpha_t$.
	\end{enumerate}
\end{lemma}
\begin{proof}
Suppose that $(A, G , \alpha)$ and $(A,G,\alpha')$ are exterior equivalent. Extend $\lambda$ to $M(M_2(A))$, and define the operator
\begin{align*}
	m_t:=\lambda_t\begin{pmatrix}
		0 & 0 \\
		1 & 0
	\end{pmatrix}.
\end{align*}
We then see that 
\begin{align*}
	m_t^* m_t = \lambda_{t} \begin{pmatrix}
		1 & 0 \\
		0 & 0 
	\end{pmatrix} =  \begin{pmatrix}
		1 & 0 \\
		0 & 0
	\end{pmatrix},
\end{align*}
since $\lambda_t \in \mathrm{Aut}(M(M_2(A)))$. Similarly $m_tm_t^* = \begin{pmatrix}
	0 & 0\\
	0 & 1
\end{pmatrix}$. It follows that $m_t = \begin{pmatrix}
	0 & 0 \\
	u_t & 0
\end{pmatrix}$
for some unitary operator $u_t \in M(A)$, $t \in G$. It is easy to verify that $u_t$ satisfies the desired properties. For the converse, define $\lambda_t$ on $M_2(A)$ by
\begin{align*}
	\lambda_t \begin{pmatrix}
		a & b \\
		c & d
	\end{pmatrix}:=
	\begin{pmatrix}
		\alpha_t(a) & \alpha_t(b) u_t^*\\
		u_t \alpha_t(c) & \alpha'_t(d)
	\end{pmatrix}
\end{align*}
and check that this is indeed an action.
\end{proof}
Traditionally, the second condition was called exterior equivalence while the first is called the \myemph{Cocycle equation}. In his thesis (\cite[2.2.5]{connesclassification}), Connes showed that whenever $p\sim q$ are $G$-invariant equivalent projections in $M(A)$ (in the traditional sense, i.e., $p = v^*v$ and $ q = v v^*$ for a partial isometry $v$), then
\begin{align*}
	\hat G (\alpha |_{pAp}) = \hat G (\alpha|_{qAq}),
\end{align*}
see also \cite[lemma 4.3]{olesenpedersen1}. This is particularly useful for us, since we obtain:
\begin{proposition}
	If $(A,G,\alpha)$ and $(A, G , \alpha')$ are exteriorly equivalent, then $\hat G(\alpha) = \hat G(\alpha')$.
\end{proposition}
\begin{proof}
	Let $p = 1 \oplus 0$ and $q = 0 \oplus 1$ in $M_n(M(A))$. Then $p \sim q$ and both are $G$-invariant, and clearly $\alpha = \gamma | _{p(M_2(A))p}$ and $\alpha' = \gamma|_{q (M_2(A))q}$, so
	\begin{align*}
		\hat G(\alpha) = \hat G ( \gamma|_{p(M_2(A)p}) = \hat G(\gamma|_{q(M_2(A))q}) = \hat G(\alpha').
	\end{align*}
\end{proof}
\begin{note}
	If $(A, G, \alpha)$ and $(B, G, \beta)$ are covariantly isomorphic $C^*$-dynamical systems, then (after identifying $A$ with $B$), we see that $(A,G, \alpha)$ and $(A, G, \beta)$ are trivially exteriorly equivalent.
\end{note}
\begin{theorem}
	Let $(A,G,\alpha)$ be a $C^*$-dynamical system. Then $\hat G(\alpha)=\hat G (\hat{\hat \alpha})$ for $(A, G, \alpha)$ and $(A \rtimes_\alpha G \rtimes_{\hat \alpha} \hat G, G, \hat{\hat \alpha})$.
	\label{dualactionspectrum}
\end{theorem}
\begin{proof}
	First, let $I$ be the trivial representation of $G$ on $\mathbb{K}(L^2(G))$. Then $\hat G(\alpha) = \hat G(\alpha \otimes I)$:
	Let $\{\xi_i\}$ be a basis of $L^2(G)$, and let $p_i$ be the minimal projection onto $\C \xi_i$. Let $\iota_i$ be the hereditary embedding of $A$ into $A \otimes \mathbb{K}(L^2(G))$, $a \mapsto a \otimes p_i$. Since $A \subseteq A \otimes \mathbb{K}(L^2(G))$ as a hereditary $C^*$-subalgebra, $\hat G(\alpha \otimes I) \subseteq \hat G(\alpha)$. (In fact it is a good exercise to show that given a projection in $p$ a $C^*$-algebra $A$, the cut-off $C^*$-algebra $pAp \subseteq A$ is hereditary, however we instead refer to \cite[Example II.7.3.14]{blackadar})
	
	For the other inclusion, note that given any non-zero $G$-invariant hereditary $C^*$-subalgebra $B \subseteq A \otimes \mathbb{K}(L^2(G))$, we have $\iota_i(A) \cap B \neq 0$ for some $i$. For else, given $0 < b \in B$, we have $b_i := (1 \otimes p_i) b (1\otimes p_i)  \in \iota_i(A) \cap B= 0$ for all $i$ (apply hereditarity of $B$ to the opposite cutoff and $b$). From this we infer that $b_i \eta \otimes \xi_i = 0$ for all $i$ and all $\eta \in H$, where $A \subseteq \mathbb{B}(H)$. Since $\{\xi_i\} $ is a basis, this implies that $b = 0$. In particular, $\Sp {\alpha \otimes I |_B} \supseteq \Sp { \alpha |_{\iota_i^{-1}(B \cap \iota_i(A))}}$, so that $\hat G(\alpha \otimes I) \supseteq \hat G(\alpha)$.

	We already know that $\hat G(\hat{\hat \alpha}) = \hat G(\alpha \otimes \mathrm{Ad} \rho)$ by Takai duality. Hence it suffices to show that
	\begin{align*}
		(A \otimes \mathbb{K}(L^2(G)), G , \alpha \otimes I) \sim_{\mathrm{ext}} (A \otimes \mathbb{K}(L^2(G)), G , \alpha \otimes \mathrm{Ad} \rho).
	\end{align*}
	To see this, let $u_t := 1 \otimes \rho_t \in M(A \otimes \mathbb{K}(L^2(G)))$. Clearly $(\alpha_s \otimes I) (u_t) = \alpha_s(1) \otimes \rho_t = 1 \otimes \rho_t$, so $u_{s+t}= u_s \alpha_s(u_t)$. Also clearly $\alpha \otimes \mathrm{Ad} \rho = \mathrm{Ad} u \circ \alpha \otimes I$, and $t \mapsto u_t$ is equally clearly strongly continuous. We conclude that 
	\begin{align*}
		\hat G ( \hat {\hat \alpha}) = \hat G ( \alpha \otimes \mathrm{Ad} \rho) = \hat G( \alpha \otimes I) = \hat G( \alpha),
	\end{align*}
	finishing the proof.
\end{proof}
Which gives us the first ingredient in determining the ideal structure of $A \rtimes_\alpha G$:
\begin{corollary}
	Let $(A,G,\alpha)$ be a $C^*$-dynamical system. Then $\sigma \in \hat G ( \alpha)$ if and only if $I \cap \hat \alpha_\sigma (I) \neq 0$ for all non-zero ideals $I \subseteq A \rtimes_\alpha G$.
	\label{olpe1.5.4}
\end{corollary}
\begin{proof}
	Apply \cref{connesideal1} to $\hat{ \hat{ \alpha}}$ and combine it with \cref{dualactionspectrum} to obtain the desired result.
\end{proof}
\begin{definition}
	\todo{include Prime $A$ in prelim} Let $G \acts \alpha A$. We say tha $A$ is \myemph{$G$-prime} if whenever $I,J$ are non-zero $G$-invariant ideals in $A$ it holds that $I \cap J \neq 0$.
\end{definition}
\begin{remark}
	Analogously with the regular definition of a prime $C^*$-algebra, the definition of $A$ being $G$-prime is equivalent to every non-zero $G$-invariant ideal of $A$ being essential or if $\alpha_t(x) A \alpha_s(y)=0$ for all $s,t \in G$ implies $x = 0$ or $y = 0$. We will not go into details with this, the details are almost identical to the ones for regular primitivity of $A$.

	Also worth noting is that if $(A,G,\alpha)$ and $(B, \Gamma, \beta)$ are covariantly isomorphic (i.e., there is a $G$-equivariant isomorphism $\varphi \colon A \to B$), then $A$ is $G$-prime if and only if $B$ is $\Gamma$-p, obviously.
\end{remark}
With this, we obtain:
\begin{lemma}
	Let $(A,G,\alpha)$ be a $C^*$-dynamical system. Then the following are equivalent:
	\begin{enumerate}
		\item $A$ is prime;
		\item (a) $A$ is $G$-prime and (b) $G(\hat \alpha ) =G$.
	\end{enumerate}
	\label{olpe1.3.4}
\end{lemma}
\begin{proof}
	That $A$ is prime implies $A$ being $G$-prime is obvious and the last implication follows directly from \cref{connesideal1}.

	Assume towards a contradiction that (a) and (b) holds and $A$ is not prime, and let $I, J$ be non-zero ideals of $A$ such that $I \cap J = 0$. If $t \in G$ is such that $\alpha_t(I) \cap J \neq 0$, then, since $-t \in G(\hat \alpha)$, it must hold that with $L := \alpha_t(I)) \cap J$, we have $L \cap \alpha_{-t}(L) \neq 0$, but $L \cap \alpha_{-t}(L) \subseteq  J \cap I = 0$, so $\alpha_t(I) \cap J = 0$. Let $I_G$ be the smallest ideal containing $\bigcap_{t \in G} \alpha_t(I)$ and note that in particular $I_g$ is a non-zero $G$-invariant ideal of $A$, hence essential by (a). Then $I_G \cap J = 0$, but since $I_G$ is essential in $A$, this implies that $J = 0$, a contradiction. Hence $A$ is a prime $C^*$-algebra.
\end{proof}
\begin{lemma}
	Let $(A, G, \alpha)$ be a $C^*$-dynamical system. Then $A$ is $G$-prime if and only if $A \rtimes_\alpha G$ is $\hat G$-prime.			
	\label{olpe1.5.7}
\end{lemma}
\begin{proof}
	It is quite easy to see that if $I,J$ are non-zero $\hat G$-invariant ideals of $A \rtimes_\alpha G$ with $I \cap J = 0$, then whenever $f \in L^1(G) \cap L^2(G)$, $\lv f \rv_1  \neq 0$ and $x \in I_+$ and $y \in J_+$ are non-zero, then by \cref{olpe1lemma2.8} we have $a := I(\lambda_f^* y \lambda_f) \in A$ and $b := I(\lambda_f^* x \lambda_f)$ since $A$ are the elements of $M(A \rtimes_\alpha G)$ satisfying Landstad's condition. It is not hard to see that $a z b = 0$ for all $z \in M(A \rtimes_\alpha G)$ (in partiular for all $z \in A \subseteq M(A \rtimes_\alpha G)$), since $\hat \alpha_{\sigma}(\lambda_f^* y \lambda_f) z \hat \alpha_{\chi}(\lambda_f^* x \lambda_f) = 0$ for all $z \in M(A \rtimes_\alpha G)$. In particular, by covariance of $\lambda$ and $\alpha$, we have
	\begin{align*}
		\alpha_s(a) z \alpha_t(b) = \lambda_s a (\lambda_{-s} z \lambda_t) b \lambda_{-t}=0,
	\end{align*}
	so if $I'$ and $J'$ are the smallest ideals generated by $\{ \alpha_s(a) \mid s \in G\}$ and $\{ \alpha_s(b) \mid s \in G\}$, respectively, which must be $G$-invariant, then $I'\cdot J' = 0$ implying that $A$ is not $G$-prime.

	For the converse, given non-zero $G$-invariant ideals $I,J$ of $A$ then $J':=J \otimes \mathbb{K}(L^2(G))$ and $I':=I \otimes \mathbb{K}(L^2(G))$ are $G$-invariant ideals of $A \otimes \mathbb{K}(L^2(G))$ with respect to the action $\alpha \otimes \mathrm{Ad} \rho$, clearly satisfying $ J'\cap I' = 0$. By Takai duality, this implies that $A \rtimes_\alpha G$ is not $\hat G$-prime. 
\end{proof}
Combining the previous round of results, we obtain a nice result:
\begin{corollary}
	For a $C^*$-dynamical system $(A,G,\alpha)$, the following are equivalent:
	\begin{enumerate}
		\item $A \rtimes_\alpha G$ is prime;
		\item (a) $A$ is $G$-prime and (b) $\hat G(\alpha) = \hat G$.
	\end{enumerate}
	\label{olpe1.5.8}
\end{corollary}

Analogously to the case of primitivity and $G$-primitivity, we introduce the notion of $G$-simplicity of a dynamical system:
\begin{definition}
	Let $(A,G,\alpha)$ be a $C^*$-dynamical system. We say that $A$ is \myemph{$G$-simple} if there are no non-trivial $G$-invariant ideals $J$ of $A$.
\end{definition}
It is clear that $G$-simple is a stronger requirement than being $G$-prime and is more closely related to simplicity of $A$. We will now see that the set of $G$-invariant ideals of $A$ is in bijection with the set of $\hat G$-invariant ideals of the crossed product $A \rtimes_\alpha G$, and list a necessary and sufficient condition for simplicity of $A \rtimes_\alpha G$ for a large class of groups, but first, we have the following:
\begin{proposition}
	Given a $C^*$-dynamical systems $(A,G,\alpha)$, it holds that $A$ is $G$-simple if and only if $A \rtimes_\alpha G$ is $\hat G$-simple.
	\label{olpe1.6.1}
\end{proposition}
\begin{proof}
	Suppose that $A \rtimes_\alpha G$ is not $\hat G$-simple and choose a non-trivial $\hat G$-invariant ideal $J$ witnessing this. Then there is some state $\varphi \in \mathcal{S}(A \rtimes_\alpha G)$ such that $\varphi(J) = 0$, i.e., by letting $\varphi$ be the state obtained by the representation with kernel $J$. Extend $\varphi$ to all of $M(A \rtimes_\alpha G)$ and note that $\varphi(A) \neq 0$: Wenever $u_i$ is an approximate identity of $A$ we have $u_i x \to x$ for all $x \in A \rtimes_\alpha G$, which by Cauchy-Schwarz gives
	\begin{align*}
	|\varphi(u_i^* x)| \leq \varphi(u_i^*u_i)\varphi(x^*x),
	\end{align*}
	implying that if $\varphi(A) = 0$ then $\varphi(x) = \lim \varphi(u_i^* x) \leq 0$ for all $x \in A \rtimes_\alpha G$ contradicting our assumptions.
	
	Let now $N$ be the subset of $A$ (by \cref{landstadcross} and \cref{olpe1lemma2.8}) given by
	\begin{align*}
		N:=\{ I(y) \mid y \in J \text{ is } \hat \alpha-\text{integrable}\},
	\end{align*}
	so that $\varphi(N) = 0$ by $\hat G$-invariance of $J$, implying that $N \subseteq A$ is not dense. $*$-linearity of $I$ shows that $N$ is a $*$-subspace, and since the action on $A$ is given by $\alpha_t(a) = \lambda_t a \lambda_{-t}$ and every element are $\hat \alpha$-invariant, we see that $a I(y) b = I(ayb) \in N$ and $\alpha_t(I(y)) = \lambda_t I(y) \lambda_{-t} = I(\lambda_t y \lambda_{-t}) \in N$ for all $I(y)\in N$, $t \in G$ and $a,b \in A$. We conclude that $N$ is an algebraic $*$-ideal of $A$, which is non-zero since $I(\lambda_f^* y \lambda_f) \in N$ for all $y \in J$ and $f \in L^1(G) \cap L^2(G)$ by \cref{olpe1lemma2.8}. Thus the closure of $N$ is a non-trivial $G$-invariant ideal of $A$.

	For the converse, if $J$ is a non-zero $G$-invariant ideal of $A$, then $J \otimes \mathbb{K}(L^2(G))$ is a non-zero $\hat G$-invariant ideal of $A \otimes \mathbb{K}(L^2(G))$ with respect to the action $\alpha \otimes \mathrm{Ad} \rho$, and the result follows from Takai duality.
\end{proof}
Allowing us to, analogously with the case of primitivity, to obtain the following:
\begin{proposition}	
	For a $C^*$-dynamical system $(A,G,\alpha)$, if $A$ is simple then
\begin{enumerate}
	\item $A \rtimes_\alpha G$ is $\hat G$-simple;
	\item $G(\hat \alpha) = G$.
\end{enumerate}
\end{proposition}
\begin{proof}
	If $A$ is simple, then $A$ is in particular $G$-simple implying that $A \rtimes_\alpha G$ is $\hat G$-simple by \cref{olpe1.6.1}. Similarly, if $A$ is simple then $A$ is prime so $G(\hat \alpha) = G$ by \cref{olpe1.3.4}.
\end{proof}
\begin{proposition}
	For a $C^*$-dynamical system $(A,G,\alpha)$, if $A \rtimes_\alpha G$ is simple then 
	\begin{enumerate}
		\item $A$ is $G$-simple;
		\item $\hat G(\alpha)= \hat G$.
	\end{enumerate}
	\label{olpe1.6.3}
\end{proposition}
\begin{proof}
	The first condition follows again from \cref{olpe1.6.1}. The second condition follows from \cref{olpe1.5.8} since simplicity implies primitivity.
\end{proof}
We proceed to show that whenever $G$ is discrete, then the converse of the above holds. For this, we will apply the fact that $G$ is discrete if and only if $\hat G$ is compact with the following:
\begin{lemma}
	Suppose that $G$ is a compact group and $G \acts \alpha A$. If $A$ is prime and $G$-simple then $A$ is simple.	
	\label{olpe1.6.4}
\end{lemma}
\begin{proof}
	ASsume that $I$ is a non-trivial ideal of $A$, and choose non-zero elements $x,y \in I_+$ with $yx=y$ as in the proof of \cref{connesideal1}. Let $0 \in U$ be a neighborhood such that 
	\begin{align*}
		\lv \alpha_s(x)-x\rv < 1, \text{ for all }s \in U,
	\end{align*}
	so that $\alpha_s(y) \in I$ for all $s \in U$, as per the proof of \cref{connesideal1}. Let $t_1,\dots,t_n \in G$ be such that $\{ U - t_i\}$ covers $G$. Recall that primitivity implies that for all non-zero $a,b \in A$ there is $z \in A$ with $azy \neq 0$ (or equivalently, $aAb = 0 $ implies $a=0$ or $b = 0$). Recursively, find $a_1,\dots,a_{n-1} \in A \\ \{0\}$ such that
	\begin{align*}
	0 \neq	y_0 = \alpha_{t_1}(y) a_1 \alpha_{t_2}(y) a_2\cdots \alpha_{t_{n-1}}(y) a_{n-1} \alpha_{t_n}(y).
	\end{align*}
	If $t \in G$, then there is $1 \leq k \leq n$ such that $t-t_k \in U$ and hence $\alpha_t(y_0) = \cdots \alpha_{t+t_k}(y) \dots \in I$, implying that the orbit $\{\alpha_s(y_0) \mid s \in G\}$ is contained in $I$. Let now $I_0$ be the smallest ideal containing this orbit, then $y_0 \in I_0 \subseteq I \neq A$ contradicting the assumption that $A$ is $G$-simple, since $I_0$ is $G$-invariant. Hence $A$ is simple.
\end{proof}
\begin{theorem}
	Let $(A, G ,\alpha)$ be a $C^*$-dynamical system with $G$ discrete, then the following are equivalent:
	\begin{enumerate}
		\item $A \rtimes_\alpha G$ is simple;
		\item (a) $A$ is $G$-simple and (b) $\hat G(\alpha) = \hat G$.
	\end{enumerate}
	\label{olpe1thm}
\end{theorem}
\begin{proof}
	The first implication is \cref{olpe1.6.3}. 
	
	For the converse, note that $A \rtimes_\alpha G$ is prime by \cref{olpe1.5.8} and by \cref{olpe1.6.1} it is also $\hat G$-simple. Since $G$ is discrete, $\hat G$ is compact and \cref{olpe1.6.4} implies that $A \rtimes_\alpha G$ is simple.
\end{proof}
We will now see that the Connes spectrum, in a way, measures how well $A$ sees ideals of $A \rtimes_\alpha G$ whenever $G$ is discrete and abelian.
\begin{definition}
	Let $A$ be a $C^*$-subalgebra of a $C^*$-algebra $B$. We say that $A$ \myemph{detects ideals} in $B$ if $J \cap A = 0$ happens if and only if $J = 0$ for each ideal $J$ of $B$. Similarly, we say that $A$ \myemph{separates ideals} in $B$ if $I \cap A = J \cap A$ implies $I=J$ whenever $I,J$ are ideals of $B$.
\end{definition}
\begin{lemma}
	Let $(A,G,\alpha)$ be a $C^*$-dynamical system and assume that for all ideals $0 \neq I \subseteq A$ and each $t \in G$ we have $I \cap \alpha_t(I) \neq 0$. Then for each non-zero ideal $0 \neq J \subseteq A$ and each compact $K \subseteq G$ there is $z \in J \backslash\{0\}$ such that $\alpha_t(z) \in J$ for all $t \in K$. 
	\label{olpe2.2.1}
\end{lemma}
\begin{proof}
	Choose $0 \neq x,y \in J_+$ such that $yx=x$ and a neighborhood $0 \in U \subseteq G$ such that $\lv \alpha_t(x)-x\rv < 1$ for all $t \in U$ so that $\alpha_t(y) \in J$ whenever $ t \in U$. Let now $t_1,\dots,t_n \in G$ be such that $K \subseteq \bigcup_{1 \leq i \leq n} (U-t_n)$, by compactness of $K$. Assume that for some $2 \leq m < n$ we can find $a_2,\dots,a_m \in A$ such that $z_1:= \alpha_{t_1}(y) \neq 0$ and $z_m = z_{m-1} a_m \alpha_{t_m}(y) \neq 0$. We claim that then there is $a_{m+1} \in A$ such that $z_{m+1}=z_{m}a_{m+1}\alpha_{t_{m+1}}(y) \neq 0$:

	Suppose not, so that $z_m A \alpha_{t_{m+1}}(y) = 0$. Then $z_m A \alpha_{t_{m+1}-t_m}(z_m) \subseteq  z_m A \alpha_{t_{m+1}}(y)=0$, by assumption. Since $z_m \neq 0$, if $I$ is the ideal generated by $z_n$, then $I \cap \alpha_{s}(I) = I \alpha_{s}(I)= 0$, contradicting our assumption. Hence we can pick $a_{m+1} \in A$ such that $z_{m+1} = z_m a_{m+1} \alpha_{t_{m+1}}(y) \neq 0$. Let $z = z_m$, so that $\alpha_t(z) \in J$ for all $t \in K$, since $t +t_m \in U$ for some $m$ so $\alpha_{t+t_{n}}(m)$ is a factor of $\alpha_t(z)$. By choosing $t_1,\dots,t_n$ smart (e.g., as elements of $-K$), we may without loss of generality assume that $0 \in K$, so that $z \in J$, finishing the proof.
\end{proof}
\begin{proposition}
Let $(A,G,\alpha)$ be a $C^*$-dynamical system and assume that for each non-zero ideal $0 \neq I \subseteq A$ it holds that $I \cap \alpha_t(I) \neq 0 $ for all $t \in G$. If $0 \neq J \subseteq A$ is an ideal invariant under a subgroup $G_0$ of $G$ with compact quotient $G/G_0$, then there is a non-zero $G$-invariant ideal $J_0 \subseteq J$.
	\label{olpe2.2.2}
\end{proposition}
\begin{proof}
	If $q$ denotes the quotient map $G \to G/G_0$, then, by compactness of the quotient, there is a finite number of pre-compact sets of $G$ whose image cover $G/G_0$, in particular, we may choose $E \subseteq G$ compact such that $G/G_0 = q(E)$. Choose $x\in J$ satisfying \cref{olpe2.2.1}. Each $s \in G$ is of the form $t+s_0$ for some $t \in K$ and $s_0 \in G_0$, since $q(E) = G/G_0$. In particular, this means that $\alpha_s(x) = \alpha_{s_0}(\alpha_t(x)) \in \alpha_{s_0}(J) \subseteq J$, by assumption and our choice of $x$. Hence the smallest ideal generated by the orbit $\{ \alpha_t(x) \mid t \in G\}$ will be a non-zero $G$-invariant ideal contained in $J$.
\end{proof}
\begin{proposition}
	Let $(A,G,\alpha)$ be a $C^*$-dynamical system with $G$ discrete. Then $\sigma \in \hat G(\alpha)$ if and only if given any non-zero ideal $J$ of $A \rtimes_\alpha G$ there is a non-zero $\hat \alpha_\sigma$-invariant ideal $J_0 \subseteq J$.
	\label{olpe2.2.4}
\end{proposition}
\begin{proof}
	We know that $\sigma \in \hat G(\alpha)$ if and only if $\hat \alpha_\sigma(I) \cap I \neq 0$ for all non-zero ideals $I\subseteq A \rtimes_\alpha G$, cf. \cref{olpe1.5.4}. Note that since $G$ is assumed discrete, $\hat G$ is compact and since $\hat G(\alpha)$ is closed in $\hat G$, this implies that $\hat G$ is compact as well. By \cref{olpe2.2.2}, since $\hat G(\alpha) /\{0\}$ is compact and the conditions are satisfied, there is a non-zero $\hat G(\alpha)$-invariant ideal $I_0\subseteq I$, hence in particular it will be $\hat \alpha_\sigma$-invariant. 
\end{proof}
We summerize, for discrete groups (though some of them are also true for non-discrete groups), our characterization of the Connes spectrum in the following theorem:
\begin{theorem}
	Let $(A,G,\alpha)$ be a $C^*$-dynamical system with $G$ discrete. Then the following are equivalent:
	\begin{enumerate}
		\item $\hat G(\alpha) = \hat G$.
		\item For each $\sigma \in \hat G$ and each non-zero ideal $I \subseteq A \rtimes_\alpha G$ we have $I \cap \hat \alpha_\sigma(I) \neq 0$.
		\item For each non-zero ideal $I \subseteq A \rtimes_\alpha G$ there is a non-zero $\hat G$-invariant ideal $I_0 \subseteq I$.
		\item $A$ detects ideals in $A \rtimes_\alpha G$.
	\end{enumerate}
	\label{olpe2.2.5}
\end{theorem}
\begin{proof}
	The implication (1) $\iff$ (2) is \cref{olpe1.5.4} and the implication (2) $\implies$ (3) is \cref{olpe2.2.2}. We show that (3) $\implies$ (4) $\implies$ (2) for discrete groups:

	Assume (3) and let $I$ be a non-zero ideal of $A \rtimes_\alpha G$, which without loss of generality may be assumed to be $\hat G$-invariant by (3). If $0 \neq x \in I_+$, then $a = \int_{\hat G} \hat \alpha_{\sigma}(x) \d \sigma$ is a non-zero element of $I_+$ which is fixed under $\hat G$. By \cite[lemma 3.2]{landstad1979towards}, for discrete groups $G$, the elements of $A \rtimes_\alpha G$ which are fixed by $\hat G$ correspond exactly to $A$, hence $a \in A \cap I$, showing (4).

	Assume (4)., then given $\sigma \in \hat G$ and a non-zero ideal $I \subseteq A \rtimes_\alpha G$, if $a \in A \cap I$, then again by \cite[lemma 3.2]{landstad1979towards} we have $a = \hat \alpha_\sigma(a)$, so $a \in I \cap \hat \alpha_\sigma(I)$, showing (2).
\end{proof}

\Cref{olpe1thm} of Olesen and Pedersen is from 1978, and was one of the first results characterizing simplicity by the means of Connes spectrum. In year after, following a paper of Gootman and Rosenberg (see \cite{gootman1979structure}) on the Generalized Effros-Hahn conjecture, Olesen and Pedersen improved it for the case where $A$ and $G$ are separable to include a larger class of groups, including $\R$ and $S^1$, based on the previous results. We include a corollary of their main theorem for completeness, and refer the reader to \cite{olesenpedersen2} for the full discussion.
\begin{corollary}[\mbox{\cite[Corollary 3.3]{olesenpedersen2}}]
	Let $(A,G,\alpha)$ be a $C^*$-dynamical system with $A$ separable and $G = \R$ or $G=S^1$. If $A$ is not primitive, then $G \rtimes_\alpha A$ is simple if and only if $A$ is $G$-simple and $\hat G(\alpha) = \hat G$.
\end{corollary}

