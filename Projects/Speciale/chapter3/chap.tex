\chapter{Simplicity Of Crossed Products}
As previously mentioned, we will now begin describing the theory, both old and new, describing the ideal structure of crossed products $A \rtimes_\alpha G$ for topological groups $G$. We will begin by introducing some new structures and concepts associated to this theory.

Since we will describe theory which only describes dynamical systems $(A,G,\alpha)$ where $G$ is abelian and also theory which describes the general case, we introduce the notation that a triple $(A, G, \alpha)$ is a $C^*$-dynamic a-system if $G \acts \alpha A$ and $G$ is abelian. 

\sssection{The Group Spectra}
\ssubsection{The Arveson and Connes spectra}
For the time being, we will untill otherwise stated assume that $G$ is abelian with dual group $\hat G$ as in the previous chapter. Recall that whenever we have a $C^*$-dynamical a-system $(A, G, \alpha)$, we also obtain a $*$-representation of $L^1(G)$ via the map $f \mapsto \alpha_f$. In fact, for each bounded Radon measure $\mu \in M(G)$ we obtain an operator $\alpha_\mu$ on $A$ satisfying
\begin{align*}
	\varphi(\alpha_f(a)) = \int_G \varphi(\alpha_t(x)) \d \mu(t), \text{ for all } a \in A \text{ and } \varphi \in A^*.
\end{align*}
We say that $\alpha_f$ and $\alpha_\mu$ are the \myemph{weak$^*$-integrals} of $f$ and $\mu$, respectively. Fore a more detailed exposition of vector-valued integration, see e.g., \cite[Appendix 3, Lemma 7.4.4]{pedersenalgauto} or \cite[Appendix A.3]{folland2016fourier}. For a group $G$ we define the subset $\cc^1(G)$ of $\cc(G)$ by
\begin{align*}
	\cc^1(G):= \{ f \in \cc(G) \mid \hat f \in \cc(\hat G)\},
\end{align*}
which is a dense ideal of $L^1(G)$, see \cite[Theorem 4.60]{folland2016fourier}.
\begin{definition}
	Let $(A, G, \alpha)$ be a $C^*$-dynamical system. For open sets $U \subseteq \hat G$, we define the \myemph{spectral $R$-subspace} $R^\alpha(U)$ to be the subspace of $A$ defined by
	\begin{align*}
		R^\alpha(U) := \overline{ \Span \left\{ \alpha_f(a) \mid a \in A, f \in \cc^1(G) \text{ with } \supp \hat f \subseteq U \right\} }^{w},
	\end{align*}
	and similarly, for each closed subset $K \subseteq \hat G$ we define the \myemph{spectral $M$-subspace} $M^\alpha(K)$ to be the subspace of $A$ defined by
	\begin{align*}
		M^\alpha(K) := \overline{\Span \left\{ x \in A \mid \alpha_f(x) = 0 \text{ for all } f \in \cc^1(G) \text{ with } \supp \hat f \subseteq K \right\}},
	\end{align*}
	i.e., as the annihilator of $R^{\alpha'}(K^c)$.
\end{definition}
By \cite[Theorem 8.1.4]{pedersenalgauto}, it follows that the $R$- and $M$-spaces are $G$-invariant subsets, and
\begin{align*}
	\bigcap_i M^\alpha(K_i) = M^\alpha(\bigcap_I K_i),
\end{align*}
for all collections of subsets $(K_i)$ of $\hat G$. This allows us to give the following definition:
\begin{definition}
	The \myemph{Arveson Spectrum} of an action $G \acts \alpha A$ is defined to be the smallest closed subset of $\hat G$ such that $M^\alpha(K) = A$, and we denote it by $\Sp \alpha$.
\end{definition}
In his dissertation (see \cite{connesclassification}), Alain Connes proved a series of equivalent characterizations of $\Sp \alpha$, which we will need. We include it without proof and instead refer the reader to \cite[Proposition 8.1.9]{pedersenalgauto} for the proof.
\begin{proposition}
	Let $(A, G, \alpha)$ be a $C^*$-dynamical a-system. Then for $\sigma \in \hat G$ the following are equivalent:
	\begin{enumerate}
		\item $\sigma \in \Sp \alpha$,
		\item For all neighborhoods $U$ of $\sigma$: $R^\alpha(U) \neq 0$,
		\item There is a net of unit vectors $(a_i) \subseteq A$ such that $\lv\alpha_t(x_i) - (t,\sigma)x_i\rv \to 0$ uniformly on compacta of $G$,
		\item For every finite Radon measure $\mu$ on $G$: $|\hat \mu(\sigma)| \leq \lv \alpha_\mu\rv$,
		\item For every $f \in L^1(G)$: $|\hat f (\sigma)| \leq \lv \alpha_f\rv$ and
		\item If $f \in L^1(G)$, $\alpha_f=0$, then $\hat f(\sigma)=0$.
	\end{enumerate}
\end{proposition}
One very common use of the above is to take the last condition as the definition:
\begin{align*}
		\Sp \alpha := \left\{ \chi \in \hat G \mid \hat f(\chi) = 0 \text{ whenever } f \in L^1(G) \text{ satisfies } \alpha_f(x) = 0 \text{ for all } x \in A  \right\}.
	\end{align*}
We will frequently switch between the above equivalent conditions.

The main motivation (in our case) for introducing the Arveson spectrum is to define the Connes Spectrum. For this, we need to introduce the concept of hereditary sub-$C^*$-algebras:
\begin{definition}
	A $C^*$-subalgebra $B\subseteq A$ is \myemph{hereditary} if for all positive $x \in A_+$ we have $x \in B_+$ whenever $y \in B_+$ with $0 \leq x \leq y$. We define $\mathscr{H}(A)$ to be the set of non-zero hereditary $C^*$-subalgebras of $A$. 
	
	If furthermore we assume that $(A, G, \alpha)$ is a $C^*$-dynamical system, we defin $\mathscr{H}^\alpha(A)$ to be the set
	\begin{align*}
		\mathscr{H}^\alpha (A) := \{ D \in \mathscr{H}(A) \mid \alpha_g(D)=D \text{ for all } g \in G\}.
	\end{align*}
\end{definition}

With this, we may define the Connes spectrum of an action $G \acts \alpha A$.
\begin{definition}
	The \myemph{Connes Spectrum} of an action $G \acts \alpha A$ is the subset $\Gamma(\alpha)$ of $\hat G$ given by
	\begin{align*}
		\Gamma( \alpha) := \bigcap_{D \in \mathscr{H}^\alpha (A)} \Sp {\alpha|_D}.	
	\end{align*}
\end{definition}
As we shall see, this subset is very important in the theory of classification of crossed products, for instance it is an invariant under exterior equivalence (which we've yet to define).
