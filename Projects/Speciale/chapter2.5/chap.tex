\sssection{An Important Example: Transformation Groups and actions on Spaces}
Recall the example of $\R \acts \theta C(S^1)$ via translation by $\theta \in \R$ from earlier. This is a more general construction of great importance which gives many great examples and ways to describe non-trivial examples of $C^*$-algebras and topological spaces. We will see that there is a correspondance between classic topological dynamic system and $C^*$-dynamical systems of abelian $C^*$-algebras and properties thereof. 

Shortly put, we will describe $C^*$-dynamical systems associated with transformation groups, properties of actions on spaces and important examples of transformation groups including $G$-boundaries and the Furstenberg boundary. 

Some of the theory discussed and covered in the following are commonly available in teaching books on the subject, for instance we use \cite{williamscrossed} as an inspiration for the first part. Some theory is newer and not-so-common in teaching books, e.g., the theory of boundary actions. We will refer to the authors written projects (when possible) on these subjects as well as original source material on the subject.

\ssubsection{Topological Dynamics}
Throughout this chapter, we will let $X$ denote an arbitrary locally compact Hausdorff space and $G$ an arbitrary locally compact group. 
\begin{definition}
	We say that $G$ acts on $X$, abbreviated \myemph{$G \acts {} X$}, if there is a jointly continuous map
	\begin{align*}
		G \times X &\to X\\
		(t,x) \mapsto &t.x \in X
	\end{align*}
	such that for each fixed $t \in G$ the map $x \mapsto t.x$ is a homeomorphism of $X$ and such that $s.(t.x) = (st).x$ and $e.x = x$ for all $s,t \in G$ and $e \in G$. Whenever $G \acts {} X$, the pair $(X,G)$ is called a \myemph{Transformation Group} and $X$ is called a \myemph{$G$-space}.
\end{definition}
\begin{remark}
	A net $(f_i) \subseteq C(X,Y)$ converges in the compact-open topology if and only if $f_i(x_i) \to f(x)$ for whenever $(x_i)$ is a net with the same index set converging to $x \in X$, see e.g. \cite[Lemma 1.30]{williamscrossed}.
\end{remark}
\begin{example}
	If $h \in \mathrm{Homeo}(X)$, then we obtain an action $\Z  \acts {} X$, $n.x := h^n(x)$, $n \in \Z$, and we may always convert $X$ into a $\Z$-space this way.
\end{example}
It is evident from the definition of transformation groups that we are interested in decribing continuous maps from $G$ to $\mathrm{Homeo}(X)$. To do this, we must describe the topology on it.
\begin{definition}
	Given topological spaces $X$ and $Y$, we endow $C(X,Y)$ with a topology called the \myemph{compact-open} topology. It is the topology generated by the subbasis of the form
	\begin{align*}
		V(K,U) := \{ f \in C(X,Y) \ \mid \ f(K) \subseteq U\},
	\end{align*}
	for $K \subseteq X$ compact and $U \subseteq Y$ open. The topology on $\mathrm{Homeo}(X)$\index{topology on $\mathrm{Homeo}(X)$} is the topology such that $(h_i) \subseteq \mathrm{Homeo}(X)$ converges to $h$ if and only if $h_i \to h$ and $h_i^{-1} \to h^{-1}$ in the compact-open topology of $C(X,X)$.
\end{definition}
And, luckily for us, there is a nice correspondance between $C^*$-dynamical systems $(C_0(X),G,\alpha)$ and Transformation Groups $(X,G)$:
\begin{proposition}
	There is a bijective correspondance between $C^*$-dynamical systems $(C_0(X),G,\alpha)$ and Transformation groups $(X,G)$ such that when $G \acts {} X$ the corresponding action $G \acts \alpha C_0(X)$ is given by
	\begin{align*}
		\alpha_s(f)(x):= f(s^{-1}.x), \text{ for } f \in C_0(X), \ s \in G \text{ and } x \in X.
	\end{align*}
	\label{dynamiccorr}
\end{proposition}
For a proof, see \cite[Lemma 1.33, 2.5 and Proposition 2.6]{williamscrossed}.

\ssubsection{Boundary Actions and the Furstenberg boundary}
The rest of this chapter is largely based on discussion and results covered authors written project (see \cite{bscp}), which covers and discusses (in a lesser generality) results of Furman and Glasner, see e.g., \cite{furman2003minimal} and \cite{glasner1976proximal} respectively. We will not go through the proofs and techincal details, and instead offer a heuristic explanation.
\begin{definition}
	An action $G \acts {} X$ is said to be \myemph{minimal} if there are no non-trivial $G$-invariant closed subsets of $X$.
\end{definition}
\begin{remark}
	It is evident that the action $G \acts {} X$ being minimal is equivalent to every orbit $G.x := \{t.x \ \mid \ t \in G\}$ being dense in $X$. This will be used repeatedly.
\end{remark}
\begin{example}
	If $\theta \in \R \backslash \Q$, then the action $\Z \acts {} S^1$, $k.e^{2\pi i t}:= e^{2 \pi i (t+k\theta)}$, is minimal.	
\end{example}
See e.g., \mbox{\cite[Example 4.6]{bscp}}. 
\begin{definition}
	A subset $Y \subseteq X$ of a $G$-space is minimal if it is non-empty, closed and $G$-invariant and contains no other non-empty closed $G$-invariant subsets.
\end{definition}
\begin{remark}
	It is evident that $Y \subseteq X$ is minimal if and only if the action restricts to a minimal action on $Y$. Moreover, an easy application of Zorn's lemma ensures that every compact $G$-space admits a minimal subset.
\end{remark}
An easy application of abstract change-of-variable and dominated convergence shows that if we let $\mathrm{Prob}(X)$ denote the set of Radon probability measures on a compact $G$-space $X$, then we obtain a continuous action on $\mathrm{Prob}(X)$ of $G$ as well (with respect to the weak$^*$-topology of $C_0(X)^*$), which is given by $t.\mu(f) := \mu(\alpha_{t^{-1}}.f)$, where $\alpha$ is the induced action on $C_0(X)$ of $G$. This leads us the following:
\begin{definition}
	Let $X$ be a compact $G$-space. We say that $G \acts {} X$ is a \myemph{strongly proximal action} if for each $\mu \in \mathrm{Prob}(X)$ we have
	\begin{align*}
		\overline{G.\mu}^{w^*} \cap \{\delta_x \ \mid \ x \in X\} \neq \emptyset.
	\end{align*}
\end{definition}
\begin{note}
	Given a compact Hausdorff space $X$, the map $x \mapsto \delta_x$ is a homeomorphism between $X$ and the point-mass measures on $X$ as a subset of $\mathrm{Prob}(X)$. With this in mind, we can think of strongly proximal actions on compact spaces to correspond to having the ability to continuously deform subsets of spaces to points.
\end{note}
\begin{definition}
	A compact $G$-space $X$ is caled a \myemph{$G$-boundary} if the action is a \myemph{boundary action} which are minimal and strongly proximal actions.
\end{definition}
For discrete countable groups, the theory of boundary actions and compact $G$-boundaries have strong influence on the classification of group $C^*$-algebras and other areas, as discovered by E. Breuillard, M. Kalantar, M. Kennedy, N. Ozawa (see e.g. \cite{breuillard2017c}). If $G$ is any (also non-discrete) group, then there is a particularly important $G$-boundary:
\begin{definition}
	The \myemph{Furstenberg boundary} of a group $G$ is the universal compact $G$-boundary, abbreviated \myemph{$\dd_F G$}, such that every other compact $G$-boundary $X$ is the continuous image of a $G$-equivariant map $\dd_F G \to X$.
\end{definition}
Both the existence and uniqueness is covered in \cite[Chapter 4]{bscp}. For discrete countable groups, properties of $C(\dd_F G) \rtimes_\alpha G$ are intertwined with properties of $G$. We will discuss this further later on, and mention as an example that the free group on two generators gives rise to a boundary action on $S^1$, where (say $\mathbb{F}_2$ is generated by $a,b$)
\begin{align*}
	a \mapsto (e^{2 \pi i t} \mapsto e^{2 \pi i (t+\theta)}) \text{  and  } b \mapsto (e^{2 \pi i t} \mapsto e^{2 \pi i t^{2}}),
\end{align*}
for $\theta \in \R\backslash \Q$, see e.g., \cite[Example 4.13]{bscp}.

As mentioned, the Furstenberg boundary has a lot of interesting and nice properties. It is an \myemph{extremally disconnected space} (= the closure of open sets are open), see e.g. \cite[Proposition 2.4]{breuillard2017c} for a newer proof or \cite[4.18]{hamana1979injective} for the original proof.

We will get back to the connection of crossed product (in particular of the form $C(X) \rtimes_r G$ for $G$-boundaries) and boundary actions, where we will examine connections between the ideal structure of $C_r^*(G)$ and $A \rtimes_r G$ for certain $C^*$-algebras.
