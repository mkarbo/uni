\chapter{Crossed Products with Topological Groups}
Suppose that you are given a $C^*$-dynamical system $(A,G,\alpha)$ with $G$ discrete. It is not hard to extend representations of $G$ and $A$ to the $*$-algebra $\cc(G,A)$, and thus create new $C^*$-algebras through completion of it. Examples include the reduced- and full crossed product of $A$ and $G$, $A \rtimes_{\alpha,r} G$ and $A \rtimes_{\alpha} G$.

However, when $G$ is not discrete, we run into issues since we can not assume that a unitary representation of $G$ is continuous in norm, which is required by classic constructions seen e.g., in \cite{brown2008c}. We will in this chapter develop the necessary theory to define the non-discrete analogues of these techniques, i.e., for when $G$ is a locally compact Hausdorff group.

We assume familiarity with basic theory of analysis in topological groups, basic representation theory of topological groups and Fourier analysis thereon and will not go lengths to explain results from this field, for a reference, we recommend the excellent books on the subject: \cite{folland2016fourier}, \cite{berg1984harmonic} and \cite{folland2013real}. 
\sssection{Integral Forms And Representation Theory}
Throughout this chapter, we let $G$ be a locally compact topological group and $A$ a $C^*$-algebra, with no added assumptions unless otherwise specified. We will write $G \acts \alpha A$ to specify that $\alpha \colon G \to \Aut(A)$ is an action of $G$ on $A$, i.e., a strongly continuous group homomorphism. We will always fix a base left Haar measure $\mu$ on $G$, whose integrand we will denote by $\d t$, and with modular function $\Delta \colon G \to (0,\infty)$. 

The main motivation of the theory we are about to develop is this: Given a covariant representation $(\pi,u,\H)$ of $(A,G,\alpha)$, we wish to construct a representation $\pi \rtimes u$ of $\cc(G,\mathbb{B}(\H))$ satisfying 
\begin{align*}
\pi \rtimes u (f) = \int \pi(f(t))u_t \d t.
\end{align*}
The problem at hand is, a priori $u_t$ is not norm continuous, and so we don't know if the integrand is integrable. We will however show that $u_t$ is strictly convergent and use this to define our integral by combining it with the previously seen fact that $M(\mathbb{B}(\H)) = \mathbb{K}_A(A_A)= \mathbb{B}(\H)$. 

For $f \in \cc(G,A)$, the map $s \mapsto \lv f(s) \rv$ is in $\cc(G)$, and we define the number $\lv f \rv_1 \in [0,\infty)$ to be
\begin{align*}
\lv f \rv_1 = \int_G \lv f(t) \rv \d t.
\end{align*}
For $A= \C$, it is well known that given $f \in \cc(G,A)$ and $\varepsilon > 0$, there is a neighborhood $V$ of $e \in G$ such that $sr^{-1} \in V$ or $s^{-1}r \in V$ implies $| f(s) - f(r)| < \varepsilon$, i.e., $f$ is left- and right uniformly continuous. The statement holds for general $A$, and the proof is identical, see e.g. \cite[Proposition 2.6]{folland2016fourier} or \cite[Lemma 1.88]{williamscrossed}
\begin{lemma}
For $f \in \cc(G,A)$ and $\varepsilon>0$, there is a neighborhood $V$ of $e \in G$ such that either $sr^{-1} \in V$ or $s^{-1}r \in V$ implies
\begin{align*}
	\lv f(s)  - f(r) \rv < \varepsilon.
\end{align*}
\label{int:lrunicont}
\end{lemma}

We will use $\mathcal{D}$ to denote an arbitrary complex Banach space, and for $f \in \cc(G)$ and $x \in \mathcal{D}$, we denote by $f \otimes x$ the map $G \ni t \mapsto f(t)x \in \mathcal{D}$. It is clear that then $f \otimes x \in \cc(G,\mathcal{D})$. We introduce a 'topology' or a name for a special kind of convergence on $\cc(G,\mathcal{D})$, which will prove useful:
\begin{definition}
We say that a net $\{f_i\}_{i \in I} \subseteq \cc(G,\mathcal{D})$ converges to a function $f$ on $\cc(G,\mathcal{D})$ in the \myemph{inductive limit topology} if $f_i \to f$ uniformly and there is a compact set $K \subseteq G$ and $i_0 \in I$ such that for $i \geq i_0$ we have $f_i \big|_{K^c} = 0$, i.e., $\mathrm{supp} f_i \subseteq K$.
\end{definition}

\begin{lemma}
Let $\mathcal{D}_0 \subseteq \mathcal{D}$ be any dense subset. Then the set
\begin{align*}
\cc(G) \odot \mathcal{D}_0 := \Span\left\{ f \otimes x \mid f \in \cc(G) \text{ and } x \in \mathcal{D}_0 \right\} 
\end{align*}
is dense in the inductive limit topology on $\cc(G,\mathcal{D})$.
\label{int:indlmdense}
\end{lemma}
\begin{proof}	
Let $\varphi \in \cc(G,\mathcal{D})$ and $\varepsilon > 0$ be arbitrary. Let $K= \mathrm{supp} \varphi$ and let $W$ denote an arbitrary compact neighborhood of $e \in G$, and pick a symmetric neighborhood $e \in V_{\varepsilon} \subseteq W$ such that $sr^{-1} \in V_\varepsilon$ implies $\lv f(s) - f(r) \rv < \frac{\varepsilon}{\mu(WK)}$. Since $K$ is compact, there is a finite set $\left\{ r_1,\dots,r_n \right\} \subseteq K$ such that $K \subseteq \bigcup_{1 \leq i \leq n} V_\varepsilon r_i$. Recall that every locally compact Hausdorff group is paracompact and the cover $K^c \cup \left( \bigcup_{1 \leq i \leq n } V_\varepsilon r_i \right)$ admits a partition of unity $\left\{ f_j \right\}_{j=0}^n$ such that $\mathrm{supp}f_0 \subseteq K^c$ and $\mathrm{supp}f_i \subseteq V_\varepsilon r_i$ for $i = 1 ,\dots,n$. 

Using this, we define
\begin{align*}
	g_\varepsilon := \sum_{i=1}^n f_i \otimes x_i,
\end{align*}
where $x_i \in \mathcal{D}_0$ such that $\lv x_i - \varphi(r_i)\rv < \frac{\varepsilon}{2\mu(WK)}$ for $i = 1,\dots,n$. Then $g_\varepsilon \in \cc(G) \odot \mathcal{D}_0$ with $\mathrm{supp} g_\varepsilon \subseteq V_\varepsilon K \subseteq W K$. The partition of unity satisfies $\sum_{i=1}^n f_i(s) \leq 1$ for all $s \in G$ with equality for $s \in K$, so in particular $\sum_{1 \leq i \leq n}f_i(s) \varphi(s) = \varphi(s)$ for all $s \in G$. Hence we see that for $s \in G$: 
\begin{align*}
	\lv\varphi(s) - g_{\varepsilon}(s)\rv = \lv \sum_{i=1}^n f_i(s)(\varphi(s) - \varphi(r_i)+\varphi(r_i)-x_i)\rv &\leq \sum_{i=1}^n f_i(s) (\lv \varphi(s) - \varphi(r_i)\rv+\lv \varphi(r_i)-x_i\rv) \\
	&\leq \frac{\varepsilon}{\mu(WK)},
\end{align*}
Since $\varepsilon$ was arbitrary and $W,K$ didn't depend on $\varepsilon$ and $ \mathrm{supp} g_\varepsilon \subseteq WK$, we see that $g_\varepsilon \to f$ in the inductive limit topology as $\varepsilon \to 0$.
\end{proof}
We will also need the following result, which can be found in \cite[Theorem 3.20, part (b, c)]{rudin1991functional}
\begin{lemma}
If $K \subseteq B$ is compact, then the convex hull convex hull $\conv(K)$ is totally bounded and hence it has compact closure.
\label{int:clconvcomp}
\end{lemma}


\begin{definition}
Let $f \in \cc(G,B)$ and $\varphi \in B^*$, the continuous dual of $B$. Then we can define a bounded linear functional $I_f$ on $B^*$ by
\begin{align*}
	L_f(\varphi) := \int_G \varphi(f(s)) \d s,
\end{align*}
since $| L_f(\varphi)| \leq \int_G \lv \varphi \rv \lv f(s) \rv \d s = \lv f \rv_1 \lv \varphi\rv$.
\end{definition}
\begin{note}	
If $(f_i) \subseteq  \cc(G,B)$ is a net which converges in the inductive limit topology to $f \in \cc(G,B)$, i.e., $f_i \to f$ uniformly on $G$ and for some $K \subseteq G$ compact we have $\supp f_i \subseteq K$ eventually (and hence also $\supp f \subseteq K$), then
\begin{align*}
\lv f_i - f\rv_1 = \int_{G} \lv f_i(s) - f(s) \rv \d s 	\leq \lv f_i - f \rv \mu(K) \to 0,
\end{align*}
so $f_i \to f$ in the $L^1$-norm, and hence by the above we see that $|L_{f_i}(\varphi) - L_f(\varphi) | \leq \lv f_i-f\rv_1 \lv \varphi \rv \to 0$ for all $ \varphi \in B^*$, i.e.,  $L_{f_i}\to L_f$ in the weak$^*$ topology of $B^{**}$.
\label{note:induct}
\end{note}

We will see that $L_f$ is equal to $\hat a \colon \varphi \to \varphi(a)$ for some unique $a \in B$. Let $\iota \colon B \to B^{**}$ denote the inclusion $\iota(a) = \hat a$, then we have
\begin{lemma}
If $f \in \cc(G,B)$, then $L_f \in \iota(B)$.
\label{int:defintegral}
\end{lemma}
\begin{proof}
Let $W$ be a compact neighborhood of $\mathrm{supp} f$, and let $K:= f(G) \cup \left\{ 0 \right\} \subseteq B$. Then $K$ is compact, so by \ref{int:clconvcomp} the set $C := \overline \conv \left\{ K \right\}$ is compact.

Fix $\varepsilon > 0$, and pick a neighborhood $V$ of $e \in G$ such that $ \lv f(s) - f(r) \rv < \varepsilon$ for $s^{-1}r \in V$. Assume without loss of generality that $\mathrm{supp} f V \subseteq W$, since we can make $V$ arbitrarily small. Using compactness of $\supp f$, pick $s_1,\dots,s_n \in \supp f$ such that $\left\{ s_i V \right\}$ covers $\supp f$.

Since $ \left\{ s_i V \right\} \subseteq W$, it has compact closure, so we may pick a partition of unity $\left\{ \varphi_i \right\}_{i=1}^n$ of $ K \subseteq \bigcup_{i=1}^n s_i V$. Using this, we define 
\begin{align*}
	g_\varepsilon (s) := \sum_{i=1}^n f(s_i) \varphi(s),
\end{align*}
so $\supp g \subseteq W$. If $s \in \left( \bigcup_{i=1}^n s_i V \right)^c$ then $f(s) = g_{\varepsilon}(s) = 0$, and otherwise we see that if $s \in s_i V$, then $s_i^{-1}s \in V$ and $\lv f(s) - f(s_i) \rv < \varepsilon$, and hence
\begin{align*}
	\lv f(s) - g_{\varepsilon} \rv  \leq \sum_{i=1}^n \varphi(s) \lv f(s)-f(s_i) \rv  < \varepsilon.
\end{align*}

Since $s \mapsto \sum_{i=1}^n \varphi(s) \leq 1$ and has compact support and $\bigcup_{i=1}^n \supp \varphi_i \subseteq W$, we have $\sum_{i=1}^n \int_G \varphi_i(s) \d s \leq \mu(W) < \infty$, so it is equal to some number $D$. For $\psi \in B^*$ we see that
\begin{align*}
	L_{g_{\varepsilon}} ( \psi ) = \int_{G} \psi\left( \sum_{i=1}^n \varphi(s) f(s_i) \right) \d s = \sum_{i=1}^n \int_G \varphi(s) \d s \psi(f(s_i)) = \sum_{i=1}^n \int_G \varphi(s) \d s \underbrace{\iota(f(s_i))}_{ \in \iota(C)} (\psi) 
\end{align*}
so $L_{g_{\varepsilon}}  \in D \iota(C) \subseteq \mu(W) \iota(C)$, since $0 \in C$ so we can shrink the scalar $\mu(W)$ to $D$. For $\varepsilon \to 0$, we see that $g_{\varepsilon} \to f$ in the inductive limit topology and $L_{g_\varepsilon} \in \mu(W) \iota(C)$. 

Since $C$ is compact and convex, it is weakly compact which by definition is equivalent to $\iota(C)$ being compact in the weak$^*$-topology of $B^{**}$, and since $g_{\varepsilon} \to f$ in the inductive limit topology, we from \ref{note:induct} we see that $L_{g_\varepsilon } \to L_f$ in the weak $^*$-topology on $B^{**}$. In particular, since $\iota(C)$ is weak$^*$-compact, it is closed and hence $\mu(W) \iota(C)$ is closed in the weak$^*$-topology so $L_f \in \mu(W) \iota(C) \subseteq \iota(B)$.
\end{proof}
And we have a lemma emphasizing important properties of this integral:
\begin{lemma}
There exists a unique linear map $I \colon \cc(G,B) \to B$,
\begin{align*}
f \mapsto \int_G f(s) \d s
\end{align*}
such that for all $f \in \cc(G,B)$ we have
\begin{align}
\varphi(\int_G f(s) \d s ) &= \int_G \varphi(f(s)) \d s, \ \text{ for all } \varphi \in B^*,\\
\lv \int_G f(s) \d s \rv &\leq \lv f \rv_1 \\
\int_G (g \otimes x) (s) \d s &=x \int_G g(s) \d s, \text{ for all } g \otimes x \in \cc(G) \odot B,
\end{align}
and if $T \colon B \to D$ is a bounded linear operator between Banach spaces. Then
\begin{align*}
T\left( \int_G f(s) \d s \right) = \int_G T(f(s)) \d s.
\end{align*}
\label{int:bochnerproperties}
\end{lemma}
\begin{proof}
Clearly letting $I(f) := \iota^{-1}(L_f)$, with $L_f$ as in \ref{int:defintegral}, then $f \mapsto I(f)$ instantly satisfies the first properties. For the last, suppose that $T \colon B \to D$ is a bounded linear operator between Banach spaces. Then $T \circ f \in \cc(G,D)$ for all $f \in \cc(G,B)$, and $\varphi \circ T \in B^*$ for all $ \varphi \in D^*$. Hence 
\begin{align*}
	\varphi \circ T \left( \int_G f(s) \d s  \right) = \int_G \varphi( T \circ f (s) ) \d s = \varphi \left( \int_G T (f(s)) \d s \right),
\end{align*}
for all $\varphi \in D^*$, finishing the proof.
\end{proof}
In light of the above, we may give the following definition:
\begin{definition}
For $f \in \cc(G,B)$, define the integral of $f$ as $I(f) := \iota^{-1}(L_f)$, which we denote by $\int_G f(s) \d s := I(f)$.
\end{definition}

In particular, if we let $A$ be a $C^*$-algebra, then we obtain the following useful result:
\begin{proposition}
For $f \in \cc(G,A)$, if $\pi \colon A \to \mathbb{B}(\H)$ is a representation of $A$, then for all $\xi , \eta \in \H$ the integral of \ref{int:bochnerproperties} satisfies
\begin{align*}
	\langle \pi\left( \int_G f(s) \d s ) \right) \xi , \eta \rangle = \int_{G} \langle \pi(f(s)) \xi , \eta \rangle \d s,
\end{align*}
and 
\begin{align*}
	\left( \int_G f(s) \d s  \right)^* = \int_G f(s)^* \d s
\end{align*}
and for $a , b \in M(A)$ we have
\begin{align*}
 a \int_G f(s) \d s b = \int_G a f(s) b \d s.		
\end{align*}
\label{int:cstarint}
\end{proposition}
\begin{proof}
Define the vector-state $\varphi_\pi \colon a \mapsto \langle \pi(a) \xi , \eta \rangle$. Then, $\varphi_\pi \in A^*$, and the first statement follows. Suppose now that $\pi$ is a faithful representation, then since $\int_G \pi(f(s)) \d s = \pi \left( \int_Gf(s) \d s \right)$, we see that the second statement holds as well when applied to $\varphi_\pi(I(f)^*)$. Finally if $a,b \in M(A)$ then
\begin{align*}
	\langle \pi\left(a \int_G f(s) \d s b \right) \xi , \eta \rangle &= \langle \overline \pi(a) \pi\left( \int_G f(s) \d s \right) \overline \pi (b)  \xi , \eta \rangle \\
	 &= \langle \pi \left( \int_G f(s) \d s \right) (\overline \pi (b) \xi ) , \overline \pi(a^*) \eta \rangle \\ 
	 &= \int_G \langle \pi(f(s)) \overline \pi (b) \xi , \overline \pi(a ^* ) \eta \rangle \d s\\
	 &= \int_G \langle \pi(af(s)b) \xi , \eta \rangle \d s\\
	 &= \langle \pi \left(  \int_G a f(s) b \d s  \right) \xi , \eta \rangle.
\end{align*}
\end{proof}

As mentioned, we will need to consider maps of the form of the form $s \mapsto \pi(f(s)) U_s$ where $f \in \cc(G,A)$, $\pi\colon  A \to \mathbb{B}(\H)$ and $U \colon G \to \mathcal{U}(\H)$ is strongly continuous. In order to use the above theorem to make sense, we will want to change $A$ out with $M_s(A)$, i.e., the multiplier algebra of $A$ equipped with the strict topology, and combine it with the following lemma:
\begin{lemma}
Let $E$ be a Hilbert $A$-module, and let $u \colon G \to \mathcal{U}(E)$ be a group homomorphism from $G$ to the group of unitary operators on $E$, i.e., operators $T \in \mathcal{L}_A(E)$ such that $T^*T=TT^*=I_E$. Then $u$ is strictly continous if and only if $u$ is strongly continuous.
\label{int:unistrictstrong}
\end{lemma}
\begin{proof}
We saw in \cref{mult:STARTSTRONG} that on bounded sets, the strict and the $*$-strong topology coincided. In particular, since $\mathcal{L}_A(E)$ is a $C^*$-algebra, the set of unitary operators is a subset of the unit sphere. Thus, it suffices to show that if $s \mapsto u_s x$ is continuous for $x$ then $s \mapsto u_s^* x$ is continuous. Let $x \in E$, then
\begin{align*}
	\lv u_s^* x - u_t^*x \rv_A^2 = \underbrace{\lv u_s^* x \rv_A^2}_{= \lv x \rv_A^2} + \underbrace{\lv u_t^*x \rv_A^2}_{= \lv x \rv_A^2} - \langle \underbrace{ \left( u_t u_s^* + u_s u_t^*  \right)}_{\stackrel{s \to t}{\longrightarrow} 2I_E} x ,x \rangle \stackrel{s \to t}{\longrightarrow} 0,
\end{align*}
showing the wanted, since $x \in E$ was arbitrary.
\end{proof}

We now generalize the result of \cref{int:cstarint} to cover the class of functions which we shall deal with later on.
\begin{theorem}
Given a $C^*$-algebra $A$, there is a unique linear map $I\colon \cc(G,M_s(A)) \to M(A)$, $f \mapsto \int_G f(s) \d s$, such that for any non-degenerate representation $\pi \colon A \to \mathbb{B}(\H)$ and all $\xi, \eta \in \H$ the identity
\begin{align}
	\langle \overline \pi \left( \int_G f(s) \d s \right) \xi , \eta \rangle = \int_{G} \langle \overline \pi(f(s)) \xi , \eta \rangle \d s,
	\label{mult:1.35}
\end{align}
and for all $f \in \cc(G,M_s(A))$
\begin{align}
	\lv I(f) \d s \rv \leq \lv f \rv _{ \infty} \mu(\supp f ).
\end{align}
Also, the identities
\begin{align}
	\int_G f(s) ^* \d s = \left( \int_G f(s) \d s \right)^*,
\end{align}
and
\begin{align}
	\int_G a f(s) b \d s = a \left( \int_G f(s) \d s  \right) b
\end{align}
hold for all $f \in \cc(G,M_s(A))$ and $a,b \in M(A)$. If $T \colon A \to M(B)$ is a non-degenerate homomorphism of $C^*$-algebras, then
\begin{align}
	\overline T \left( \int_G f(s) \d s\right) = \int_G \overline T (f(s)) \d s.
\end{align}

\label{int:multstrictintegral}
\end{theorem}
\begin{proof}
If $I$ exists, then letting $\pi$ be a faithful representation of $A$ we see that $I$ must be unique by the first identity. For  $f \in \cc(G,M_s(A))$ and $a \in A$, it is easy to see that if $f_a \colon s \mapsto f(s) a$, then $f_a \in \cc(G,A)$. This allows us to use \cref{int:cstarint} to define the integral $\int_G f(s) a \d s$ for all $f \in \cc(G,M_s(A))$ and $a \in A$. Define $L_f \colon A \to A$ by
\begin{align*}
	L_f(a) = \int_G f(s) a \d s, \ \text{ for } a \in A.
\end{align*}
Then $L_f \in \mathcal{L}_A(A_A)$, for if $a,b \in A$ then \cref{int:cstarint} implies that
\begin{align*}
	\langle L_f a , b \rangle_{A_A} = (L_f a )^* b = \int_G (f(s) a)^* \d s b = \int_G a^* f(s)^* \d s = a^* \int_G f(s)^* b \d s = \langle a , L_{f^*} b\rangle_{A_A}.
\end{align*}
So $L_{f^*}$ is an adjoint of $L_f$. Let $\int_G f(s) \d s := L_f$ for all $f \in \cc(G,M_s(A)$, then $L_f \in \mathcal{L}_A(A_A) = M(A)$, and it is not hard to see that the assignment $f \mapsto L_f$ is linear.

Again, by \cref{int:cstarint}, we see that a non-degenerate representation $\pi \colon A \to \mathbb{B}(\H)$ with $a \in A$ and $\xi, \eta \in \H$ that
\begin{align*}
	\langle \overline \pi \left( \int_G f(s) \d s \right) \pi(a) \xi , \eta \rangle =\langle \pi \left(f(s) a \d s \right) \xi , \eta \rangle &= \int_G \langle \pi(f(s) a) \xi , \eta \rangle \d s\\
	&= \int_G \langle \overline \pi (f(s)) \pi(a) \xi , \eta \rangle \d s,
\end{align*}
which shows \cref{mult:1.35}, since $\pi$ is non-degenerate. The analogues of \cref{int:cstarint} follows similarly. Let $a \in A$ with $\lv a \rv \leq 1$, then by \cref{int:cstarint} applied to $f_a \in \cc(G,A)$, we see that
\begin{align*}
	\lv \int_G f(s) \d s a \rv  \leq \lv f_a \rv_1 = \int_G \lv f(s) a \rv \d s \leq \int_G 1_{\supp f} \lv f \rv_{\infty} \d s = \mu(\supp f) \lv f \rv_{\infty},
\end{align*}
Hence $\lv \int_G f(s) \d s \rv = \sup_{\lv a \rv \leq 1} \lv \int_G f(s) \d s a\rv \leq \lv f \rv_{\infty} \mu(\supp f)$. Let $T \colon A \to M(B)$ be a non-degenerate homomorphism, with unique extension $\overline T \colon M(A) \to M(B)$, and hence for all $ a \in A$ and $b \in B$ we have
\begin{align*}
	\overline T\left(  \int_G f(s) \d s \right) T(a) b = T\left( \int_G f(s) a \d s  \right)b= \int_{G} T(f(s)a ) b \d s &= \int_G \overline T(f(s)) T(a)b \d s \\
	&=\int_G \overline T(f(s)) \d s T(a)b,
\end{align*}
hence $\overline T \left( \int_G f(s) \d s \right) = \int_G \overline T(f(s)) \d s$ by non-degeneracy.
\end{proof}
In the construction of the general crossed product and other $C^*$-algebras associated to a $C^*$-dynamical system $(A,G,\alpha)$ we will need to deal with the algebra $\cc(G,A)$ quite a lot. We will want to develop a sort of twisted-convolution incorporating the action $\alpha$ of $G$. However, in order to do this properly, we will need a form of Fubini's theorem for the above:
\begin{lemma}
For $f \in \cc(G \times G  , B)$, the function
\begin{align*}
	t \mapsto \int_G f(t,s) \d s
\end{align*}
is an element of $\cc(G,B)$.
\label{int:singleint}
\end{lemma}
\begin{proof}
It is easy to see that if $F \colon G \to B$ is the map $t \mapsto \int_G f(t,s) \d s$, then $ \supp F$ is compact. We show continuity. Suppose that $x_i \to x \in G$ and let $ \varepsilon > 0$. Pick compact $K_1,K_2 \subseteq G$ covering the support of $f$, i.e., such that $ \supp f \subseteq K_1 \times K_2$. Note that $f(x_i,y) \to f(x,y)$ uniformly: If not, then we can find a net $(r_j)\subseteq G$ and $\varepsilon_0 > 0$ such that
\begin{align*}
	\lv f(x_i,r_j) - f(x,r_j) \rv \geq \varepsilon_0,
\end{align*}
for all $i,j$. This shows that $r_j \in K_2$ for all $j$, hence we may assume that $r_j \to r \in K_2$ (up to passing to a subnet). But then for sufficiently large $i,j$ we would have $\lv f(x_i,r_j) - f(x,r_j) \rv < \varepsilon_0$. Hence we can assume that $\lv F(x_i) - F(x) \rv < \frac{\varepsilon}{ \mu(K_2)}$. Then, 
\begin{align*}
	\lv \int_G f(x_i,s) \d s - \int_G f(x,s) \d s \rv \leq \int_{K_2} \lv f(x_i,s) - f(x,s) \rv \ d s + \int_{K_2^c} \lv f(x_i,s) - f(x,s) \rv \d s  \leq \varepsilon,
\end{align*}
as $\supp f \subseteq K_1 \times K_2$, so $F \in \cc(G,B)$.
\end{proof}
\begin{corollary}
Replacing the integral of \cref{int:singleint} with $\int_G f(t,s) \d t$ the statement still holds.
\end{corollary}
\begin{theorem}[Fubini's for $C^*$-valued integrals]
Suppose that $f \in \cc(G \times G, M_s(A))$, then 
\begin{align*}
	s \mapsto \int_G^{M_s(A)} f(s,t) \d t \text{ and } s \mapsto \int_G^{M_s(A)} f(t,s) \d t
\end{align*}
are elements of $\cc(G,M_s(A))$ and moreover
\begin{align*}
	\int_G^{M_s(A)} \int_G^{M_s(A)} f(s,t) \d s \d t = \int_G^{M_s(A)} \int_G^{M_s(A)} f(s,t) \d t \d s.
\end{align*}
\label{int:fubini}
\end{theorem}
\begin{proof}
Define for a fixed $s$, the map $t \mapsto f(s,t)$ is in $\cc(G,M_s(A)$, hence we can define $\int_G f(s,t) \d t$ and $\int_G f(t,s) \d t$ as in \Cref{int:multstrictintegral}. Recall that $M(A) = M(\mathbb{K}_A(A_A))$, so the strict topology is induced by multiplication operators on $A$, so it suffices to show that $s \mapsto a \left(\int_Gf(s,t) \d t\right)$ and $s \mapsto \left(\int_G f(s,t) \d t\right) a $ are in $\cc(G,A)$ for all $ a \in A$ for the first claim. By \Cref{int:multstrictintegral}, we have 
\begin{align*}
a \left( \int_G f(s,t) \d t \right) = \int_G a f(s,t) \d t,
\end{align*}
for all $a \in A$ and $s \in G$, and similarly $\left( \int_G f(s,t) \d t \right)a = \int_G f(s,t) a \d t$. The maps $(s,t) \mapsto a f(s,t)$ and $(s,t) \mapsto f(s,t) a$ are in $\cc(G \times G , A)$, clearly, so \Cref{int:singleint} ensures that $s \mapsto \int_G^{M_s(A)} f(s,t) \d t$ and $s \mapsto \int_G ^{M_s(A)} f(t,s) \d t$ are in $\cc(G,M_s(A)$.

For the last claim, we now see that the double integrals are well-defined \Cref{int:multstrictintegral}, so by letting $A$ be faithfully representted on $\mathbb{B}(\H)$, we may obtain the equality using the regular Fubini's theorem for scalar valued integrals (see e.g., \cite{schilling}) applied pointwise to an orthonormal basis through the inner-product.
\end{proof}

%***************************************************************************************************************************************
%***************************************************************************************************************************************
%***************************************************************************************************************************************
%***************************************************************************************************************************************
%***************************************************************************************************************************************
%***************************************************************************************************************************************

\ssection{Crossed Products}
In analysis, an object of interest is the group algebra $L^1(G)$ for locally compact groups $G$. One can turn it into a $*$-algebra by showing that the convolution gives rise to a well-defined multiplication, and one can use the modular function to define an involution. Analogously to this, we do the same for $A$ valued functions:
\begin{definition}
Suppose that $G \acts\alpha A$. We let 
\begin{align*}
	\cc(G,A):=\{ f \colon G \to A \mid f \text{ continuous and compactly supported}\}.
\end{align*}
For $f,g \in \cc(G,A)$, we define the $\alpha$-twisted convolution of $f$ and $g$ at $t \in G$ by
\begin{align*}
	(f \ast_\alpha g)(t):= \int_G  f(s) \alpha_s(g(s^{-1}t)) \d s,
\end{align*}
which is well defined since the map $(s,t) \mapsto f(s) \alpha_s g(s^{-1}t)$ is in $\cc(G \times G , A)$ and \cref{int:singleint} applies. Moreover, we define an $\alpha$-twisted involution of $f \in \cc(G,A)$ at $t \in G$ by
\begin{align*}
	f^*(t):=\Delta(t^{-1}) \alpha_t(f(t^{-1})^*).
\end{align*}
Straight forward calculations similar to the one classic ones show that $\cc(G,A)$ becomes a $*$-algebra with multiplication given by the $\alpha$-twisted convolution and involution given by the $\alpha$-twisted involution. We denote the associated $*$-algebra by $\cc(G,A,\alpha)$.

To $f \in \cc(G,A,\alpha)$, we associate the number $\lv f \rv_1 \in [0, \infty)$ by
	\begin{align*}
		\lv f \rv_1 := \int_G \lv f(t) \rv \d t.
	\end{align*}
Straightforward calculations show that $f \mapsto \lv f \rv_1$ is a norm, and we denote the completion of $\cc(G,A,\alpha)$ in $\lv \cdot \rv_1$ by $L^1(G,A,\alpha)$.
\end{definition}

For discrete and second-countable groups $G$ acting on a $C^*$-algebra $A$ via an action $\alpha$, we may easily construct a $*$-representation of $\cc(G,A)$ from a covariant representation $(\pi,u)$ by letting $\pi \rtimes u \colon \cc(A,G,\alpha) \to \mathbb{B}(\H)$ be the map
\begin{align*}
\sum_{g \in G} a_g \delta_g \mapsto \pi(a_g) u_g,
\end{align*}
however as discussed previously, this is not enough when $G$ is locally compact, since we can't assume that $u$ was continuous with respect to the norm on $\mathbb{B}(\H)$. However, in the previous section we constructed the tools to construct the desired generalization:
\begin{lemma}
Suppose that $G \acts \alpha A$ with $G$ locally compact. If $(\pi,u)$ is a covariant representation of the associated $C^*$-dynamical system, then for all $f \in \cc(G,A)$, the map
\begin{align*}
	s \mapsto \pi(f(s)) u_s, \text{ for }s \in G
\end{align*}
is in $\cc(G,\mathbb{B}_s(\H))$, where $\mathbb{B}_s(\H)$ indicates that we endow $\mathbb{B}\left(\H \right)$ with the strict topology of $ \mathbb{K}(\H)$.
\label{cross:integrandcont}
\end{lemma}
\begin{proof}
By assumption, $u$ is strongly continuous, which is equivalent to being strictly continuous by \Cref{int:unistrictstrong}. Since $f \in \cc(G,A)$, we see that $\pi \circ f \in \cc(G,\mathbb{B}(\H))$, and being norm continuous implies strict continuity, so $s \mapsto \pi(f(s)) u_s$ is an element of $\cc(G,\mathbb{B}_s(\H))$.
\end{proof}
The above and \ref{int:multstrictintegral} shows that given a covariant representation $(\pi,u)$ of $G \acts \alpha A$, there is a well-defined linear map $\pi \rtimes u \colon \cc(G,A) \to \mathbb{B}(\H)$ given by
\begin{align*}
\pi \rtimes u (f) = \int_G^{\mathbb{B}(\H)} \pi(f(s)) u_s \d s,
\end{align*}
And we shall see that this representation has a lot of nice properties:

\begin{proposition}
Suppose that $(\pi,u)$ is a covariant representation of a dynamical system $G \acts \alpha A$, with $G$ locally compact. Then the map $\pi \rtimes u$ is a $*$-representation of $\cc(G,A,\alpha)$ on $\mathbb{B}(\H)$ which is norm-decreasing. We call $\pi \rtimes u$ the \myemph{integrated form} \emph{of $\pi$ and $u$}. If $\pi$ is non-degenerate, then $\pi \rtimes u$ is as well.
\label{cross:integform}
\end{proposition}
\begin{proof}
Let $\xi , \eta \in \H$ and $f \in \cc(G,A)$, then
\begin{align*}
	 |\langle \pi \rtimes u (f) \xi , \eta \rangle| \leq \int_G | \langle \pi(f(s)) u_s \xi , \eta \rangle| \d s \leq \int_G \lv \pi(f(s)) \rv \lv \xi\rv \lv \eta \rv \d s \leq \lv f \rv_1 \lv \xi \rv \lv \eta \rv
\end{align*}
Since $\xi, \eta$ can be any unit vector, we have $\lv \pi \rtimes u (f) \rv \leq \lv f \rv_1$ for all $f \in \cc(G,A)$.

Let $f \in \cc(G,A,\alpha)$, then
\begin{align*}
	\pi \rtimes u(f)^* = \int_G^{\mathbb{B}(\H)}  u_s^* \pi(f(s))^* \d s &= \int_G^{\mathbb{B}(\H)} u_{s^{-1}} \pi(f(s)^*) \d s\\
	&= \int_G^{\mathbb{B}(\H)} \Delta(s)^{-1} u_{s} \pi(f(s^{-1})^*) \d s \\
	&= \int_G^{\mathbb{B}(\H)} \Delta(s)^{-1} \pi(\alpha_s(f(s^{-1})^*)) u_s \d s\\
	&= \int_G^{\mathbb{B}(\H)} \pi\left( \underbrace{\alpha_s\left( \Delta(s)^{-1}f(s^{-1})^* \right)}_{=f^*(s)} \right) u_s \d s\\
	&= \int_G^{\mathbb{B}(\H)} \pi(f^*(s)) u_s \d s\\
	&= \pi \rtimes u(f^*)
\end{align*}
and if $f,g \in \cc(G,A,\alpha)$ then using left-invariance of $\d s$, \Cref{int:cstarint} and \Cref{int:fubini}, we see that
\begin{align*}
	\pi \rtimes u(f\ast_{\alpha}g) &= \int_G^{\mathbb{B}(\H)} \pi \left( f \ast_\alpha g(s) \right)u_s \d s\\
	&= \int_G^{\mathbb{B}(\H)} \pi\left( \int_G^A f(t) \alpha_t g(t^{-1}s) \d t \right)u_s \d s\\
	&= \int_G^{\mathbb{B}(\H)}\int_G^{\mathbb{B}(\H)} \pi(f(t)) \pi(\alpha_t(g(t^{-1}s))u_{t t^{-1}s} \d t \d s\\
	&= \int_G^{\mathbb{B}(\H)} \int_G ^{\mathbb{B}(\H)} \pi(f(t)) u_t \pi(g(t^{-1}s)) u_{t^{-1}s}\d s \d t\\
	&= \int_G^{\mathbb{B}(\H)} \pi(f(t)) u_t \int_G^{\mathbb{B}(\H)} \pi(g(t^{-1}s )) u_{t^{-1}s} \d s \d t\\
	&= \int_G ^{\mathbb{B}(\H)} \pi(f(t)) u_t \int_G ^{\mathbb{B}(\H)} \pi(g(s)) u_s \d s \d t\\
	&= \int_G ^{\mathbb{B}(\H)} \pi(f(t)) u_t \d t \int_G ^{\mathbb{B}(\H)} \pi(g(s)) u_s \d s \\
	&= \pi \rtimes u (f) \pi\rtimes u (g),
\end{align*}
as wanted. Assuming non-degeneracy of $\pi$, for $h \in \H$ and $\varepsilon>0$ we can pick $a \in A$, $\lv a \rv =1$ such that $\lv \pi(a) h - h \rv < \varepsilon$. Being a unitary representation, we may choose a neighborhood $e \in V \subseteq G$ such that $\lv u_s h - h\rv < \varepsilon$ for $s \in V$. Pick $0 \leq \varphi \in \cc(G)$ such that $\int_G \varphi(s) \d s = 1$ and $\supp \varphi \subseteq V$. Then, $f := \varphi \otimes a \in \cc(G,A)$ satisfies for $k \in \H$, $\lv k \rv = 1$, that
\begin{align*}
\left| \langle \pi \rtimes u(f)h-h,k\rangle \right| &= 	\left| \int_G \varphi(s) \langle \pi(a)u_s h-h,k \rangle\right|\\
& \leq \int_G \varphi(s) (|\langle \pi(a) (u_s h-h),k\rangle|+|\langle \pi(a)h-h,k\rangle| ) \d s \\
&< 2 \varepsilon,
\end{align*}
showing that $\lv \pi \rtimes u(f)h-h\rv \leq 2 \varepsilon$.
\end{proof}

Now, given a locally compact topological group $G$, let $\lambda \colon G \to \U(L^2(G))$ be the left-regular representation $\langle \lambda_g \xi, \eta\rangle = \int_G \xi(g^{-1}s)\overline{\eta(s)}\d s$ for $g \in G$ and $\xi, \eta \in L^2(G)$. Even though $L^2(G)$ consists of equivalence classes (unless e.g., $G$ is discrete), we will abbreviate $\lambda_g \xi(s) := \xi(g^{-1}s)$. Given a Hilbert space $\H$, we will use $\lambda_g$ to denote the operator $I \otimes \lambda_g$ on $\H \otimes L^2(G) \cong L^2(G,\H)$. With this, we have the following:
\begin{lemma}
Suppose that $G \acts \alpha A$, and $\pi$ is a representation of $A$ on $\H$. Define $\tilde{\pi}(a) \colon L^2(G,H) \to L^2(G,H)$ by 
\begin{align*}
	\tilde{\pi}(a)\xi(r) = \pi(\alpha_{r^{-1}}(a))\xi(r), \text{ for } a \in A, \ r \in G \text{ and } \xi \in L^2(G,H).
\end{align*}
Then $(\tilde{\pi},\lambda)$ is a covariant representation of $G \acts \alpha A$ on $L^2(G,\H)$.
\label{cross:regularrep}
\end{lemma}
\begin{proof}
Let $a \in A$ and $r \in G$. Then for all $\xi \in L^2(G,\H)$ and $s \in G$:
\begin{align*}
	\lambda_g (\tilde{\pi}(a) \lambda_{g^{-1}} \xi)(s) = \pi(\alpha_{g^{-1}r}^{-1}(a))\lambda_g^* \xi(g^{-1}s) = \pi(\alpha_{r^{-1}}\alpha_g(a))\xi (g g^{-1} s) = \tilde{\pi}(\alpha_g(a))\xi(s),
\end{align*}
as wanted.
\end{proof}
\begin{corollary}
To every $C^*$-dynamical system $G \acts \alpha A$, there exists a covariant representation.
\end{corollary}
In light of the above, we may define the following important class of covariant representations of a $C^*$-dynamical system:
\begin{definition}
Given $G \acts \alpha A$, then for any representation $\pi$ of $A$ on $\H$, the above covariant representation is called the \myemph{induced regular representation}, and we will write $\mathrm{Ind}_e^G(\pi) := (\tilde{\pi}, \lambda)$. Any representation of the above form is called a \myemph{regular representation}.
\end{definition}
As we shall see, the integrated form of the above plays a crucial role in the theory regarded in this thesis, but first, we shall need the following lemma:
\begin{lemma}
Suppose that $G \acts \alpha A$, and $\pi$ is a representation of $A$ on $\H$. Then $\mathrm{Ind}_e^G \pi$ is non-degenerate if $\pi$ is non-degenerate.
\label{cross:regrepnondeg}
\end{lemma}
\begin{proof}
Recall that the set of functions $ f \otimes \eta \colon r \mapsto f(r) \eta$ with $f \in \cc(G)$ and $\eta \in \H$ has dense linear span in $L^2(G,\H)$, and since $\pi$ is non-degenerate, the set $\{ f \otimes \pi(a) \eta \mid \eta \in \H, \ f \in \cc(G) \text{ and } a \in A \}$ has dense linear in $L^2(G, \H)$ as well. Let $\varepsilon > 0$ and let $(e_i)$ be an approximate identity of $A$. By the previous remark and the fact that $\tilde{\pi}$ is a linear contraction, it suffices to see that 
\begin{align*}
	\tilde{\pi}(e_i) (f \otimes \pi(a) \xi)(r)=f(r) \pi(\alpha_{r}^{-1}(e_i)a) \xi   \stackrel 2 \to f(r) \pi(a) \xi,
\end{align*}
for $f \in \cc(G)$, $\xi \in \H$ and $a \in A$ and $r \in \supp f$. The argument thus boils down to showing that $\alpha_r^{-1}(e_i) a \to a$, which an easy argument of contradiction using compactness of $\supp f$ and the approximate identity property passing to subnets shows that this is true.
\end{proof}
\begin{remark}
In particular, if $\pi$ is a non-degenerate representation of $A$, then the integrated form, $\tilde{\pi} \rtimes \lambda$, of $\mathrm{Ind}_e^G \pi$ is non-degenerate by \Cref{cross:integform}. This is not the only property which carries over from the map $\mathrm{Ind}_e^G \pi \mapsto \tilde{\pi} \rtimes \lambda$, but before we further comment on this, we must introduce some new maps
\end{remark}
\begin{definition}
Suppose that $G \acts \alpha A$. Let $f \otimes a  \in \cc(G,A)$ be the map $r \mapsto f(r)a$. For $s \in G$, let $\iota_G(s)$ be the linear extension of the map $\lambda_s \otimes \alpha_s \colon f \otimes a \mapsto \lambda_s f \otimes \alpha_s(a)$ to all of $\cc(G,A)$, i.e., $\iota_G(s) h(t) = \alpha_s(h(s^{-1}t))$ for $h \in \cc(G,A)$.
\end{definition}
And, we have the following handy lemma:
\begin{lemma}
Suppose that $(\pi,u)$ is a covariant representation of a $C^*$-dynamical system $G \acts \alpha A$, and let $\pi \rtimes u$ denote the representation of $\cc(G,A,\alpha)$. Then, for $f \in \cc(G,A)$ and $r \in G$ it holds that
\begin{align*}
	 u_r^*  \circ \pi \rtimes u \circ \iota_G(r) (f) =   \pi \rtimes u (f)
\end{align*}
\label{cross:iotaG}
\end{lemma}
\begin{proof}
It follows from the calculation
\begin{align*}
\pi \rtimes u (\iota_G(r)f) = \int_G \pi(\iota_G(r) f(s) ) u_s \d s &= \int_G \pi(\alpha_r(f(r^{-1}s))) u_ru_{r^{-1}s} \d s\\
&= \int_G u_r \pi(f(r^{-1}t)) u_{r^{-1}s} \d s \\
&= u_r \int_G  \pi(f(s)) u_s \d s \\
&= u_r \circ \pi \rtimes u (f).
\end{align*}
\end{proof}
\begin{lemma}
Suppose that $G \acts \alpha A$ and $\pi$ is a faitful representation of $A$ on $\H$. Then the integrated form, $\tilde{\pi} \rtimes \lambda$, of $\mathrm{Ind}_e^G (\pi)=(\tilde \pi , \lambda)$ is a faithful representation of $\cc(G,A,\alpha)$.
\label{cross:regfaith}
\end{lemma}
\begin{proof}
Let $0 \neq f \in \cc(G,A)$, and let $r \in G$ witness this, i.e, $f(r) \neq 0$. By \Cref{cross:iotaG} we can replace $f$ by $\iota_G(r^{-1})f$ and assume that $r=e$. So without loss of generality we assume $r = e$, and let $\xi, \eta \in \H$ witness faithfulness of $\pi$, i.e., such that $\langle \pi ( f(e)) \xi, \eta \rangle \neq 0$. Pick an open neighborhood $e \in V$ such that for $s,r \in V$ we have 
\begin{align*}
	| \langle \pi(\alpha_r^{-1}(f(s))) \xi  - \pi(f(e)))\xi , \eta \rangle|  < \frac{| \langle \pi (f(e)) \xi , \eta\rangle|}{3}.
\end{align*}
Let $W \subseteq V$ be a symmetric neighborhood such that $W^2 \subseteq V$, and let $\varphi \in \cc(G)$ such that $\varphi \geq 0$ and $\supp \varphi \subseteq W$ and
\begin{align*}
\int_G \int_G \varphi(s^{-1}r) \varphi(r) \d r \d s = 1.
\end{align*}
Let $\zeta = \varphi \otimes \xi$ and $\psi = \varphi \otimes \eta$ be the above defined elements of $L^2(G,H)$. Then
\begin{align*}
\langle \tilde{ \pi } \rtimes \lambda (f) \zeta, \psi \rangle  =  \int_G \langle \tilde{\pi}(f(s)) \lambda_s \zeta , \psi\rangle \d s &= \int_G \int_G \langle \tilde{\pi} f(s) \lambda_s \zeta(r) , \psi(r) \rangle \d r \d s\\
&= \int_G \int_G \langle \pi(\alpha_{r^{-1}}f(s)) \zeta\left( r^{-1}s \right) \psi(s) \rangle \d r \d s\\
&= \int_G \int_G \varphi(r^{-1}s) \varphi(r) \langle \pi(\alpha_{r^{-1}}f(s)) \xi , \eta \rangle \d r \d s
\end{align*}
and hence for all $r,s \in W$ we have
\begin{align*}
\left | \langle\tilde{\pi} \rtimes \lambda (f) \zeta , \psi \rangle - \langle \pi(f(e)) \zeta , \psi \rangle\right | &=  \left |  \int_G \int_G\varphi(r^{-1}s) \varphi (s) (\langle \pi(\alpha_{r^{-1}} f(s)) \xi , \eta \rangle - \langle \pi(f(e)) \xi , \eta \rangle  \d r \d s \right | \\
&\leq \int_G \int_G \varphi(r^{-1}s ) \varphi (s) | \langle \pi(\alpha_{r^{-1}}(f(s))) \xi , \eta\rangle - \langle \pi(f(e)) \xi , \eta \rangle | \d r \d s \\
&< \int_G \int_G \varphi(r^{-1}s) \varphi(s) \frac{| \langle \pi(f(e)) \xi, \eta \rangle|}{3} \d r \d s \\
&= \frac{| \langle \pi(f(e)) \xi , \eta \rangle|} {2},
\end{align*}
and in particular, it follows that $\tilde{\pi} \rtimes \lambda (f) \neq 0$.
\end{proof}

With the above result in hand, we can finally give meaning to a universal norm for the general convolution algebra $\cc(G,A,\alpha)$. Like in the discrete case, we define the \myemph{universal norm} of $f \in \cc(G,A,\alpha)$ to be non-negative real number
\begin{align*}
\lv f \rv_u &:= \sup\left\{ \lv \pi \rtimes u (f)\rv \mid  \pi \rtimes u \text{ is a covariant represenation of }(A,G ,\alpha) \right\}\\
&\leq \lv f \rv_1,
\end{align*}
and it is easy to see that the assignment $\cc(G,A,\alpha) \ni f \mapsto \lv f \rv_u$ is a norm: It clearly satisfies $\lv f + g \rv_u \leq \lv f \rv_u + \lv g \rv_u$ and $\lv cf \rv_u = |c| \lv f \rv_u$ for $f,g \in \cc(G,A,\alpha)$ and $c \in \C$. If $\lv f \rv_u = 0$ then $\tilde{\pi} \rtimes \lambda (f) = 0$ for all faithful $\pi$, which implies by \Cref{cross:regfaith} that $f = 0$. Moreover, we see that $\cc(G,A,\alpha)$ is a pre $C^*$-algebra, since $\pi \rtimes u$ is a homomorphism for all covariant representations $(\pi,u)$, and thus $\lv \pi \rtimes u (f^*\ast f)\rv = \lv \pi \rtimes u(f)^* \circ \pi \rtimes u(f)\rv = \lv \pi \rtimes u (f)\rv^2$, so that $\lv f^* \ast f\rv_u = \lv f \rv_u^2$. 
\begin{note}
While it indeed is not possible to take the supremum over a class, it can be shown that it is enough to take the supremum over cyclic covariant representations, which is indeed a set - this is due to the fact that every non-degenerate representation of a $^*$-algebra is a direct sum of cyclic representations, see e.g. \cite[Theorem 5.1.3]{murphy2014c}. For now, we stick with the following:
\end{note}
\begin{lemma}
Let $G \acts \alpha A$ be a $C^*$-dynamical system and let $(\pi,u)$ be a covariant representation on $\H$. Let 
\begin{align*}
	V_\pi = \overline \Span\left\{ \pi(a) \xi \mid a \in A \text{ and } \xi \in \H \right\} \subseteq \H,
\end{align*}
then $V_\pi$ is invariant under both $\pi$ and $u$ by covariance, and if $\pi^{V_\pi}$ and $ u^{V_\pi}$ denotes the corresponding subrepresentations on $V_{\pi}$, we have for all $f \in \cc(G,A,\alpha)$ 
\begin{align*}
	\lv \pi ^{V_\pi} \rtimes u^{V_\pi}(f) \rv = \lv \pi \rtimes u (f) \rv,
\end{align*}
so that
\begin{align*}
	\lv f \rv_u = \sup\left\{ \lv \pi \rtimes u(f) \rv \mid  (\pi,u) \text{ is a non-degenerate covariant representation of } G \acts \alpha A\right\}.
\end{align*}
\label{cross:essnorm}
\end{lemma}
\begin{proof}
This is a standard proof, and we refer to \cite[52]{williamscrossed} for the proof.
\end{proof}

This allows us to define the general crossed product as the completion of $\cc(G,A,\alpha)$:
\begin{definition}
For a $C^*$-dynamical system $(A,G,\alpha)$, we define the \myemph{full crossed product} (or simply the crossed product) of $A$ and $G$ to be norm closure of $\cc(G,A,\alpha)$ in the norm $\lv \cdot \rv_u$, and we will denote it by $A \rtimes_\alpha G$.
\end{definition}
\begin{remark}
Another (equivalent) formulation of the crossed product is to consider the representation of $L^1(G,A)$ given by the direct sum of all non-degenerate covariant representations of $L^1(G,A)$. Again, one can reformulate the above to only consider cyclic covariant representations, and use the fact that every representation of a $C^*$-algebra is a sum of cyclic representations.
\end{remark}
The perhaps most trivial example is in fact not a degenerate example, for it gives us a way to define the full group $C^*$-algebra of a locally compact group $G$:
\begin{example}
Recall the definition of the full group $C^*$-algebra $C^*(G)$, which is given as the closure of $\cc(G)$ (or $L^1(G)$) coming from the direct sum of all non-degenerate representations of the Banach $^*$-algebra $L^1(G)$. This is equivalent to the construction above when we let $G \acts \C$ be the trivial action $1$. 
\end{example}
Another fundamental example is the following:
\begin{example}
Let $G = \Z$ and let $\theta \in \R$. Define an action of $\Z$ on $C(S^1)$ by $\alpha_\theta(n)f(w) = f(e^{-2\pi \theta n} w)$. The corresponding crossed product $C(S^1) \rtimes_{\alpha_\theta} \Z$ is called the \myemph{(Ir)rational Rotation Algebra} (depending on wether $\theta$ is irrational or rational).
\end{example}
In fact, for $\theta$ irrational, we obtain one of the prototypes of simple crossed product algebras, which we recall is the topic of this thesis. Nice!

Inconvenient as it is, there is no guarantee that $A \rtimes_\alpha G$ contains a copy of $A$ or $G$ except under certain additional conditions on $A$ and $G$. However, we can always embed $A$ and $G$ into the Multiplier Algebra $M(A \rtimes_\alpha G)$ of $A \rtimes_\alpha G$:
\begin{lemma}
Suppose that $G \acts \alpha A$ is a $C^*$-dynamical system. Define for $a \in A$ the map $\iota_A(a) \colon A \rtimes_\alpha G \to A \rtimes_\alpha G$ to be the extension of the map defined on $\cc(G,A,\alpha)$ by
\begin{align*}
	\iota_A(a) f(s) = af(s), \text{ for } f \in \cc(G,A,\alpha) \text{ and } s \in G.
\end{align*}
Then $\iota_A \colon A \to M(A \rtimes_\alpha G)$, $a \mapsto \iota_A(a)$, is a faithful non-degenerate $^*$-representation, satisfying
\begin{align*}
	\overline{\pi \rtimes u } \ \iota_A(a) = \pi(a),
\end{align*}
for all covariant representations $(\pi,u)$ of $G \acts \alpha A $.
\end{lemma}
\begin{proof}
First, for covariant representations $(\pi,u)$ of $G \acts \alpha A$, we see that
\begin{align*}
\pi \rtimes u(\iota_A(a) f) = \int_G \pi(a) \pi(f(s)) u_s \d s = \pi(a) \pi \rtimes u (f),
\end{align*}
for all $f \in \cc(G,A,\alpha)$ and $a \in A$, so $\lv \iota_A(a) f \rv \leq \lv a \rv \lv f \rv$. Hence $\iota_A(a)$ extends to an operator, which we also denote by $\iota_A(a)$. If $a \in A$ and $f \in \cc(G,A,\alpha)$, then 
\begin{align*}
\left( \iota_A(a) f \right)^*(s) = \Delta(s^{-1}) \alpha_s(f(s^{-1})^*) \alpha_s(a^*) = f^*(s) \alpha_s(a^*), \text{ for }s \in G,
\end{align*}
implying that
\begin{align*}
\left( \iota_A(a) f \right)^* \ast g (s) = \int_G f^*(r) \alpha_r(a^*) \alpha_r(g(r^{-1}s)) \d r &= \int_G f^*(r) \alpha_r \left(  (\iota_A(a^*) g)(r^{-1}s)\right) \d r \\
&= f^* \ast(\iota_A(a^*) g)(s).
\end{align*}
We conclude that $\iota_A(a^*)$ is an adjoint (with respect to the canonical $A \rtimes_\alpha G)$-valued inner product) on the dense sub$*$-algebra $\cc(G,A,\alpha) \subseteq A \rtimes_\alpha G$, hence everywhere by continuity, i.e., $\iota_A(a) \in M(A \rtimes_\alpha G)$. 

For injectivity, let $\pi$ be any faithful, non-degenerate representation of $A$, and let $\overline{\tilde \pi \rtimes \lambda}$ denote the extension of $\tilde \pi \rtimes \lambda$ to $M(A \rtimes_\alpha G)$. By \ref{cross:integform} and the above, we see that $\tilde \pi(a) = \overline{\tilde \pi \rtimes \lambda } \circ \iota_A(a)$, hence $\iota_A(a) = 0 \implies \tilde \pi(a) = 0 \implies a = 0$, since $\tilde \pi$ is faithful.

For non-degeneracy, recall that the set of functions $\varphi \rtimes b$ for $\varphi \in \cc(G)$ and $b \in A $ span a dense subset of $\cc(G,A)$ in the inductive limit norm and hence in $L^1$-norm by \cref{int:indlmdense}. This means that this set also spans a dense subset of $A \rtimes_\alpha G$ in the universal norm. Also, for $a \in A$, we see that $\iota_A(a) \varphi \rtimes b = \varphi \rtimes ab$, from which we conclude that the set $\iota_A(A) A \rtimes_\alpha G$ is dense, finishing the proof.
\end{proof}
\begin{lemma}
There is an injective and strictly continuous group homomorphism
\begin{align*}
	\iota_G \colon G \to \mathcal{U} M(A \rtimes_\alpha G),
\end{align*}
such that for $f \in \cc(G,A,\alpha) \subseteq A \rtimes_\alpha G$ we have
\begin{align*}
	\iota_G(s) f(t) = \alpha_s (f(s^{-1}t)),
\end{align*}
for all $s,t \in G$, and if $(\pi,u)$ is a covariant representation of $G \acts \alpha A$, then
\begin{align*}
	\overline{\pi \rtimes u} \iota_G(s) =u_s,
\end{align*}
for $s \in G$.
\end{lemma}
Before we commence the proof of the above, we want to point out that the above to lemmas ensures that for each covariant representation $(\pi , u)$ of $G \acts \alpha A$ we obtain the following commutative diagram
\begin{equation}
\begin{tikzcd}
	G \arrow[rr, "\iota_G", hook] \arrow[rrd, "u"'] && M(A \rtimes_\alpha G) \arrow[d, "\overline{\pi \rtimes u }"] && A \arrow[ll,"\iota_A"',hook] \arrow[lld, "\pi"]\\
	& &\mathbb{B}(\H)&&
\end{tikzcd}
\label{diag:iotacom}
\end{equation}
And we shall see that this pair is in fact a covariant pair. But first, the proof of the above lemma:
\begin{proof}
Let $\iota_G(s)$ be defined on $\cc(G,A,\alpha)$ as earlier. By \Cref{cross:iotaG}, we see that
\begin{align*}
	\lv \iota_G(r) f \rv_u = \lv f \rv_u,
\end{align*}
for all $f \in \cc(G,A,\alpha)$, and so it extends to $A \rtimes_\alpha G$. Let $s,r \in G$, then for $f \in \cc(G,A,\alpha)$, we see that
\begin{align*}
	(\iota_G(r) f(s))^* = \Delta(s^{-1}) \alpha_s ( (\alpha_r(f(r^{-1}s^{-1}))^*)= \Delta(s^{-1}) \alpha_{sr}(f(r^{-1}s^{-1})^*),
\end{align*}
which, combined with the fact that $\Delta(s^{-1}) \d s$ is the integrand of a right Haar measure, can be used to show that $\iota_G(r)^* = \iota_G(r^{-1})$, so that $\iota_G(r) \in M(A \rtimes_\alpha G)$. In fact, since clearly $\iota_G(rs) = \iota_G(r) \circ \iota_G(s)$ for $s,r \in G$, it is a unitary element.

Let $(\pi,u)$ be any non-degenerate covariant representation of $G \acts \alpha A$. By \Cref{cross:iotaG}, we see that $u_s  = \overline{ \pi \rtimes u} \iota_G(s)$ for all $ s \in G$. Moreover, for faithful representations $\pi$ (which also implies that $\tilde \pi$ is faithful) we have that the corresponding representation $\tilde \pi \rtimes \lambda$ is faithful on $\cc(G,A,\alpha)$. This implies that for $s \neq e$ we have $\mathrm{id} \neq \lambda_s = \overline{\tilde \pi \rtimes \lambda} \iota_G(s)$ for $s \neq e$, hence $\iota_G(s) \neq \mathrm{id}$ as well, so it is injective.
	

For continuity, let $f \in \cc(G,A,\alpha)$. We will see that $\iota_G$ is strongly continuous, which by \Cref{int:unistrictstrong} is equivalent to being strictly continuous: Let $(s_i)\subseteq G$, $s_i \to e$. Then we claim that $\iota_G(s_i) x \to x$ for $x \in A \rtimes_\alpha G$. To see this, note that we know that the set of functions $\varphi \otimes a$, $\varphi \in \cc(G)$ and $a \in A$, spans a dense subset of $\cc(G,A)$, hence it suffices to show that $\iota_G(s_i) \varphi \otimes a \stackrel{L^1}{\to} \varphi \otimes a$, but
\begin{align*}
	\lv \iota_G(r_{i}) \varphi \otimes a - \varphi \otimes a \rv_1 &= \int_G \lv \varphi(r_{i}^{-1}s) \alpha_{r_i}(a) - \varphi(r_i^{-1}s) a + \varphi(r_{i}^{-1}a) a - \varphi(s) a \rv_A \d s\\ 
	& \leq \lv \alpha_{r_i}(a) - a \rv \lv \varphi \rv_1 + \lv \lambda_{r_i} \varphi - \varphi \rv_1 \lv a \rv,
\end{align*}
which goes to $0$, by uniform continuity of $\varphi$ and the fact that $\alpha_{r_i}(a) \to a$ for all $ a\in A$ as $r_i \to e$.
\end{proof}
And, we summarize the above in the following
\begin{proposition}
For a $C^*$-dynamical system $G \acts \alpha A$, there exists a non-degenerate faithful $^*$-homomorphsim 
\begin{align*}
\iota_A \colon A \to M(A \rtimes_\alpha G),	
\end{align*}
and an injective strictly continuous unitary representation 
\begin{align*}
	\iota_G \colon G \to \mathcal{U}M(A \rtimes_\alpha G),
\end{align*}
such that $\iota_G(r) f(s) = \alpha_r(f(r^{-1}s))$  and $\iota_A(a) f(s) = af(s)$ for $f \in \cc(G,A,\alpha)$ and $r,s \in G$ and $a \in A$. Moreover, the pair $(\iota_A,\iota_G)$ is covariant with respect to $\alpha$, i.e., 
\begin{align*}
	\iota_A(\alpha_r (a) ) = \iota_G(r)  \iota_A(a) \iota_G(r)^*,
\end{align*}
and for non-degenerate covariant representations $(\pi,u)$ we have
\begin{align*}
	\overline{\pi \rtimes u} \circ \iota_A(a) = \pi(a) \text{ and } \overline{\pi \rtimes u } \circ \iota_G (s) = u_s.
\end{align*}
\label{cross:iotaprop}
\end{proposition}
\begin{proof}
The only thing left to be shown is the covariance relation, but this follows immediately from the following calculation:
\begin{align*}
	\iota_G(r) \iota_A(a) \iota_G(r)^* f(s) = \alpha_r(\iota_A(a) \iota_G(r)^{*}f(r^{-1}s)) = \alpha_r(a) \alpha_r ( \alpha_{r^{-1}}f(r r^{-1}s))= \iota_A(\alpha_r(a)) f(s),
\end{align*}
for $f \in \cc(G,a,\alpha)$, $a \in A$ and $s,r \in G$.
\end{proof}
We now embark on our final task in our construction and picture painting of the general crossed product $A \rtimes_ \alpha G$, namely we wish to properly characterize its representation theory. We are going to see that $A \rtimes_\alpha G$ is the $C^*$-algebra whose representation theory is in a one-to-one correspondance with the covariant representations of $(A,G,\alpha)$ via the map $(\pi,u) \mapsto \pi \rtimes u$, but before we finalize the description of the representation theory of $A \rtimes_\alpha G$, we need to extend the map $\iota_G$ to $C^*(G)$, the full group $C^*$-algebra of $G$, since this will provide us with a useful characterization of a dense subset of $A \rtimes_\alpha G$, so without further ado:
\begin{lemma}
There exists a $^*$-homomorphsim $\tilde \iota_G \colon C^*(G) \to M(A \rtimes_\alpha G)$ such that
\begin{align*}
	\tilde \iota_G(z) = \int_G z(s) \iota_G(s) \d s, 
\end{align*}
for all $z \in \cc(G)$
\label{cross:2.35}
\end{lemma}
\begin{proof}
For all $z \in \cc(G)$, the map $s \mapsto z(s) \iota_G(s)$ is strictly continuous, and we may define the integral by \Cref{int:multstrictintegral}. Let $z,w \in \cc(G)$, then
\begin{align*}
	\tilde \iota_G(z \ast w) &= \int_G \int_G z(r)w(r^{-1}s) \d r \iota_G(s) \d s \\
	&= \int_G \int_G z(r) w(r^{-1}s) \iota_G(s) \d s \d r\\
	&= \int_G \int_G z(r) w(s) \iota_G(rs) \d s \d r\\
	&= \int_G z(r) \iota_G(r) \d r \int_G w(s) \iota_G(s) \d s \\
	&= \tilde \iota_G(z) \tilde \iota_G(w),
\end{align*}
so if it is well-defined and continuous on $C^*(G)$, it will be a $^*$-homomorphism, since the integral also respects taking adjoints. 

Let $\pi$ be any faithful and non-degenerate representation of $A \rtimes_\alpha G$, so that its extension $\overline \pi $ to $M(A \rtimes_ \alpha G)$ is faithful aswell.

Consider the composition $u \colon s \mapsto \overline{\pi}( \iota_G(s))$, which is a unitary valued homomorphism. It is strongly continuous: Suppose that $f \in \cc(G,A)$ and $\xi \in \H_\pi$, then for $s \in G$ we see that $\overline{ \pi} (\iota_G(s)) \pi(f) \xi = \pi(\iota_G(s) f) \xi$, and it follows that  $s \mapsto u_s \xi$ is continuous for all $\xi \in \H$ by continuity of $s \mapsto \iota_G(s)f$ and non-degeneracy of $\pi$. Hence:
\begin{align*}
\overline \pi(\tilde \iota_G(z)) = \overline \pi \left(  \int_G z(s) \iota_G(s) \d s \right) &=\int_G z(s) \overline \pi(\iota_G(s)) \d s \\
&= u(z),
\end{align*}
Where the last line is the representation of $L^1(G)$ correpsonding the the unitary representation $\overline \pi \circ \iota_G$ of $G$, which we shall discuss later on. In particular, this implies that
\begin{align}
	\lv \tilde \iota_G(z)\rv = \lv u(z)\rv \leq \lv z \rv,
\end{align}
for all $z \in \cc(G)$, so it extends to a $^*$-homomorphism $C^*(G) \to M(A \rtimes_\alpha G)$, which we will also denote by $\iota_G$.
\end{proof}
\begin{corollary}
Let $G \acts \alpha A$ be a $C^*$-dynamical system and let $a \in A$, $f,h \in \cc(G,A)$ and $\varphi \in \cc(G)$. Then
\begin{align*}
\iota_A(a) \iota_G(\varphi) &= \varphi \otimes a,\\
\int_G \iota_A(f(s)) \iota_G(s)h  \d s &= f \ast h \text{ and }\\
\int_G \iota_A(f(s)) \iota_G(s) \d s & = f,
\end{align*}
where we've identified $\cc(G,A,\alpha) \subseteq A \rtimes_\alpha G \subseteq M(A \rtimes_\alpha G)$.
\label{cross:iotaresults}
\end{corollary}
\begin{proof}
Let $(\pi, u)$ be a non-degenerate covariant representation of $G \acts \alpha A$ and notice that
\begin{align*}
	\overline{\pi \rtimes u}\left( \int_G \iota_a(f(s)) \iota_G(s) \d s  \right) = \pi \rtimes u (f),
\end{align*}
which combined with \cref{cross:essnorm} shows the first and last identity. The second equality follows from the last and  \cref{int:multstrictintegral}.
\end{proof}
\todo{swap around the above and below}
\begin{remark}
In particular, by the above and \cref{int:indlmdense} we see that the set of functions of the form $\iota_A(a) \iota_G(z)$ for $a \in A$ and $z \in \cc(G)$ span a dense subset of $A \rtimes_\alpha G$. This will be useful in the following:
\end{remark}
\begin{theorem}
Let $G \acts \alpha A$ be a $C ^*$-dynamical system and let $\H$ be a Hilbert space. If $L \colon A \rtimes_\alpha G \to \mathbb{B}(\H)$ is a non-degenerate $*$-representation, then there exists a non-degenerate covariant representation $(\pi,u)$ of $G \acts \alpha A$ into $ \mathbb{B}(\H)$ such that $L= \pi \rtimes u$ which can be realised by 
\begin{align*}
	u_s = \overline L(\iota_G(s)) \text{ and } \pi(a) = \overline L(\iota_A(a)),
\end{align*}
for all $s \in G$ and $a \in A$. In fact, there is a bijective correspondance between covariant representations of $G \acts \alpha A$ and $*$-representations of $A \rtimes_\alpha G$.
\label{cross:onetoonecor}
\end{theorem}
\begin{proof}
We already have seen the assignment $(\pi,u) \mapsto \pi \rtimes u$. For the converse, suppose that $L$ is a non-degenerate representation of $A \rtimes_\alpha G$. Let $\pi, u$ be defined by the above. It is clear that $\pi$ is a $*$-representation of $A$ on $\mathbb{B}(\H)$ and similarly that $u$ is a strongly continuous unitary-valued group representation on $\mathbb{B}(\H)$, since $\iota_G$ is strictly and hence strongly continuous. To see that $(\pi,u)$ is non-degenerate, let $(e_i)$ be an approximate unit for $A$. Then $\iota_A(e_i) \to 1$ strictly in $M(A \rtimes_\alpha G)$, and so $\pi(e_i) = \overline L(\iota_A(e_i)) \to I$ strictly, so in particular $\pi(e_i) \xi \to \xi$ for all $\xi \in \H$ and $\pi$ is non-degenerate. Now, let $s \in G$ and $a \in A$, then
\begin{align*}
	u_s \pi(a) u_s^* = \overline L(\iota_G(s) ) \overline L (\iota_A(a)) \overline L(\iota_G(s))^* = \overline L( \iota_A(\alpha_s(a))) = \pi(\alpha_{s}(a)),
\end{align*}
so the pair is a proper non-degenerate covariant representation of $G \acts \alpha A$. Finally, using \cref{int:multstrictintegral}, we see that
\begin{align*}
\pi \rtimes u(\iota_A(a) \iota_G(z)) &= \pi(a) \circ \overline{\pi \rtimes u}(\iota_G(z)) = \overline L(\iota_A(a)) \int_G z(s) \overline L(\iota_G(s)) \d s \\
&= \overline L(\iota_A(a)) \overline L(\iota_G(z)) \\
&= \overline L(\iota_A(a) \iota_G(z)),
\end{align*}
for all $z \in \cc(G)$ and $a \in A$, which by the above remark ensures that $L = \pi \rtimes u$ on all of $A \rtimes_ \alpha G$.
\end{proof}

\ssection{Properties of Crossed products}
We want to examine some interesting properties of Crossed products, such as what happens when $G$ is amenable or how can we concretize the statespace $\mathcal{S}(A\rtimes_\alpha G)$. To do this, we will need to recall the usual definition of a positive definite function $G \to \C$ and extend it to the case of positive definite functions $G \to A^*$. The following is from \cite{pedersenalgauto}

\begin{definition}
Suppose that $G \acts \alpha A$ is a dynamical system, then a bounded continuous (with respect to the norm on $A^*$) function $\varphi \colon G \to A^*$ is \myemph{positive definite} if any of the following equivalent statements hold:
\begin{enumerate}
	\item There is a positive functional $\psi \in (A \rtimes_\alpha G)^*$ such that $\varphi(t)(a) = \psi(\iota_A(a) \iota_G(t))$ for all $t \in G$ and $a \in A$,
	\item $\sum_{i,j} \varphi (s_i^{-1}s_j)(\alpha_{s_{i}^{-1}}(x_{i}^* x_{j})) \geq 0 $ for all finite sets $s_1,\dots,s_n \in G$ and $x_1,\dots,x_n \in A$,
	\item $\sum_{i,j}\int_G \varphi(t)(x_{i}^*\alpha_t(x_j))(f_i^* \times f_j)(t) \d t \geq 0$ for all finite sets $f_1,\dots,f_n \in \cc(G,A,\alpha)$ and $x_1,\dots,x_n \in A$, and
	\item $\int \varphi(t)(f^* \rtimes f)(t) \d t \geq 0$ for all $f \in \cc(G,A,\alpha)$.
\end{enumerate}
\end{definition}
\begin{remark}
Note that in the above, a priori without explanation the first definition does not make sense, since $\iota_A(a) \iota_G(t)$ might not be an element of $A \rtimes_\alpha G$, however, since states of $A \rtimes_\alpha G$ correspond to non-degenerate representations which extend to $M(A \rtimes_\alpha G)$, we may in fact extend each positive functional $\psi$ on $A \rtimes_\alpha G$ to all of $M(A \rtimes_\alpha G)$. We will also denote this extension by $\psi$. 
\end{remark}
For $A = \C$ and $\alpha = 1$ is the trivial representation, the above becomes the regular result of about equivalent definitions of positive definite functions on $G$, as described in \cite{folland2016fourier} or in \cite{pedersenalgauto} 

The first definition is the general motivation behind the above for us, since we will be dealing with a class of functions arising this way, which leads to the following definition:
\begin{definition}
Suppose that $G \acts \alpha A$. For each $\varphi \in (A \rtimes_\alpha G)^*$, define the map $\Gamma_\varphi \colon G \to A^*$, given by
\begin{align*}
	\Gamma_\varphi(t)(a) :=\varphi(\iota_A(a) \iota_G(t)) \in \C,
\end{align*}
for $t \in G $ and $a \in A$. The set of functions $G \to A^*$ defined in this way is denoted by $B(A \rtimes_ \alpha G)$, and the set of $\Gamma_\varphi \in B(A \rtimes_ \alpha G)$ for $\varphi \geq 0$,  is denoted by $B_+(A \rtimes_\alpha G)$. We denote by $B_+^1(A \rtimes_\alpha G)$ the functions such that $\lv \Gamma_\varphi (e) \rv = 1$ and $\varphi \geq 0$. \todo{bounded using C-S}
\end{definition}
\begin{remark}
Note that we may use the fact that every non-degenerate representation of $A \rtimes_\alpha G$ is of the form $\pi \rtimes u$ for some covariant, non-degenerate representation $(\pi,u)$ of $G \acts \alpha A$ to show that for all $\varphi \in \mathcal{S}(A \rtimes_\alpha G)$ we have
\begin{align}
	\varphi(y) = \int_G \Gamma_\varphi(t)(y(t)) \d t
	\label{eq:gammafun}
\end{align}
for all $y \in \cc(G,A,\alpha) \subseteq A \rtimes_\alpha G$: To $\varphi$ we associate a GNS triple $(\pi_\varphi,\xi, \H)$ with $\pi_{\varphi}$ non-degenerate. By \cref{cross:onetoonecor}, we see that $\pi_\varphi$ is of the form $\pi \rtimes u$ for a suitable covariant representation $(\pi,u)$, and so
\begin{align*}
\varphi(y) = \langle \pi_\varphi(y) \xi, \xi \rangle = \langle \pi \rtimes u(y) \xi, \xi\rangle &= \int_G \langle \pi(y(t)) u_t \xi, \xi \rangle \d t\\
&= \int_G \langle \overline{\pi \rtimes u }(\iota_A(y(t))) \overline{\pi \rtimes u } (\iota_G(t)) \xi , \xi \rangle \d t\\
&= \int_G \langle \overline{\pi \rtimes u} (\iota_A(y(t)) \iota_G(t)) \xi , \xi \rangle\d t\\
&= \int_G \varphi(\iota_A(y(t)) \iota_G(t)) \d t \\
&= \int_G \Gamma_\varphi (t) (y(t)) \d t.
\end{align*}
In particular, density of $\cc(G, A,\alpha) \subseteq A \rtimes_\alpha G$ implies by the above that the assignment $\mathcal{S}(A \rtimes_\alpha G) \ni \varphi \mapsto \Gamma_\varphi \in B_+^1(A \rtimes_\alpha G)$ is a bijection. Combining this with linearity of the assignment, i.e., that $\Gamma_{\alpha \varphi + \beta \psi}= \alpha \Gamma_\varphi + \beta \Gamma_\psi$ for $\alpha, \beta \in \C$ and $\varphi, \psi \in (A \rtimes_\alpha G)^*$, this shows that the assignment also induces a bijection between $(A \rtimes_\alpha G )^*$ and $B(A \rtimes_\alpha G)$.
\todo{Continuity of $\Gamma_\varphi$ using cauchy-schwarz}
\end{remark}
For $A = \C$ again and $\alpha = 1$, we may represent $G$ on $M(C^*G)$ via $t \mapsto \delta_t$, and then $\Gamma_\varphi$ becomes the map $t \mapsto \varphi(\delta_t)$ for $ \varphi \in C^*(G)^*$ which leads to the following result (see \cite[7.1.11]{pedersenalgauto}):
\begin{proposition}
For $A= \C$ and $\alpha = 1$, the isomorphism above is a homeomorphism from $\mathcal{S}(C^*(G))$ with the weak$^*$-topology onto the set of positive-definite functions on $G$ with $\Gamma_\varphi(e)=1$ equipped with the topology of uniform convergence on compacta.
\label{mult:OGstatehomeo}
\end{proposition}

Generalizing the above, we introduce a topology on $B_+^1(A \rtimes_\alpha G)$, which extends the topology of uniform convergence on compacta of $G$ for the general case of $G \acts \alpha A$:
\begin{definition}
We define \myemph{the topology of weak$^*$ uniformly convergence on compacta} on $B_+^1(A \rtimes_\alpha G)$ to be the topology satisfying that a net $\Gamma_{\varphi_i} \subseteq B_+^1(A \rtimes_\alpha G)$ converges to $\Gamma_{\varphi} \in B_+^1(A \rtimes_\alpha G)$ if and only if for each $a \in A$, the restriction of the map
\begin{align*}
	t \mapsto \Gamma_{f_i}(t)(a)
\end{align*}
to any compact subset $K \subseteq G$ is uniformly convergent to the map $t \mapsto \Gamma_\varphi(t)(a)$ restricted to $K$. A basis for the topology consists of the sets
\begin{align*}
	V(\varepsilon, f, K, a) :=\left\{ f' \in B_+^1(A \rtimes_\alpha G) \mid \sup_{t \in K}| f'(t)(a) - f(t)(a) | < \varepsilon \right\},
\end{align*}
where $f \in B_{+}^1(A \rtimes_\alpha G)$, $\varepsilon > 0$, $a \in A$ and $K \subseteq G$ is compact.
\end{definition}

And this allows us to generalize the above result to the general case:
\begin{proposition}
The map $\mathcal{S}(A \rtimes_\alpha G) \ni \varphi \mapsto \Gamma_\varphi \in B_+^1(A \rtimes_\alpha G)$ is a homeomorphism with respect to the weak$^*$-topology on $\mathcal{S}(A \rtimes_\alpha G)$ and the topology of weak$^*$ convergence uniformly on compacta of $G$.	
\end{proposition}
\begin{proof}
We have already established the bijectivity. For the last part, suppose that we have a convergent net $(\varphi_i) \subseteq \mathcal{S}(A \rtimes _\alpha  G)$	with limit $\varphi \in \mathcal{S}(A \rtimes_\alpha G)$. We will in the following also write $\Gamma_\varphi$ to denote the extension from $A$ to $A^1$. For $x,y \in A^1$, define the map $\Psi_{\varphi}(x,y) \colon G \to \C$ by $t \mapsto \Gamma_\varphi(x^* \alpha_t(y))$. The polarization identity and linearity ensures that if for all $x \in A^1$ we have $\Psi_{\varphi_i}(x,x) \to \Psi_{\varphi}(x,x)$ in the topology of convergence uniformly on compacta then $\Psi_{\varphi_i}(x,y) \to \Psi_{\varphi}(x,y)$ as well. By \Cref{mult:OGstatehomeo}, we see that $\Psi_{\varphi_i}(x,x) \to \Psi_{\varphi}(x,x)$ in the topology of convergence uniformly on compacta of $G$, and setting $y = 1_{A^1}$, we see that $\Gamma_{\varphi_i} \to \Gamma_\varphi$.

Conversely, suppose that $\Gamma_{\varphi_i} \to \Gamma_{\varphi}$, then in particular $\varphi_{i}(f \otimes x) \to \varphi(f \otimes x)$ for all $x \in A$ and $f \in \cc(G)$ by \Cref{eq:gammafun} and hence on all of $A \rtimes_\alpha G$, since the functions $f \otimes x$ span a dense subset of $A \rtimes_\alpha G$.
\end{proof}

For discrete groups, it is well-known \todo{add cite Brown Ozawa} that if $G$ is amenable (or more generally acts amenably on $A$), i.e., has a left-invariant mean $m \in \ell^\infty(G)$, then there is only one crossed product of $G$ and any $C^*$-algebra $A$ (if $G$ is not amenable, but admits an amenable action on $A$, then the statement is true for crossed products with $A$ as well). We will see that this is true for locally compact groups as well - in particular, it is true for abelian groups, c.f. \cite[appendix G]{bekka2008kazhdan}.

Alas, we continue:
\begin{lemma}
Suppose that $G \acts \alpha A$ and $(\pi,\H)$ is a cyclic representation of $A$ with respect to a set $(\xi_i) \subseteq \H$, and suppose that $(\eta_j) \subseteq L^2(G)$ is cyclic for the left-regular representation $\lambda$ of $G$. Then the set of $\xi_i \otimes \eta_j$ is cyclic for the representation $(\tilde \pi \rtimes \lambda, L^2(G,\H))$.
\label{mult:cyclic1}
\end{lemma}
\begin{proof}
We will show that the orthogonal complement of $\sum_{i,j} (A \rtimes_\alpha G)(\xi_i \otimes \eta_j)$ is $0$. Suppose that $\xi$ is orthogonal to $\tilde \pi \rtimes \lambda(x) \xi_i \otimes \eta_j$ for all $i,j$. Then for all $i,j$ and $a \otimes f \in \cc(G,A) \subseteq A \rtimes_\alpha G$ we have
\begin{align*}
	0 &= \langle \tilde \pi \rtimes \lambda (a \otimes f) (\xi_i \otimes \eta_j) , \xi \rangle_{L^2}\\
	&= \int_G \langle \tilde \pi \rtimes \lambda (a \otimes f) (\xi_i \otimes \eta_j)(s) , \xi(s)\rangle_\H \d s \\
	&= \int_G \int_G \langle \pi(\alpha_{t^{-1}}\left( (a \otimes f)(s) \right) \lambda_t \eta_j(s) \xi_i , \xi \rangle_\H \d t \d s\\
	&= \int_G \int_G \langle \pi\left( \alpha_{t^{-1}} (a) \right) f(s) \eta_j(t^{-1}s) \xi_i , \xi \rangle_\H \d t\d s\\
	&= \int_G \int_G \langle \pi( \alpha_{t^{-1}}(a) ) \xi_i , \xi \rangle_H f(s) \eta_j(t^{-1}s) \d t \d s \\
	&= \int_G \langle \pi(\alpha_{t^{-1}}(a)) \xi_i , \xi \rangle_\H (f \ast \eta_j)(s) \d s.
\end{align*}
Now, let $E \subseteq G$ be any compact set. We will argue that $\xi = 0$ by showing that $\xi = 0$ on any arbitrarily large compact subset $F$ of $E$: Let $\varepsilon > 0$, and use a Banach space-valued analogue of Lusin's theorem (e.g. as in \cite[Appendix B]{williamscrossed}) to pick $F \subseteq E$ compact with $\mu(E\backslash F) < \varepsilon$ and such that $\xi|F \in \cc(F,H)=C(F,H)$. Cyclicity of $\left\{ \eta_j \right\}_{j \in J}$ implies that the set $\{f \ast \eta_j \mid j \in J \text{ and } f \in \cc(G)\}$ spans a dense subset of $C_c(G)$ (ee e.g. \cite[Proposition 3.33]{folland2016fourier}), so
\begin{align*}
	\int_{F} \langle \pi(\alpha_{t^{-1}}(x)) \xi_i , \xi(t) \rangle_H f(t) \d t = 0,
\end{align*}
for all $f \in C(F)$, from which we conclude that $F \ni t \mapsto \langle \pi(\alpha_{t^{-1}}(x)) \xi_i , \xi(t) \rangle=0$, and since $\xi_i$ was cyclic for $\pi$, this means that $\xi|_F = 0$. In particular, since $\varepsilon > 0$ was arbitrary, this means that $\xi|_E = 0$ and since $E \subseteq G$ was arbitrary, this means that $\xi = 0$, finishing the proof.
\end{proof}

We now will go through a result which will allow us to describe regular representations pointwise through approximations of states of a certain form:
\begin{proposition}
Let $(A,G,\alpha)$ be a $C^*$-dynamical system. Given $\varphi \in \mathcal{S}(A)$, for each $z \in \cc(G,A,\alpha)$ such that $\varphi(z^* \ast z(e)) = 1$, there is a vector state $\tilde \varphi_z \in \mathcal{S}(A \rtimes_\alpha G)$ of the representation $(\tilde \pi_{\varphi} \rtimes \lambda , L^2(G,\H_\varphi))$ satisfying
\begin{align*}
	\tilde \varphi_z (f) =\varphi\left( (z^* \ast f \ast z)(e) \right), \ \text{ for all } f \in \cc(G,A,\alpha).
\end{align*}
Moreover, the set of vector states contained in $(\tilde \pi_{\varphi} \rtimes \lambda , L^2(G,\H_\varphi))$ is a norm limit of vector states of this form.
\label{mult:vectorstateapprox}
\end{proposition}
\begin{proof}
For $z \in \cc(G,A,\alpha)$\todo{$C_0$ or $\cc$} with $t \mapsto \lv z(t) \rv \in L^2(G)$, define
\begin{align*}
	\xi_z(t) = \pi_{\varphi}\left( \alpha_{t^{-1}}(z(t) \right)\xi_\varphi,
\end{align*}
so that $\xi_z \in L^2(G,\H_\varphi)$. A few calculations then shows that 
\begin{align*}
	\langle \tilde \pi_\varphi \rtimes \lambda (y^* \ast y) \xi_z , \xi_z \rangle  = \lv \tilde \pi_\varphi \rtimes \lambda(y) \xi_z \rv ^2 &= \int_G \lv  \pi_\varphi ( \alpha_{t^{-1}}(y \ast z)(t) \xi_\varphi \rv^2 \d t \\
	&=\int_G \varphi\left( (y \ast z)^*(t) \alpha_t( (y \ast z)( t^{-1}) ) \right) \d t \\
	&= \varphi ( \underbrace{(y  \ast z )^*}_{= z^* \ast y^*} \ast (y \ast z) (e)) = \tilde\varphi_{z} (y^* \ast y),
\end{align*}
for all $y \in \cc(G,A,\alpha)$. By linearity, it follows that $\tilde \varphi_z$ is a vector state on $A \rtimes_\alpha G$ of $(\tilde \pi \rtimes \lambda, L^2(G,\H_\varphi))$.

For the last claim, we show that the set $\{ \xi_z \mid z \in \cc(G,A,\alpha)\}$ is dense in $L^2(G,\H_\varphi)$: Note that for $z \in \cc(G,A,\alpha)$, $f \in \cc(G)$ and $t \in G$ we have
\begin{align*}
	\tilde \pi_\varphi(z(s)) (\lambda_s (f \otimes \xi_\varphi))(t) = \pi_\varphi(\alpha_t^{-1}(z(s)))f(s^{-1}t) \xi_\varphi,
\end{align*}
which implies that 
\begin{align*}
	\tilde \pi_\varphi \rtimes \lambda (z) (f \otimes \xi_\varphi) (t) = \xi_{z \ast f}(t),
\end{align*}
where $z \ast f(s) = \int_G z(s) f(s^{-1}t) \d s \in A$. \todo{fix $\ast$'s} By cyclicity of $\xi_\varphi$ and \Cref{mult:cyclic1}, since $\cc(G) \subseteq L^2(G)$ is dense, we see that the set $\{ f \otimes \xi_\varphi \mid f \in \cc(G)\}$ is cyclic for $\tilde \pi_\varphi \rtimes \lambda$, which finishes the proof.
\end{proof}


Recall the definition of the universal representation $(\pi_u,\H_u)$ of a $C^*$-algebra $A$. The induced representation $\mathrm{Ind}(\pi_u)$ gives rise to the reduced crossed product of a $C^*$-dynamical system $(A,G,\alpha)$:
\begin{definition}
For a $C^*$-dynamical system $(A,G,\alpha)$, we define the \myemph{reduced crossed product} of $G$ and $A$ to be the algebra
\begin{align*}
	A \rtimes_{\alpha,r} G := \overline{\tilde \pi_u \rtimes \lambda (\cc(G,A,\alpha))},
\end{align*}
where $\pi_u$ is the universal representation of $A$. Equivalently, we may define $A \rtimes_{\alpha , r} G$ as the image of $A \rtimes_\alpha G$ of the extension of $\tilde \pi_u \rtimes \lambda$ to $A \rtimes_\alpha G$. 
\end{definition}
\begin{remark}
We will soon see that the above is actually independent of the choice of the representation $\pi_u$ of $A$, and may just as well be replaced with any faithful representation $\pi$ of $A$.
\end{remark}
We need define yet another subset of functionals on $A \rtimes_\alpha G$, one which closely resembles vector states of $\tilde \pi_u \rtimes \lambda$:
\begin{definition}
The set of functionals $\psi \in (A \rtimes_\alpha G)^*$ of the form
\begin{align*}
	\psi(x) = \sum_{i} \langle \tilde \pi_u \rtimes \lambda (x) \xi_i , \eta_i\rangle,
\end{align*}
where $(\xi_i)\subseteq L^2(G,\H_u)$ and $(\eta_i) \subseteq L^2(G,\H_u)$ have finite $\ell^2$-norm, i.e., $ \sum_i \lv \eta_i\rv^2 < \infty$ and $\sum_{i} \lv \xi_i \rv^2 < \infty$, is easily seen to be a norm-closed subspace of $(A \rtimes_\alpha G)^*$. We define the set 
\begin{align*}
	A(A \rtimes_\alpha G) := \{\Gamma_\psi \in B(A \rtimes_\alpha G) \mid \psi(x) = \sum_{i} \langle \tilde \pi_u \rtimes \lambda(x) \xi_i , \eta_i \rangle , \ (\xi_i), \ (\eta_i) \text{ is } \ell^2-\text{summable}\}.
\end{align*}
\end{definition}
\begin{remark}
It is an easy consequence of the commutativity of \Cref{diag:iotacom} that if $\varphi \in \mathcal{S}(A)$ satisfies the condition of \Cref{mult:vectorstateapprox} with respect to some $z \in \cc(G,A,\alpha)$, then 
\begin{align*}
\Gamma_{\tilde \varphi_z}(t)(x) = \varphi_z(\iota_A(x)  \iota_G(t)) = \langle \overline{\pi \rtimes \lambda(\iota_A(x) \iota_G(t))} \xi_z , \xi_z\rangle & = \int_G \varphi(z^*(s)\alpha_s(x)\alpha_{st}(z\left( st \right)^{-1})) \d s \\
&= \varphi\left( \int_G z^*(s) \alpha_s(x) \alpha_{st}(z(st)^{-1})\d s \right),
\end{align*}
and in particular, we have $\Gamma_{\tilde \varphi_z} \in A_+(A \rtimes_\alpha G)$. Another consequence of \Cref{mult:vectorstateapprox} is that every function in $A_+(A \rtimes_\alpha G)$ is a norm limit of linear combinations of functions $\Gamma_{\tilde \varphi_z}$, for states $\varphi \in \mathcal{S}(A)$. This is because $\pi_u$ is the direct sum of cyclic representations of $A$.
\end{remark}
For a representation $(\pi,\H)$ of $A$, we will also use $\tilde \pi \rtimes \lambda$ to denote the extension from $\cc(G,A,\alpha)\subseteq A \rtimes_{\alpha,r}G$ to all of $A \rtimes_{\alpha, r} G$ (doable, since the norm $\pi_u$ is the universal representation of $A$, hence the norm will be bounded by it). With this, we may prove the following essential theorem in our description of the reduced crossed product: \todo{rephrase}
\begin{theorem}
For a $C^*$-dynamical system $(A,G,\alpha)$, given any faithful representation $(\pi,\H)$ of $A$, it holds that the associated integrated form $\tilde \pi \rtimes \lambda$ will be faithful on $A \rtimes_{\alpha, r} G$.
\label{mult:reducedfaithful}
\end{theorem}
\begin{proof}
\todo{Is this the proper way of describing states contained in $\tilde \pi \rtimes_r \lambda$, or should we use the identification using predual, instead of using composition with $\sigma$}
For this particular proof, we will let $\tilde \pi \rtimes_r \lambda$ denote the corresponding representation of $A \rtimes_{\alpha,r}G$.

We will show that $\tilde \pi_u \rtimes_r \lambda$ is \myemph{weakly contained} in $\tilde \pi \rtimes_r \lambda$ for all faithful representations $\pi$ of $A$, i.e., that $\ker \tilde \pi \rtimes_r \lambda \subseteq \ker \pi_u \rtimes_r \lambda = 0$. By \cite[80]{dixmier1969c}, this is equivalent to showing that the set of vector states of $\tilde \pi \rtimes_r \lambda$ is weak$^*$ dense in the set of states in $\tilde \pi_u \rtimes_r \lambda$.

Let $E$ be the set of states contained in $(\pi,\H)$, which will be weak$^*$ dense in $\mathcal{S}(A)$ by faithfulness of $\pi$. For now, let $\sigma$ denote the surjective $*$-representation $A \rtimes_\alpha G \to A \rtimes_{\alpha,r}G$. We then let $F$ denote set of functions
\begin{align*}
	F = \left\{  \Gamma_\psi \in A_+^1(A\rtimes_\alpha G) \mid \Gamma_\psi = \Gamma_{\varphi \circ \sigma} \text{ for some state } \varphi \text{ contained in } \tilde \pi \rtimes_r \lambda \right\}
\end{align*}

Note that the set of vector states in $\tilde \pi_u \rtimes_r \lambda$ is exactly the set of functionals which give rise to $A_+^1(A \rtimes_\alpha G)$, by definition and the fact that we may view $A \rtimes_{\alpha,r} G$ as the image of $A \rtimes_\alpha G$ under $\tilde \pi_u \rtimes \lambda$.

Hence if $F \subseteq A_+^1(A \rtimes_\alpha G)$ is dense in the topology of weak$^*$ convergence uniformly on compacta, then, by the isomorphism between the statespace of $A \rtimes_\alpha G$ and $B(A \rtimes_\alpha G)$, we're done.

By the above remark and \Cref{mult:vectorstateapprox}, it is enough to consider approximations of the functions of the form
\begin{align*}
	\Gamma_{\tilde \varphi_z}(t) (x) =\varphi\left( \int_G z^*(s) \alpha_s(x) \alpha_{st}(z\left( (st)^{-1} \right)\d s \right),
\end{align*}
for suitable $\varphi \in \mathcal{S}(A)$ and $z \in \cc(G,A,\alpha)$, i.e., with $\varphi(z^* \ast z(e)) = 1$, since they span a dense subset of $A_+^1(A \rtimes_\alpha G)$: Let such $\varphi$ and $z$ be given. By density of $E$, pick a convergent net $\varphi_i \to \varphi$ in the weak$^*$-topology. This implies that $\Gamma_{\tilde {\varphi_i}_z}(t) \to \Gamma_{\tilde \varphi_z}(t)$ weak$^*$, uniformly on compacta of $G$. Since $\varphi_i \in E$, the corresponding representation $(\pi_{\varphi_i},H_{\varphi_i})$ will be a sub-representation of $(\pi,H)$, from which we conclude that $\Gamma_{\tilde {\varphi_i}_z} \in F$, finishing the proof.
\end{proof}
In particular, this means that we may always consider $A \rtimes_{\alpha,r}G$ as the completion of $\cc(G,A,\alpha)$ under the regular representation induced by a faithful representation of $A$, which will be convenient, as we shall see, for in a large case of groups there is only one crossed product. The following useful lemma can be found in \cite[Lemma 7.7.6]{pedersenalgauto}:
\begin{lemma}
If $\Gamma_\varphi \in B_+^1(A \rtimes_\alpha G)$ has compact support, then $\Gamma_\varphi \in A_+^1(A \rtimes_\alpha G)$
\label{crossed:compactposdef}
\end{lemma}

We recall one of the several equivalent definition of amenability for locally compact groups $G$:
\begin{definition}
A group $G$ is \myemph{amenable} if there is a net $(f_i) \subseteq L^2(G)$ such that $(f_ \ast \tilde f_i)$ converges uniformly on compacta of $G$ to $1$, where $\tilde f (t) = \overline f(t^{-1})$ for $f \in L^2(G)$ and $t \in G$.
\end{definition}
See e.g., \cite[Appendix G]{bekka2008kazhdan}, \cite[Proposition 7.3.7 and 7.3.8]{pedersenalgauto} for the above and more characterizations. 

With this, we obtain the following:
\begin{theorem}
If $G$ is amenable, then whenever $G \acts \alpha A$, we have $A \rtimes_\alpha G \cong A \rtimes_{\alpha,r}G$.
\label{cross:amenable}
\end{theorem}
\begin{proof}
By applying the above to $G \acts 1 \C$ as the trivial action, we obtain a net $\Psi_i \in B_{+}(\C \rtimes_\alpha G)$ with compact support convergin uniformly on compacta to $1$, by amenability of $G$ and \cite[Lemma 7.2.4]{pedersenalgauto}. It is easy to see that the pointwise product $\Psi_i \Gamma_\varphi \in B_+^1(A \rtimes_\alpha G)$ for all $\varphi \in \mathcal{S}(A \rtimes_\alpha G)$, and since $\Psi_i$ has compact support, we see that $\Psi_i \Gamma_\varphi \in A_+(A \rtimes_\alpha G)$ and $\Psi_i \Gamma_\varphi \to \Gamma_\varphi$ in the topology of weak$^*$ uniformly convergence on compacta of $G$. Hence, $A_+(A \rtimes_\alpha G) \subseteq B_+(A \rtimes_\alpha G)$ is dense. In particular, this implies that if $\pi$ denotes the universal faithful representation of $A \rtimes_\alpha G$, then $\ker \tilde \pi_u \rtimes \lambda \subseteq \ker \pi = 0$, by weak containment (\cite[80]{dixmier1969c}), so $A \rtimes_\alpha G \cong \tilde \pi_u \rtimes \lambda( A \rtimes_\alpha G) = A \rtimes_{\alpha,r}G$.
\end{proof}
Combining the above , we obtain
\begin{corollary}
If $G \acts \alpha A$ with $G$ amenable, then $A \rtimes_\alpha G = \overline{\tilde \pi \rtimes \lambda (\cc(G,A,\alpha))}$ for any faithful representation $\pi$ of $A$.
\end{corollary}
The above allows us to concretely work with the crossed products of a large class of groups which include finite groups, abelian groups, compact groups and others. This will be very useful.

For a fixed group $G$, the crossed product constructions described above are in fact functors from the category of $G$-algebras $A$ to the category of $C^*$-algebras, where the objects in $G$-algebras are $C^*$-algebras $A$ equipped with an action $G \acts \alpha A$ and morphism given by $G$-equivariant $*$-morphism, i.e., $\varphi \colon A \to B$ for $G$-algebras is a morphism if the diagram

\begin{equation}
\begin{tikzcd}
	A \arrow[r, "\varphi"] \arrow[d, "\alpha_g"'] & B  \arrow[d, "\beta_g"]\\
	A \arrow[r, "\varphi"'] & B
\end{tikzcd}
\end{equation}
commutes for every $g \in G$. This can bee seen by the following:
\begin{proposition}
Suppose that $G \acts \alpha A$ and $G \acts \beta B$ and $\varphi \colon A \to B$ is a $G$-equivariant morphism. Then the map $\tilde \varphi \colon \cc(G,A,\alpha) \to \cc(G,B,\beta)$, $\tilde \varphi (f)(s) := \varphi(f(s))$, extends to a $*$-morphism $A \rtimes_\alpha G \to B \rtimes_\beta G$.
\end{proposition}
\begin{proof}
It is clear that $\tilde \varphi$ is a $*$-morphism. To see that it extends, we show that $\lv \tilde \varphi(f) \rv \leq \lv f \rv$, where the norms comes from the embeddings $\cc(G,A,\alpha) \subseteq A \rtimes_\alpha G$ and $\cc(G,B,\beta) \subseteq B \rtimes_\beta G$. For this, let $(\pi,u)$ be a covariant representation of $G \acts \beta B$. Then, clearly $(\pi \circ \varphi , u)$ is a covariant representation of $G \acts \alpha A$, and moreover, if $f \in \cc(G,A,\alpha)$, then
\begin{align*}
	(\pi \circ \varphi) \rtimes u(f) = \int_G \pi(\varphi(f(s))) u_s \d s = \int _G \pi (\tilde \varphi(f)(s)) u_s \d s  =  \pi \rtimes u \circ \tilde \varphi (f),
\end{align*}
so $\pi \circ \varphi \rtimes u = \pi \rtimes u \circ \tilde \varphi$, and hence 
\begin{align*}
	\lv \pi \rtimes u \circ \tilde \varphi(f)\rv = \lv \pi \circ \varphi \rtimes u (f) \rv \leq \lv f \rv,
\end{align*}
since the integrated form of a covariant representation is norm decreasing. Hence, since the norm of $ \tilde \varphi(f)$ is the supremum of integrated forms of covariant representations of $G \acts \beta B$, we see that $\tilde \varphi$ extends to a morphism $A \rtimes_\alpha G \to B \rtimes_\beta G$.
\end{proof}
The above proof applies to the case of the reduced crossed product. This is easily seen by applying the above proof to regular representations $(\tilde \pi, \lambda)$ rather than any covariant representation. The remaining properties for a functor are easily verified, i.e., preserves composition and sends the identity map to the identity map.

\ssection{The case of Abelian groups and Takai Duality}
In the 70's, Hiroshi Takai published a paper (see \cite{takai1975duality}) showing that the crossed product construction in a way captures the duality of locally compact abelian groups, in the sense of $\hat{\hat G} \cong G$, where $\hat G$ denotes the dual group of $G$. He showed that taking the crossed product twice with $G$ (in the sense that $\hat G \acts {\hat \alpha} A \rtimes_\alpha G$), one obtains an isomorphism 
\begin{align*}
A \otimes \mathbb{K}(L^2(G)) \cong (A \rtimes_\alpha G) \rtimes_{\hat \alpha} \hat G,
\end{align*}
and in fact that both of these are $G$-dynamical systems and the isomorphism is covariant. \todo{revisit if really raeburn or if pedersen} We will follow Raeburns take on this result, found in \cite{raeburn1988crossed}, since it is more aligned with the results we've developed so far.
\begin{note}
In the above, and in the rest, unless otherwise stated, we will use $A \otimes B$ between two $C^*$-algebras to mean the maximal tensor product, and we recall that for a commutative $C^*$-algebra, the maximal and minimal tensorproducts coincide, i.e., they are nuclear. For this, see e.g., \cite{brown2008c}.
\end{note}

For this, we first show that the left-hand side is a $C^*$-dynamical system for any (also non-abelian) locally compact group $G$:
\begin{proposition}
Let $G \acts \alpha A$ and let $\rho$ denote the right-regular representation of $G$ on $L^2(G)$, $\rho_t f(s) = f(st) \Delta(t)^{\frac12}$. Then $A \otimes \mathbb{K}(L^2(G)), G, \alpha \otimes \mathrm{Ad} \rho)$ is a $C^*$-dynamical system.
\end{proposition}
\begin{proof}
The proof relies on the fact that in a Von Neumann algebra $\mathcal{W}$ (or Borel $*$-algebra $B$) equipped with a strongly continuous action $G \acts \alpha \mathcal{W}$, the set of elements $x \in \mathcal{W}$ such that $t \mapsto \alpha_t(x)$ is norm-continuous forms a $C^*$-algebra which is invariant under $G$, see e.g., \cite[Proof of Lemma 7.5.1]{pedersenalgauto} or \cite[Proposition III.3.2.4]{blackadar}. \todo{easy proof for $C^*$-algebra as well, no need to cite}.

With this in mind, let $(\pi,u)$ be a covariant representation of $(A, G, \alpha)$ on a Hilbert space $H$ with $\pi$ faithful. Define a representation $\sigma$ of $G$ on $H \otimes L^2(G)$ by
\begin{align*}
\sigma = u \otimes \rho,
\end{align*}
so that $\sigma$ is a strongly continuous unitary representation. For $x \in A \subseteq \mathbb{B}(H)$ and $y \in \mathbb{K}(L^2(G))$ we see that
\begin{align*}
	\sigma_t (x \otimes y) \sigma_t^* = \alpha_t(x) \otimes \rho_t y \rho_t^* \in A \otimes \mathbb{K}(L^2(G)) \subseteq \mathbb{B}(H \otimes L^2(G)).
\end{align*}
Hence $A \otimes \mathbb{K}(L^2(G))$ is invariant under $\mathrm{Ad}_\sigma$. Moreover, for finite dimensional $y \in \mathbb{K}(L^2(G))$, the map $t \mapsto \mathrm{Ad}_{\sigma_t}(y)$ is norm continuous. The closure of $F = \Span\{a \otimes y \mid a \in A, \ y \text{ finite rank}\}$ equals $A \otimes \mathbb{K}(L^2(G))$. The $C^*$-algebra of elements $x\otimes y \in A \otimes \mathbb{K}(L^2(G))$ such that $t \mapsto \sigma_t x \otimes y\sigma_t^*$ is continuous and $\mathrm{Ad}_{\sigma_s}$ invariant for all $s$ and contains $F$ hence equals $A \otimes \mathbb{K}(L^2(G))$. Letting $\alpha \otimes \mathrm{Ad}_\rho := \mathrm{Ad}_{\sigma}$, we see that $(A \otimes \mathbb{K}(L^2(G)), G , \alpha \otimes \mathrm{Ad}_\rho)$ is a $C^*$-dynamical system.
\end{proof}

The way we show Takai duality is that we will show that the $C^*$-dynamical system defined in the above proposition is covariantly isomorphic to another dynamical system, and then finally showing an isomorphism between this system and the dual system when $G$ is abelian. We show the first step in this in the following:
\begin{theorem}
Let $G \acts \alpha A$. Then there is a $C^*$-dynamical system $(C_0(G,A) \rtimes_{\gamma,r} G , G, \sigma )$ where we define the actions $G \acts \gamma C_0(G,A)$ and $G \acts \sigma C_0(G,A)$ by
\begin{align*}
	\sigma_s (f)(t) = f(ts) \text{ and } \gamma_s f(t) = \alpha_s(f(s^{-1}t)),
\end{align*}
for $f \in C_0(G,A)$ and $s,t \in G$. This system is covariantly isomorphic to $(A \otimes \mathbb{K}(L^2(G)), G , \alpha \otimes \mathrm{Ad}_\rho)$.
\label{takai:compactiso}
\end{theorem}
\begin{proof}
\todo{Add as example the dynamical systems} Represent $A$ faithfully on a Hilbert space $\H$, and define a representation of $C_0(G,A)$ on $L^2(G,A)$ as multiplication operators, i.e.,
\begin{align*}
	\pi(f) \xi (s) = f(s)\xi(s), 
\end{align*}
for $f \in C_0(G,A)$, $\xi \in L^2(G,A)$ and $s \in G$. This is easily seen to be a faithful representation, so by \Cref{mult:reducedfaithful} we obtain a faithful representation $\tilde \pi \rtimes \lambda$ of $C_0(G,A) \rtimes_{\gamma,r} G$ on $L^2(G,L^2(G,H)) \cong L^2(G \times G, H)$. Consider $\cc(G \times G,A) \subseteq \cc(G,C_0(G,A))$, via the map $(f \colon (x,y) \mapsto f(x,y)) \mapsto (x \mapsto (y \mapsto f(x,y)))$, and note that there is not necessarily equality (e.g., for the case $G=  \R$ and $A = \C$). For $z \in \cc(G \times G, A)$, we see that
\begin{align*}
	\tilde \pi \rtimes \lambda (z) \xi(s,t) &= \int_G \alpha_{s^{-1}}(z(r,st)) \xi(r^{-1}s,t) \d r.
\end{align*}
Define the operator $w$ on $L^2(G\times G,H)$ by
\begin{align*}
	w \xi(s,t) \xi(st,t) \Delta(t)^{\frac12},
\end{align*}
then $w$ is a unitary operator with adjoint $w^* \xi(s,t) = \xi(st^{-1},t) \Delta(t)^{-\frac12}$. We then calculate for $z \in \cc(G \times G, A)$
\begin{align*}
	w^*  (\tilde \pi \rtimes \lambda (z)) w (\xi) (s,t) = \int_G \alpha_{ts^{-1}} (z(r,s)) w \xi(r^{-1} s t^{-1},t) \d r \Delta(t)^{-\frac12} = \int_G \alpha_{ts^{-1}}(z(r,s)) \xi(r^{-1}s,t) \d r,
\end{align*}
for all $\xi \in L^2(G \times G, H)$ and $s,t \in G$. It is not hard to see that any function $f \in \cc(G \times G , A)$ can be approximated by linear combinations of maps of the form $(s,t) \mapsto z(s)g(t)\alpha_t(x)$ for $z,g \in \cc(G)$ and $x \in A$, and hence also by functions of the form
\begin{align*}
	f(s,t) = \alpha_{t}(x) z(s^{-1}t)g(t) \Delta(s^{-1}t),
\end{align*}
for fixed $z,g \in \cc(G)$ and $x \in A$. Applying $\mathrm{Ad}_{w}$ to this we get
\begin{align*}
	w^* (\tilde \pi \rtimes \lambda(f)) w(\xi)(s,t) = \int_G \alpha_{ts^{-1}}(f(r,s)) \xi(r^{-1}s,t) \d r &= \int_G \alpha_{ts^{-1}}(\alpha_s(x) z(r^{-1}s)g(s) \Delta(r^{-1}s)) \xi(r^{-1}s,t) \d r \\
	&= \int_G \alpha_t(x) z(r^{-1}s) g(s) \xi(r^{-1}s,t) \Delta(r^{-1}s)  \d r\\
	&= \alpha_t(x) g(s) \int_G z(r) \xi(r,t) \d r.
\end{align*}
Consider the faithful representation $\check \pi$ of $A$ on $L^2(G,H)$ by $\check \pi(x) \xi(s) = \alpha_s(x) \xi(s)$. Given $K \in \mathbb{K}(H)$ and $x \in A$, if $U$ denotes the unitary operator identifying $L^2(G) \otimes L^2(G,H)$ with $L^2(G, L^2(G, H)$, then for $h \in L^2(G)$ and $\xi \in L^2(G,H)$ we see that
\begin{align*}
	U (\check \pi(x) \otimes K (h \otimes \xi))(s)(t) = \alpha_s(x) h(s) K \xi(t),
\end{align*}
so after identifying $L^2(G,L^2(G,H))$ with $L^2(G \times G,H)$, we see that for $f$ as above:
\begin{align*}
	w^* (\tilde \pi \rtimes \lambda (f)) w  = \check \pi \otimes p_{zg} \in A \otimes \mathbb{K}(L^2(G)) \subseteq \mathbb{B}(L^2(G \times G, H)),
\end{align*}
where is $p_{z,g}$ the rank-one operator $\eta \mapsto \langle \eta , \overline z \rangle g$, $\eta \in L^2(G)$.
In particular, this implies that 
\begin{align*}
	A \otimes \mathbb{K}(L^2(G)) \subseteq w^* (\tilde \pi \rtimes \lambda (C_0(G,A) \rtimes_{\lambda, r} G))w,
\end{align*}
since $\check \pi$ is faithful and $\mathbb{K}(L^2(G))$ is the closure of the span of rank-one operators. Since we may approximate $\cc(G \times G, A)$ by linear combinations of forms $(s,t)  \alpha_s(x)  f(r^{-1}s)g(s) \Delta(r^{-1}s)$, and $\cc(G \times G, A)$ is dense in $C_0(G,A) \rtimes_{\lambda, r}G$, the above inclusion is an equality, so the two algebras are spatially isomorphic with respect to $w$.

Denote also by $\sigma_t$ the representation of $G$ on $\cc(G,C_0(G,A))$, $\sigma_t f(x)(y) = f(s)(yt)$. Then $\sigma_t$ extends to all of $C_0(G,A) \rtimes_{\lambda, r}G$ by continuity and density. If $f$ is as above, then
\begin{align*}
	w^* (\tilde \pi \rtimes \lambda \sigma_r (f) )w \xi(s,t) &= \alpha_{tr}(x)\underbrace{g(sr) \Delta(r)}_{= \rho_r g(s) \Delta(r) ^{\frac12}} \int_G \underbrace{z(qr)}_{= \rho_{r}z(q) \Delta(r)^{-\frac12}} \xi(q,t) \d q\\
	&= \alpha_{t} ( \alpha_r(x)) \rho_{r} g(s) \int_G \rho_r z(q) \xi(q,t) \d q\\ 
	&= ( \check \pi(\alpha_r(x)) \otimes \rho_r p_{z,g} \rho_{r}^*) \xi(s,t)\\
	&= ( \check \pi(\alpha_r(x)) \otimes \mathrm{Ad}_{\rho_r}p_{z,g}) \xi(s,t).
\end{align*}
In particular, after identifying $\check \pi (A) $ with $A$, we have
\begin{align*}
	w^* (\tilde \pi \rtimes \lambda(\sigma_r(f)))w = (\alpha \otimes \mathrm{Ad}_{\rho})_r(x \otimes p_{z,g}),
\end{align*}
and hence by density, we see that $(C_0(G,A) \rtimes_{\lambda , r}, G , \sigma)$ is covariantly isomorphic to $(A \otimes \mathbb{K}(L^2(G)), G, \alpha \otimes \mathrm{Ad}_{\sigma})$ via $w$, in the sense that $w$ implements a $G$-equivariant spatial isomorphism between the two systems. \todo{define category of $C^*$-systems and covariant isomorphism}
\end{proof}
Applying the above to the case where $A = \C$ and $\alpha = 1$, we obtain the Stone-Von Neumann theorem for reduced crossed products:
\begin{corollary}
Let $\mathrm{lt}$ denote the representation of $G$ on $C_0(G)$ given by left-translation, i.e., $\mathrm{lt}_s f(t) = f(s^{-1}t)$. Then $\mathrm{lt}$ corresponds to $\gamma$ in the above case with $G \acts 1 \C$, so
\begin{align*}
	C_0(G) \rtimes_{\mathrm{lt},r} G \cong \mathbb{K}(L^2(G)) \otimes \C \cong \mathbb{K}(L^2(G)).
\end{align*}
\end{corollary}
It is worth noting that the result also holds for the full crossed product, however, we mention it without proof. The inclined reader is refered to \cite[Theorem 4.24]{williamscrossed} or \cite[§C.6 ]{williamsmorita}. In particular, this implies that
\begin{align*}
C_0(G) \rtimes_{\mathrm{lt},r} G \cong C_0(G) \rtimes_{\mathrm{lt}}G,
\end{align*}
for all groups $G$.

In the following, we will assume that $G$ in addition to being locally compact is abelian, and we will use $+$ to denote the binary operation. We briefly recall the definition of the dual group $\hat G$ and associated notions. For a better exposition on these topics, we again refer the reader to the excellent work \cite{folland2016fourier}. 
\begin{definition}
For a locally compact group $G$, the dual group $\hat G$ is the group of continuous homomorphisms $\xi \colon G \to S^1$, where $S^1$ denotes the circle. For $\xi \in \hat G$, we will write $(t,\xi) := \xi(t)$. An alternative and equivalent definition of $\hat G$ is as the set of irreducible unitary representations of $G$, which by commutativity of $G$ are all one-dimensional, or we may identify $\hat G$ with the spectrum, $\sigma(L^1(G))$, of the group algebra via the pairing
\begin{align}
	\xi(f):=\int_G (t , \xi) f(t) \d t,
	\label{abel:fourier}
\end{align}
for $f \in L^1(G)$ and $\xi \in \hat G$. 
\end{definition}
Usually, one will examine the Fourier transformation of $M(G)$, where $M(G)$ is the space of bounded Radon Measures on $G$. It is useful to recall also that $L^1(G) \subseteq M(G)$ by assigning $f$ to $f \cdot \d x$, where $\d x$ is the integrand of the fixed Haar measure on $G$. However, will instead consider the inverse Fourier transformaton:
\begin{definition}
The inverse Fourier transformation of a measure $\mu \in M(G)$ is the bounded continuous map on $\hat G$ given by 
\begin{align*}
\hat \mu(\xi) := \int_G	(t, \xi) \d \mu(t).
\end{align*}
For $f \in L^1(G) \subseteq M(G)$, we have
\begin{align*}
	\hat f(\xi) := \int_G (t, \xi) f(t) \d t.
\end{align*}
Note that $\hat \mu \in C_b(\hat G)$ and $\hat f \in C_0(\hat G)$ for $\mu \in M(G)$ and $f \in L^1(G)$.
\end{definition}
And it will be worth noting that there is a topological duality between $G$ and $\hat G$, in the following sense:
\begin{proposition}
A group $G$ is discrete if and only if $\hat G$ is compact and vice versa.
\end{proposition}
It will also be important to note that there is a one-to-one correspondance between unitary representations of $G$ and representations of $M(G)$ as a Banach $*$-algebra:
\begin{proposition}
There is a bijective correspondance between unitary representations of $G$ and representations of $M(G)$ which restricts to a non-degenerate representation of $L^1(G)$. If $u$ is a unitary representation of $G$ on $\H$, then the associated representation of $L^1(G)$ is given by
\begin{align*}
	u_f(\xi) = \int_G u_s \xi f(s) \d s,
\end{align*}
in the weak sense, i.e., that 
\begin{align*}
	\langle u_f(\xi), \eta \rangle = \int_G \langle u_s \xi, \eta \rangle f(s) \d s,
\end{align*}
for $\xi, \eta \in \H$.
\end{proposition}

For the rest of this section, we will assume that $G$ is any abelian locally compact group, and we shall think of $A \rtimes_\alpha G$ as operators on $L^2(G,\H_u)$, where $\H_u$ is the universal Hilbert space for $A$.
\begin{definition}
Suppose that $G \acts \alpha A$. We define the dual action $\hat \alpha$ of $\hat G$ on $A \rtimes_\alpha G$, which is defined on the dense subset $\cc(G,A,\alpha)$ by
\begin{align*}
	\hat \alpha_\xi y(t) =  (t,\xi) y(t),
\end{align*}
for $y \in \cc(G,A,\alpha)$, $\xi \in \hat G$ and $t \in G$.
\end{definition}
To see that this in fact is a well-defined action, we use the following:
\begin{lemma}
Let $G \acts \alpha A$. Define a unitary representation $u$ of $\hat G$ on $L^2(G,H)$ by
\begin{align*}
	u_\chi \xi(t) = (t, \chi) \xi(t).
\end{align*}
Then, with $\hat \alpha_\chi$ as above, it holds that
\begin{align*}
	\hat \alpha_\chi = \mathrm{Ad} u_\chi,
\end{align*}
so that $(A \rtimes_\alpha G, \hat G, \hat \alpha)$ becomes a $C^*$-dynamical system.
\label{dualaction}
\end{lemma}
\begin{proof}
Let $y \in \cc(G,A,\alpha)$, so that $y \xi(s) = \int_G \alpha_{-s}(y(t))\xi(t-s) \d s$. Then
\begin{align*}
	\mathrm{Ad} u_{\chi}(y) \xi(t) &= (t, \chi) \int_G \alpha_{-t}(y(s)) (t, - \chi)(-s,-\chi) \xi(t-s) \d s\\
	&= \int_G \alpha_{-t}( (s,\chi) y(s)) \xi(t-s) \d s \\
	&= \hat \alpha_{\chi}y \xi(s).
\end{align*}
Hence, since $\hat \alpha_\chi y \in \cc(G,A,\alpha)$, and conjugation by unitaries is an isometry, we may define 
\begin{align*}
	\hat \alpha_\chi := (A \rtimes_\alpha G \ni x \mapsto \mathrm{Ad} u_\chi x \in A \rtimes_\alpha G) \in \mathrm{Aut}(A \rtimes_\alpha G),
\end{align*}
and we will think of it as the automorphism satisfying $\hat \alpha_\chi y(s) = (s,\chi) y(s)$ for $y \in \cc(G,A,\alpha)$.
\end{proof}
The above example is one of the motivations for the notion of a $G$-product:
\begin{definition}
	Let $G$ be a group with dual group $\hat G$. We say that a $C^*$-algebra $A$ is a \myemph{$G$-product} if:
	\begin{enumerate}
		\item There is a homomorphism $\lambda \colon G \to \mathcal{U}(M(A))$ such that the map $t \mapsto \alpha_t y$, $y \in A$ is continuous,
		\item There is a homomorphism $\hat \alpha \colon \hat G \to \mathrm{Aut}(A)$ such that $(A, \hat G, \hat \alpha )$ is a $C^*$-dynamical system satisfying
			\begin{align*}
				\hat{\alpha_\chi} \lambda_t = (t,\chi) \lambda_t,
			\end{align*}
			for all $t \in G$ and $\chi \in \hat G$.
	\end{enumerate}
	Moreover, we say that $x \in M(A)$ satisfies \myemph{Landstad's conditions} if:
	\begin{enumerate}
		\item $\hat \alpha_{\chi}(x) = x$ for all $\chi \in \hat G$,
		\item $ x \lambda_f \in A$ and $\lambda_f x \in A$ for all $f \in L^1(G)$,
		\item the map $t \mapsto \lambda_t x \lambda_{t}^*$ is continuous.
	\end{enumerate}
\end{definition}

\begin{lemma}
Suppose that $G \acts \alpha A$. Then with the above action, it holds that $G \rtimes_\alpha A$ is a $G$-product, and each element of $A \subseteq M(A \rtimes_\alpha G)$ satisfies Landstad's condition.
\end{lemma}
\begin{proof}
Let $u$ and $\hat \alpha$ be as above. Let $\lambda_t$ be the operator on $L^2(G,H)$ given by
\begin{align*}
	\lambda_t \xi(s) = \xi(s-t).
\end{align*}
Then we may view $\lambda$ as a map $G \to \mathcal{U}(M(A \rtimes_\alpha G))$, $\lambda_t y(s) = \alpha_t (y(s-t))$ for $y \in \cc(G,A)\subseteq A \rtimes_\alpha G$, by \cref{cross:iotaprop} such that $t \mapsto \lambda_t y$ is continuous for all $y \in A \rtimes_\alpha G$. Combining this with the calculations in \cref{dualaction}, we see that $A \rtimes_\alpha G$ is a $G$-product, since
\begin{align*}
	u_\chi \lambda_s u_{-\chi} \xi(t) = (t,\chi) (t-s, - \chi) \xi(t-s) = (s, \chi) \lambda_s \xi(t),
\end{align*}
for all $\xi \in L^2(G,H)$, $s,t \in G$ and $\chi  \in \hat G$. For the last claim, we see that if $a \in A \subseteq M(A \rtimes_\alpha G)$, then $\hat \alpha_\chi a = u_\chi a u_\chi^* = a$, since $a y(t) = \alpha_t(a) y(t)$ for all $y \in \cc(G,A,\alpha)$, so
\begin{align*}
	\hat \alpha_\chi (a)y(t) = (t, \chi) \alpha_{-t}(a) (t,-\chi) y(t) = \alpha_{-t}(a) y(t) = ay(t).
\end{align*}

If $f \in \cc(G)$ and $x \in A$, then for all $\xi \in L^2(G,H)$ and $s \in G$ we see that
\begin{align*}
	x \lambda_f \xi(s) = \int_G f(t) x \lambda_t \xi(s) \d t = \int_G f(t) \alpha_{-t}(x) \lambda_t \xi(s) \d t = \int_G (f \otimes x)(t) \lambda_t \xi(s) \d t = (f \otimes x )\xi(s),
\end{align*}
where we have identified $f \otimes x \in \cc(G,A,\alpha)\subseteq A \rtimes_\alpha G$. In particular $x \lambda_f \in A \rtimes_\alpha G$ and likewise $\lambda_f x \in A \rtimes_\alpha G$ for $f \in \cc(G)$ and hence for $f \in L^1(G)$, by dominance of the norm on $A \rtimes_\alpha G$.
\end{proof}

Now, we are going to show the theorem of Takai by combining the previous results with the following: If $G \acts \alpha A$, we let $\iota$ denote the trivial action of $\hat G$ on $A$. Then
\begin{lemma}
For $G \acts \alpha A$, there is an isomorphism $A \rtimes_\iota \hat G \cong C_0(G,A)$. Moreover, the isomorphism is spatially implemented by a unitary $u$.
\label{takai:trivcrossiso}
\end{lemma}
\begin{proof}
Let $C_0(G,A)$ be faithfully represented on $L^2(G,H)$ as multiplication operators $f \xi(s) = f(s) \xi(s)$, where $A$ is faithfully represented on $H$. By the Plancheral Theorem, we have $L^2(G) \cong L^2(\hat G)$ via the (proper) Fourier transformation, and so we may define the isometry $u \colon L^2(G,H) \to L^2(\hat G,H)$ by
\begin{align*}
	u \xi(\chi) = \int_G \xi(t) \overline{(t, \chi)} \d t, \text{ for } \xi \in L^2(G,H) \text{ and } \chi \in \hat G,
\end{align*}
so that 
\begin{align*}
	u^* \eta (t) = \int_{\hat G} \eta(\chi) (t, \chi) \d \chi, \text{ for } \eta \in L^2(\hat G,H) \text{ and } t \in G,
\end{align*}
where the integrals are defined in the $L^2$-sense. Then, for all $f \in \cc(\hat G, A,\iota)$ we see that
\begin{align*}
	f u \xi(\chi) = \int_{\hat G} \iota_{-\sigma}(f(\chi)) \lambda_\sigma (u \xi)(\chi) \d \sigma &= \int_{\hat G} f(\chi) \int_G \xi(t) \overline{(t,\chi - \sigma)} \d t \d \sigma\\
	&= \int_{\hat G}\int_G f(\chi)(t, \sigma) \xi(t) \overline{(t, \chi)} \d t \d \sigma\\
	&= \int_{\hat G} \hat f(t) \xi(t) \overline{(t, \chi)} \d t\\
	&= \int_{\hat G} (\hat f \xi)(t) \overline{(t, \chi)} \d t\\
	&= u \hat f \xi(\chi)
\end{align*}
implying that $u^* f u = \hat f \in C_0(G, A)$. By density of $\cc(\hat G, A, \iota) \subseteq A \rtimes_\iota \hat G$, it follows that $u$ implements the desired isomorphism.
\end{proof}
We are now ready to define the $C^*$-dynamical system which we will show is covariantly isomorphic to $(A \otimes \mathbb{K}(L^2(G)), G, \alpha \otimes \mathrm{Ad} \lambda)$ and $\left( (A \rtimes_\alpha G)\rtimes_{\hat \alpha} \hat G, G, \hat{\hat \alpha} \right)$:
\begin{proposition}
Suppose that $G \acts \alpha A$. Define for each $t \in G$ the map $\beta_t \colon \cc( \hat G, A, \iota) \to \cc(\hat G, A, \iota)$ by
\begin{align*}
	\beta_t f(\chi) = \overline{ ( t, \chi)} \alpha_t(f(\chi)),	
\end{align*}
for $f \in \cc(\hat G, A, \iota)$ and $\chi \in \hat G$. Then $\beta_t$ extends to an automorphism of $A \rtimes_{\iota} \hat G$ such $t \mapsto \beta_t$ is an action of $G$ on $A \rtimes_\iota \hat G$. With this $\beta$, we have a covariant isomorphism $(A \rtimes_\iota \hat G) \rtimes_\beta G \cong (A \rtimes_\alpha G) \rtimes_{\hat \alpha} \hat G$. \todo{reformulate, you don't show covariant isomorphism}
\end{proposition}
\begin{proof}
Define $\gamma_t \colon C_0(G,A) \to C_0(G,A)$ to be the diagonal action
\begin{align*}
\gamma_t y(s) = \alpha_t (y(s-t)),
\end{align*}
here diagonal comes from the fact that $\gamma_t = \alpha_t \otimes \mathrm{lt}_t$, when we identify $C_0(G,A) \cong C_0(G) \otimes A$\todo{add diagonal to earlier introduction}. Let $u$ be as in \cref{takai:trivcrossiso}, $\hat{\beta_s(y)} \eta(t) = u^*yu$. Then, if $\xi \in L^2(G,H)$ we see that
\begin{align*}
\hat{\beta_s(y)}\xi(t) = \hat{\beta_s(y)}(t) \xi(t) = \int_{\hat G} (t, \sigma) \beta_s(y)(\sigma) \xi(t) \d \sigma &= \int_{\hat G} (t-s,\sigma) \alpha_s(y(\sigma)) \xi(t) \d \sigma
\end{align*}
and we also see that 
\begin{align*}
\gamma_s (\hat y) \xi(t) = \alpha_{s}(\hat y(t-s)) \xi(t) = \int_{\hat G} (t-s, \sigma) \alpha_s (y (\sigma)) \xi(t) \d \sigma = \hat{\beta_s(y)} \xi(t),
\end{align*}
In particular, this allows us to extend $\beta_t$ to an automorphism of $A \rtimes_\iota \hat G$ such that the associated dynamical system is covariantly isomorphic to $(C_0(G,A),G,\gamma)$ via $u$.

We now show that the system $(\hat G \rtimes_\iota  A , G, \beta)$ is covariantly isomorphic to $(G \rtimes_\alpha A, \hat G, \hat \alpha)$. For this, we note that an easy calculation shows for $x \in \cc(\hat G \times G, A)$ and $\xi \in L^2(\hat G \times G, H)$ that
\begin{align*}
x \xi(t, \chi)=\int_G \int_{\hat G} \alpha_{-t}(x(t, \sigma)) \xi(t-s,\chi-\sigma)(t,\sigma)\d \sigma \d s,	
\end{align*}
and similarly for $y \in \cc(G \times \hat G, A)$ and $\eta \in L^2(G \times \hat G, H)$ we have
\begin{align*}
y \eta(\chi,t) = \int_{\hat G} \int_G \alpha_{-t}(y(\sigma,s)) \xi(\chi-\sigma, t-s) \overline{(t,\sigma)} \d s \d \sigma,
\end{align*}
for all $\chi \in \hat G$ and $t \in G$. Define an isomorphism $\Phi \colon \cc(G \times \hat G, A) \to \cc(\hat G\times G , A)$ and the isometry $w \colon L^2(G \times \hat G, H) \to L^2(\hat G \times G, H)$ by
\begin{align*}
\Phi(y)(\chi,t) = (t,\chi) y(t,\chi) \text{ and } w \xi(\chi,t) = \overline{(t,\chi)} \xi(t,\chi)	
\end{align*}
so that
\begin{align*}
\Phi^{-1}(x)(t,\chi) = \overline{(t,\chi)} x(\chi,t) \text{ and } w^* \eta(t,\chi) = (t,\chi) \eta(\chi,t)	
\end{align*}
for $t \in G$, $\chi \in \hat G$. Note now that for $\sigma, \chi \in \hat G$ and $s,t \in G$ we have
\begin{align*}
(s,\sigma) \overline{(t-s,\chi- \sigma)}\overline{(s,\sigma)} = \overline{(t,\chi)}(t,\sigma),
\end{align*}
hence for $x \in \cc(G \times \hat G, A)$ and $\xi \in L^2(G \times \hat G,H)$ we get using the above identities:
\begin{align*}
\Phi(x) w \xi(\chi,t)&= \int_{\hat G} \int_G \alpha_{-t}(x(s, \sigma)) \xi(t-s, \sigma- \chi)(s, \sigma) \overline{(t-s,\sigma-\chi)}\overline{(s,\sigma)} \d s \d \sigma\\
&= \overline{(t,\chi)} \int_{\hat G} \int_G \alpha_{-t}(x(s,\sigma)) \xi(t-s,\sigma-\chi)(t,\sigma) \d s \d \sigma\\
&= w x \xi(t,\chi),
\end{align*}
so that $w^* \Phi(x) w = x$ for all $x \in \cc(G \times \hat G, A)$. By density, it follows that the two algebras $(G \rtimes_\alpha A) \rtimes_{\hat \alpha} \hat G $ and $(A \rtimes_{\iota} \hat G)\rtimes_{\beta} G$ are isomorphic.
\end{proof}
We are now ready to prove the main theorem by combining the above:
\begin{theorem}[Takai Duality]
Suppose that $G \acts \alpha A$. Then the double dual system $\left( (A \rtimes_\alpha G) \rtimes_{\hat \alpha} \hat G, G, \hat{\hat{\alpha}} \right)$ is covariantly isomorphic to the $C^*$-dynamical system $(A \otimes \mathbb{K}(L^2(G)), G, \alpha \otimes \mathrm{Ad} \rho)$.	
\label{takai:takai}
\end{theorem}
\begin{proof}
We already have the following diagram of isomorphisms
\begin{equation*}
	\begin{tikzcd}
		A \otimes \mathbb{K}(L^2(G)) \ar[r,"\cong"] & C_0(G,A) \rtimes_{\gamma} G \ar[r, "\cong"] & (A \rtimes_\iota \hat G) \rtimes_\beta G \ar[r, "\cong"] & (A \rtimes_\alpha G) \rtimes_{\hat \alpha} \hat G
	\end{tikzcd}
\end{equation*}
Moreover, the dynamical systems $(A \otimes \mathbb{K}(L^2(G), G, \alpha \otimes \mathrm{Ad} \rho)$ and $(C_0(G,A) \rtimes_{\gamma} G, G, \sigma)$ are covariantly isomorphic, c.f. \cref{takai:compactiso}. We finish the proof by showing that $(C_0(G,A) \rtimes_{\gamma} G, G, \sigma)$ is covariantly isomorphic to $( (A \rtimes_{\alpha} G)\rtimes_{\hat \alpha} \hat G, G, \hat{\hat{\alpha}})$, which is done by showing the relation on the generating set of functions $z \in \cc(G,C_0(G,A))$, 
\begin{align*}
	z(s,t) =x f(s) \hat g(t),
\end{align*}
for some $x \in A$, $f \in \cc(G)$ and $g \in \cc(\hat G)$. Denote by $\check z \in \cc(G \times \hat G, A)$ the image of $z$ under the isomorphism above, so that
\begin{align*}
	\check z(s,\chi) = x f(s) g(\chi),
\end{align*}
for $(s,\chi) \in G \times \hat G$, and note that the map $g_r \colon \chi \mapsto g(\chi)(r,\chi)$ is mapped to the function $\hat g_r \colon t \mapsto \hat g(r+t)$. Denote by $\Phi$ the last isomorphism in the above, so that $\Phi(\check z) \in \cc(\hat G\times  G, A)$ is the map
\begin{align*}
	\Phi(\check z)(\chi,s) = (s,\chi) \check z(s,\chi) = x f(s) g(\chi) (s,\chi).
\end{align*}
In particular, for the action $G \acts \sigma C_0(G,A) \rtimes_{\gamma} G$ is the action $\sigma_r y(s,t) = y(s,t+r)$ for $y \in \cc(G \times G,A)$, then we see that
\begin{align*}
	\sigma_r z(s,t) = x f(s) + \hat g(t+r) = x f(s) \hat g_r(t),
\end{align*}
whence we see that
\begin{align*}
	\check{ \sigma_r z}(s,\chi) = x f(s) g(\chi) (r,\chi),
\end{align*}
so that
\begin{align*}
\Phi(\check{ \sigma_r z})(\chi,s) = x f(s) g(\chi)(s,\chi)(r,\chi) = \Phi(\check z)(\chi,s) (r,\chi).
\end{align*}
Note also, that if $G \acts {\hat{\hat{\alpha}}} (A \rtimes_\alpha G) \rtimes_{\hat{\alpha}} \hat G$ is the dual of the dual action, then for $y \in \cc(\hat G \times G, A) \subseteq (A \rtimes_\alpha G) \rtimes_{\hat{\alpha}} \hat G$ then
\begin{align*}
	\hat{\hat{\alpha}}_r y(\chi,t) = (r,\chi) y(\chi,t),
\end{align*}
whence we see that $\hat{\hat{\alpha}}_r(\Phi(\check z))= \Phi(\check{\sigma_r (z)})$, and since $z$ generates $C_0(G,A) \rtimes_\gamma G$, this shows the desired covariant isomorphism of the systems $(C_0(G,A) \rtimes_\gamma G, G, \sigma)$ and $( (A \rtimes_\alpha G)\rtimes_{\hat{\alpha}} \hat G, G, \hat{\hat{\alpha}})$.
\end{proof}

Before we begin our discussion of simplicity in crossed product in the next chapter, we will actually go back to the concept of a $G$-product and properties thereof.
\begin{definition}
	An element $y \in M(A)_+$, for a $G$-product $A$, is said to be \myemph{$\hat \alpha $-integrable} if there is an element $I(y) \in M(A)_+$ such that
	\begin{align*}
		\varphi(I(y)) = \int_G \varphi(\hat \alpha_\chi (y)) \d \chi,
	\end{align*}
	for all $\varphi \in A^*$.
\end{definition}


\todo{rewrite since more added above} We end this chapter with the promise that the above result actual has meaningfull implications, even if they are not apparent just yet. In particular, in a few papers of Olesen and Pedersen, see e.g., \cite{olesenpedersen1}, \cite{olesenpedersen2} and \cite{olesenpedersen3}, Takai duality is used to describe a necessary and sufficient condition for simplicity of $A \rtimes_\alpha G$ in terms of a certain subgroup of $\hat G$ and certain properties of the action $G \acts \alpha A$. These particular articles established the foundation of much of the theory which we will examine in the next chapter.
