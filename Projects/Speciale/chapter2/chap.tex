\chapter{Crossed Products with Topological Groups}
Suppose that you are given a $C^*$-dynamical system $(A,G,\alpha)$ with $G$ discrete. It is not hard to extend representations of $G$ and $A$ to the $*$-algebra $\C_C(G,A)$, and thus create new $C^*$-algebras through completion of it, examples include the reduced- and full crossed product of $A$ and $G$, $A \rtimes_{\alpha,r} G$ and $A \rtimes_{\alpha} G$.

However, when $G$ is not discrete, we run into issues since we can not assume that a unitary representation of $G$ is continuous in norm, which is required by classic constructions seen e.g., in \cite{brown2008c}. We will in this chapter develop the necessary theory to define the non-discrete analogues of these techniques, i.e., for when $G$ is a locally compact Hausdorff group.

We assume familiarity with basic theory of analysis in topological groups, basic representation theory of topological groups and Fourier analysis thereon and will not go lengths to explain results from this field, for a reference, we recommend the excellent books on the subject: \cite{folland2016fourier}, \cite{berg1984harmonic} and \cite{folland2013real}. 
\sssection{Integral Forms And Representation Theory}
Throughout this chapter, we let $G$ be a locally compact topological group and $A$ a $C^*$-algebra, with no added assumptions unless otherwise specified. We will write $G \acts \alpha A$ to specify that $\alpha \colon G \to \Aut(A)$ is an action of $G$ on $A$, i.e., a strongly continuous group homomorphism. We will always fix a base left Haar measure $\mu$ on $G$, whose integrand we will denote by $\d t$, and with modular function $\Delta \colon G \to (0,\infty)$. 

The main motivation of the theory we are about to develop is this: Given a covariant representation $(\pi,u,\H)$ of $(A,G,\alpha)$, we wish to construct a representation $\pi \rtimes u$ of $\cc(G,\mathbb{B}(\H))$ satisfying 
\begin{align*}
	\pi \rtimes u (f) = \int \pi(f(t))u_t \d t.
\end{align*}
The problem at hand is, a priori $u_t$ is not norm continuous, and so we don't know if the integrand is integrable. We will however show that $u_t$ is strictly convergent and use this to define our integral by combining it with the previously seen fact that $M(\mathbb{B}(\H)) = \mathbb{K}_A(A_A)= \mathbb{B}(\H)$. 

For $f \in \cc(G,A)$, the map $s \mapsto \lv f(s) \rv$ is in $\cc(G)$, and we define the number $\lv f \rv_1 \in [0,\infty)$ to be
\begin{align*}
	\lv f \rv_1 = \int_G \lv f(t) \rv \d t.
\end{align*}
For $A= \C$, it is well known that given $f \in \cc(G,A)$ and $\varepsilon > 0$, there is a neighborhood $V$ of $e \in G$ such that $sr^{-1} \in V$ or $s^{-1}r \in V$ implies $| f(s) - f(r)| < \varepsilon$, i.e., $f$ is left- and right uniformly continuous. The statement holds for general $A$, and the proof is identical, see e.g. \cite[Proposition 2.6]{folland2016fourier} or \cite[Lemma 1.88]{williamscrossed}
\begin{lemma}
	For $f \in \cc(G,A)$ and $\varepsilon>0$, there is a neighborhood $V$ of $e \in G$ such that either $sr^{-1} \in V$ or $s^{-1}r \in V$ implies
	\begin{align*}
		\lv f(s)  - f(r) \rv < \varepsilon.
	\end{align*}
	\label{int:lrunicont}
\end{lemma}

We will use $\mathcal{D}$ to denote an arbitrary complex Banach space, and for $f \in \cc(G)$ and $x \in \mathcal{D}$, we denote by $f \otimes x$ the map $G \ni t \mapsto f(t)x \in \mathcal{D}$. It is clear that then $f \otimes x \in \cc(G,\mathcal{D})$. We introduce a 'topology' or a name for a special kind of convergence on $\cc(G,\mathcal{D})$, which will prove useful:
\begin{definition}
	We say that a net $\{f_i\}_{i \in I} \subseteq \cc(G,\mathcal{D})$ converges to a function $f$ on $\cc(G,\mathcal{D})$ in the \emph{inductive limit topology} if $f_i \to f$ uniformly and there is a compact set $K \subseteq G$ and $i_0 \in I$ such that for $i \geq i_0$ we have $f_i \big|_{K^c} = 0$, i.e., $\mathrm{supp} f_i \subseteq K$.
\end{definition}

\begin{lemma}
Let $\mathcal{D}_0 \subseteq \mathcal{D}$ be any dense subset. Then the set
\begin{align*}
	\cc(G) \odot \mathcal{D}_0 := \Span\left\{ f \otimes x \mid f \in \cc(G) \text{ and } x \in \mathcal{D_0} \right\} 
\end{align*}
is dense in the inductive limit topology on $\cc(G,\mathcal{D})$.
	\label{int:indlmdense}
\end{lemma}
\begin{proof}	
	Let $\varphi \in \cc(G,B)$ and $\varepsilon > 0$ be arbitrary. Let $K= \mathrm{supp} \varphi$ and let $W$ denote an arbitrary compact neighborhood of $e \in G$, and pick \todo{add ref} a symmetric neighborhood $e \in V_{\varepsilon} \subseteq W$ such that $sr^{-1} \in V$ implies $\lv f(s) - f(r) \rv < \frac{\varepsilon}{\mu(WK)}$. Since $K$ is compact, there is a finite set $\left\{ r_1,\dots,r_n \right\} \subseteq K$ such that $K \subseteq \bigcup_{1 \leq i \leq n} V r_i$. Recall that every locally compact Hausdorff group is the disjoint union of $\sigma$-compact subspaces, hence we can pick a partition of unity $\left\{ f_j \right\}_{j=0}^n$ of $K^c \cup \left( \bigcup_{1 \leq i \leq n} Vr_i\right)=G$ such that $\mathrm{supp}f_0 \subseteq K^c$ and $\mathrm{supp}f_i \subseteq V r_i$ for $i = 1 ,\dots,n$. Define
	\begin{align*}
		g_\varepsilon := \sum_{i=1}^n f_i \otimes x_i,
	\end{align*}
	where $x_i \in B_0$ such that $\lv x_i - \varphi(r_i)\rv < \frac{\varepsilon}{2\mu(WK)}$ for $i = 1,\dots,n$. Then $g_\varepsilon \in \cc(G) \odot \mathcal{B}_0$, and the partition of unity satisfies $\sum_{i=1}^n f_i(s) \leq 1$ for all $s \in G$ and is equal to $1$ for $s \in K$. Hence
	\begin{align*}
		\lv\varphi(s) - g_{\varepsilon}\rv = \lv \sum_{i=1}^n f_i(s)(\varphi(s) - \varphi(r_i)+\varphi(r_i)-x_i)\rv &\leq \sum_{i=1}^n f_i(s) (\lv \varphi(s) - \varphi(r_i)\rv+\lv \varphi(r_i)-x_i\rv) \\
		&\leq \frac{\varepsilon}{\mu(WK)}.
	\end{align*}
	Since $\varepsilon$ was arbitrary and $W,K$ didn't depend on $\varepsilon$ and $ \mathrm{supp} g_\varepsilon \subseteq WK$, we see that $g_\varepsilon \to f$ in the inductive limit topology as $\varepsilon \to 0$.
\end{proof}
We will also need the following result, which can be found in \cite[Theorem 3.20, part (b, c)]{rudin1991functional}
\begin{lemma}
	If $K \subseteq B$ is compact, then the convex hull convex hull $\conv(K)$ is totally bounded and hence it has compact closure.
	\label{int:clconvcomp}
\end{lemma}


\begin{definition}
	Let $f \in \cc(G,B)$ and $\varphi \in B^*$, the continuous dual of $B$. Then we can define a bounded linear functional $I_f$ on $B^*$ by
	\begin{align*}
		L_f(\varphi) := \int_G \varphi(f(s)) \d s,
	\end{align*}
	since $| L_f(\varphi)| \leq \int_G \lv \varphi \rv \lv f(s) \rv \d s = \lv f \rv_1 \lv \varphi\rv$.
\end{definition}
\begin{note}	
If $(f_i) \subseteq  \cc(G,B)$ is a net which converges in the inductive limit topology to $f \in \cc(G,B)$, i.e., $f_i \to f$ uniformly on $G$ and for some $K \subseteq G$ compact we have $\supp f_i \subseteq K$ eventually (and hence also $\supp f \subseteq K$), then
\begin{align*}
	\lv f_i - f\rv_1 = \int_{G} \lv f_i(s) - f(s) \rv \d s 	\leq \lv f_i - f \rv \mu(K) \to 0,
\end{align*}
so $f_i \to f$ in the $L^1$-norm, and hence by the above we see that $|L_{f_i}(\varphi) - L_f(\varphi) | \leq \lv f_i-f\rv_1 \lv \varphi \rv \to 0$ for all $ \varphi \in B^*$, i.e.,  $L_{f_i}\to L_f$ in the weak$^*$ topology of $B^{**}$.
\label{note:induct}
\end{note}

We will see that $L_f$ is equal to $\hat a \colon \varphi \to \varphi(a)$ for some unique $a \in B$. Let $\iota \colon B \to B^{**}$ denote the inclusion $\iota(a) = \hat a$, then we have
\begin{lemma}
	If $f \in \cc(G,B)$, then $L_f \in \iota(B)$.
	\label{int:defintegral}
\end{lemma}
\begin{proof}
	Let $W$ be a compact neighborhood of $\mathrm{supp} f$, and let $K:= f(G) \cup \left\{ 0 \right\} \subseteq B$. Then $K$ is compact, so by \ref{int:clconvcomp} the set $C := \overline \conv \left\{ K \right\}$ is compact.

	Fix $\varepsilon > 0$, and pick a neighborhood $V$ of $e \in G$ such that $ \lv f(s) - f(r) \rv < \varepsilon$ for $s^{-1}r \in V$. Assume without loss of generality that $\mathrm{supp} f V \subseteq W$, since we can make $V$ arbitrarily small. Using compactness of $\supp f$, pick $s_1,\dots,s_n \in \supp f$ such that $\left\{ s_i V \right\}$ covers $\supp f$.
	
	Since $ \left\{ s_i V \right\} \subseteq W$, it has compact closure, so pick a partition of unity $\left\{ \varphi_i \right\}_{i=1}^n$ of $ K \subseteq \bigcup_{i=1}^n s_i V$. Define 
	\begin{align*}
		g_\varepsilon (s) := \sum_{i=1}^n f(s_i) \varphi(s),
	\end{align*}
	so $\supp g \subseteq W$. If $s \in \left( \bigcup_{i=1}^n s_i V \right)^c$ then $f(s) = g_{\varepsilon}(s) = 0$, and otherwise we see that if $s \in s_i V$, then $s_i^{-1}s \in V$ and $\lv f(s) - f(s_i) \rv < \varepsilon$, and hence
	\begin{align*}
		\lv f(s) - g_{\varepsilon} \rv  \leq \sum_{i=1}^n \varphi(s) \lv f(s)-f(s_i) \rv  < \varepsilon.
	\end{align*}

	Since $s \mapsto \sum_{i=1}^n \varphi(s) \leq 1$ and has compact support and $\bigcup_{i=1}^n \supp \varphi_i \subseteq W$, we have $\sum_{i=1}^n \int_G \varphi_i(s) \d s \leq \mu(W) < \infty$, so it is equal to some number $D$. For $\psi \in B^*$ we see that
	\begin{align*}
		L_{g_{\varepsilon}} ( \psi ) = \int_{G} \psi\left( \sum_{i=1}^n \varphi(s) f(s_i) \right) \d s = \sum_{i=1}^n \int_G \varphi(s) \d s \psi(f(s_i)) = \sum_{i=1}^n \int_G \varphi(s) \d s \underbrace{\iota(f(s_i))}_{ \in \iota(C)} (\psi) 
	\end{align*}
	so $L_{g_{\varepsilon}}  \in D \iota(C) \subseteq \mu(W) \iota(C)$, since $0 \in C$ so we can shrink scalar $\mu(W)$ to $D$. For $\varepsilon \to 0$, we see that $g_{\varepsilon} \to f$ in the inductive limit topology and $L_{g_\varepsilon} \in \mu(W) \iota(C)$. 

	Since $C$ is compact and convex, it is weakly compact which by definition is equivalent to $\iota(C)$ being compact in the weak$^*$-topology of $B^{**}$, and since $g_{\varepsilon} \to f$ in the inductive limit topology, we from \ref{note:induct} we see that $L_{g_\varepsilon } \to L_f$ in the weak $^*$-topology on $B^{**}$. In particular, since $\iota(C)$ is weak$^*$-compact, it is closed and hence $\mu(W) \iota(C)$ is closed in the weak$^*$-topology so $L_f \in \mu(W) \iota(C) \subseteq \iota(B)$.
\end{proof}
And we have a lemma emphasizing important properties of this integral:
\begin{lemma}
There exists a unique linear map $I \colon \cc(G,B) \to B$,
\begin{align*}
	f \mapsto \int_G f(s) \d s
\end{align*}
such that for all $f \in \cc(G,B)$ we have
\begin{align}
	\varphi(\int_g f(s) \d s ) &= \int_G \varphi(f(s)) \d s, \ \text{ for all } \varphi \in B^*,\\
	\lv \int_G f(s) \d s \rv &\leq \lv f \rv_1 \\
	\int_G g \otimes x (s) \d s &=x \int_G g(s) \d s, \text{ for all } g \otimes x \in \cc(G) \odot B,
\end{align}
and if $T \colon B \to D$ is a bounded linear operator, then
\begin{align*}
	T\left( \int_G f(s) \d s \right) = \int_G T(f(s)) \d s.
\end{align*}
	\label{int:bochnerproperties}
\end{lemma}
\begin{proof}
	Clearly letting $I(f) := \iota^{-1}(L_f)$, with $L_f$ as in \ref{int:defintegral}, then $f \mapsto I(f)$ is which instantly satisfies all the properties but the last. Suppose that $T \colon B \to D$ is a bounded linear operator between Banach spaces. Then $T \circ f \in \cc(G,D)$ for all $f \in \cc(G,B)$, and $\varphi \circ T \in B^*$ for all $ \varphi \in D^*$. Hence 
	\begin{align*}
		\varphi \circ T \left( \int_G f(s) \d s  \right) = \int_G \varphi( T \circ f (s) ) \d s = \varphi \left( \int_G T (f(s)) \d s \right),
	\end{align*}
	for all $\varphi \in D^*$, which separates points points of $D$, finishing the proof.
\end{proof}
In light of the above, we may give the following definition:
\begin{definition}
	For $f \in \cc(G,B)$, define the integral of $f$ $I(f) := \iota^{-1}(L_f)$, which we denote by $\int_G f(s) \d s := I(f)$.
\end{definition}

In particular, if we let $A$ be a $C^*$-algebra, then we obtain the following useful result:
\begin{proposition}
	For $f \in \cc(G,A)$, if $\pi \colon A \to \mathbb{B}(\H)$ is a representation of $A$, then for all $\xi , \eta \in \H$ the integral of \ref{int:bochnerproperties} satisfies
	\begin{align*}
		\langle \pi\left( \int_G f(s) \d s ) \right) \xi , \eta \rangle = \int_{G} \langle \pi(f(s)) \xi , \eta \rangle \d s,
	\end{align*}
	and 
	\begin{align*}
		\left( \int_G f(s) \d s  \right)^* = \int_G f(s)^* \d s
	\end{align*}
	and for $a , b \in M(A)$ we have
	\begin{align*}
	 a \int_G f(s) \d s b = \int_G a f(s) b \d s.		
	\end{align*}
	\label{int:cstarint}
\end{proposition}
\begin{proof}
	Define the vector-state $\varphi_\pi \colon a \mapsto \langle \pi(a) \xi , \eta \rangle$. Then, $\varphi_\pi \in A^*$, and the first statement follows. Suppose now that $\pi$ is a faithful representation, then since $\int_G \pi(f(s)) \d s = \pi \left( f(s) \d s \right)$, we see that the second statement holds as well when applied to $\varphi_\pi(I(f)^*)$. Finally if $a,b \in M(A)$ then
	\begin{align*}
		\langle \pi\left(a \int_G f(s) \d s b \right) \xi , \eta \rangle &= \langle \overline \pi(a) \pi\left( \int_G f(s) \d s \right) \overline \pi (b)  \xi , \eta \rangle \\
		 &= \langle \pi \left( \int_G f(s) \d s \right) (\overline \pi (b) \xi ) , \overline \pi(a^*) \eta \rangle \\ 
		 &= \int_G \langle \pi(f(s)) \overline \pi (b) \xi , \overline \pi(a ^* ) \eta \rangle \d s\\
		 &= \int_G \langle \pi(af(s)b) \xi , \eta \rangle \d s\\
		 &= \langle \pi \left(  \int_G a f(s) b \d s  \right) \xi , \eta \rangle.
	\end{align*}
\end{proof}

As mentioned, we will need to consider maps of the form of the form $s \mapsto \pi(f(s)) U_s$ where $f \in \cc(G,A)$, $\pi A \to \mathbb{B}(\H)$ and $U \colon G \to \mathcal{U}(\H)$ is strongly continuous. In order to use the above theorem to give sense then, we will want to change $A$ out with $M_s(A)$, i.e., the multiplier algebra of $A$ equipped with the strict topology, and combine it with the following lemma:
\begin{lemma}
	Let $E$ be a Hilbert $A$-module, and let $u \colon G \to \mathcal{U}(E)$ be a group homomorphism from $G$ to the group of unitary operators on $E$, i.e., operators $T \in \mathcal{L}_A(E)$ such that $T^*T=TT^*=I_E$. Then $u$ is strictly continous if and only if $u$ is strongly continuous.
	\label{int:unistrictstrong}
\end{lemma}
\begin{proof}
	We saw in \cref{mult:STARTSTRONG} that on bounded sets, the strict and the $*$-strong topology coincided. In particular, since $\mathcal{L}_A(E)$ is a $C^*$-algebra, the set of unitary operators is a subset of the unit circle. Thus, it suffices to show that if $s \mapsto u_s x$ is continuous for $x$ then $s \mapsto u_s^* x$ is continuous. Let $x \in E$, then
	\begin{align*}
		\lv u_s^* x - u_t^*x \rv_A^2 = \underbrace{\lv u_s^* x \rv_A^2}_{= \lv x \rv_A^2} + \underbrace{\lv u_t^*x \rv_A^2}_{= \lv x \rv_A^2} - \langle \underbrace{ \left( u_t u_s^* + u_s u_t^*  \right)}_{\stackrel{s \to t}{\longrightarrow} 2I_E} x ,x \rangle \stackrel{s \to t}{\longrightarrow} 0,
	\end{align*}
	showing the wanted, since $x \in E$ was arbitrary.
\end{proof}

We now generalize the result of \cref{int:cstarint} to cover the class of functions which we shall deal with later on.
\begin{theorem}
	Given a $C^*$-algebra $A$, there is a unique linear map $I\colon \cc(G,M_s(A)) \to M(A)$, $f \mapsto \int_G f(s) \d s$, such that for any non-degenerate representation $\pi \colon A \to \mathbb{B}(\H)$ and all $\xi, \eta \in \H$ the identity
	\begin{align}
		\langle \overline \pi \left( \int_G f(s) \d s \right) \xi , \eta \rangle = \int_{G} \langle \overline \pi(f(s)) \xi , \eta \rangle \d s,
		\label{mult:1.35}
	\end{align}
	and for all $f \in \cc(G,M_s(A))$
	\begin{align}
		\lv I(f) \d s \rv \leq \lv f \rv _{ \infty} \mu(\supp f ).
	\end{align}
	Also, the identities
	\begin{align}
		\int_G f(s) ^* \d s = \left( \int_G f(s) \d s \right)^*,
	\end{align}
	and
	\begin{align}
		\int_G a f(s) b \d s = a \left( \int_G f(s) \d s  \right) b
	\end{align}
	hold for all $f \in \cc(G,M_s(A))$ and $a,b \in M(A)$. If $T \colon A \to M(B)$ is a non-degenerate homomorphism of $C^*$-algebras, then
	\begin{align}
		\overline T \left( \int_G f(s) \d s\right) = \int_G \overline T (f(s)) \d s.
	\end{align}

	\label{int:multstrictintegral}
\end{theorem}
\begin{proof}
	If $I$ exists, then letting $\pi$ be a faithful representation of $A$ we see that $I$ must be unique by the first identity. For  $f \in \cc(G,M_s(A))$ and $a \in A$, it is easy to see that if $f_a \colon s \mapsto f(s) a$, then $f_a \in \cc(G,A)$. This allows us to use \cref{int:cstarint} to define the integral $\int_G f(s) a \d s$ for all $f \in \cc(G,M_s(A))$ and $a \in A$. Define $L_f \colon A \to A$ by
	\begin{align*}
		L_f(a) = \int_G f(s) a \d s, \ \text{ for } a \in A.
	\end{align*}
	Then $L_f \in \mathcal{L}_A(A_A)$, for if $a,b \in A$ then \cref{int:cstarint} implies that
	\begin{align*}
		\langle L_f a , b \rangle_{A_A} = (L_f a )^* b = \int_G (f(s) a)^* \d s b = \int_G a^* f(s)^* \d s = a^* \int_G f(s)^* b \d s = \langle a , L_{f^*} b\rangle_{A_A}.
	\end{align*}
	So $L_{f^*}$ is an adjoint of $L_f$. Let $\int_G f(s) \d s := L_f$ for all $f \in \cc(G,M_s(A)$, then $L_f \in \mathcal{L}_A(A_A) = M(A)$, and it is not hard to see that the assignment $f \mapsto L_f$ is linear.

	Again, by \cref{int:cstarint}, we see that a non-degenerate representation $\pi \colon A \to \mathbb{B}(\H)$ with $a \in A$ and $\xi, \eta \in \H$ that
	\begin{align*}
		\langle \overline \pi \left( \int_G f(s) \d s \right) \pi(a) \xi , \eta \rangle =\langle \pi \left(f(s) a \d s \right) \xi , \eta \rangle &= \int_G \langle \pi(f(s) a) \xi , \eta \rangle \d s\\
		&= \int_G \langle \overline \pi (f(s)) \pi(a) \xi , \eta \rangle \d s,
	\end{align*}
	which shows \cref{mult:1.35}, since $\pi$ is non-degenerate. The analogues of \cref{int:cstarint} follows similarly. Let $a \in A$ with $\lv a \rv \leq 1$, then by \cref{int:cstarint} applied to $f_a \in \cc(G,A)$, we see that
	\begin{align*}
		\lv \int_G f(s) \d s a \rv  \leq \lv f_a \rv_1 = \int_G \lv f(s) a \rv \d s \leq \int_G 1_{\supp f} \lv f \rv_{\infty} \d s = \mu(\supp f) \lv f \rv_{\infty},
	\end{align*}
	Hence $\lv \int_G f(s) \d s \rv = \sup_{\lv a \rv \leq 1} \lv \int_G f(s) \d s a\rv \leq \lv f \rv_{\infty} \mu(\supp f)$. Let $T \colon A \to M(B)$ be a non-degenerate homomorphism, with unique extension $\overline T \colon M(A) \to M(B)$, and hence for all $ a \in A$ and $b \in B$ we have
	\begin{align*}
		\overline T\left(  \int_G f(s) \d s \right) T(a) b = T\left( \int_G f(s) a \d s  \right)b= \int_{G} T(f(s)a ) b \d s &= \int_G \overline T(f(s)) T(a)b \d s \\
		&=\int_G \overline T(f(s)) \d s T(a)b,
	\end{align*}
	hence $\overline T \left( \int_G f(s) \d s \right) = \int_G \overline T(f(s)) \d s$ by non-degeneracy.
\end{proof}
In the construction of the general crossed product and other $C^*$-algebras associated to a $C^*$-dynamical system $(A,G,\alpha)$ we will need to deal with the algebra $\cc(G,A)$ quite a lot. We will want to develop a sort of twisted-convolution incorporating the action $\alpha$ of $G$. However, in order to do this properly, we will need a form of Fubini's theorem for the above:
\begin{lemma}
For $f \in \cc(G \times G  , B)$, the function
	\begin{align*}
		s \mapsto \int_G f(s,t) \d t
	\end{align*}
	is an element of $\cc(G,B)$.
	\label{int:singleint}
\end{lemma}
\begin{proof}
	It is easy to see that if $F \colon G \to B$ is the map $s \mapsto \int_G f(s,t) \d t$, then $ \supp F$ is compact. We show continuity. Suppose that $x_i \to x \in G$ and let $ \varepsilon > 0$. Pick compact $K_1,K_2 \subseteq G$ covering the support of $f$, i.e., such that $ \supp f \subseteq K_1 \times K_2$. Note that $f(x_i,y) \to f(x,y)$ uniformly: If not, then we can find a net $(r_j)\subseteq G$ and $\varepsilon_0 > 0$ such that
	\begin{align*}
		\lv f(x_i,r_j) - f(x,r_j) \rv \geq \varepsilon_0,
	\end{align*}
	for all $i,j$. This shows that $r_j \in K_2$ for all $j$, hence we may assume that $r_j \to r \in K_2$ (up to passing to a subnet). But then for sufficiently large $i,j$ we would have $\lv f(x_i,r_j) - f(x,r_j) \rv < \varepsilon_0$. Hence we can assume that $\lv F(x_i) - F(x) \rv < \frac{\varepsilon}{ \mu(K_2)}$. Then, 
	\begin{align*}
		\lv \int_G f(x_i,s) \d s - \int_G f(x,s) \d s \rv \leq \int_{K_2} \lv f(x_i,s) - f(x,s) \rv \ d s + \int_{K_2^c} \lv f(x_i,s) - f(x,s) \rv \d s  \leq \varepsilon,
	\end{align*}
	as $\supp f \subseteq K_1 \times K_2$, so $F \in \cc(G,B)$.
\end{proof}
\begin{corollary}
	Replacing the integral of \cref{int:singleint} with $\int_G f(t,s) \d t$ the statement still holds.
\end{corollary}
\begin{theorem}[Fubini's for $C^*$-valued integrals]
	Suppose that $f \in \cc(G \times G, M_s(A))$, then 
	\begin{align*}
		s \mapsto \int_G^{M_s(A)} f(s,t) \d t \text{ and } s \mapsto \int_G^{M_s(A)} f(t,s) \d t
	\end{align*}
	are elements of $\cc(G,M_s(A))$ and moreover
	\begin{align*}
		\int_G^{M_s(A)} \int_G^{M_s(A)} f(s,t) \d s \d t = \int_G^{M_s(A)} \int_G^{M_s(A)} f(s,t) \d t \d s.
	\end{align*}
	\label{int:fubini}
\end{theorem}
\begin{proof}
Define for a fixed $s$, the map $t \mapsto f(s,t)$ is in $\cc(G,M_s(A)$, hence we can define $\int_G f(s,t) \d t$ and $\int_G f(t,s) \d t$ as in \Cref{int:multstrictintegral}. Recall that $M(A) = M(\mathbb{K}_A(A_A))$, so the strict topology is induced by multiplication operators on $A$, so it suffices to show that $s \mapsto a \left(\int_Gf(s,t) \d t\right)$ and $s \mapsto \left(\int_G f(s,t) \d t\right) a $ are in $\cc(G,A)$ for all $ a \in A$ for the first claim. By \Cref{int:multstrictintegral}, we have 
\begin{align*}
	a \left( \int_G f(s,t) \d t \right) = \int_G a f(s,t) \d t,
\end{align*}
for all $a \in A$ and $s \in G$, and similarly $\left( \int_G f(s,t) \d t \right)a = \int_G f(s,t) a \d t$. The maps $(s,t) \mapsto a f(s,t)$ and $(s,t) \mapsto f(s,t) a$ are in $\cc(G \times G , A)$, clearly, so \Cref{int:singleint} ensures that $s \mapsto \int_G^{M_s(A)} f(s,t) \d t$ and $s \mapsto \int_G ^{M_s(A)} f(t,s) \d t$ are in $\cc(G,M_s(A)$.

For the last claim, we now see that the double integrals are well-defined \Cref{int:multstrictintegral}, so by letting $A$ be faithfully representted on $\mathbb{B}(\H)$, we may obtain the equality using the regular Fubini's theorem for scalar valued integrals (see e.g., \cite{schilling}) applied pointwise to an orthonormal basis through the inner-product.
\end{proof}

%***************************************************************************************************************************************
%***************************************************************************************************************************************
%***************************************************************************************************************************************
%***************************************************************************************************************************************
%***************************************************************************************************************************************
%***************************************************************************************************************************************

\ssection{Crossed Products}
In analysis, an object of interest is the group algebra $L^1(G)$ for locally compact groups $G$. One can turn it into a $*$-algebra by showing that the convolution gives rise to a well-defined multiplication, and one can use the modular function to define an involution. Analogously to this, we do the same for $A$ valued functions:
\begin{definition}
	Suppose that $G \acts\alpha A$. We let 
	\begin{align*}
		\cc(G,A):=\{ f \colon G \to A \mid f \text{ continuous and compactly supported}\}.
	\end{align*}
	For $f,g \in \cc(G,A)$, we define the $\alpha$-twisted convolution of $f$ and $g$ at $t \in G$ by
	\begin{align*}
		(f \ast_\alpha g)(t):= \int_G  f(s) \alpha_s(g(s^{-1}t)) \d s,
	\end{align*}
	which is well defined since the map $(s,t) \mapsto f(s) \alpha_s g(s^{-1}t)$ is in $\cc(G \times G , A)$ and \cref{int:singleint} applies. Moreover, we define an $\alpha$-twisted involution of $f \in \cc(G,A)$ at $t \in G$ by
	\begin{align*}
		f^*(t):=\Delta(t^{-1}) \alpha_t(f(t^{-1})^*).
	\end{align*}
	Straight forward calculations similar to the one classic ones show that $\cc(G,A)$ becomes a $*$-algebra with multiplication given by the $\alpha$-twisted convolution and involution given by the $\alpha$-twisted involution. We denote the associated $*$-algebra by $\cc(G,A,\alpha)$.
	
	To $f \in \cc(G,A,\alpha)$, we associate the number $\lv f \rv_1 \in [0, \infty)$ by
		\begin{align*}
			\lv f \rv_1 := \int_G \lv f(t) \rv \d t.
		\end{align*}
	Straightforward calculations show that $f \mapsto \lv f \rv_1$ is a norm, and we denote the completion of $\cc(G,A,\alpha)$ in $\lv \cdot \rv_1$ by $L^1(G,A,\alpha)$.
\end{definition}

For discrete and second-countable groups $G$ acting on a $C^*$-algebra $A$ via an action $\alpha$, we may easily construct a $*$-representation of $\cc(G,\A)$ from a convariant representation $(\pi,u)$ by letting $\pi \rtimes u \colon \cc(A,G,\alpha) \to \mathbb{B}(\H)$ be the map
\begin{align*}
	\sum_{g \in G} a_g \delta_g \mapsto \pi(a_g) u_g,
\end{align*}
however as discussed previously, this is not enough when $G$ is locally compact, since we can't assume that $u$ was continuous with respect to the norm on $\mathbb{B}(\H)$. However, in the previous section we constructed the tools to construct the desired generalization:
\begin{lemma}
	Suppose that $G \acts \alpha A$ with $G$ locally compact. If $(\pi,u)$ is a covariant representation of the associated $C^*$-dynamical system, then for all $f \in \cc(G,A)$, the map
	\begin{align*}
		s \mapsto \pi(f(s)) u_s, \text{ for }s \in G
	\end{align*}
	is in $\cc(G,\mathbb{B}_s(\H))$, where $\mathbb{B}_s(\H)$ indicates that we endow $\mathbb{B}\left(\H \right)$ with the strict topology of $ \mathbb{K}(\H)$.
	\label{cross:integrandcont}
\end{lemma}
\begin{proof}
	By assumption, $u$ is strongly continuous, which is equivalent to being strictly continuous by \Cref{int:unistrictstrong}. Since $f \in \cc(G,A)$, we see that $\pi \circ f \in \cc(G,\mathbb{B}(\H))$, and being norm continuous implies strict continuity, so $s \mapsto \pi(f(s)) u_s$ is an element of $\cc(G,\mathbb{B}_s(\H))$.
\end{proof}
The above and \ref{int:multstrictintegral} shows that given a covariant representation $(\pi,u)$ of $G \acts \alpha A$, there is a well-defined linear map $\pi \rtimes u \colon \cc(G,A) \to \mathbb{B}(\H)$ given by
\begin{align*}
	\pi \rtimes u (f) = \int_G^{\mathbb{B}(\H)} \pi(f(s)) u_s \d s,
\end{align*}
And we shall see that this representation has a lot of nice properties:

\begin{proposition}
	Suppose that $(\pi,u)$ is a covariant representation of a dynamical system $G \acts \alpha A$, with $G$ locally compact. Then the map $\pi \rtimes u$ is a $*$-representation of $\cc(G,A,\alpha)$ on $\mathbb{B}(\H)$ which is norm-decreasing. We call $\pi \rtimes u$ the \myemph{integrated form} \emph{of $\pi$ and $u$}. If $\pi$ is non-degenerate, then $\pi \rtimes u$ is as well.
	\label{cross:integform}
\end{proposition}
\begin{proof}
	Let $\xi , \eta \in \H$ and $f \in \cc(G,A)$, then
	\begin{align*}
		 |\langle \pi \rtimes u (f) \xi , \eta \rangle| \leq \int_G | \langle \pi(f(s)) u_s \xi , \eta \rangle| \d s \leq \int_G \lv \pi(f(s)) \rv \lv \xi\rv \lv \eta \rv \d s \leq \lv f \rv_1 \lv \xi \rv \lv \eta \rv
	\end{align*}
	Since $\xi, \eta$ can be any unit vector, we have $\lv \pi \rtimes u (f) \rv \leq \lv f \rv_1$ for all $f \in \cc(G,A)$.

	Let $f \in \cc(G,A,\alpha)$, then
	\begin{align*}
		\pi \rtimes u(f)^* = \int_G^{\mathbb{B}(\H)}  u_s^* \pi(f(s))^* \d s &= \int_G^{\mathbb{B}(\H)} u_{s^{-1}} \pi(f(s)^*) \d s\\
		&= \int_G^{\mathbb{B}(\H)} \Delta(s)^{-1} u_{s} \pi(f(s^{-1})^*) \d s \\
		&= \int_G^{\mathbb{B}(\H)} \Delta(s)^{-1} \pi(\alpha_s(f(s^{-1})^*)) u_s \d s\\
		&= \int_G^{\mathbb{B}(\H)} \pi\left( \underbrace{\alpha_s\left( \Delta(s)^{-1}f(s^{-1})^* \right)}_{=f^*(s)} \right) u_s \d s\\
		&= \int_G^{\mathbb{B}(\H)} \pi(f^*(s)) u_s \d s\\
		&= \pi \rtimes u(f^*)
	\end{align*}
	and if $f,g \in \cc(G,A,\alpha)$ then using left-invariance of $\d s$, \Cref{int:cstarint} and \Cref{int:fubini}, we see that
	\begin{align*}
		\pi \rtimes u(f\ast_{\alpha}g) &= \int_G^{\mathbb{B}(\H)} \pi \left( f \ast_\alpha g(s) \right)u_s \d s\\
		&= \int_G^{\mathbb{B}(\H)} \pi\left( \int_G^A f(t) \alpha_t g(t^{-1}s) \d t \right)u_s \d s\\
		&= \int_G^{\mathbb{B}(\H)}\int_G^{\mathbb{B}(\H)} \pi(f(t)) \pi(\alpha_t(g(t^{-1}s))u_{t t^{-1}s} \d t \d s\\
		&= \int_G^{\mathbb{B}(\H)} \int_G ^{\mathbb{B}(\H)} \pi(f(t)) u_t \pi(g(t^{-1}s)) u_{t^{-1}s}\d s \d t\\
		&= \int_G^{\mathbb{B}(\H)} \pi(f(t)) u_t \int_G^{\mathbb{B}(\H)} \pi(g(t^{-1}s )) u_{t^{-1}s} \d s \d t\\
		&= \int_G ^{\mathbb{B}(\H)} \pi(f(t)) u_t \int_G ^{\mathbb{B}(\H)} \pi(g(s)) u_s \d s \d t\\
		&= \int_G ^{\mathbb{B}(\H)} \pi(f(t)) u_t \d t \int_G ^{\mathbb{B}(\H)} \pi(g(s)) u_s \d s \\
		&= \pi \rtimes u (f) \pi\rtimes u (g),
	\end{align*}
	as wanted. Assuming non-degeneracy of $\pi$, for $h \in \H$ and $\varepsilon>0$ we can pick $a \in A$, $\lv a \rv =1$ such that $\lv \pi(a) h - h \rv < \varepsilon$. Being a unitary representation, we may choose a neighborhood $e \in V \subseteq G$ such that $\lv u_s h - h\rv < \varepsilon$ for $s \in V$. Pick $0 \leq \varphi \in \cc(G)$ such that $\int_G \varphi(s) \d s = 1$ and $\supp \varphi \subseteq V$. Then, $f := \varphi \otimes a \in \cc(G,A)$ satisfies for $k \in \H$, $\lv k \rv = 1$, that
	\begin{align*}
	\left| \langle \pi \rtimes u(f)h-h,k\rangle \right| &= 	\left| \int_G \varphi(s) \langle \pi(a)u_s h-h,k \rangle\right|\\
	& \leq \int_G \varphi(s) (|\langle \pi(a) (u_s h-h,k\rangle|+|\langle \pi(a)h-h,k\rangle| ) \d s \\
	&< 2 \varepsilon,
	\end{align*}
	showing that $\lv \pi \rtimes u(f)h-h\rv \leq 2 \varepsilon$.
\end{proof}

Now, given a locally compact topological group $G$, let $\lambda \colon G \to \U(L^2(G))$ be the left-regular representation $\langle \lambda_g \xi, \eta\rangle = \int_G \xi(g^{-1}s)\overline{\eta(s)}\d s$ for $g \in G$ and $\xi, \eta \in L^2(G)$. Even though $L^2(G)$ consists of equivalence classes (unless e.g., $G$ is discrete), we will abbreviate $\lambda_g \xi(s) := \xi(g^{-1}s)$. Given a Hilbert space $\H$, we will use $\lambda_g$ to denote the operator $I \otimes \lambda_g$ on $\H \otimes L^2(G) \cong L^2(G,\H)$. With this, we have the following:
\begin{lemma}
	Suppose that $G \acts \alpha A$, and $\pi$ is a representation of $A$ on $\H$. Define $\tilde{\pi}(a) \colon L^2(G,H)$ by 
	\begin{align*}
		\tilde{\pi}(a)\xi(r) = \pi(\alpha_{r^{-1}}(a))\xi(r), \text{ for } a \in A, \ r \in G \text{ and } \xi \in L^2(G,H).
	\end{align*}
	Then $(\tilde{\pi},\lambda)$ is a covariant representation of $G \acts \alpha A$ on $L^2(G,\H)$.
	\label{cross:regularrep}
\end{lemma}
\begin{proof}
	Let $a \in A$ and $r \in G$. Then for all $\xi \in L^2(G,\H)$ and $s \in G$:
	\begin{align*}
		\lambda_g (\tilde{\pi}(a) \lambda_{g^{-1}} \xi)(s) = \pi(\alpha_{g^{-1}r}^{-1}(a))\lambda_g^* \xi(g^{-1}s) = \pi(\alpha_{r^{-1}}\alpha_g(a))\xi (g g^{-1} s) = \tilde{\pi}(\alpha_g(a))\xi(s),
	\end{align*}
	as wanted.
\end{proof}
\begin{corollary}
To every $C^*$-dynamical system $G \acts \alpha A$, there exists a covariant representation.
\end{corollary}
In light of the above, we may define the following important class of covariant representations of a $C^*$-dynamical system:
\begin{definition}
	Given $G \acts \alpha A$, then for any representation $\pi$ of $A$ on $\H$, the above covariant representation is called the \myemph{induced regular representation}, and we will write $\mathrm{Ind}_e^G(\pi) := (\tilde{\pi}, \lambda)$. Any representation of the above form is called a \myemph{regular representation}.
\end{definition}
As we shall see, the integrated form of the above plays a crucial role in the theory regarded in this thesis, but first, we shall need the following lemma:
\begin{lemma}
	Suppose that $G \acts \alpha A$, and $\pi$ is a representation of $A$ on $\H$. Then $\mathrm{Ind}_e^G \pi$ is non-degenerate if $\pi$ is non-degenerate.
	\label{cross:regrepnondeg}
\end{lemma}
\begin{proof}
	First note that the set $ f \otimes \eta \colon r \mapsto f(r) \eta$ with $f \in \cc(G)$ and $\eta \in \H$ has dense linear span in $L^2(G,\H)$, and since $\pi$ is non-degenerate, the set $\{ f \otimes \pi(a) \eta \mid \eta \in \H, \ f \in \cc(G) \text{ and } a \in A \}$ has dense linear in $L^2(G, \H)$ as well. Let $\varepsilon > 0$ and let $(e_i)$ be an approximate identity of $A$. By the previous remark and the fact that $\tilde{\pi}$ is a linear contraction, it suffices to see that 
	\begin{align*}
		\tilde{\pi}(e_i) (f \otimes \pi(a) \xi)(r)=f(r) \pi(\alpha_{r}^{-1}(e_i)a) \xi   \stackrel 2 \to f(r) \pi(a) \xi,
	\end{align*}
	\todo{perhaps elaborate using proper integrals}for $f \in \cc(G)$, $\xi \in \H$ and $a \in A$ and $r \in \supp f$. The argument thus boils down to showing that $\alpha_r^{-1}(e_i) a \to a$, which an easy argument of contradiction using compactness of $\supp f$ and the approximate identity property passing to subnets shows that this is true.
\end{proof}
\begin{remark}
	In particular, if $\pi$ is a non-degenerate representation of $A$, then the integrated form, $\tilde{\pi} \rtimes \lambda$, of $\mathrm{Ind}_e^G \pi$ is non-degenerate by \Cref{cross:integform}. This is not the only property which carries over from the map $\mathrm{Ind}_e^G \pi \mapsto \tilde{\pi} \rtimes \lambda$, but before we further comment on this, we must introduce some new maps
\end{remark}
\begin{definition}
	Suppose that $G \acts \alpha A$. Let $f \otimes a  \in \cc(G,A)$ be the map $r \mapsto f(r)a$. For $s \in G$, let $\iota_G(s)$ be the linear extension of the map $\lambda_s \otimes \alpha_s \colon f \otimes a \mapsto \lambda_s f \otimes \alpha_s(a)$ to all of $\cc(G,A)$, i.e., $\iota_G(s) h(t) = \alpha_s(h(s^{-1}t))$ for $h \in \cc(G,A)$.
\end{definition}
And, we have the following handy lemma:
\begin{lemma}
	Suppose that $(\pi,u)$ is a covariant representation of a $C^*$-dynamical system $G \acts \alpha A$, and let $\pi \rtimes u$ denote the representation of $\cc(G,A,\alpha)$. Then, for $f \in \cc(G,A)$ and $r \in G$ it holds that
	\begin{align*}
		 u_r  \circ \pi \rtimes u \circ \iota_G(r) (f) =   \pi \rtimes u (f)
	\end{align*}
	\label{cross:iotaG}
\end{lemma}
\begin{proof}
It follows from the calculation
\begin{align*}
	\pi \rtimes u (\iota_G(r)f) = \int_G \pi(\iota_G(r) f(s) ) u_s \d s &= \int_G \pi(\alpha_r(f(r^{-1}s))) u_ru_{r^{-1}s} \d s\\
	&= \int_G u_r \pi(f(r^{-1}t)) u_{r^{-1}s} \d s \\
	&= u_r \int_G  \pi(f(s)) u_s \d s \\
	&= u_r \circ \pi \rtimes u (f).
\end{align*}
\end{proof}
\begin{lemma}
	Suppose that $G \acts \alpha A$ and $\pi$ is a faitful representation of $A$ on $\H$. Then the integrated form, $\tilde{\pi} \rtimes \lambda$, of $\mathrm{Ind}_e^G (\pi)=(\tilde \pi , \lambda)$ is a faithful representation of $\cc(G,A,\alpha)$.
	\label{cross:regfaith}
\end{lemma}
\begin{proof}
	Let $0 \neq f \in \cc(G,A)$, and let $r \in G$ witness this, i.e, $f(r) \neq 0$. By \Cref{cross:iotaG} we can replace $f$ by $\iota_G(r^{1})f$ and assume that $r=e$, so without loss of generality we assume $r = e$. Let $\xi, \eta \in \H$ witness faithfulness of $\pi$, i.e., such that $\langle \pi ( f(e)) \xi, \eta \rangle \neq 0$. Pick an open neighborhood $e \in V$ such that for $s,r \in V$ we have 
	\begin{align*}
		| \langle \pi(\alpha_r^{-1}(f(s))) \xi  - \pi(f(e)))\xi , \eta \rangle|  < \frac{| \langle \pi (f(e)) \xi , \eta\rangle|}{3}.
	\end{align*}
Let $W \subseteq V$ be a symmetric neighborhood such that $W^2 \subseteq V$, and let $\varphi \in \cc(G)$ such that $\varphi \geq 0$ and $\supp \varphi \subseteq W$ and
\begin{align*}
	\int_G \int_G \varphi(s^{-1}r) \varphi(r) \d r \d s = 1.
\end{align*}
Let $\zeta = \varphi \otimes \xi$ and $\psi = \varphi \otimes \eta$, then
\begin{align*}
	\langle \tilde{ \pi } \rtimes \lambda (f) \zeta, \psi \rangle  =  \int_G \langle \tilde{\pi}(f(s)) \lambda_s \zeta , \psi\rangle \d s &= \int_G \int_G \langle \tilde{\pi} f(s) \lambda_s \zeta(r) , \psi(r) \rangle \d r \d s\\
	&= \int_G \int_G \langle \pi(\alpha_{r^{-1}}f(s)) \zeta\left( r^{-1}s \right) \psi(s) \rangle \d r \d s\\
	&= \int_G \int_G \varphi(r^{-1}s) \varphi(r) \langle \pi(\alpha_{r^{-1}}f(s)) \xi , \eta \rangle \d r \d s
\end{align*}
and hence for all $r,s \in W$ we have
\begin{align*}
	\left | \tilde{\pi} \rtimes \lambda (f) \zeta , \psi \rangle - \langle \pi(f(e)) \zeta , \psi \rangle\right | &=  \left |  \int_G \int_G\varphi(r^{-1}s) \varphi (s) (\langle \pi(\alpha_{r^{-1}} f(s)) \xi , \eta \rangle - \langle \pi(f(e)) \xi , \eta \rangle  \d r \d s \right | \\
	&\leq \int_G \int_G \varphi(r^{-1}s ) \varphi (s) | \langle \pi(\alpha_{r^{-1}}(f(s))) \xi , \eta\rangle - \langle \pi(f(e)) \xi , \eta \rangle | \d r \d s \\
	&< \int_G \int_G \varphi(r^{-1}s) \varphi(s) \frac{| \langle \pi(f(e)) \xi, \eta \rangle|}{3} \d r \d s \\
	&= \frac{| \langle \pi(f(e)) \xi , \eta \rangle|} {3},
\end{align*}
and in particular, it follows that $\tilde{\pi} \rtimes \lambda (f) \neq 0$.
\end{proof}

With the above result in hand, we can finally give meaning to a universal norm for the general convolution algebra $\cc(G,A,\alpha)$. Like in the discrete case, we define the \emph{universal norm} of $f \in \cc(G,A,\alpha)$ to be non-negative real number
\begin{align*}
	\lv f \rv_u &:= \sup\left\{ \lv \pi \rtimes u (f)\rv \mid  \pi \rtimes u \text{ is a covariant represenation of }(A,G ,\alpha) \right\}\\
	&\leq \lv f \rv_1,
\end{align*}
and it is easy to see that the assignment $\cc(G,A,\alpha) \ni f \mapsto \lv f \rv_u$ is a norm: It clearly satisfies $\lv f + g \rv_u \leq \lv f \rv_u + \lv g \rv_u$ and $\lv cf \rv_u = |c| \lv f \rv_u$ for $f,g \in \cc(G,A,\alpha)$ and $c \in \C$. If $\lv f \rv_u = 0$ then $\tilde{\pi} \rtimes \lambda (f) = 0$ for all faithful $\pi$, which implies by \Cref{cross:regfaith} that $f = 0$. Moreover, we see that $\cc(G,A,\alpha)$ is a pre $C^*$-algebra, since $\pi \rtimes u$ is a homomorphism for all covariant representations $(\pi,u)$, and thus $\lv \pi \rtimes u (f^*\ast f)\rv = \lv \pi \rtimes u(f)^* \circ \pi \rtimes u(f)\rv = \lv \pi \rtimes u (f)\rv^2$, so that $\lv f^* \ast f\rv_u = \lv f \rv_u^2$. 
\begin{note}
	While it indeed is not possible to take the supremum over a class, it can be shown that it is enough to take the supremum over cyclic covariant representations, which is a set. For now, we stick with the following:
\end{note}
\begin{lemma}
	Let $G \acts \alpha A$ be a $C^*$-dynamical system and let $(\pi,u)$ be a covariant representation on $\H$. Let 
	\begin{align*}
		V_\pi = \overline \Span\left\{ \pi(a) \xi \mid a \in A \text{ and } \xi \in \H \right\} \subseteq \H,
	\end{align*}
	then $V$ is invariant under both $\pi$ and $u$, and if $\pi^{V_\pi}$ and $ u^{V_\pi}$ denotes the corresponding subrepresentations on $V^{\pi}$, we have for all $f \in \cc(G,A,\alpha)$ 
	\begin{align*}
		\lv \pi ^{V_\pi} \rtimes u^{V_\pi}(f) \rv = \lv \pi \rtimes u (f) \rv,
	\end{align*}
	so that
	\begin{align*}
		\lv f \rv_u = \sup\left\{ \lv \pi \rtimes u(f) \rv \mid  (\pi,u) \text{ is a non-degenerate covariant representation of } G \acts \alpha A\right\}.
	\end{align*}
	\label{cross:essnorm}
\end{lemma}
\begin{proof}
	This is a standard proof, and we refer to \cite[52]{williamscrossed} for the proof.
\end{proof}

This allows us to define the general crossed product as the completion of $\cc(G,A,\alpha)$:
\begin{definition}
	For a $C^*$-dynamical system $(A,G,\alpha)$, we define the \myemph{full crossed product} (or simply the crossed product) of $A$ and $G$ to be norm closure of $\cc(G,A,\alpha)$ in the norm $\lv \cdot \rv_u$, and we will denote it by $A \rtimes_\alpha G$.
\end{definition}
\begin{remark}
	Another (equivalent) formulation of the crossed product is to consider the representation of $L^1(G,A)$ given by the direct sum of all non-degenerate covariant representations of $L^1(G,A)$. Again, one can reformulate the above to only consider cyclic covariant representations, and use the fact that every representation of a $C^*$-algebra is a sum of cyclic representations.
\end{remark}
Inconvenient as it is, there is no guarantee that $A \rtimes_\alpha G$ contains a copy of $A$ or $G$ except under certain additional conditions on $A$ and $G$. However, we can always embed $A$ and $G$ into the Multiplier Algebra $M(A \rtimes_\alpha G)$ of $A \rtimes_\alpha G$:
\begin{lemma}
	Suppose that $G \acts \alpha A$ is a $C^*$-dynamical system. Define for $a \in A$ the map $\iota_A(a) \colon A \rtimes_\alpha G \to A \rtimes_\alpha G$ to be the extension of the map defined on $\cc(G,A,\alpha)$ by
	\begin{align*}
		\iota_A(a) f(s) = af(s), \text{ for } f \in \cc(G,A,\alpha) \text{ and } s \in G.
	\end{align*}
	Then $\iota_A \colon A \to M(A \rtimes_\alpha G)$, $a \mapsto \iota_A(a)$, is a faithful non-degenerate $^*$-representation, satisfying
	\begin{align*}
		\overline{\pi \rtimes u } \iota_A(a) = \pi(a),
	\end{align*}
	for all covariant representations $(\pi,u)$ of $G \acts \alpha A $.
\end{lemma}
\begin{proof}
First, for covariant representations $(\pi,u)$ of $G \acts \alpha A$, we see that
\begin{align*}
	\pi \rtimes u(\iota_A(a) f) = \int_G \pi(a) \pi(f(s)) u_s \d s = \pi(a) \pi \rtimes u (f),
\end{align*}
for all $f \in \cc(G,A,\alpha)$ and $a \in A$, so $\lv \iota_A(a) f \rv \leq \lv a \rv \lv f \rv$. Hence $\iota_A(a)$ extends to an operator, which we also denote by $\iota_A(a)$. If $a \in A$ and $f \in \cc(G,A,\alpha)$, then 
\begin{align*}
	\left( \iota_A(a) f \right)^*(s) = \Delta(s^{-1}) \alpha_s(f(s^{-1})^*) \alpha_s(a^*) = f^*(s) \alpha_s(a^*), \text{ for }s \in G,
\end{align*}
and hence we see that 
\begin{align*}
	\left( \iota_A(a) f \right)^* \ast g (s) = \int_G f^*(r) \alpha_r(a^*) \alpha_r(g(r^{-1}s)) \d r &= \int_G f^*(r) \alpha_r \left(  (\iota_A(a^*) g)(r^{-1}s)\right) \d r \\
	&= f^* \ast(\iota_A(a^*) g)(s),
\end{align*}
so $\iota_A(a^*)$ is an adjoint on a dense sub$*$-algebra (of the $C^*$-algebra seen as a Hilbert module over itself), hence everywhere by continuity, so $\iota_A(a) \in M(A \rtimes_\alpha G)$. For injectivity, let $\pi$ be any faithful, non-degenerate representation of $A$, and let $\overline{\tilde \pi \rtimes \lambda}$ denote the extension of $\tilde \pi \rtimes \lambda$ to $M(A \rtimes_\alpha G)$. By \ref{cross:integform} and the above, we see that $\tilde \pi(a) = \overline{\tilde \pi \rtimes \lambda } \circ \iota_A(a)$, hence $\iota_A(a) = 0 \implies \tilde \pi(a) = 0 \implies a = 0$, since $\tilde \pi$ is faithful.

For non-degeneracy, note that \ref{int:indlmdense} implies that the span of $\varphi \otimes ab$ is dense in $A \rtimes_\alpha G$, since the denseness in the inductive limit topology also implies that the span is dense in the $L^1$-norm on $\cc(G,A)$, and clearly $\varphi \otimes ab \in \{ \iota_A(a) x \mid\ a \in A, \ x \in A \rtimes_\alpha G\}$.
\end{proof}
\begin{lemma}
	There is an injective and strictly continuous group homomorphism
	\begin{align*}
		\iota_G \colon G \to \mathcal{U} M(A \rtimes_\alpha G),
	\end{align*}
	such that for $f \in \cc(G,A,\alpha) \subseteq A \rtimes_\alpha G$ we have
	\begin{align*}
		\iota_G(s) f(t) = \alpha_s (f(s^{-1}t)),
	\end{align*}
	for all $s,t \in G$, and if $(\pi,u)$ is a covariant representation of $G \acts \alpha A$, then
	\begin{align*}
		\overline{\pi \rtimes u} \iota_G(s) =u_s,
	\end{align*}
for $s \in G$, where $\overline {\pi \rtimes u}$
\end{lemma}
Before we commence the proof of the above, we want to point out that the above to lemmas ensures that for each covariant representation $(\pi , u)$ of $G \acts \alpha A$ we obtain the following commutative diagram
\begin{equation}
	\begin{tikzcd}
		G \arrow[rr, "\iota_G", hook] \arrow[rrd, "u"'] && M(A \rtimes_\alpha G) \arrow[d, "\overline{\pi \rtimes u }"] && A \arrow[ll,"\iota_A"',hook] \arrow[lld, "\pi"]\\
		& &\mathbb{B}(\H)&&
	\end{tikzcd}
\end{equation}
And we shall see that they are in fact a covariant pair. But first, the proof of the above lemma:
\begin{proof}
	Let $\iota_G(s)$ be defined on $\cc(G,A,\alpha)$ as earlier. By \Cref{cross:iotaG}, we see that
	\begin{align*}
		\lv \iota_G(r) f \rv_u = \lv f \rv_u,
	\end{align*}
	for all $f \in \cc(G,A,\alpha)$, and so extends to $A \rtimes_\alpha G$. Let $s,r \in G$, then for $f \in \cc(G,A,\alpha)$, we see that
	\begin{align*}
		(\iota_G(r) f(s))^* = \Delta(s^{-1}) \alpha_s ( (\alpha_r(f(r^{-1}s^{-1}))^*) \Delta(s^{-1}) \alpha_{sr}(f(r^{-1}s^{-1})^*),
	\end{align*}
	which, combined with the fact that $\Delta(s^{-1}) \d s$ is the integrand of a right Haar measure, can be used to show that $\iota_G(r)^* = \iota_G(r^{-1})$, so that $\iota_G(r) \in M(A \rtimes_\alpha G)$. In fact, since clearly $\iota_G(rs) = \iota_G(r) \circ \iota_G(s)$ for $s,r \in G$, it is a unitary element.

	Let $(\pi,u)$ be any non-degenerate covariant representation of $G \acts \alpha A$. By \Cref{cross:iotaG}, we see that $u_s  = \overline{ \pi \rtimes u} \iota_G(s)$ for all $ s \in G$, and if $\pi$ is a faithful representation of $A$, then considering  $\mathrm{Ind} \pi = (\tilde \pi, \lambda)$, we see that $\iota_G(s) \neq \mathrm{id}$ if $s \neq e$. For continuity, let $f \in \cc(G,A,\alpha)$. We will see that $\iota_G$ is strongly continuous, which by \Cref{int:unistrictstrong} is equivalent to being strictly continuous. We thus show that if $(s_i)\subseteq G$, $s_i \to e$, then $\iota_G(s_i) x \to x$ for $x \in A \rtimes_\alpha G$. We know that the set of functions $\varphi \otimes a$, $\varphi \in \cc(G)$ and $a \in A$, spans a dense subset of $\cc(G,A)$, hence it suffices to show that $\iota_G(s_i) \varphi \otimes a \stackrel{L^1}{\to} \varphi \otimes a$, but
	\begin{align*}
		\lv \iota_G(r_{i}) \varphi \otimes a - \varphi \otimes a \rv_1 &= \int_G \lv \varphi(r_{i}^{-1}s) \alpha_{r_i}(a) - \varphi(r_i^{-1}s) a + \varphi(r_{i}^{-1}a) a - \varphi(s) a \rv_A \d s\\ 
		& \leq \lv \alpha_{r_i}(a) - a \rv \lv \varphi \rv_1 + \lv \lambda_{r_i} \varphi - \varphi \rv_1 \lv a \rv,
	\end{align*}
	which goes to $0$, by uniform continuity of $\varphi$ and the fact that $\alpha_{r_i}(a) \to a$ for all $ a\in A$ as $r_i \to e$.
\end{proof}
And, we summarize the above in the following
\begin{proposition}
	For a $C^*$-dynamical system $G \acts \alpha A$, there exists a non-degenerate faithful $^*$-homomorphsim 
	\begin{align*}
	\iota_A \colon A \to M(A \rtimes_\alpha G),	
	\end{align*}
	and an injective strictly continuous unitary representation 
	\begin{align*}
		\iota_G \colon G \to \mathcal{U}M(A \rtimes_\alpha G),
	\end{align*}
	such that $\iota_G(r) f(s) = \alpha_r(f(r^{-1}s))$  and $\iota_A(a) f(s) = af(s)$ for $f \in \cc(G,A,\alpha)$ and $r,s \in G$ and $a \in A$. Moreover, the pair $(\iota_A,\iota_G)$ is covariant with respect to $\alpha$, i.e., 
	\begin{align*}
		\iota_A(\alpha_r (a) ) = \iota_G(r)  \iota_A(a) \iota_G(r)^*,
	\end{align*}
	and for non-degenerate covariant representations $(\pi,u)$ we have
	\begin{align*}
		\overline{\pi \rtimes u} \circ \iota_A(a) = \pi(a) \text{ and } \overline{\pi \rtimes u } \circ \iota_G (s) = u_s.
	\end{align*}
	\label{cross:iotaprop}
\end{proposition}
\begin{proof}
	The only thing left to be shown is the covariance relation, but this follows immediately from the following calculation:
	\begin{align*}
		\iota_G(r) \iota_A(a) \iota_G(r)^* f(s) = \alpha_r(\iota_A(a) \iota_G(r)^{*}f(r^{-1}s)) = \alpha_r(a) \alpha_r ( \alpha_{r^{-1}}f(r r^{-1}s))= \iota_A(\alpha_r(a)) f(s),
	\end{align*}
	for $f \in \cc(G,a,\alpha)$, $a \in A$ and $s,r \in G$.
\end{proof}
We now embark on our final task in our construction and picture painting of the general crossed product $A \rtimes_ \alpha G$, namely we wish to properly characterize its representation theory. We are going to see that $A \rtimes_\alpha G$ is the $C^*$-algebra whose representation theory is in a one-to-one correspondance with the covariant representations of $(A,G,\alpha)$ via the map $(\pi,u) \mapsto \pi \rtimes u$. To see this, we note first that as one usually does when dealing with representations of $C^*$-algebras and groups, we decompose them into subrepresentations:
\begin{corollary}
	Let $G \acts \alpha A$ be a $C^*$-dynamical system and let $a \in A$, $f,h \in \cc(G,A)$ and $\varphi \in \cc(G)$. Then
\begin{align*}
	\iota_A(a) \iota_G(\varphi) &= \varphi \otimes a,\\
	\int_G \iota_A(f(s)) \iota_G(s)h  \d s &= f \ast h \text{ and }\\
	\int_G \iota_A(f(s)) \iota_G(s) \d s & = f,
\end{align*}
where we've identified $\cc(G,A,\alpha) \subseteq A \rtimes_\alpha G \subseteq M(A \rtimes_\alpha G)$.
	\label{cross:iotaresults}
\end{corollary}
\begin{proof}
	Let $(\pi, u)$ be a non-degenerate covariant representation of $G \acts \alpha A$ and notice that
	\begin{align*}
		\overline{\pi \rtimes u}\left( \int_G \iota_a(f(s)) \iota_G(s) \d s  \right) = \pi \rtimes u (f),
	\end{align*}
	which combined with \cref{cross:essnorm} shows the last and first identity. The second inequality follows from the last and  \cref{int:multstrictintegral}.
\end{proof}
\todo{swap around}
Before we finalize the description of the representation theory of $A \rtimes_\alpha G$, we need to extend the map $\iota_G$ to $C^*(G)$, the full group $C^*$-algebra of $G$, since this will provide us with a useful characterization of a dense subset of $A \rtimes_\alpha G$, so without further ado:
\begin{lemma}
	There exists a group homomorphsim $\tilde \iota_G \colon C^*(G) \to M(A \rtimes_\alpha G)$ such that
	\begin{align*}
		\tilde \iota_G(z) = \int_G z(s) \iota_G(s) \d s, 
	\end{align*}
	for all $z \in \cc(G)$
	\label{cross:2.35}
\end{lemma}
\begin{proof}
	For all $z \in \cc(G)$, the map $s \mapsto z(s) \iota_G(s)$ is strictly continuous, and we may define the integral by \Cref{int:multstrictintegral}. Let $z,w \in \cc(G)$, then
	\begin{align*}
		\tilde \iota_G(z \ast w) &= \int_G \int_G z(r)w(r^{-1}s) \d r \iota_G(s) \d s \\
		&= \int_G \int_G z(r) w(r^{-1}s) \iota_G(s) \d s \d r\\
		&= \int_G \int_G z(r) w(s) \iota_G(rs) \d s \d r\\
		&= \int_G z(r) \iota_G(r) \d r \int_G w(s) \iota_G(s) \d s \\
		&= \tilde \iota_G(z) \tilde \iota_G(w),
	\end{align*}
	so if it is well-defined and continuous on $C^*(G)$, it will be a homomorphism. Let $\pi$ be any faithful and non-degenerate representation of $A \rtimes_\alpha G$, so that its extension $\overline \pi $ to $M(A \rtimes_ \alpha G)$ is faithful aswell. Consider the composition $s \mapsto \overline{\pi} \circ \iota_G(s)$, which is a unitary valued homomorphism. It is in fact strongly continuous: Suppose that $f \in \cc(G,A)$ and $\xi \in \H_\pi$, then for $s \in G$ we see that $\overline{ \pi} (\iota_G(s)) \pi(f) \xi = \pi(\iota_G(s) f) \xi$, and it follows that $s \mapsto \overline{\pi}(\tilde \iota_G(s)) \xi$ is continuous for all $\xi \in \H$ by continuity of $s \mapsto \iota_G(s)f$ and non-degeneracy of $\pi$. Thus
	\begin{align}
		\lv \tilde \iota_G(z)\rv \leq \lv z \rv,
	\end{align}
	for all $z \in \cc(G)$, so it extends to a unitary-valued homomorphism $C^*(G) \to M(A \rtimes_\alpha G)$, which we will also denote by $\iota_G$
\end{proof}
\begin{remark}
In particular, by the above and \cref{int:indlmdense} we see that the set of functions of the form $\iota_A(a) \iota_G(z)$ for $a \in A$ and $z \in \cc(G)$ span a dense subset of $A \rtimes_\alpha G$. This will be useful in the following:
\end{remark}
\begin{theorem}
	Let $G \acts \alpha A$ be a $C ^*$-dynamical system and let $\H$ be a Hilbert space. If $L \colon A \rtimes_\alpha G \to \mathbb{B}(\H)$ is a non-degenerate $*$-representation, then there exists a non-degenerate covariant representation $(\pi,u)$ of $G \acts \alpha A$ into $ \mathbb{B}(\H)$ such that $L= \pi \rtimes u$ which can be realised by 
	\begin{align*}
		u_s = \overline L(\iota_G(s)) \text{ and } \pi(a) = \overline L(\iota_A(a)),
	\end{align*}
	for all $s \in G$ and $a \in A$. In fact, there is a bijective correspondance between covariant representations of $G \acts \alpha A$ and $*$-representations of $A \rtimes_\alpha G$.
\end{theorem}
\begin{proof}
	We already have seen the assignment $(\pi,u) \mapsto \pi \rtimes u$. For the converse, suppose that $L$ is a non-degenerate representation of $A \rtimes_\alpha G$. Let $\pi, u$ be defined by the above. It is clear that $\pi$ is a $*$-representation of $A$ on $\mathbb{B}(\H)$ and similarly that $u$ is a strongly continuous unitary-valued group representation on $\mathbb{B}(\H)$, since $\iota_G$ is strictly and hence strongly continuous. To see that $(\pi,u)$ is non-degenerate, let $(e_i)$ be an approximate unit for $A$. Then $\iota_A(e_i) \to 1$ strictly in $M(A \rtimes_\alpha G)$, and so $\pi = \overline L(\iota_A(e_i)) \to I$ strictly, so in particular $\pi(e_i) \xi \to \xi$ for all $\xi \in \H$ and $\pi$ is non-degenerate. Now, let $s \in G$ and $a \in A$, then
	\begin{align*}
		u_s \pi(a) u_s^* = \overline L(\iota_G(s) ) \overline L (\iota_A(a)) \overline L(\iota_G(s))^* = \overline L( \iota_A(\alpha_s(a))) = \pi(\alpha_{s}(a)),
	\end{align*}
	so the pair is a proper non-degenerate covariant representation of $G \acts \alpha A$. Finally, using \cref{int:multstrictintegral}, we see that
\begin{align*}
	\pi \rtimes u(\iota_A(a) \iota_G(z)) &= \pi(a) \circ \overline{\pi \rtimes u}(\iota_G(z)) = \overline L(\iota_A(a)) \int_G z(s) \overline L(\iota_G(s)) \d s \\
	&= \overline L(\iota_A(a)) \overline L(\iota_G(z)) \\
	&= \overline L(\iota_A(a) \iota_G(z)),
\end{align*}
for all $z \in \cc(G)$ and $a \in A$, which by the above remark ensures that $L = \pi \rtimes u$ on all of $A \rtimes_ \alpha G$.
\end{proof}

\ssection{Properties of Crossed products}
It is well-known that for abelian, discrete and second countable groups $G$, the crossed product coincides with the reduced crossed product.
