\chapter{Integral Forms And Representation Theory}
Suppose that you are given a discrete $C^*$-dynamical system $(A,G,\alpha)$, here descrete menas that $G$ is discrete. It is not hard to extend representations of $G$ and $A$ to the group ring $\C_C(G,A)$, and thus create new $C^*$-algebras through completion of it. However, when $G$ is not discrete, we run into issues since we can not assume that a unitary representation of $G$ is continuous in norm, which is required by classic constructions seen e.g., in \cite{brown2008c}. We will in this chapter develop the necessary theory to define the non-discrete analogues of these techniques.

We assume familiarity with basic theory of analysis in topological groups, basic representation theory of topological groups and Fourier analysis thereon and will not go lengths to explain results from this field, for a reference, we recommend the excellent books on the subject: \cite{folland2016fourier}, \cite{berg1984harmonic} and \cite{folland2013real}.


