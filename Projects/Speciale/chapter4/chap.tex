\chapter{Actions And Automorphisms}
In this chapter, we will go through theory describing automorphisms on a $C^*$-algebra $A$ and actions of groups $G$ on $C^*$-algebras. The goal of this chapter is to characterize the ideal structure of crossed products with discrete groups through properties of the associated action. We will cover a theorem of Archbold and Spielberg (see \cite{archbold1994topologically}) stating that the reduced crossed product $A \rtimes_{\alpha,r} G$ is simple whenever the action of $G$ is topologically free, which is something we will describe later on. After that we will cover some important consequences of this result, including an important result in the classification of group $C^*$-algebras.
\sssection{Action Properties}
We are going to discuss the concept of topologically free actions $G \acts \alpha A$ and its implications on the crossed product. In order to do this, we introduce notation and discuss some results on the spectrum of a general $C^*$-algebra as well as the primitive ideal space of $A$:
\begin{definition}
	For a $C^*$-algebra $A$, we define the \myemph{primitive ideal space} of a $C^*$-algebra $A$ to be the set
	\begin{align*}
		\Prim A := \left\{ \ker \pi \mid \pi \text{ is an irreducible representation of } A \right\}.
	\end{align*}
	Furthermore, we define the \myemph{spectrum of a $C^*$-algebra} $A$ to be the set
	\begin{align*}
		\hat A := \{[\pi]_u \mid \pi \text{ is an irreducible representation of } A \}.
	\end{align*}
where $[\cdot]_u$ denotes the class of unitarily equivalent representations.

If $\pi_1 \sim_u \pi_2$, then $\ker \pi_1 = \ker \pi_2$ so the map $\kappa \colon \hat A \to \Prim A$, $[\pi]_u \mapsto \ker \pi$, is surjective and well-defined. We endow $\Prim A$ with a topology called the \myemph{hull-kernel topology} defined such that for $S = \{s_i\}_{i \in I} \subseteq \Prim A$ we have
\begin{align*}
	\overline S := \{ p \in \Prim A \mid \cap_{i \in I} s_i \subseteq p\}.
\end{align*}
To see that this is in fact a topology, we refer the reader to \cite[Appendix A.2]{williamsmorita} for the details. We also endow $\hat A$ with the corresponding initial topology stemming from the surjective map $\kappa$.
\end{definition}
\begin{remark}
	For a commutative $C^*$-algebra (i.e., $A = C_0(X)$), the spectrum of $A$ is the Gelfand Space of $A$, i.e., non-zero $*$-homomorphisms $A \to \C$, since irreducible representations of commutative spaces are one-dimensional. Moreover, in this case it holds that $\hat A \cong \Prim A$.
\end{remark}
These objects interact nicely with dynamical systems (we already saw this for $A= C_0(X)$): Whenever $G \acts \alpha A$ then we may define the \myemph{transposed action of $G$} on both $\hat A$ and $\Prim A$ (as topological spaces) by
\begin{align*}
	\alpha_r^*([\pi]_u) := [\pi \circ \alpha_{r^{-1}}] \text{ and } \alpha_{r}^*(\ker \pi) = \ker \pi \circ \alpha_{r^{-1}},
\end{align*}
for all irreducible representations $\pi$ of $A$. These are indeed actions on topological spaces, i.e., the corresponding maps $G \times \hat A \to \hat A$ and $G \times \Prim A \to \Prim A$ are jointly continuous, see e.g., \cite[Lemma 7.1]{williamsmorita}.
\begin{definition}
	For an action $G \acts \alpha A$ we say that the action is \myemph{topologically free} if for all $t_1,\dots,t_n \in G\backslash\{e\}$ the set $\bigcap_{i=1}^n \left\{ [\pi]_u \in \hat A | \alpha_{t_i}^*([\pi]_u)\neq [\pi]_u \right\}$ is dense in $\hat A$.
\end{definition}
\begin{remark}
	If $A=C_0(X)$, then the action $G \acts \alpha A$ if and only if for all $e \neq t \in G$, the set of fixed points of the induced action $G \acts{} X$ have empty interior or equivalently dense complement. Moreover, define 
	\begin{align*}
		G_x^\circ := \{t \in G \ \mid \ t|_U = \mathrm{id} \text{ for some neighborhood } U \text{ of } x\},
	\end{align*}
	which is clearly a normal subgroup of $G$, then the action is topologically free if and only $G_x^\circ = \{e\}$ for all $x$: If $t.y= y$ for all $y \in U$ of some neighborhood $U$ of $x$, then the fixpoint set of $t$ has non-empty interior.
\end{remark}
\begin{definition}
	We say that an action $G \acts {} X$ is free if every stabilizer is trivial, i.e., if for each $x \in X$ we have $G_x = \{e\}$.
\end{definition}
\begin{remark}
	In \cite{olesenpedersen3}, Olesen and Pedersen characterized topological freeness of actions $\Z \acts {\alpha} A$, $\alpha \in \mathrm{Aut}(A)$, in terms of the Connes spectrum (and many other characterizations). They showed the equivalence of the following:
	\begin{enumerate}[nosep]
		\item For each $n \neq 0$, the set $\{ [\pi]_u \in \hat A \ \mid \ [\pi \circ \alpha^{n}]_u = [\pi]_u\}$ is has empty interior,
		\item $\hat \Z(\alpha) = \hat \Z = S^1$.
	\end{enumerate}
	The result is \cite[Theorem 10.4]{olesenpedersen3}, and the proof goes beyond the scope of this thesis. However, we include it to highlight that the theory developed in the previous chapter served as a stepping stone towards newer results which we will cover in this chapter.
\end{remark}
\begin{lemma}
	Suppose that $G \acts {} X$ is a minimal action. If $G_x^\circ = \{e\}$ for some $x \in X$, then $G_y^\circ = \{e\}$ for all $y \in X$, i.e., the action is topologically free.
	\label{topfreeminimal}
\end{lemma}
\begin{proof}
	Let $x \in X$ be such that $G_x^\circ = \{e\}$. Let $y \in X$ and suppose towards a contradiction that $s \in G_y^\circ\{e\}$. Let $U \subseteq X$ witness this, i.e, be open with $y \in U$ and $s|_U=\mathrm{id}$. By minimality of the action, there is $t \in G$ such that $x \in tU$. Then $tst^{-1}.(t.v) =  ts.v = t.v$ for all $v \in U$, so $tst^{-1}|_{tU}=  \mathrm{id}$. By assumption, this implies that $tst^{-1} = e$, which forces $s = e$ (if $t = e$, then $s \in G_x^\circ = \{e\}$ and if $t \neq e$ then $s = e$), contradicting the assumption that $s \neq e$. 
\end{proof}
\begin{corollary}
	The action of $G$ on $\dd_F G$ (the Furstenberg boundary) is topologically free if and only if it is free.
\end{corollary}
\begin{proof}
	This follows directly from the earlier remark that $\dd_F G$ is extremally disconnected, for $(\dd_F G)_s$ is clopen by a theorem of Frolik, see e.g., \cite[Theorem 3.1]{frolik1971maps}, which states that the set of fixed points of a homeomorphism $f \colon X \to X$, where $X$ is a compact extremally disconnected space, is clopen in $X$.
\end{proof}
\begin{definition}
	For a $C^*$-algebra $A$, we say that $\alpha \in \mathrm{Aut}(A)$ is \myemph{properly outer} if for all $\alpha$-invariant non-zero ideals $I \subseteq A$ and all inner automorphisms $\beta$ of I it holds that
	\begin{align*}
		\lv \alpha|_I - \beta \rv = 2.
	\end{align*}
\end{definition}
Elliot and Olesen + Pedersen showed in \cite[Theorem 3.2]{elliott1980some} and \cite[Theorem 7.2]{olesenpedersen3}, respectively, that for discrete groups $G$ with $G \acts \alpha A$ minimal, a sufficient condition for $C^*$-simplicity of $A \rtimes_{\alpha,r} G$ for a large class of $C^*$-algebras was that $\alpha_t$ be properly outer for $t \neq e$. Archbold and Spielberg later improved on this result in \cite{archbold1994topologically}. We will go through their result and some recent applications.
\begin{proposition}[\mbox{\cite[Proposition 1]{archbold1994topologically}}]
	Let $(A,G,\alpha)$ be a $C^*$-dynamical system. If $\alpha$ is topologically free, then $\alpha_t$ is properly outer for all $t \neq e$.
\end{proposition}
\begin{remark}
	The proof is based on theory of derivations of $C^*$-algebras, which we will not go into detail with. Shortly put, a derivation of a $C^*$-algebra $A$ is a linear map on $A$ which behaves similar to derivation of differentiable functions. For an exhibition on this, see e.g., \cite[Chapter 8.6]{pedersenalgauto} (or the books they refer to, \cite{sakai1978recent} and \cite{bratteli2012operator}, by Sakai and Bratteli et al., respectively.) 

	The proof is based on a result of Kadison-Ringrose, which characterizes proper outerness with not never locally being the composition of an inner automorphism composed with the exponential of a derivation, see \cite{kadison1967derivations} or \cite[Theorem 6.6 (i) and (ii)]{olesenpedersen3}.
\end{remark}
\begin{definition}
	Let $A$ denote an arbitrary $C^*$-algebra with two representations $\pi$ and $\tau$ on Hilbert spaces $\H_\pi$ and $\H_\tau$. We say that $\pi$ and $\tau$ are \myemph{disjoint representations} if there is no subrepresentations of one which is unitarily equivalent to a non-zero subrepresentation of the other.
\end{definition}
\begin{remark}
	The above definition is the (to the author's knowledge) original definition. We will use the following fact: If $\pi$ and $\tau$ are disjoint if and only if $1 \oplus 0$ is an element of the strong operator closure of $(\pi \oplus \tau) (A)$, i.e., there is some bounded net $(x_i) \subseteq A$ with $\pi(x_i) \to 1$ and $\tau(x_i) \to 0$ in their strong operator topology, and this happens if and only if $\pi \not \sim _u \tau$. See e.g. \cite[Proposition 2.1.4]{arveson2012invitation} for a proof. 
\end{remark}
\sssection{Consequences of Topological Freeness}
Recall that we may identify $G$ as a subset of the unitary group of $M(A \rtimes_\alpha G)$. We denote this identification by $u_t$, $t \in G$. 
\begin{lemma}
	Let $G$ be discrete and $G \acts \alpha A$ and let $\pi$ be any non-degenerate representation of $A$. If there is $t \in G$ such that $\pi \circ \alpha_t^{-1}$ and $\pi$ are disjoint representations, then for all completely positive extension $\varphi$ of $\pi$ to $A \rtimes_\alpha G$ of norm $1$ it holds that $\varphi(a u_t) = 0$ for all $a \in A$ and $u_t \in G \subseteq M(A \rtimes_\alpha G)$.
	\label{ASlem1}
\end{lemma}
\begin{proof}
	Let $(x_i) \subseteq A$ be a bounded net such that $\pi  \circ \alpha_{t}^{-1}(x_i) \to 1$ in the strong operator topology and $\pi(x_i) \to 0$. Let $\varphi$ be a completely positive extension of $\pi$ to $A  \rtimes_\alpha G$ of norm one. Then, for $a \in A$ and $t \in G$:
	\begin{align*}
		\varphi(a u_t) = \lim_{\text{SOT}} \varphi(x_i )\varphi(a u_t) = \lim_{\text{SOT}} \varphi(u_t \alpha_{t}^{-1}(x_i a )) = \lim_{\text{SOT}} \varphi(u_t) \varphi(\alpha_t^{-1}(x_i))\varphi(\alpha_{t}^{-1}(a)) = 0,
	\end{align*}
	as $A \subseteq \mathrm{mult}(\varphi)$. \todo{Move to prelim} see \cite[Proposition 1.5.7 p.t (2)]{brown2008c}.
\end{proof}
With this, we obtain the following:
\begin{theorem}
	Let $(A,G,\alpha)$ be a $C^*$-dynamical system, $G$ discrete. If $\alpha$ is topologically free, then $A$ detects ideals in $A \rtimes_{\alpha}G$ modulo the kernel of the canonical surjection $A \rtimes_\alpha G \to A \rtimes_{\alpha,r}G$. 
	\label{ASthm1}
\end{theorem}
\begin{proof}
	\todo{Include in prelim}
	Let $I_r$ denote the kernel of the canonical surjection $ \pi_r \colon A \rtimes_\alpha G \to A \rtimes_{\alpha , r}G$ and let $E$ denote the canonical faithful conditional expectation (see e.g., \cite[Proposition 4.1.9]{brown2008c} \todo{include in Prelim}) given by
	\begin{align*}
	E_r \colon A \rtimes_{\alpha,r} G &\to A, \\ 
		c_c(G,A) \ni \sum_{g \in G} a_g  u_g &\mapsto a_e \in A,
	\end{align*}
	and let $E = E_r \circ \pi_r$ denote the canonical conditional expectation $A \rtimes_{\alpha }G \to A$.
	
	Suppose that $I$ is an ideal of $A \rtimes_\alpha G$ with $A \cap I=0$ and $I \not \subseteq I_r$. Since $E_r$ is faithful, let $a \in I_+$ with $E_r(a) \neq 0$. By density of $\cc(G,A)$ in $A \rtimes_\alpha G$, pick $b = \sum_{g \in F} b_g u_g$ for some finite set $F \subseteq G$ with
	\begin{align*}
		\lv a - b \rv < \frac{\lv E (a) \rv}{2}.
	\end{align*}
	By assumption, the set $P := \bigcap_{e \neq t \in F}\left\{ p \in \hat A \mid \alpha_{t}^*(p) \neq p \right\}$ is dense in $\hat A$. Let $[\pi] \in P$. Let $\pi'$ denote the composition
	\begin{equation*}
		\begin{tikzcd}
			A + I \ar[r, "q"] & (A+I)/I \ar[r, "\cong"] & A \ar[r, "\pi"]  & \mathbb{B}(\H_\pi),
		\end{tikzcd}
	\end{equation*}
	where $q$ is the quotient map. We are going to extend $\pi'$ to a completely positive map on $A \rtimes_\alpha G$ in the following way:	For $x \in A \rtimes_\alpha G$, let $\tau(x) := P _e(\tilde \pi \rtimes \lambda(x)) P_e$, where $P_e$ is the minimal projection $\H_\pi \otimes \ell^2(G) \to \H_\pi \otimes \C \delta_{e}$, then $\tau \colon A \rtimes_\alpha G$ is a completely positive map extending $\pi'$ (after identifying $\H_\pi$ with $\H_\pi \otimes \C \delta_e$) to all of $A \rtimes_\alpha G$; if $a \in A$ and $s,t \in G$ then
	\begin{align*}
		\tau(au_s) \xi \otimes \delta_t = \begin{cases}
			\pi(a) \xi \otimes \delta_e & \text{ if } s = t = e,\\
			0 & \text{ else}
		\end{cases}
	\end{align*}
	for all $\xi \in \H_\pi$. If $x \in I$, then $P_e \tilde \pi \rtimes_r \lambda (x) P_e = 0$ as $A \cap I = 0$. In particular, $\tau$ is completely positive and $\lv \tau \rv = \lim \lv \tau(e_i)\rv = 1\lv \pi \rv = 1$, where $e_i$ is an approximate identity of $A$. 
	
	For $t \in F\backslash\{e\}$ the representations $\pi$ and $\pi \circ \alpha_{t}^{-1}$ are disjoint, since $[\pi] \in P$. By \cref{ASlem1} we then see that
	\begin{align*}
		\tau(b) = \sum_{t \in F} \tau(b_t u_t) = \tau(b_e) = \tau(E(b)).
	\end{align*}
	Since $\tau(I) = 0$, we have
	\begin{align*}
		\lv \pi(E(b))\rv = \lv \tau(b) \rv = \lv \tau(b-a)\rv \leq \lv b-a\rv.
	\end{align*}
	As $P$ is dense in $\hat A$, we see that $\lv E(b)\rv \leq \lv b-a\rv$ yielding the contradiction
	\begin{align*}
		\lv E(a) \rv \leq \lv E(a-b)\rv + \lv E(b)\rv \leq 2 \lv a-b\rv <  \lv E(a)\rv,
	\end{align*}
	so $I \subseteq I_r$.
\end{proof}
\begin{corollary}
	If $G$ is amenable, then $A$ detects ideals in $A \rtimes_\alpha G$ whenever $\alpha$ is topologically free.
\end{corollary}

If we further assume that $A$ is abelian, then the converse holds as well:
\begin{theorem}
	Let $X$ be a locally compact Hausdorff space. Given a $C^*$-dynamical system $(C_0(X),G,\alpha)$ with $G$ discrete, if $C_0(X)$ detect ideals in $C_0(X) \rtimes_{\alpha}G$ modulo $I_r$, then the action is topologically free.
	\label{ASthm2}
\end{theorem}
\begin{proof}
	Let $x \in X$, and consider the Hilbert space $\ell^2(G.x)$. We define the covariant pair $(\pi_x,u)$ for $(C_0(X),G,\alpha)$ by
	\begin{align*}
		\pi_x(f) \delta_{t.x} = f(t.x) \delta_{t.x} \text{ and } u_s \delta_{t.x} = \delta_{st.x},
	\end{align*}
	for $f \in C_0(X)$ and $t,s \in G$. That they are representations is easy to verify, we show covariance instead:
	\begin{align*}
		u_s \pi_x(f) u_{s^{-1}} \delta_{t.x} = u_{s} f(s^{-1}t.x)\delta_{s^{-1}t.x} = f(s^{-1}t.x) \delta_{t.x} = s.f(t.x)\delta_{t.x} = \pi_x(s.f)\delta_{t.x},
	\end{align*}
	so that we obtain a representation $\pi_x \rtimes u_x$ of $C_0(X) \rtimes_\alpha G$ on $\ell^2(G.x)$. Let $I = \bigcap_{x \in  X} \ker \pi_{x}$. If $f \in I$, then $f(t.x) = 0$ for all $t \in G$ (in particular for $t=e$) and $x \in  X$, so $f = 0$. Hence $I \cap C_0(X) = 0$.
	
	Suppose that $G \acts \alpha C_0(X)$ is not topologically free. Then there is $e \neq s \in G$ such that $V_s=\left\{ y \in X  \mid s.y \neq y \right\}$ is not dense in $X$. Let $0 \neq f \in C_0(X)$ with $\supp f \subseteq \overline{V_s}^c$. Let $t \in G$ and note the following:
	\begin{itemize}[nosep]
	\item If $t.x \in \supp f$ then $st.x = t.x$ and we see that
	\begin{align*}
		\pi_x\rtimes u_x(f-fu_s)\delta_{tx} = f(t.x) \delta_{t.x} - f(st.x)\delta_{st.x}= 0.
	\end{align*}
	\item If $st.x  \in \supp f$, then $t.x = s^{-1}st.x = st.x$ as well.
	\item If $t.x \not \in \supp f$ then $st.x \not \in \supp f$ and we see that
	\begin{align*}
		\pi_x\rtimes u_x(f-fu_s)\delta_{tx} = f(t.x) \delta_{t.x} - f(st.x)\delta_{st.x}= 0.
	\end{align*}
\end{itemize}
	 Thus $f-f u_s \in I \subseteq I_r$ as $x \in X$ was arbitrary. Since $E(f u_s) = E_r(\pi_r(fu_s)) =0$, we see that 
	\begin{align*}
		0 \neq f = E(f-fu_s) = E_r (\pi(f-fu_s))= E_r(0) =0,
	\end{align*}
	a contradiction. We conclude that the action of $C_0(X)$ is topologically free.
\end{proof}
\begin{corollary}
	Suppose that $G \acts \alpha C_0(X)$ with $G$ discrete. Then $C_0(X) \rtimes_\alpha G$ is simple if and only if the action is minimal, topologically free and $I_r = 0$.
\end{corollary}
\begin{proof}
	This follows directly from \cref{ASthm1}, \cref{ASthm2} and \cref{ASlem1}.
\end{proof}
Once again, this gives an easy proof of simplicity of $C(S^1) \rtimes_{\theta} \Z$ for $\theta$ irrational, since the action is very clearly topologically free (there are no fixed points on $S^1$ for irrational rotation), and we already know that it is minimal and $I_r = 0$.

\sssection{A Relation between Ideals of $C_r^*(G)$ and $A \rtimes_{\alpha,r} G$ and a theorem of four}
From here on, let $G$ denote a discrete countable group and $X$ a compact $G$-space. In the following, we will discuss recent developments regarding the classification of $C^*$-simple groups. In particular results based on theory of boundary actions and topological freeness from earlier will be reviewd and their relations to simplicity of associated crossed products $A \rtimes_{\alpha,r} G$. 

We say that a group $G$ is \myemph{$C^*$-simple} if its reduced group is $C^*$-simple, and we say that $G$ has the \myemph{unique trace property} if the canonical faithful tracial state $\tau_0 \colon f \mapsto \langle f\delta_e  , \delta_e\rangle_{\ell^2(G)}$ is the only tracial state on $C_r^*(G)$. It was an open question for a long time wether having the unique tracial state was equivalent to being $C^*$-simple. In 2014, using theory of boundary actions, this was shown by to false by Le Boudec (see \cite[Thoerem A]{le2017c}). However, $C^*$-simplicity does imply unique tracial state which is equivalent to having no non-trivial amenable normal subgroups (see e.g. the author's project on this subject: \cite[Chapter 5]{bscp}). The breakthrough which started this development was due to Kennedy and Kalantar (see \cite{kalantar2017boundaries}).

It is quite a strong condition on a group $G$ to be $C^*$-simple, for example discrete amenable (in particular abelian) groups are not $C^*$-simple. Groups which are $C^*$-simple include the class of Powers groups (an homage to R.T. Powers' proof on simplicity of the free group on two generators, see \cite{powers1975simplicity} or \cite[chapter 3]{bscp}) which includes the free groups on finite number of generators.

In order to show our desired results, we need to cover a bit of theory on conditional expectations and crossed products. For instance whenever $N$ is a normal subgroup of a discrete group $G$, then the inclusion $N \subseteq G$ gives rise to a conditional expecation $E_{N}^G \colon C_r^*(G) \to C_r^*(N)$ given by $\lambda_s \mapsto 1_{N}(s) \lambda_s$, see e.g. \cite[Corollary 2.5.12]{brown2008c}. Moreover, supposing that $A = C(X)$, then for all $x \in X$, the associated conditional expectation $E_{G_x}^G \colon C_r^* (G) \to C_r^*(G_x)$ extends to a conditional expecation
\begin{align}
	E_x \colon C(X) \rtimes_{\alpha,r} G &\to C_r^*(G_x), \\
	f \lambda_s &\mapsto f(x) E_{G_x}^G(\lambda_s).
	\label{eq:stabcond}
\end{align}

The following result shows the close relationship between topological freeness of actions and $C^*$-simplicty of groups with the ideal structure of crossed products of $G$-boundaries. It is (to the author's knowledge) due to \cite{breuillard2017c}, however we follow a proof of Ozawa (see e.g., \cite{ozawa2014lecture}). The result is heavily inspired by the proof of the Archbold-Spielberg result covered earlier.
\begin{proposition}
	Let $G \acts {} X$ be a minimal action on a compact Hausdorff space $X$. If the stabilizer group $G_x$ is $C^*$-simple for some $x \in X$, then the corresponding reduced crossed product $C(X) \rtimes_{r} G$ is simple.
	\label{ozawa15pt1}
\end{proposition}
\begin{proof}
	Suppose that $I \subseteq C(X) \rtimes_r G$ is a non-trivial closed ideal and let $x \in X$. Let $E$ denote the canoncial faithful conditional expectation $C(X) \rtimes_r G \to C(X)$. Then $E(I) \neq \{0\}$ as $E$ is faithful which implies that $E(I)$ is dense in $C(X)$ by minimality of $G \acts{}X$, as $\overline{E(I)}$ is a non-empty $G$-invariant ideal of $C(X)$ so $\overline{E(I)} = C(X)$.

Let $\tau_0$ denote the canoncial faithful tracial state on $C_r^*(G)$ let $f \in C(X)$ with $f(x) \neq 0$. Then 
\begin{align*}
	\tau_0 (E_x(f \lambda_e))= f(x) = \delta_x(E(f\lambda_e)),
\end{align*}
so $E_x(I) \neq \{0\}$ in $C_r^*(G_x)$. 

We claim that $E_x(I)$ is not dense in $C_r^*(G_x)$, and similar to the proof of \cref{ASthm1}, we consider state on $C(X) + I$ given by the composition:
	\begin{align*}
		C(X) + I \to (C(X)+I)/I \cong C(X) \stackrel {\delta_x} \to \C,
	\end{align*}
	where we used that the proper ideal $C(X) \cap I$ of $C(X)$ must be $0$ by minimality of $G \acts {} X$. Let $\varphi_x$ denote any state extending the above composition with $\varphi_x(I) =0$.
	
	The result will follow once we show that $\varphi_x = \varphi_x \circ E_x$, for then $E_x(I)$ can not be dense: If $\varphi_x = \varphi_x \circ E_x$ and $E_x(I)$ is dense in $C_r^*(G_x)$, then 
	\begin{align*}
		\varphi_x(C(X) \rtimes_r G) = \varphi_x (E_x(C(X) \rtimes_r G)) = \varphi_x(C^*_r(G_x))=\{0\},
	\end{align*}
a contradiction. To see that $\varphi_x = \varphi_x \circ E_x$ it suffices to show that $\varphi_x (\lambda_s) = 0$ for $s \not \in G_x$: If $s \in G_x$, then as $C(X)$ is in the multiplicative domain of $\varphi_x$, it holds for each $f \in C(X)$ that
	\begin{align*}
		\varphi_x(f \lambda_s) =  \varphi_x( f(x) \lambda_s)  =  \varphi_x (E_x( f\lambda_s)).
	\end{align*}
	So assume that $s \not \in G_x$, i.e., that $s.x \neq x$. Using Uryhson's lemma we pick a function $h \in C(X)$ with $h(x) = 1$ and $h(s^{-1}.x) = 0$. Then
	\begin{align*}
		\varphi_x(\lambda_s) =\varphi_x(h)\varphi_x(\lambda_s) \varphi_x(h)= \varphi_x(h\lambda_sh) = \varphi_x(h (s.h)\lambda_s)=f(x)f(s^{-1}.x) \varphi_x(\lambda_s) = 0,
	\end{align*}
	finishing the proof.
\end{proof}
\begin{remark}
	Given a group $G$, if $N$ is an amenable normal subgroup of $G$, then the quotient map $\C( G) \to \C \left(G/N\right)$ extends to a $*$-homomorphism $\pi_N \colon C_r^*(G) \to C_r^*\left( G/N \right)$ such that
\begin{equation}
	\begin{tikzcd}
		\C G \ar[r,"q"] \ar[d,"\lambda_G"'] & \C\left( G/N \right) \ar[d, "\lambda_{G/N}"]\\
		C_r^*(G) \ar[r,"\pi_N"'] & C_r^*\left( G/N \right)
	\end{tikzcd}
\end{equation}
is commutative with the vertical maps being the left-regular representations, see e.g. \cite[Proposition 3]{de2007simplicity} for a proof. In general, the associated representation of $G$ as operators on $\ell^2(G/N)$ given by composition of the above is often refered to as the \myemph{quasi-regular representation}. We will use this result in a moment.
\label{quotientextend}
\end{remark}
We have seen that topological freeness implies simplicity of $A \rtimes_{\alpha,r} G$ for $G$-simple $C^*$-algebras. We now prepare to cover a result from the same paper as above allows us to determine topological freeness from properties of $G$ and $C(X)\rtimes_\alpha G$, but first we need a lemma which will allow us to describe a useful representation:
\begin{lemma}
	Suppose that $X$ is a minimal $G$-space such that for some $x \in X$ the normal subgroup $G_x^\circ$ is amenable. Then there is a representation $\Pi_x$ of $C(X) \rtimes_r G$ on $\ell^2(G/G_x^\circ)$ such that
	\begin{align*}
		\Pi_x(f)\delta_{[t]} = f(t.x)\delta_{[t]} \text{  and  } \Pi_x(\lambda_s)\delta_{[t]} = \delta_{[st]},
	\end{align*}
	for $s,t \in G$ and $f \in C(X)$.
	\label{CXGrep}
\end{lemma}
\begin{proof}
	For the remainder of this proof, we will let $\lambda^x$ denote the representation of $G$ on $C_r^*(G/G_x^\circ)$ from \cref{quotientextend} and $\lambda$ denote the left-regular representation of $G$. Observe that we may define a $*$-homomorphism $\pi_x \colon C(X) \to \ell^\infty(G/G_x^\circ)$, given by $\pi_x(f)([t])=f(t.x)$: It is well-defined, for if $[s]=[t] \in G/G_x^\circ$, then 
	\begin{align*}
		\pi_x(f)([t])=f(t.x) = f(s.x) = \pi_x(f)([s]),
	\end{align*}
	for all $s,t \in G$ and $f \in C(X)$ as $s^{-1}t.x = x$ if and only if $s.x=t.x$. It is easy to verify that it is a $*$-homomorphism. In fact, it is injective, for if $\pi_x(f) = \pi_x(g)$, then $f(t.x)=g(t.x)$ for all $t \in G$, which by minimality implies that $f=g$.
	
	Redefine now $\pi_x$ to denote the associated representation of $C(X)$ as multiplication operators on $\ell^2(G/G_x^\circ)$ and note that for all $t,s \in G$ and $f \in C(X)$ we have
	\begin{align*}
		\lambda^x_t \pi_x (f) \lambda^x_{t^{-1}} \delta_{[s]} = f(t^{-1}s.x)\lambda^x_{t^{-1}} \delta_{[ts]} = f(t^{-1}s.x) \delta_{[s]} = t.f(s.x) \delta_{[t]} = \pi_x(t.f) \delta_{[s]},
	\end{align*}
	so that the pair $(\lambda^x, \pi_x, \ell^2(G/G_x^\circ))$ is a covariant pair for the $C^*$-dynamical system $G \acts {} C(X)$. By Fell's absorbtion principle (see \cite[Proposition 4.1.7]{brown2008c} or \cite[25]{approxtalk}), we may identify
	\begin{align*}
		C(X) \rtimes_r G =C^*\left(\{\lambda^x_t \otimes  \lambda_t \}_{t \in G} , \{\pi_x(f) \otimes 1 \}_{f \in C(X)}\right) \subseteq \mathbb{B}(\ell^2(G/G_x^\circ)) \otimes C_r^*(G),
	\end{align*}
	where $f$ is mapped to $\pi_x(f) \otimes 1$ and $\lambda_t$ is mapped to $\lambda^x_t \otimes \lambda_t$. Now, since $G$ is discrete and countable, $C_r^*(G) \cong C^*(\{\lambda_t\}_{t \in G})$, and by \cref{quotientextend} we obtain a $*$-homomorphism 
\begin{align*}
	 C^*(\{\lambda_t\}_{t \in G}) \to C^*(\{\lambda_t^x\}_{t \in G}), \quad \lambda_t\mapsto  \lambda_t^x,
\end{align*}
for all $t \in G$. Combining everything, we obtain a representation $\sigma$ of $C(X) \rtimes_r G$ on $\mathbb{B}(\ell^2(G/G_x^\circ)) \otimes \mathbb{B}(\ell^2(G/G_x^\circ))$ such that
\begin{align*}
	\sigma_x(f) = \pi_x(f) \otimes 1 \text{  and  } \sigma_x(f)(\lambda_t)=\lambda_t^x \otimes \lambda_t^x.	
\end{align*}
Let $\iota \colon \mathbb{B}(\ell^2(G/G_x^\circ)) \otimes \mathbb{B}(\ell^2(G/G_x^\circ)) \subseteq  \mathbb{B}(\ell^2(G/G_x^\circ) \otimes \ell^2(G/G_x^\circ))$ be the canonical unique embedding of $C^*$-algebras and let $u$ be the unitary operator embedding $\ell^2(G/G_x^\circ)$ in the diagonal space of $\ell^2(G/G_x^\circ) \otimes \ell^2(G/G_x^\circ)$, i.e., $u \delta_{[t]} = \delta_{[t]} \otimes \delta_{[t]}$.

Then the $^*$-homomorphism $\Pi_x \colon C(X) \rtimes_r G \to \mathbb{B}(\ell^2(G/G_x^\circ))$ given by $\Pi_x(a) := u^* \iota(\sigma_x(a)) u $ has the desired properties.
\end{proof}
\begin{proposition}
	Suppose that $X$ is a minimal $G$-space and $C(X) \rtimes_r G$ is simple. If there is some $x \in X$ such that the normal subgroup $G_x^\circ$ of $G$ is amenable, then the action $G \acts { } X$ is topologically free. 
	\label{ozawa15pt2}
\end{proposition}
\begin{proof}
	First, let $\Pi_x$ denote the representation of \cref{CXGrep} and suppose that $s \in G_x^\circ$ and let $U$ be a neighborhood of $x$ witnessing this, i.e, be such that $s|_U = \mathrm{id}$, and let $f \in C(X)\backslash\{0\}$ with $\supp f \subseteq U$. By the same argument as in the proof of \cref{ASthm2}, if $t.x \in \supp f$ then $st.x=t.x$ and if $t.x \not \in \supp f$ then $st.x \not \in \supp f$, so
	\begin{align*}
		\Pi_x\left(f ( 1-\lambda_s)) \right) \delta_{[t]} = f(t.x) \delta_{[t]} -f(st.x) \delta_{[st]}  = 0.
	\end{align*}
	Since we assumed simplicity of $C(X) \rtimes_r G$, this implies that $s = e$ and hence $G_x^\circ $ is trivial whence the minimality assumption finishes the proof.
\end{proof}
	In order to prove the main theorem of this section, we need two results which, arguably, is outside the scope of this thesis. We include them with reference to proofs.
\begin{lemma}[\mbox{\cite[Lemma 3.1]{hamana1985injective}}]
	Let $A$ and $B$ be unital $C^*$-algebras and $G$ a discrete group such that $G \acts \alpha A$ and $G \acts \beta B$. If $\psi \colon A \to B$ is a $G$-equivariant unital completely positive map, then there a $G$-equivariant unital completely positive map 
	\begin{align*}
		\tilde \psi \colon A \rtimes_{\alpha,r}  G &\to B \rtimes_{\alpha,r} G\\
		a \lambda_s \mapsto \psi(a) \lambda_s 
	\end{align*}
	which extends $\psi$. Moreover, $\psi$ is an isometry or surjection if and only if $\tilde \psi$ is and similarly $\psi$ is a $*$-homomorphism if and only if $\tilde \psi$ is.
	\label{hamanalift}
\end{lemma}
\begin{lemma}[\mbox{\cite[Theorem 6 + Corollary 7]{ozawa2014lecture}}]
	whenever $A$ is a unital $G$-$C^*$-algebra, then there is a $G$-equivariant unital completely positive map of $A$ into $C(\dd_F G)$ extending the map sending $1_A$ to $1_{C(\dd_F G)}$, which is $G$-equivariant as $G$ act by automorphisms on $A$ and $C(\dd_F G)$. 
	\label{1lifthamana}
\end{lemma}
With these results, we continue.
\begin{proposition}
	Let $G \acts {} A$ with $1 \in A$ and let $X$ be a $G$-boundary. If $I\subseteq A \rtimes_r G$ is a non-trivial ideal, then the ideal generated by $I$ in $(A \otimes C(X))\rtimes_r G$ is proper.
	\label{ozawprop17}
\end{proposition}
\begin{proof}
	Consider the embedding of $A \rtimes_r G$ in $(A \otimes C(X)) \rtimes_r G$,  $a \lambda_s \mapsto (a \otimes 1) \lambda_s$, induced by the embedding $A \subseteq A \otimes C(X)$, $a \mapsto a \otimes 1$. Let $\pi$ be a non-faithful representation of $A \rtimes_r G$ on $\mathbb{B}(\H)$ with kernel $I$, and let $\overline \pi$ denote a unital completely positive extension $(A \otimes C(X)) \rtimes_r G \to \mathbb{B}(\H)$. As $A \rtimes_r G$ is contained in the multiplicative domain of $\overline \pi$, the bimodule property implies that $\overline \pi(axb) = \pi(a) \overline \pi(x) \pi(b)$ for all $a,b \in A \rtimes_r G$ and $x \in (A \otimes C(X)) \rtimes_r G$, so that the image of $\pi(A)$ and $\overline \pi(C(X))$ commute, since
	\begin{align*}
		(a \otimes 1 )\lambda_s (1 \otimes f ) \lambda_e = (1 \otimes f) \lambda_e (a \otimes 1) \lambda_s,
	\end{align*}
	for all $ a \in A$, $f \in C(X)$ and $s \in G$ and the bimodule property for $\overline \pi$ of $A \rtimes_r G$ mentioned earlier. 

	Define an action of $G$ on $C^*(\overline \pi(C(X)))$ by inner automorphisms $s \mapsto \mathrm{Ad} \overline \pi(\lambda_s)$. By the virtue of \cref{1lifthamana} let $\psi$ denote any $G$-equivariant unital completely positive map from $C^*(\overline \pi (C(X))) \to C(\dd_F G)$. By a result of Furstenberg, see e.g. \cite[lemma 4.16]{bscp}, the composition $\psi \circ \overline \pi|_{C(X)}$ must be the unique $G$-equivariant injective $*$-homomorphism, i.e., the inclusion of $C(X)$ in $C(\dd_F G)$ induced by the quotient map $\dd_F G \to X$ stemming from the universal property of $\dd_F G$.

	In particular $\psi$ is then a $*$-homomorphism: The multiplicative domain of $\psi$ contains the image of $\overline \pi(C(X))$ and by linearity and continuity it must be all of $C^*(\overline \pi (C(X)))$. 

	We now consider the kernel $K := \ker \psi$, which is $G$-invariant, and the $C^*$-algebra
	\begin{align*}
		B := C^*(\overline \pi ( ( A \otimes C(X)) \rtimes_r G)) = \overline{ C^*(\overline \pi(C(X))) \cdot \pi(A \rtimes_r G)},
	\end{align*}
	where the second equality comes from the earlier remark that $A \rtimes_r G$ is contained in the multiplicative domain of $\overline \pi$. In particular, the ideal generated by $K$ in $B$ is
	\begin{align*}
		K' :=\overline{K \cdot \pi(A \rtimes_r G)}.
	\end{align*}
	Let $x \in B$ and let $(e_i)$ denote an approximate identity of $K'$. If $e_i x \to x$, then $x \in K'$, hence $x \in K'$ if and only if $e_i x\to x$ so $K' \cap C^*(\overline \pi(C(X))) = K$. 
	
	Let $q$ denote the quotient map $B \to B/K'$, and note that the composition $q \circ \overline \pi$ will be multiplicative when restricted to $A \otimes C(X)$ as $A \rtimes_r G$ is contained in the multiplicative domain of $\overline \pi$. Moreover, $q$ will be $G$-equivariant with respect to $\mathrm{Ad} \lambda_s$, $s \in G$. 

	We conclude that the ideal $\ker (q \circ \overline \pi)$ contains $I$ and is proper since the non-zero ideal $K$ of $C^*(\overline \pi(C(X)))$ is proper, hence the ideal generated by $I$ will be non-trivial and proper.
\end{proof}
\begin{definition}
	Let $B$ be a $G$-$C^*$-algebra and $K \subseteq B$ be a $G$-invariant ideal of $B$. We then define an ideal \myemph{$K \bar \rtimes_r G$} of $B \rtimes_r G$ by
	\begin{align*}
		K \bar \rtimes_r G := \{b \in B \rtimes_r G \ \mid \ E(b \lambda_s^* ) \in K \text{ for all } s \in G\},
	\end{align*}
	where $E$ is the canonical conditional expectation onto $B$. It clearly contains $K \rtimes_r G$ as an ideal.
\end{definition}
\begin{remark}
	That $K \bar \rtimes_r G$ is a closed ideal comes by the virtue of \cref{hamanalift}: If $E$ (respectively $E_K$) denotes the canonical conditional expectation on $B \rtimes_r G$ (respectively $B/K \rtimes_r G$), then the quotient map $\pi \colon B \to B/K$ extends to a surjective $*$-homomorphism $\Pi \colon B \rtimes_r G \to B/K \rtimes_r G$ such that $\Pi(a \lambda_s) = [a] \lambda_s$. The kernel of $\Pi$ is $K \bar \rtimes_r G$, and $\Pi$ satisfies
	\begin{align*}
		E_K \circ \Pi =  \pi \circ E.
	\end{align*}
	\label{condidealrem}
\end{remark}
\begin{proposition}
	Let $A$ be a unital $C^*$-algebra, $G \acts {} {A}$, $X$ be a free $G$-boundary. Supppose that $J$ is a closed ideal in $(A \otimes C(X)) \rtimes_r G$ and define $J_A := J \cap (A \otimes C(X))$. Then we have inclusions of ideals
	\begin{align*}
		 J_A \rtimes_r G \subseteq J \subseteq J_A \bar \rtimes_r G.
	\end{align*}
	\label{ozawprop18}
\end{proposition}
\begin{proof}
	We begin by noting that $J_A$ will is a $G$-invariant ideal: If $s \in G$ and $y \in J_A$ then $s.y = \lambda_s y \lambda_s^*  \in J$ and as $A \otimes C(X)$ is a $G$-$C^*$-algebra also $s.y \in A \otimes C(X)$ whenever $y \in A \otimes C(X)$.
	
	Now, let $x \in X$ be arbitrary and define $J_A^x := \mathrm{id}_A \otimes \delta_x(J_A) \subseteq A \otimes \C \cong A$, where $\delta_x \colon f \mapsto f(x)$ for $f \in C(X)$. Then $J_A^x$ is an ideal of $A$, however it could be improper. We then obtain a surjective $*$-homomorphism $\pi_x \colon (A \otimes C(X)) / J_A \to A / J_A^x$ such that the diagram
	\begin{equation*}
		\begin{tikzcd}
			0 \ar[r] & J_A \ar[r] \ar[d, "\mathrm{id}\otimes \delta_x"] & A \otimes C(X) \ar[d, "\mathrm{id}\otimes \delta_x"] \ar[r,"q"] &(A \otimes C(X)) /J_A \ar[r] \ar[d,dashed, "\pi_x"] & 0\\
			0 \ar[r] & J_A^x \ar[r] & A \ar[r, "q_x"] & A/J_A^x \ar[r] & 0
		\end{tikzcd}
	\end{equation*}
	commutes with exact rows. We claim that if $y \in \cap_{x \in X} \ker \pi_x$ then $y = 0$. To see this, let $\varepsilon > 0$ and pick some irreducible representation $\pi$ of $(A \otimes C(X))/J_A$ on a Hilbert space $\H$ such that $\lv \pi(a) \rv \geq \lv a \rv - \varepsilon$.
	
	Every element of $(A \otimes C(X))/J_A$ commutes with elements of the form $[1 \otimes f]_{J_A}$, $f \in C(X)$, hence $\pi\left( (1 \otimes C(X))/J_A \right) \subseteq \pi(A \otimes C(X))/J_A)' \cong  \C 1_\H$, as $\pi$ is irreducible so the commutant is trivial. This implies that the composition
	\begin{equation*}
		\begin{tikzcd}
			 C(X) \ar[r] & 1 \otimes C(X) \ar[r] & (1 \otimes C(X))/J_A \ar[r, "\pi"] & \C 1_\H 
		\end{tikzcd}
	\end{equation*}
	defines a character on $C(X)$ and thus is of the form $\pi([1 \otimes f]) = \delta_x(f)1_\H$, $f \in C(X)$, for some $x \in X$. Define yet another representation $\pi_A^x$ by the composition 
	\begin{equation*}
		\begin{tikzcd}
			A \ar[r] &A \otimes 1  \ar[r] & (A \otimes 1)/J_A \ar[r, "\pi"] & \mathbb{B}(\H),
		\end{tikzcd}
	\end{equation*}
	i.e., $\pi_A^x \colon a \mapsto \pi([a \otimes 1])$. This gives us the following commutative diagram
	\begin{equation*}
		\begin{tikzcd}
			A \otimes C(X) \ar[rr, "q"] \ar [dd, "\mathrm{id}\otimes \delta_x"] \ar[rrdd, "\pi_A \otimes \delta_x"] && (A \otimes C(X))/J_A \ar[dd, "\pi"]\\ 
			 & & \\
			A \ar[rr, "\pi_A^x"] && \mathbb{B}(\H)
		\end{tikzcd}
	\end{equation*}
	Finishing the jigsaw puzzle, if $a \in A \otimes C(X)$ is such that $\pi_x([a]_{J_A})=0$, then $\mathrm{id} \otimes \delta_x (a) \in J_a^x$. Pick a representative $b \in J_A$ such that $\mathrm{id} \otimes \delta_x (a) = \mathrm{id} \otimes \delta_x (b)$. Then
	\begin{align*}
		\pi([a]) = \pi( q(a)) =  \pi(q(b)) = \pi([b])=0.
	\end{align*}
	Letting $\varepsilon \to 0$, we obtain that $[a] = 0$ whenever all $\pi_x([a]) = 0$, as the norm of $A$ is given by the supremum over norms under irreducible representations. 
	
	The rest of the proof follows analogously to the proof of the Archbold-Spielberg result: Let $x \in X$ such that $J_A^x \neq A$, and assume $A/J_A^x \subseteq \mathbb{B}(\H)$ via some faithful representation $\pi$. Consider the composition
	\begin{equation}
		\begin{tikzcd}
			J + A \otimes C(X) \ar[r, "Q"] & (J+ A \otimes C(X))/J \cong  (A \otimes C(X))/J_A \ar[r, "\pi_x"] & A/J_A^x \ar[r, "\pi"] &\mathbb{B}(\H),
		\end{tikzcd}
	\end{equation}
	 and use Arveson's theorem to extend the above composition to a unital completely positive map $\Phi_x \colon (A \otimes C(X)) \rtimes_r G \to \mathbb{B}(\H)$. We claim that $\Phi_x \circ E = \Phi_x$, where $E$ is the canonical conditional expectation onto $A \otimes C(X)$: This is true, for $A \otimes C(X)$ is in the multiplicative domain of $\Phi_x$ and 
	\begin{align*}
		\Phi_x(1 \otimes f) = \pi_x(q(1 \otimes f)) = f(x).
	\end{align*}
	
	Let $s \in G\backslash\{e\}$. By assumption that the action of $G$ on $X$ being free, we have $s.x\neq x$ so there is some $h \in C(X)$, $ 0 \leq h \leq 1$, such that $h(x) = 1$ and $h(s^{-1}.x) = 0$. Then, we see that
	\begin{align*}
		\Phi_x( \lambda_s) = \Phi_x(h) \Phi_x(\lambda_s) \Phi_x(h) = \Phi_x ( h \lambda_s h ) = \Phi_x (h (s.h) \lambda_s) = h(x)h(s^{-1}.x) \Phi_x(\lambda_s) = 0,		
	\end{align*}
	finishing the claim, as $s$ was arbitrary different from $e$. It follows that $\Phi_x \circ E = \Phi_x$ and in particular 
	\begin{align*}
		\pi_x(Q(E(J))) = \Phi_x(E(J)) = \Phi_x(J) = 0	
	\end{align*}
	for all $x \in X$ so $E(J) \subseteq J_A$. By construction, $E(J) \subseteq J_A$ implies that $J \subseteq J_A \bar \rtimes_r G$, as wanted. That $J_A \rtimes_r G \subseteq J$ is obvious.
\end{proof}
We will need the following small lemma for the big theorem of this section, and we note that the lemma holds in a more general setting if we weaken the requirement a bit, however, this is out of our scope.
\begin{lemma}
	For $G$ discrete, each stabilizer subgroup $G_x$ for $x \in \dd_F G$ is amenable.
	\label{stabfurstamen}
\end{lemma}
For a proof, see \cite[lemma 11]{ozawa2014lecture}. With these results in hand, we may prove a (super substantial) result, again due to \cite{breuillard2017c} and \cite{kalantar2017boundaries}:
\begin{theorem}
	For a discrete group $G$, the following are equivalent:
\begin{enumerate}
		\item $G$ is $C^*$-simple.
		\item $C(X) \rtimes_r G$ is simplie for all $G$-boundaries $X$.
		\item $C(\dd_F G) \rtimes_r G$ is simple.
		\item The action $G \acts {} \dd_F G$ is free.
		\item The action $G \acts {} \dd_F G$ is topologically free.
		\item There is some $G$-boundary $X$ such that the action is topologically free.
		\item Whenever $X$ is a compact minimal $G$-space, amenability of $G_x$ for some $x \in X$ implies topological freeness of the action on $X$.
		\item If $G \acts \alpha A$ with $A$ unital, then $A \rtimes_{\alpha,r} G$ is simple whenever $A$ is $G$-simple.
	\end{enumerate}
	\label{breulcsimple}
\end{theorem}
\begin{proof}
	$1 \implies 2$: Let $G$ be a $C^*$-simple group and $X$ a $G$-boundary. Let $I$ be an ideal of $C(X) \rtimes_r G$, and let $\pi$ denote the quotient map onto $A:= C(X) \rtimes_r G /I$, so that $I = \ker \pi$. We then have the following observations:
	\begin{itemize}
		\item By simplicity of $C_r^*(G)$, we may consider the canoncial tracial state as a map $\tau_0 \colon \pi(C^*_r(G)) \to \C$, $\pi(\lambda_s) \mapsto \tau_0(\lambda_s)$, which will be $G$-invariant with respect to $\mathrm{Ad} \lambda_s$.
		\item Identifying $\C$ with $\C 1 \subseteq C(\dd_F G)$, we may use \cref{1lifthamana} to extend $\tau_0$ to a $G$-equivariant completely positive map $\Phi \colon \pi(C(X) \rtimes_r G) \to C(\dd_F G)$.
		\item As previously, $\Phi \circ \pi$ restricts to the unique $G$-equivariant map $C(X) \to C(\dd_F G)$, so it must be the inclusion map. In particular, this implies that $C(X)$ is in the multiplicative domain of $\Phi \circ \pi$.
	\end{itemize}
	With these observations, we see that
	\begin{align*}
		\Phi (\pi(f \lambda_s)) = \delta_{e}(s) \Phi(\pi(f)) = E(f)	
	\end{align*}
	where $E$ is the canonical conditional expectation $C(X) \rtimes_r G$ and where we have identified $C(X) \subseteq C(\dd_F G)$. As $E$ is faithful, so is $\pi$. We conclude that $C(X) \rtimes_r G$ is simple.
	$2 \implies 3$: Follows as $\dd_F G$ is a $G$-boundary.
	$3 \iff 4 \iff 5$: This follows from \cref{stabfurstamen}, which implies that $C(\dd_F G) \rtimes_r G$ is simple if and only if the action is topologically free, by the Archbold-Spielberg result. as $\dd_F G$ is extremally disconnected, the action is free if and only if it is topologically free.

	$3 \implies 8$: Consider a unital $G$-simple $G$-$C^*$-algebra $A$. Let $I$ be a proper ideal of $A \rtimes_r G$. Let $J$ be the ideal of $(A \otimes C(\dd_F G)) \rtimes_r G$ generated by $I$ after identifying $A \rtimes_r G \subseteq (A \otimes C(\dd_F G) ) \rtimes_r G$. By \cref{ozawprop17}, $J$ is a proper ideal and by \cref{ozawprop18}, the $G$-invariant ideal $J_A = J \cap (A \otimes C(\dd_F G))$ satisfies
	\begin{align*}
		J_A \rtimes_r G \subseteq J \subseteq J_A \bar \rtimes_r G.
	\end{align*}
	We also identify $A$ with its image in $(A \otimes C(\dd_F G)) \rtimes_r G$. Then $I_A := J \cap A$ is a $G$-equivariant ideal of $A$ which is $G$-invariant as the inclusion $A \subseteq (A \otimes C(\dd_F G)) \rtimes_r G$ is $G$-equivariant. Recall that we assumed $G$-simplicity of $A$ so $I_A = \{0\}$, implying that $I_A \bar \rtimes_r G = 0$, for if $x \in I_A \bar \rtimes_r G$ then $x^*x \in I_A \bar \rtimes_r G$, which happens if and only if $E(x^*x) \in I_A$, where $E$ is the canonical faithful conditional expectation $A \rtimes_r G \to A$. As $I_A = 0$ and $E$ is faithful, this implies that $x^*x = 0$. 

	We finish the proof by showing that $I \subseteq I_A \bar \rtimes_r G$: Let $y \in I \subseteq J \subseteq J_A \bar \rtimes_r G$, and note that we then obtain a commutative diagram with exact rows
	\begin{equation*}
		\begin{tikzcd}
			0 \ar[r] &  I_A \bar \rtimes_r G \ar[d, hook] \ar[r] & A \rtimes_r G \ar[r] \ar[d, hook] & (A / I_A)\rtimes_r G  \ar[r] \ar[d, hook]  & 0\\
			0  \ar[r] & J_A \bar \rtimes_r G  \ar[r] & (A \otimes C(X)) \rtimes_r G \ar[r] & \left((A \otimes C(X))/J_A\right) \rtimes_r G \ar[r] & 0
		\end{tikzcd}
	\end{equation*}
	as $I_A \subseteq J_A$. Then, since $y \in A \rtimes_r G$ and $y \in J_A \bar \rtimes_r G$, it follows that $y \in I_A \bar \rtimes_r G = \{0\}$ by the above diagram. As $y$ was arbitrary, this shows that $I = \{0\}$.

	$8 \implies 1$: Trivial when applied to the trivial $G$-boundary $\{\cdot\}$.

	$8 \implies 7 \implies 6 \implies 5$: follows from \cref{ozawa15pt1} and \cref{ozawa15pt2}.
\end{proof}
\begin{corollary}
	If $G$ is $C^*$-simple, then $C(X) \rtimes_{r}G$ is simple for all compact minimal $G$-spaces $X$.
\end{corollary}
\begin{note}
	As mentioned in the original article, the above result (and the rest of the paper) provided a much needed take on the theory of $C^*$-simplicity of discrete groups. Prior to this, the most common way of checking $C^*$-simplicity was the technique initially found by Powers (see \cite{powers1975simplicity} or the author's take on the subject \cite[chapter 3]{bscp}) in his proof for $C^*$-simplicity of $\mathbb{F}_2$. As mentioned, the new results of \cite{breuillard2017c} provided a route to an answer of the long lasting question of equivalence of $C^*$-simplicity and the unique trace property.
\end{note}
We include also a result of the late Uffe Haagerup, which was based on the above article as well, and characterized $C^*$-simplicity and uniqueness of trace in terms of convex hulls and states and an averaging property similar to that of Powers' original paper. For a proof, see the original article \cite{haagerup2015new} or the author's discussion and covering of the result \cite[Chapter 5]{bscp}:
\begin{theorem}
	Let $G$ be a discrete countable group with canonical tracial state $\tau_0$ on $C_r^*(G)$. Then the following are equivalent:
	\begin{enumerate}
		\item $G$ is $C^*$-simple,
		\item $G$ admits a free boundary action,
		\item $\tau_0 \in\overline{ \left\{ s.\varphi \ \mid \ s \in G \right\}}^{w^*}$ for all states $\varphi $ on $C_r^*(G)$, where $s.\varphi(a) = \varphi(\lambda_s a \lambda_s^*)$,
		\item $\tau_0 \in \overline{\mathrm{conv}}^{w^*}\left\{ s . \varphi \ \mid \ s \in G \right\}$, for all states $\varphi$ on $C_r^*(G)$,
		\item for all $t_1,\dots,t_m \in G\backslash\{e\}$ and $\varepsilon > 0$ there exists $s_1,\dots,s_n \in G$ such that
			\begin{align*}
				\lv \frac1n \sum_{j=1}^n \lambda_{s_jt_ks_j^{-1}}\rv < \varepsilon,
			\end{align*}
			for $k=1,\dots,m$,
		\item $C_r^*(G)$ has the Dixmier property.
	\end{enumerate}
	\label{uffecsimple}
\end{theorem}
We end this chapter (in fact this thesis) by saying that, to a certain degree, the above result have been developed further even, and the theory is still finding new applications and developments today. The consequence is that mathematicians now have a much stronger toolkit for determening $C^*$-simplicity of discrete groups than ten years ago.

