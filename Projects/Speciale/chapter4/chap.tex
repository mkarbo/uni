\chapter{Actions And Automorphisms}
In this chapter, we will go through theory describing automorphisms on a $C^*$-algebra $A$ and actions of groups $G$ on $C^*$-algebras, including applications in describing the ideal structure of crossed products. A motivation for including and describing this theory is the result of Archbold And Spielberg (see \cite{archbold1994topologically}), which showed that for a large class of groups and $C^*$-algebras, a certain condition on the action $G \acts {} A$ implies simplicity of the associated (reduced) crossed product $A \rtimes G$ ($A \rtimes_{r} G$).

\sssection{Action Properties}
We are going to discuss a property that an action $G \acts \alpha A$ can have called topological freeness and it's implications on the crossed product. In order to do this, we introduce notation and discuss some results on the spectrum of a general $C^*$-algebra as well as the primitive ideal space of $A$:
\begin{definition}
	For a $C^*$-algebra $A$, we define the \myemph{primitive ideal space} of a $C^*$-algebra $A$ to be the set
	\begin{align*}
		\Prim A := \left\{ \ker \pi \mid \pi \text{ is an irreducible representation of } A \right\}.
	\end{align*}
	Furthermore, we define the \myemph{spectrum of a $C^*$-algebra} $A$ to be the set
	\begin{align*}
		\hat A := \{[\pi]_u \mid \pi \text{ is an irreducible representation of } A \}.
	\end{align*}
where $[\cdot]_u$ denotes the class of unitarily equivalent representations.

If $\pi_1 \sim_u \pi_2$, then $\ker \pi_1 = \ker \pi_2$ so the map $\kappa \colon \hat A \to \Prim A$, $[\pi]_u \mapsto \ker \pi$, is surjective and well-defined. We endow $\Prim A$ with a topology called the \myemph{hull-kernel topology} defined such that for $S = (s_i)_{i \in I} \subseteq \Prim A$ we have
\begin{align*}
	\overline S := \{ p \in \Prim A \mid \cap_{i \in I} s_i \subseteq p\}.
\end{align*}
To see that this is in fact a topology, we refer to \cite[Appendix A.2]{williamsmorita}. We also endow $\hat A$ with the corresponding initial topology stemming from the surjective map $\kappa$.
\end{definition}
\begin{remark}
	For a commutative $C^*$-algebra (i.e., $A = C_0(X)$), the spectrum of $A$ is the Gelfand Space of $A$, i.e., non-zero $*$-homomorphisms $A \to \C$, since irreducible representations of commutative spaces are one-dimensional. Moreover, in this case it holds that $\hat A \cong \Prim A$.
\end{remark}
These objects interact nicely with dynamical systems (we already saw this for $A= C_0(X)$): Whenever $G \acts \alpha A$ then we may define the \myemph{transposed action of $G$} on both $\hat A$ and $\Prim A$ (as topological spaces) by
\begin{align*}
	\alpha_r^*([\pi]_u) := [\pi \circ \alpha_{r^{-1}}] \text{ and } \alpha_{r}^*(\ker \pi) = \ker \pi \circ \alpha_{r^{-1}},
\end{align*}
for all irreducible representations $\pi$ of $A$. These are indeed actions on topological spaces (i.e., the corresponding maps out of $G \times \hat A \to \hat A$ and $G \times \Prim A \to \Prim A$ are jointly continuous cf. \cite[Lemma 7.1]{williamsmorita}).
\begin{definition}
	For an action $G \acts \alpha A$ we say that the action is \myemph{Topologically free} if for all $t_1,\dots,t_n \in G\backslash\{e\}$ the set $\bigcap_{i=1}^n \left\{ [\pi]_u \in \hat A | \alpha_{t_i}^*([\pi]_u)\neq [\pi]_u \right\}$ is dense in $\hat A$.
\end{definition}
\begin{remark}
	It is easy to see that for $A=C_0(X)$, then an action is topologically free if and only if for all $e \neq t \in G$, the set of fixpoint of the induced action $G \acts{} X$ have empty interior or equivalently dense complement. Moreover, define 
	\begin{align*}
		G_x^\circ := \{t \in G \ \mid \ t|_U = \mathrm{id} \text{ for some neighborhood } U \text{ of } x\},
	\end{align*}
	which is clearly a normal subgroup of $G$, then the action is topologically free if and only $G_x^\circ = \{e\}$ for all $x$: If $t.y= y$ for all $y \in U$ of some neighborhood $U$ of $x$, then the fixpoint set of $t$ has non-empty interior.
\end{remark}
\begin{definition}
	We say that an action $G \acts {} X$ is free if every stabilizer is trivial, i.e., if for each $x \in X$ the stabilizer $G_x :=\{g \in G \ \mid \ g.x =x\}$ is trivial.
\end{definition}
\begin{lemma}
	Suppose that $G \acts {} X$ is a minimal action. If $G_x^\circ = \{e\}$ for some $x \in X$, then $G_y^\circ = \{e\}$ for all $y \in X$, i.e., the action is topologically free.
	\label{topfreeminimal}
\end{lemma}
\begin{proof}
	Let $x \in X$ be such that $G_x^\circ = \{e\}$. Suppose towards a contradiction that $s \in G_y^\circ$, $y \neq e$. Let $U \subseteq X$ witness this, i.e, be open with $y \in U$ and $s|_U=\mathrm{id}$. By minimality of the action, there is $t \in G$ such that $x \in tU$. Then $tst^{-1}(tv) =  tsv = tv$ for all $v \in U$, so $tst^{-1}|_{tU}=  \mathrm{id}$. By assumption, this implies that $tst^{-1} = e$, which forces $s = e$ (if $t = e$, then $s \in G_x^\circ$ and if $t \neq e$ then $s = e$), a contradiction on our assumption.
\end{proof}
\begin{corollary}
	The action of $G$ on $\dd_F G$ (the Furstenberg boundary) is topologically free if and only if it is free.
\end{corollary}
\begin{proof}
	This follows directly from the earlier remark that $\dd_F G$ is extremally disconnected, since the $(\dd_F G)_s$ is clopen by a theorem of Frolik, see e.g., \cite[Theorem 3.1]{frolik1971maps}, which says that the set of fixed points of a homeomorphism $f \colon X \to X$, where $X$ is a compact extremally disconnected space, is clopen in $X$.
\end{proof}
\begin{definition}
	For a $C^*$-algebra $A$, we say that $\alpha \in \mathrm{Aut}(A)$ is \myemph{properly outer} if for all $\alpha$-invariant non-zero ideals $I \subseteq A$ and all inner automorphisms $\beta$ of I it holds that
	\begin{align*}
		\lv \alpha|_I - \beta \rv = 2.
	\end{align*}
\end{definition}
Elliot and Olesen \& Pedersen showed in \cite[Theorem 3.2]{elliott1980some} and \cite[Theorem 7.2]{olesenpedersen3}, respectively, that for discrete groups $G$ with $G \acts \alpha A$ minimal, a sufficient condition for $C^*$-simplicity of $A \rtimes_{\alpha,r} G$ for a large class of $C^*$-algebras was that $\alpha_t$ be properly outer for $t \neq e$. Archbold and Spielberg later improved on this result in \cite{archbold1994topologically}. We will go through their result. 
\begin{proposition}[\mbox{\cite[Proposition 1]{archbold1994topologically}}]
	Let $(A,G,\alpha)$ be a $C^*$-dynamical system. If $\alpha$ is topologically free, then $\alpha_t$ is properly outer for all $t \neq e$.
\end{proposition}
\begin{remark}
	The proof is based on theory of derivations of $C^*$-algebras, which we will not go into detail with. Shortly put, a derivation of a $C^*$-algebra is a linear map which behaves similar to derivation of differentiable functions. For an exhibition on this, see e.g., \cite[Chapter 8.6]{pedersenalgauto} (or the books they refer to \cite{sakai1978recent} and \cite{bratteli2012operator} by Sakai and Bratteli et al., respectively.) 

	The proof is based on a result of Kadison-Ringrose, which characterizes proper outerness with not, in a sense, never locally looking like an inner automorphism composed with the exponential of a derivation, see \cite{kadison1967derivations} or \cite[Theorem 6.6 (i) and (ii)]{olesenpedersen3}.
\end{remark}

Recall that for $G$ discrete, it holds that $A \subseteq A \rtimes_\alpha G$, and that if $A^1 = A \oplus \C$ is the canonical unitalization, then we have inclusions $ A \subseteq A \rtimes_\alpha G  \subseteq A^1 \rtimes_\alpha G \subseteq M(A \rtimes_\alpha G)$, where we abuse the notation $\alpha$ to mean $\alpha \oplus 1$ on $A^1 \rtimes_\alpha G$.
\begin{definition}
	Let $A$ denote an arbitrary $C^*$-algebra with two representations $\pi$ and $\tau$ on $H_\pi$ and $H_\tau$. We say that $\pi$ and $\tau$ are \myemph{disjoint representations} if there is no subrepresentations of one which is unitarily equivalent to a non-zero subrepresentation of the other.
\end{definition}
\begin{remark}
	The above definition is the (to the authors knowledge) original definition. We will use the following fact: If $\pi$ and $\tau$ are disjoint if and only if $1 \oplus 0$ is an element of the strong operator closure of $\pi \oplus \tau (A)$, i.e., there is some net $(x_i) \subseteq A$ bounded with $\pi(x_i) \to 1$ and $\tau(x_i) \to 0$ in the strong operator topology if and only if  $\pi \not \sim _u \tau$, see e.g. \cite[Proposition 2.1.4]{arveson2012invitation}. 
\end{remark}
\sssection{Consequences of Topological Freeness}
Recall that we may identify $G$ as a subset of the unitary group of $M(A \rtimes_\alpha G)$. We denote this identification by $u_t$, $t \in G$. 
\begin{lemma}
	Let $G$ be discrete and $G \acts \alpha A$ and $\pi$ is a non-degenerate representation of $A$. If there is $t \in G$ such that $\pi \circ \alpha_t^{-1}$ and $\pi$ are disjoint representations, then for all completely positive extension $\varphi$ of $\pi$ to $A \rtimes_\alpha G$ of norm $1$ it holds that $\varphi(a u_t) = 0$ for all $a \in A$ and $u_t \in G \subseteq M(A \rtimes_\alpha G)$.
	\label{ASlem1}
\end{lemma}
\begin{proof}
	Let $(e_i) \subseteq A$ be a bounded net such that $\pi  \circ \alpha_{t}^{-1}(e_i) \to 1$ in the strong operator topology and $\pi(e_i) \to 0$. Let $\varphi$ be a unital completely positive extension of $\pi$ to $A^1  \rtimes_\alpha G$ of norm one, possible by non-degeneracy of $\pi$, see e.g. \cite[Proposition 2.2.1]{brown2008c}. Then, for $a \in A$ and $t \in G$:
	\begin{align*}
		\varphi(a u_t) = \lim_{\text{SOT}} \varphi(x_i )\varphi(a u_t) = \lim_{\text{SOT}} \varphi(u_t \alpha_{t}^{-1}(x_i a )) = \lim_{\text{SOT}} \varphi(u_t) \varphi(\alpha_t^{-1}(x_i))\varphi(\alpha_{t}^{-1}(a)) = 0,
	\end{align*}
	by the bimodule property, see \cite[Proposition 1.5.7 p.t (2)]{brown2008c}.
\end{proof}
With this, we obtain the following:
\begin{theorem}
	Let $(A,G,\alpha)$ be a $C^*$-dynamical system, $G$ discrete. If $\alpha$ is topologically free, then $A$ detects ideals in $A \rtimes_\alpha G$ modulo the kernel of the canonical surjection $A \rtimes_\alpha G \to A \rtimes_{\alpha,r}G$.
	\label{ASthm1}
\end{theorem}
\begin{proof}
	Let $I_r$ denote the kernel of the canonical surjection $ \pi_r \colon A \rtimes_\alpha G \to A \rtimes_{\alpha , r}G$ and let $E$ denote the canonical faithful conditional expectation (see e.g., \cite[Proposition 4.1.9]{brown2008c}) given by
	\begin{align*}
	E_r \colon A \rtimes_{\alpha,r} G &\to A, \\ 
		c_c(G,A) \ni \sum_{g \in G} a_g  u_g &\mapsto a_e \in A,
	\end{align*}
	and let $E = E_r \circ \pi_r$ denote the canonical conditional expectation $A \rtimes_{\alpha }G \to A$.
	
	Suppose that $I$ is an ideal of $A \rtimes_\alpha G$ with $A \cap I=0$ and $I \not \subseteq I_r$. Since $E_r$ is faithful, there is $a \in I_+$ such that $E(a) \neq 0$, e.g., as the pre-image of a non-zero positive element of the image of $I$ in $A \rtimes_{\alpha,r}G$. By density of $\cc(G,A)$ in $A \rtimes_\alpha G$, pick $b = \sum_{g \in F} b_g u_g$ for some finite set $F \subseteq G$ with
	\begin{align*}
		\lv a - b \rv < \frac{\lv E (a) \rv}{2}.
	\end{align*}
	By assumption, the set $P := \bigcap_{e \neq t \in F}\left\{ p \in \hat A \mid \alpha_{t}^*(p) \neq p \right\}$ is dense in $\hat A$. Let $[\pi] \in P$. Pick a c.c.p extension of the composition $\pi \circ q$, where $q \colon A+I \to (A+I)/I\cong A$ is the quotient map, e.g., by using unitizations and \cite[Proposition 2.2.1]{brown2008c} repeatedly and restricting to the desired domains. Denote the extension by $\tau$, which has $\lv \tau \rv = \lim \lv \tau(e_i)\rv = 1\lv \pi \rv = 1$, where $e_i$ is an approximate identity of $A \rtimes_\alpha G$ contained in $A$. 
	
	Recall that $[\pi] \in P$ if and only if for $t \in F\backslash\{e\}$ the representations $\pi$ and $\pi \circ \alpha_{t}^{-1}$ are disjoint (by the remark following the definition), implying by \cref{ASlem1} that
	\begin{align*}
		\tau(b) = \sum_{t \in F} \tau(b_t u_t) = \tau(b_e) = \tau(E(b)).
	\end{align*}
	Note that if $x \in I$, then $\tau(x) = 0$, since then $\tau(x) = \pi \circ q(x) = \pi(0) =0$. Hence
	\begin{align*}
		\lv \pi(E(b))\rv = \lv \tau(b) \rv = \lv \tau(b-a)\rv \leq \lv b-a\rv.
	\end{align*}
	Since $P$ is dense in $\hat A$, and $\hat A$ corresponds to states of $A$, we see that $\lv E(b)\rv \leq \lv b-a\rv$, giving the contradiction
	\begin{align*}
		\lv E(a) \rv \leq \lv E(a-b)\rv + \lv E(b)\rv \leq 2 \lv a-b\rv <  \lv E(a)\rv,
	\end{align*}
	so $I \subseteq I_r$.
\end{proof}
\begin{corollary}
	If $G$ is amenable, then $A$ detects ideals in $A \rtimes_\alpha G$ whenever $\alpha$ is topologically free.
\end{corollary}

If we further assume that $A$ is abelian, then the converse holds as well:
\begin{theorem}
	Let $(A,G,\alpha)$ be a $C^*$-dynamical system with $A$ abelian and $G$ discrete. Then the converse of the implication in \cref{ASthm1} holds.
	\label{ASthm2}
\end{theorem}
\begin{proof}
	Since $A$ is commutative, we have $A \cong C_0(\hat A)$, by Gelfand theory. Let $x \in \hat A$. Recall that \todo{Add in prelim about top actions} actions of $G$ on locally compact Hausdorff spaces correspond to actions on $C_0(X)$, via the formula $t.f(x) :=  \alpha_t(f)(x) := f(t^{-1}.x)$, here $t.x$ denotes the image of the homeomorphism $t \colon X \to X$. Consider the Hilbert space $\ell^2(\left\{ t.x \mid t \in G\} \right\}$. We define the covariant pair $(\pi_x,u)$ for $(A,G)$ by
	\begin{align*}
		\pi_x(f) \delta_{t.x} = f(t.x) \delta_{t.x} \text{ and } u_s \delta_{t.x} = \delta_{st.x},
	\end{align*}
	for $f \in C_0(\hat A)$ and $t,s \in G$. That they are representations is easy to verify, we show covariance instead:
	\begin{align*}
		u_s \pi_x(f) u_{s^{-1}} \delta_{t.x} = u_{s} f(s^{-1}t.x)\delta_{s^{-1}t.x} = f(s^{-1}t.x) \delta_{t.x} = s.f(t.x) = \pi_x(s.f)\delta_{t.x},
	\end{align*}
	so that we obtain a representation $\pi_x \rtimes u_x$ of $A \rtimes_\alpha G$ on $\ell^2(G.x)$ satisfying the same relations (here we identify $G$ with its image in $M(A \rtimes_\alpha G)$). Let $I = \bigcap_{x \in \hat A} \ker \pi_{x}$. If $f \in I$, then $f(t.x) = 0$ for all $t \in G$ (in particular for $t=e$) and $x \in \hat A$, so $f = 0$. Hence $I \cap A = 0$.
	
	Suppose that $G \acts \alpha A$ is not topologically free. Then there is $e \neq s \in G$ such that $V=\left\{ y \in \hat A \mid s.y \neq y \right\}$ is not dense in $\hat A$. Let $0 \neq f \in C_0(\hat A)$ with $\supp f \subseteq \overline{V}^c$. Whenever $t.x \in \supp f$, we have $st.x = t.x$, so
	\begin{align*}
		\pi_x\rtimes u_x(f-fu_s)\delta_{tx} = f(t.x) \delta_{t.x} - f(st.x)\delta_{st.x}= 0,
	\end{align*}
	and if $t.x \not \in \supp(f)$, then (as $s.x=x \iff x = s^{-1}.(s.x)=s^{-1}.x$) we see that $st.x \not \in \supp f$, and hence $\pi_{x} \rtimes u_x (f-f u_s) = 0$ again. Thus $f-f u_s \in I \subseteq I_r$, since $x$ was arbitrary. Since $E(f u_s) = E_r(\pi_r(fu_s)
) =0$, we see that 
	\begin{align*}
		0 \neq f = E(f-fu_s) = E_r (\pi(f-fu_s))= E_r(0) =0,
	\end{align*}
	a contradiction. We conclude that the action of $A$ is topologically free.
\end{proof}

\begin{corollary}
	Suppose that $G \acts \alpha C_0(X)$ with $G$ discrete. Then $A \rtimes_\alpha G$ is simple if and only if the action is minimal, topologically free and $I_r = 0$.
\end{corollary}
\begin{proof}
	This follows directly from \cref{ASthm1}, \cref{ASthm2} and \cref{ASlem1}.
\end{proof}
Once again, this gives an easy proof of simplicity of $C(S^1) \rtimes_{\theta} \Z$ for $\theta$ irrational, since the action is very clearly topologically free (there are no fixpoints on $S^1$ for irrational rotation), and we already know that it is minimal and $I_r = 0$.

\sssection{A Relation between Ideals of $C_r^*(G)$ and $A \rtimes_{\alpha,r} G$}.
From here on, let $G$ denote a discrete countable group and $X$ a compact $G$-space. In the following, we will discuss recent developments regarding the classification of $C^*$-simple groups, in particular the results based on theory of boundary actions and topological freeness discussed earlier, and its relation to simplicity of associated crossed products $A \rtimes_{\alpha,r} G$. 

We say that a group $G$ is \myemph{$C^*$-simple} if its reduced group is $C^*$-simple, and we say that $G$ has the \myemph{unique trace property} if the canonical faithful tracial state $\tau_0 \colon f \mapsto \langle \delta_e f , \delta_e\rangle_{\ell^2(G)}$ is the only tracial state on $C_r^*(G)$. It was an open question for a long time wether having the unique tracial state was equivalent to being $C^*$-simple. In 2014, using theory of boundary actions, this was shown by to false by Le Boudec (see \cite[Thoerem A]{le2017c}). However, $C^*$-simplicity does imply unique tracial state which is equivalent to having no non-trivial amenable normal subgroups (see e.g. the authors project on this subject: \cite[Chapter 5]{bscp}).

It is quite a strong condition on a group $G$ to be $C^*$-simple, since as mentioned amenable groups are not $C^*$-simple (nor do they have any amenable subgroups, see again \cite[Chapter 5]{bscp}). Groups which are $C^*$-simple include the class of Powers groups (an homage to R.T. Powers proof on simplicity of the free group on two generators, see \cite{powers1975simplicity} or \cite[chapter 3]{bscp}) which includes the free groups on finite number of generators.

In order to show our desired results, we need to cover a bit of theory on conditional expectations and crossed products. For instance whenever $N$ is a normal subgroup of a discrete group $G$, then the inclusion $N \subseteq G$ gives rise to a conditional expecation $E_{N}^G \colon C_r^*(G) \to C_r^*(N)$ given by $\lambda_s \mapsto 1_{N}(s) \lambda_s$, see e.g. \cite[Corollary 2.5.12]{brown2008c}. Moreover, supposing that $A = C(X)$, then for all $x \in X$, the associated conditional expectation $E_{G_x}^G \colon C_r^* (G) \to C_r^*(G_x)$ (here $G_x$ is the stabilizer subgroup) extends to a condotional expecation
\begin{align}
	E_x \colon C(X) \rtimes_{\alpha,r} G &\to C_r^*(G_x), \\
	f \lambda_s &\mapsto f(x) E_{G_x}^G(\lambda_s).
	\label{eq:stabcond}
\end{align}

The following result shows the close relationship between topological freeness of actions and $C^*$-simplicty of groups with the ideal structure of crossed products of $G$-boundaries. It is (to the authors knowledge) due to \cite{breuillard2017c}, however we follow a proof of Ozawa (see e.g., \cite{ozawa2014lecture}). The result is heavily inspired by the proof of the Archbold-Spielberg result covered earlier.
\begin{proposition}
	Let $G \acts {} X$ be a minimal action. If the stabilizer group $G_x := \{ t \in G \ \mid \ t.x =x\}$ is $C^*$-simple for some $x \in X$, then the corresponding reduced crossed product $C(X) \rtimes_{\alpha,r} G$ is simple.
\end{proposition}
\begin{proof}
	Suppose that $I \subseteq C(X) \rtimes_r G$ is a non-trivial closed ideal and let $x \in X$. Let $E$ denote the canoncial faithful conditional expectation $C(X) \rtimes_r G \to C(X)$, $f \lambda_s \mapsto f$ if $s  = e$ and $0$ else (see e.g., \cite[proposition 4.1.9]{brown2008c}). Then $E(I) \neq 0$, since $E$ is faithful and by minimality of $G \acts {} X$ it is dense in $C(X)$. If $\tau_r$ denotes the canoncial faithful tracial state on $C_r^*(G)$ and let $f \in E(I)$ with $E(f) \neq 0$ and $f(x) \neq 0$. Then $\tau_r (E_x(f \lambda_e))= f(x) = \delta_x(E(f\lambda_e))$, so that $E_x(I)$ is non-zero in $C_r^*(G_x)$. We claim that $E_x(I)$ is not dense in $C_r^*(G_x)$. Similar to the proof of \cref{ASthm1}, we consider state on $C(X) + I$ given by the composition:
	\begin{align*}
		C(X) + I \to (C(X)+I)/I \cong C(X) \stackrel {\delta_x} \to \C,
	\end{align*}
	where we used that the proper ideal $C(X) \cap I$ of $C(X)$ must be $0$ by minimality of $G \acts {} X$. Let $\varphi_x$ denote any state extending the above composition with $\varphi_x(I) =0$. The result will follow one we show that $\varphi_x = \varphi_x \circ E_x$, for then $\varphi_x (E_x(I)) = \varphi_x(I) = 0 $ implying that $\varphi_x= 0$ on all of $C(X) \rtimes_{\alpha,r}G$ if $E_x(I)$ is dense in $C_r^*(G_x)$. 

	To see that $\varphi_x = \varphi_x \circ E_x$ it suffices to show that $\varphi_x (\lambda_s) = 0$ for $s \not \in G \backslash G_x$: If not, then since $C(X)$ is in the multiplicative domain of $\varphi_x$ it holds that $\varphi_x(f \lambda_s) =  \varphi_x( f(x) \lambda_s)  =  \varphi_x (E_x( f\lambda_s))$. So assume that $s.x \neq x$ and choose a function $h \in C(X)$ such that $h(x) = 1$ and $h(s^{-1}.y) = 0$ for all $y \in \supp h$ (using that $G$ is a normal space and applying Uryhson's lemma). In particular, this implies that 
	\begin{align*}
		\varphi_x(\lambda_s) = \varphi_x(h\lambda_sh) = \varphi_x(h \alpha_s(h)\lambda_s)   = 0,
	\end{align*}
	finishing the proof.
\end{proof}
We have seen that topological freeness implies simplicity of $A \rtimes_{\alpha,r} G$ for $G$-simple $C^*$-algebras. The following result from the same paper as above allows us to determine topological freeness from properties of $G$ and $C(X)\rtimes_\alpha G$:
\begin{proposition}
	Suppose that $X$ is a minimal $G$-space and $C(X) \rtimes_\alpha G$ is simple. If there is some $x \in X$ such that the normal subgroup $G_x^\circ$ of $G_x$ is amenable, then the action $G \acts { } X$ is topologically free.
\end{proposition}
\begin{proof}
	Before we can begin our work to prove the result, we need make a few remarks: Since $G_x^\circ$ is amenable, the quotient map $q \C G \to \C(G/G_x^\circ)$ extends to a $*$-homomorphism $\pi \colon C_r^*(G) \to C_r^* (G / G_x^\circ)$ such that
\begin{equation}
	\begin{tikzcd}
		\C G \ar[r,"q"] \ar[d,"\lambda_G"'] & \C\left( G/G_x^\circ \right) \ar[d, "\lambda_{G/G_x^\circ}"]\\
		C_r^*(G) \ar[r,"\pi"'] & C_r^*\left( G/G_x^\circ \right)
	\end{tikzcd}
\end{equation}
see e.g. \cite[Proposition 3]{de2007simplicity}. Note also that we may define a $*$-representation $\pi_x$ of $C(X)$ on $\ell^2(G/G_x^\circ)$ as multiplication operators of the form
\begin{align*}
	\pi_x(f) \delta_{[t]} = f(tp) \delta_{[t]},
\end{align*}
for $f \in C(X)$ and $t \in G$, induced by the corresponding $*$-homomorphism into $\ell^\infty(G/G_x^\circ)$, which can be check to be injective.

With this, we may define a $^*$-representation $\pi$ of $C(X) \rtimes_r G$ on $\ell^2(G/G_x^{\circ})$ by
	\begin{align*}
		\pi(f\lambda_s) \delta_{[t]} = f(spx) \delta_{[st]},
	\end{align*}
	for $f \in C(X)$ and $s \in G$ and $[t] \in G/G_x^{\circ}$. An easy computation shows that the state 
	\begin{align*}
		C(X) \rtimes_r G\to \C, \ a \mapsto \langle \pi_x(a) \delta_{[e]} , \delta_{[e]}\rangle_{\ell^2(G/G_x^{\circ})},	
	\end{align*}
	is actually the map $a \mapsto \langle E_{G_x^{\circ}} (E_x(a)) \delta_e , \delta_e\rangle_{\ell^2(G_x^{\circ})}$. In particular, this implies that $\pi$ is continuous. Let $s \in G_x^\circ$ and let $U$ be a neighborhood of $x$ witnessing this, i.e, be such that $s|_U = \mathrm{id}$ and pick $f \in C(X)$, $f \neq 0$ with $\supp f \subseteq U$. If $f(tx) \neq 0$, then $tx \in U$ so $[st] = [t]$. Hence $\pi\left( ( 1-\lambda_s)f) \right)= 0$, implying that $\lambda_s = \lambda_e$ by injectivity of $\pi$. \todo{revisit, perhaps do contraposition}
\end{proof}

\begin{note}
	In order to prove the main theorem of this section, we will need a corollary result saying that $C(\dd _F G)$ is injective as a $G$-algebra (as an object in the category of $G$-$C^*$-algebras): whenever $A$ is a unital $C^*$-algebra and $G \acts A$, then there is a $G$-morphism of $A$ into $C(\dd_F G)$. For a proof, see e.g. \cite[Theorem 6 + Corollary 7]{ozawa2014lecture}.
\end{note}
\begin{definition}
	
\end{definition}
\begin{proposition}
	
\end{proposition}

With these results in hand, we may prove a (super substantial) result, again due to \cite{breuillard2017c}:
\begin{theorem}
	For a discrete group $G$, the following are equivalent:
	\begin{enumerate}
		\item $G$ is $C^*$-simple.
		\item $C(\dd_F G) \rtimes_r G$ is simple.
		\item There is some $G$-boundary $X$ such that the action is topologically free.
		\item Whenever $X$ is a compact minimal $G$-space, amenability of $G_x$ for some $x \in X$ implies topological freeness of the action on $X$.
		\item If $G \acts \alpha A$ with $A$ unital, then $A \rtimes_{\alpha,r} G$ is simple whenever $A$ is $G$-simple.
	\end{enumerate}
	\label{breulcsimple}
\end{theorem}
\begin{corollary}
	If $G$ is $C^*$-simple, then $C(X) \rtimes_{r}G$ is simple for all compact minimal $G$-spaces $X$.
\end{corollary}
\begin{note}
	As mentioned in the original article, the above result (and the rest of the paper) provided a much needed take on the theory of $C^*$-simplicity of discrete groups. Prior to this, the most common way of checking $C^*$-simplicity was the technique initially found by Powers (see \cite{powers1975simplicity} or the authors take on the subject \cite[chapter 3]{bscp}) in his proof for $C^*$-simplicity of $\mathbb{F}_2$. As mentioned, the new results of \cite{breuillard2017c} provided a route to an answer of the long lasting question of equivalence of $C^*$-simplicity and the unique trace property.
\end{note}
We include also a result of the late Uffe Haagerup, which was based on the above article as well, and characterized $C^*$-simplicity and uniqueness of trace in terms of convex hulls and states. For a proof, see the original article \cite{haagerup2015new} or the authors discussion and covering of the result \cite[Chapter 5]{bscp}:
\begin{theorem}
	Let $G$ be a discrete countable group with canonical tracial state $\tau_0$ on $C_r^*(G)$. Then the following are equivalent:
	\begin{enumerate}
		\item $G$ is $C^*$-simple,
		\item $G$ admits a fre boundary action,
		\item $\tau_0 \in \left\{ s.\varphi \ \mid \ s \in G \right\}$ for all states $\varphi $ on $C_r^*(G)$, where $s.\varphi(a) = \varphi(\lambda_s a \lambda_s^*)$,
		\item $\tau_0 \in \overline{\mathrm{conv}}^{w^*}\left\{ s . \varphi \ \mid \ s \in G \right\}$, for all states $\varphi$ on $C_r^*(G)$,
		\item for all $t_1,\dots,t_m \in G\backslash\{e\}$ and $\varepsilon > 0$ there exists $s_1,\dots,s_n \in G$ such that
			\begin{align*}
				\lv \frac1n \sum_{j=1}^n \lambda_{s_jt_ks_j^{-1}}\rv < \varepsilon,
			\end{align*}
			for $k=1,\dots,m$,
		\item $C_r^*(G)$ has the Dixmier property.
	\end{enumerate}
	\label{uffecsimple}
\end{theorem}

