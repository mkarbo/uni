\chapter{Preliminaries and Prerequisites}
Throughout this thesis, we will assume familiarity with quite a bit of theory regarding topological spaces, group theory and operator theory. We will in this chapter introduce standard notation and basic results and concepts which will be used throughout this project. We include, for the more advanced results, a reference, while we refer the reader to \cite{brown2008c,zhu,folland2013real,blackadar} for the rest.
\begin{itemize}
	\item We will use $G$ to denote a group, which we assume is equipped with a locally compact Hausdorff topology making $G$ a \myemph{topological group}, i.e., a topology such that multiplication is jointly continuous and inversion is continuous.
	\item We will use $e$ to denote the neutral element of $G$. If $G$ is abelian, we will use $0$ instead.
	\item We will use $\H$ to denote an arbitary Hilbert space.
	\item If $\H$ and $\mathcal{K}$ are Hilbert spaces, then we denote by $\H \otimes \mathcal{K}$ the Hilbert space obtained by completion of the algebraic tensor product $\H \odot \mathcal{K}$ with respect to the inner product
		\begin{align*}
			\langle \sum_{i}\xi_i \otimes \eta_i, \sum_j \xi'_j \otimes \eta'_j\rangle = \sum_{i,j} \langle \xi_i, \xi'_j\rangle \langle \eta_i, \eta_j'\rangle. 
		\end{align*}
		For more on the tensor product of $C^*$-algebras and Hilbert spaces, see \cite[chapter 3.1-3.3]{brown2008c}.
\end{itemize}
A $C^*$-algebra $A$ is a complex Banach algebra which is also a $*$-algebra satisfying the $C^*$-identity; 
\begin{align*}
\lv x^*x \rv = \lv x\rv^2, \quad \text{for all } a \in A
\end{align*}
We use $A$ to denote an arbitary $C^*$-algebra. If $A$ is unital then we denote the unit of $A$ by $1_{A}$ or simply $1$ if no confusion may arrise. 
\begin{itemize}
\item A $C^*$-algebra $A$ is simple if it contains no non-trivial proper closed two-sides ideals.
\item For a $C^*$-algebra $A$ and $a \in A$ we say that
\begin{itemize}
\item $a$ is \emph{self-adjoint} if $a^*=a$. The set of all self-adjoint elements of $A$ is denoted by $(A)_{\text{s-a}}$,
\item $a$ is \emph{positive} if $a=c^*c$ for some $c \in A$. We write $a \geq 0$ if and only if $a$ is positve. The \emph{positive cone} of all positive elements in $A$ is denoted by $(A)_{+}$,
\item $a$ is a \emph{projection} if $a=a^2=a^*$, i.e. idempotent and self-adjoint,
\item $a$ is \emph{unitary} if $a^*a=aa^*=1_{A}$. The group of all unitary elements of $A$ is denoted by $\mathcal{U}(A)$.
\end{itemize}
\item A \emph{$*$-homomorphism} $\Phi \colon A \to B$ between unital $C^*$-algebras $A , B$ is a unital, multiplicative and linear map which carries involution. We say that two $C^*$-algebras $A,B$ are isomorphic if there is a bijective $*$-homomorphism between them.
\item We will by $X$ and $Y$ denote locally compact Hausdorf spaces, and we denote by $C_0(X)$ the $C^*$-algebra of continuous functions $f \colon X \to \C$ equipped the supremum norm and involution $f \mapsto \overline{f}$.
\item Every commutative $C^*$-algebra is isomorphic to the $C^*$-algebra of continuous functions which vanishes at infinity, $C_0(K)$, on some locally compact Hausdorff space $K$.  
\item A linear map $\varphi \colon A \to B$ between $C^*$-algebras is said to be positive if it carries positive elements to positive elements, i.e., if $\varphi\left( (A)_{+} \right) \subseteq \left(B\right)_+$. Hence a positive continuous linear functional $\varphi \in A^*$ is a functional such that $\varphi\left( \left( A \right)_+\right) \subseteq \R_+$. By \cite[Theorem 13.5][79]{zhu}, a functional $\varphi$ on $A$ is positive if and only if it is bounded and $\lv \varphi \rv = \varphi(1)$. For a positive linear functional $\varphi$ on $A$ we say that
\begin{itemize}
\item $\varphi$ is a \emph{state} if $\varphi(1)=1=\lv \varphi \rv$. The space of all states on $A$ is called the state space of $A$, and is denoted by $\mathcal{S}(A)$.
\item $\varphi$ is said to be \emph{faithful} if $\varphi(x) > 0$ for all non-zero positive elements $x \in (A)_+$.
\item $\varphi$ is said to be a \emph{trace} is it is a state and it is tracial, i.e., if $\varphi(ab)=\varphi(ba)$ for all $a,b \in A$ and $\varphi(1)=1$.
\end{itemize}
\item We denote by $\mathbb{B}(\H)$ the $C^*$-algebra of bounded linear operators on a Hilbert space $\H$ and we denote by $\mathbb{K}(\H)$ the compact operators on $\H$, which is a two-sided closed ideal of $\mathbb{B}(\H)$.
\item A representation of a $C^*$-algebra on a Hilbert space $\H$ is a $*$-homomorphism $\pi \colon A \to \mathbb{B}(\H)$. For any $C^*$-algebra $A$, there is a Hilbert space $\H$ such that $A$ is faithfully represented on it by the GNS-construction \cite[Thoerem 14.4][87]{zhu}
\item A linear map $\varphi \colon A \to B$ between $C^*$-algebras is \myemph{contractive completely positive} if for all $n \geq 1$, the map $\varphi_n \colon M_n(A) \to M_n(B)$ defined by
\begin{align*}
\varphi_n([a_{ij}]):=[\varphi(a_{ij})], \ [a_{ij}] \in M_n(A),
\end{align*}
is positive and contractive.
\item Given a bounded linear functional $\omega$ on a unital $C^*$-algebra $A$, we define its adjoint $\omega^*$ by $\omega^*(a)=\overline{\omega(a^*)}$ for $a \in A$.
\item Every bounded linear functional $\omega$ on a unital $C^*$-algebra $A$ can be decomposed into $\mathrm{Re}\omega+i \mathrm{Im}\omega$, where $\mathrm{Re}\omega, \mathrm{Im}\omega$ are the self-adjoint linear functionals defined by
\begin{align*}
\mathrm{Re}\omega=\frac{\omega+\omega^*}{2}, \ \mathrm{Im}\omega = \frac{\omega-\omega^*}{2i}.
\end{align*}
\item Every self-adjoint linear functional $\gamma$ on $A$ has a unique \emph{Jordan decomposition}, there exists two positive linear functionals $\gamma_{\pm}$ on $A$ such that $\gamma=\gamma_+-\gamma_-$ which satisfy $\lv \gamma \rv = \lv \gamma_+\rv + \lv \gamma_-\rv$, and they are unique.
\item As a consequence of the above, we see that every bounded linear functional can be decomposed into a linear combination of four states.
\end{itemize}
We will also need a few important results:
\begin{definition}
	Let $A$ be a unital $C^*$-algebra. An operator subsystem $E$ is a closed self-adjoint subspace of $A$ containing the identity.
\end{definition}
\begin{lemma}[\mbox{\cite[1.5.14 - 1.5.16]{brown2008c}}]
Let $E \subseteq A$ be an operator subsystem of a unital $C^*$-algebra $A$. Then all positive linear functionals on $E$ extends to a positive linear functional on $A$ of same norm. In particular, this applies to states.
\end{lemma}
\begin{theorem}[\mbox{\myemph{Arveson's Extension Theorem}}]
	Let $A$ be a unital $C^*$-algebra and $E \subseteq A$ an operator subsystem. Then every contractive completely positive map $\varphi \colon E \to \mathbb{B}(\H)$ extends to a contractive completely positive map $\tilde \varphi \colon A \to \mathbb{B}(\H)$
\end{theorem}
For a proof of the above, see \cite[Theorem 1.6.1]{brown2008c}.
\begin{definition}
	Let $A$ and $B$ be $C^*$-algebras and $\varphi \colon A \to B$ a contractive completely positive map. The set
	\begin{align*}
		A_\varphi := \{a \in A \ \mid \ \varphi(a^*a)=\varphi(a)^*\varphi(a) \text{ and } \varphi(a a^*) = \varphi(a) \varphi(a)^*\},
	\end{align*}
	is called the \myemph{multiplicative domain} of $\varphi$ in $A$.
\end{definition}
It turns out that the above behaves exceptionally well:
\begin{proposition}[\mbox{\cite[Proposition 1.5.7]{brown2008c}}]	
	Let $A$ and $B$ be $C^*$-algebras and $\varphi \colon A \to B$ a contractive completely positive map. Then
	\begin{enumerate}
		\item (\myemph{Schwarz Inequality}) The inequality $\varphi(a)^*\varphi(a) \leq \varphi(a^*a)$ holds for all $a \in A$.
		\item(\myemph{Bimodule Property}) If $a \in A_\varphi$, then
			\begin{align*}
				\varphi(ba)=\varphi(b)\varphi(a) \text{ and } \varphi(ab)=\varphi(a)\varphi(b),
			\end{align*}
			for all $b \in A$.
		\item $A_\varphi$ is a $C^*$-subalgebra of $A$.
	\end{enumerate}
\end{proposition}
A very important result is the result of Tomiyama, which characterizes something called conditional expectations in a very nice way.
\begin{definition}
	Let $A \subseteq B$ be $C^*$-algebras. A projection from $B$ onto $A$ is a linear map $E \colon B \to A$ such that $E(a)=a$ for all $a \in A$. A \myemph{conditional expectation} from $B$ onto $A$ is a contractive completely positive projection $E \colon B \to A$ such that
	\begin{align*}
		E(axa')=a E(x) a',
	\end{align*}
	for every $a,a' \in A$ and $x \in B$.
\end{definition}
\begin{theorem}[Tomiyama's theorem]
	Let $A \subseteq B$ be $C^*$-algebras and $E$ a projection from $B$ onto $A$. Then the following are equivalent:
	\begin{enumerate}
		\item $E$ is a conditional expectation,
		\item $E$ is contractive completely positive,
		\item $E$ is contractive.
	\end{enumerate}
\end{theorem}
For a proof of the above, see \cite[Theorem 1.5.10]{brown2008c}.
\begin{definition}
	A \myemph{unitary representation} of a group $G$ is a strongly continuous group homomorphism $u \colon G \to \mathcal{U}(\H)$, $t \mapsto u_t: = u(t)$, of a Hilbert space $\H$. That is, for each $\xi \in \H$, the map $t \mapsto u_t \xi$ is continuous. 
\end{definition}

