\chapter{Introduction}
In this thesis, we set out with a goal describing the ideal structure of certain $C^*$-algebras associated to $C^*$-dynamical systems called \textit{crossed products}. The construction is not new, and can be quite difficult to work with in full generality. This also means that the classification of crossed product $C^*$-algebras can be quite hard, if one has no knowledge of the corresponding ideal structure. The crossed product, in a sense, generalizes the group $C^*$-algebras, as we shall see. It provides many interesting non-trivial examples of $C^*$-algebras, and has played a crucial role in the theory of classifying and describing group $C^*$-algebras.

In general, it is not always easy to determine whether a general $C^*$-algebra $A$ is simple. If $A$ is unital and has a faithful tracial state, then one can check if $A$ has the \text{Dixmier property}:
\begin{defnonum}
	A unital $C^*$-algebra $A$ has the \myemph{Dixmier property} if for all $a \in A$, the set
	\begin{align*}
		\overline{\mathrm{conv}\{ uau \ \mid \ u \in \mathcal{U}(A)\}} \cap \C 1_{A} \neq \emptyset.
	\end{align*}
\end{defnonum}
For instance, if $G$ is a discrete group, then the reduced group $C^*$-algebra, $C_r^*(G)$, has unique faithful tracial state. Essentially checking for the Dixmier property is how the original proof of Powers (\cite{powers1975simplicity}) showing that the free group on two generators, $\mathbb{F}_2$, where $C^*$-simple. However, the general crossed product is frequently non-unital and might not have a faithful tracial state. So we need another approach to this problem. In this thesis, we will discuss and cover results, old and new, which describes exactly when the crossed product is simple.

In the first chapter, we cover some theory concerning \textit{Hilbert modules}, which can be thought of as a generalization of Hilbert spaces. In particular, we will describe the theory of unitizations, one of which is the important \textit{Multiplier algebra} of a $C^*$-algebra. This is necesarry for the general crossed product construction, for given a $C^*$-dynamical system $(A,G,\alpha)$ with $G$ not discrete, we can not a priori give meaning to representations of the $*$-algebra, $\cc(G,A)$, which we use to define the crossed product.

In the second chapter, we use the Multiplier algebra construction to construct and characterize the crossed product $A \rtimes_{\alpha} G$ of a $C^*$-dynamical system $(A,G,\alpha)$. We also discuss two important cases: The case where $G$ is abelian and the case where $A$ is. For the case of $G$ abelian, we cover a duality result called \textit{Takai Duality} based on the \textit{Pontrjagin Duality} result saying that $\hat{\hat G} \cong G$, where $\hat G$ is the dual group of $G$. For the case where $A$ is abelian, we briefly discuss topological dynamics and end with a short survey of \textit{boundary actions}, which play a crucial role later on.

In the third chapter, we describe some invariants associated to $C^*$-dynamical systems with abelian groups, called the group spectra. The chapter relies on the Takai duality theory we described in the previous chapter, and for certain groups we get a complete characterization of simple crossed products. In doing so, we obtain a well-known example of a simple crossed product, which is the crossed product associated to irrational rotation on the circle.

In the final chapter, we discuss theory of actions of groups on $C^*$-algebras and automorphisms of $C^*$-algebras. We describe how properties of the action of a group on a $C^*$-algebra influences the ideal structure of the associated crossed product when $G$ is discrete, in particular we show the result of Archbold and Spielberg (\cite{elliott1980some}), linking topological free system with simple crossed products. Finally, we apply the theory developed thus far to show the result of \cite{breuillard2017c}, which links the theory of boundary actions of $G$, $C^*$-simplicity of $G$ and simplicity of certain crossed products with $G$.
