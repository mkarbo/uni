\documentclass[10pt,twoside,openany,final]{memoir}
\usepackage[utf8]{inputenc}
\usepackage[pass]{geometry}
%\usepackage[T1]{fontenc}
\usepackage[english]{babel}
\usepackage{amsmath}
\usepackage{showkeys}
\usepackage{amsfonts}
\usepackage{amsthm}
\usepackage{amssymb} 
\usepackage[usenames,dvipsnames]{xcolor}
\usepackage{amssymb}
\usepackage{graphicx}
%\usepackage{flexisym}
\usepackage{xr}
\usepackage[all]{xy}
\usepackage{tikz-cd}
%\usepackage{hyperref}
\usepackage[style=alphabetic,backend=bibtex]{biblatex}
\usepackage{filecontents}
\usepackage[english, status=draft]{fixme}
\fxusetheme{color}
\usepackage{imakeidx} 
\usepackage{hyperref}
\usepackage{cleveref} 
\usepackage{tensor}
\usepackage[backgroundcolor=cyan]{todonotes}
\usepackage{wallpaper}
\usepackage{titlesec}
%\usepackage[titleloc]{appendix}
\usepackage{faktor}
\usepackage{xfrac}
\usepackage{mathabx}
\usepackage{mathrsfs}
\usepackage{enumitem}
\setsecnumdepth{subsection}
\maxtocdepth{subsection}


\makeatletter
	\def\SK@def\Cref#1{\SK@\SK@@ref{#1}\SK@Cref{#1}}%
\makeatother

%\usepackage{courier}
%\renewcommand*\familydefault{\ttdefault} %% Only if the base font of the document is to be typewriter style

\usepackage[T1]{fontenc}

%\renewcommand{\ttdefault}{pcr}

%\usepackage{mathabx} 

\makeindex[columns=2, options=-s master, intoc=true]

%\makeindex[columns=3, title=Alphabetical Index]

\titleformat{\chapter}[display]
{\center\normalfont\bfseries}{}{0pt}{\Large}
%\renewcommand*\cftappendixname{Appendix  }

\newcommand{\apptitle}{\section*{\huge\centering\normalfont\scshape \textbf{Appendix \thechapter}}}
\addtolength{\textwidth}{30pt}
\addtolength{\foremargin}{-30pt}
\checkandfixthelayout
\renewcommand\chaptermarksn[1]{}
\newcommand{\ssection}[1]{%
\newpage%
\section[#1]{\centering\normalfont\scshape \textbf{#1}}}

\newcommand{\sssection}[1]{%
\section[#1]{\centering\normalfont\scshape \textbf{#1}}}
\addtolength{\textwidth}{30pt}
\addtolength{\foremargin}{-30pt}
\checkandfixthelayout

\newcommand{\ssubsection}[1]{%
\subsection[#1]{\centering\normalfont\scshape \textbf{#1}}}
\checkandfixthelayout

\newcommand{\psection}[1]{%
\subsection*[#1]{\centering\normalfont\scshape \textbf{#1}}}
\checkandfixthelayout


\setlength{\parindent}{1em}
\setlength{\parskip}{1em}
\renewcommand{\baselinestretch}{1.2}

\usepackage{scalerel,stackengine}
\stackMath
\newcommand\reallywidehat[1]{%
\savestack{\tmpbox}{\stretchto{%
  \scaleto{%
    \scalerel*[\widthof{\ensuremath{#1}}]{\kern-.6pt\bigwedge\kern-.6pt}%
    {\rule[-\textheight/2]{1ex}{\textheight}}%WIDTH-LIMITED BIG WEDGE
  }{\textheight}% 
}{0.5ex}}%
\stackon[1pt]{#1}{\tmpbox}%
}
\parskip 1ex

\newtheoremstyle{break}
{\topsep}{\topsep}
{\itshape}{}
{\bfseries}{}
{\newline}{}
\theoremstyle{definition}
\newtheorem{theorem}{Theorem}[chapter]
\numberwithin{theorem}{section}
\newtheorem{lemma}[theorem]{Lemma}
\newtheorem{proposition}[theorem]{Proposition}
\newtheorem{corollary}[theorem]{Corollary}
\newtheorem{definition}[theorem]{Definition}
\newtheoremstyle{Break}
{\topsep}{\topsep}
{}{}
{\bfseries}{}
{\newline}{}
\theoremstyle{Break}
\newtheorem{example}[theorem]{Example}
\newtheorem{remark}[theorem]{Remark}
\newtheorem{note}[theorem]{Note}
%\setcounter{secnumdepth}{0}
\usepackage{xpatch}
\xpatchcmd{\proof}{\ignorespaces}{\mbox{}\\\ignorespaces}{}{}
%\newenvironment{Proof}{\proof \mbox{} \\ \\ *}{\endproof}

\chapterstyle{thatcher}

\newenvironment{abst}{\rightskip1in\itshape}{}

\makepagestyle{abs}
\makeevenhead{abs}{}{}{}
\makeoddhead{abs}{}{}{}
\makeevenfoot{abs}{}{\scshape I }{}
\makeoddfoot{abs}{}{\scshape  I }{}
%\makeheadrule{abs}{\textwidth}{\normalrulethickness}
%\makefootrule{abs}{\textwidth}{\normalrulethickness}{\footruleskip}
\pagestyle{abs}


\makepagestyle{cont}
\makeevenhead{cont}{}{}{}
\makeoddhead{cont}{}{}{}
\makeevenfoot{cont}{}{\scshape II }{}
\makeoddfoot{cont}{}{\scshape  II }{}
%\makeheadrule{abs}{\textwidth}{\normalrulethickness}
%\makefootrule{abs}{\textwidth}{\normalrulethickness}{\footruleskip}
\pagestyle{cont}

\newcommand{\lv}{\left\lVert}
\newcommand{\rv}{\right\rVert}


\renewcommand\chaptermarksn[1]{}
\nouppercaseheads
\createmark{section}{left}{shownumber}{}{\space}
\makepagestyle{dut}
\makeevenhead{dut}{\scshape\rightmark}{}{\scshape\leftmark}
\makeoddhead{dut}{\scshape\leftmark}{}{\scshape\rightmark}
\makeevenfoot{dut}{}{\scshape $-$ \thepage\ $-$}{}
\makeoddfoot{dut}{}{\scshape $-$ \thepage\ $-$}{}
\makeheadrule{dut}{\textwidth}{\normalrulethickness}
\makefootrule{dut}{\textwidth}{\normalrulethickness}{\footruleskip}
\pagestyle{dut}

\makepagestyle{chap}
\makeevenhead{chap}{}{}{}
\makeoddhead{chap}{}{}{}
\makeevenfoot{chap}{}{\scshape $-$ \thepage\ $-$}{}
\makeoddfoot{chap}{}{\scshape $-$ \thepage\ $-$}{}
\makefootrule{chap}{\textwidth}{\normalrulethickness}{\footruleskip}
\copypagestyle{plain}{chap}


\DeclareMathOperator{\supp}{supp}
\DeclareMathOperator{\Ext}{Ext}
\DeclareMathOperator{\Aut}{Aut}
\DeclareMathOperator{\Ran}{Ran}
\DeclareMathOperator{\Prob}{Prob}
\DeclareMathOperator{\conv}{conv}
\DeclareMathOperator{\AR}{AR}
\DeclareMathOperator{\Homeo}{Homeo}




\newcommand{\R}{\mathbb{R}}
\newcommand{\C}{\mathbb{C}}
\newcommand{\N}{\mathbb{N}}
\newcommand{\mbr}{(X,\mathcal{A})}
\newcommand{\Z}{\mathbb{Z}}
\newcommand{\Q}{\mathbb{Q}}
\newcommand{\F}{\mathcal{F}}
\newcommand{\A}{\mathcal{A}}
\newcommand{\cc}{C_c}
\newcommand{\PP}{\mathcal{P}}
\newcommand{\B}{\mathcal{B}}
\newcommand{\dd}{\partial}
\newcommand{\ee}{\epsilon}
\newcommand{\la}{\lambda}
\renewcommand{\H}{\mathcal{H}}
\newcommand{\pp}{\text{Prob}}
\newcommand{\U}{\mathcal{U}}
\newcommand*{\diff}{\mathop{}\!\mathrm{d}}
\newcommand{\Span}{\mathrm{span}}
\newcommand{\K}{\mathbb{K}}
\newcommand{\acts}[1]{\stackrel{#1}{\curvearrowright}}
%\def\acts{\curvearrowright}
\renewcommand{\d}{\mathrm{d}}
\newcommand{\Sp}[1]{\mathrm{Sp}(#1)}
\newcommand{\Prim}[1]{\mathrm{Prim}(#1)}

\makeatletter
\newcommand{\Spvek}[2][r]{%
\gdef\@VORNE{1}
\left(\hskip-\arraycolsep%
\begin{array}{#1}\vekSp@lten{#2}\end{array}%
\hskip-\arraycolsep\right)}

\def\vekSp@lten#1{\xvekSp@lten#1;vekL@stLine;}
\def\vekL@stLine{vekL@stLine}
\def\xvekSp@lten#1;{\def\temp{#1}%
\ifx\temp\vekL@stLine
\else
\ifnum\@VORNE=1\gdef\@VORNE{0}
\else\@arraycr\fi%
#1%
\expandafter\xvekSp@lten
\fi}
\makeatother
\pagenumbering{roman}

%\let\oldemph\emph
%\renewcommand{\emph}[1]{\index{#1}}


\newcommand{\myemph}[1]{\index{#1}\emph{#1}}%

    \makeatletter
        \def\@idxitem{\par\hangindent 2em}
	    \makeatother




\defbibheading{bibnumbered}[\refname]{\chapter{#1}}


\renewcommand{\hat}[1]{\widehat{#1}}
\renewcommand{\check}[1]{\widecheck{#1}}
