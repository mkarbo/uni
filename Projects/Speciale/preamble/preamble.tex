\documentclass[10pt,twoside,openany,final]{memoir}
\usepackage[utf8]{inputenc}
\usepackage[pass]{geometry}
\usepackage[T1]{fontenc}
\usepackage[english]{babel}
\usepackage{amsmath}
\usepackage{amsfonts}
\usepackage{amsthm}
\usepackage[usenames,dvipsnames]{xcolor}
\usepackage{tikz}
\usepackage{amssymb}
\usepackage{graphicx}
\usepackage{flexisym}
\usepackage{hyperref}
\usepackage{xr}
\usepackage[all]{xy}
\usepackage{tikz-cd}
\usepackage{hyperref}
\usepackage[style=alphabetic,backend=bibtex]{biblatex}
\usepackage{filecontents}
\usepackage[english, status=draft]{fixme}
\fxusetheme{color}
\usepackage{cleveref} 
\usepackage[backgroundcolor=cyan]{todonotes}
\usepackage{wallpaper}
\usepackage{faktor}


\checkandfixthelayout

\begin{filecontents}{bibtest.bib}
@article{bko,
  title={C*-simplicity and the unique trace property for discrete groups},
  author={Breuillard, Emmanuel and Kalantar, Mehrdad and Kennedy, Matthew and Ozawa, Narutaka},
  journal={arXiv preprint arXiv:1410.2518},
  year={2014}
}

@book{blackadar,
  title={Operator algebras: theory of C*-algebras and von Neumann algebras},
  author={Blackadar, Bruce},
  volume={122},
  year={2006},
  publisher={Springer Science \& Business Media}
}

@article{powers1975simplicity,
  title={Simplicity of the C*-algebra associated with the free group on two generators},
  author={Powers, Robert T and others},
  journal={Duke Math. J},
  volume={42},
  number={1},
  pages={151--156},
  year={1975}
}


@book{zhu,
  title={An introduction to operator algebras},
  author={Zhu, Kehe},
  volume={9},
  year={1993},
  publisher={CRC press}
}

@article{de2007simplicity,
  title={On simplicity of reduced C*-algebras of groups},
  author={De La Harpe, Pierre},
  journal={Bulletin of the London Mathematical Society},
  volume={39},
  number={1},
  pages={1--26},
  year={2007},
  publisher={Oxford University Press}
}

@misc{rudin1991functional,
  title={Functional analysis. International series in pure and applied mathematics},
  author={Rudin, Walter},
  year={1991},
  publisher={McGraw-Hill, Inc., New York}
}

@book{brown2008c,
  title={C*-algebras and finite-dimensional approximations},
  author={Brown, Nathanial Patrick and Ozawa, Narutaka},
  volume={88},
  year={2008},
  publisher={American Mathematical Soc.}
}

@book{folland2013real,
  title={Real analysis: modern techniques and their applications},
  author={Folland, Gerald B},
  year={2013},
  publisher={John Wiley \& Sons}
}

@article{furman2003minimal,
  title={On minimal, strongly proximal actions of locally compact groups},
  author={Furman, Alex},
  journal={Israel Journal of Mathematics},
  volume={136},
  number={1},
  pages={173--187},
  year={2003},
  publisher={Springer}
}

@article{le2015c,
  title={C*-simplicity and the amenable radical},
  author={Le Boudec, ADRIEN},
  journal={arXiv preprint arXiv:1507.03452},
  year={2015}
}
@book{MI,
  title={Measure theory},
  author={Hansen, Ernst},
  year={2006},
  publisher={University of Copenhagen}
}

@article{haagerup2015new,
  title={A new look at C*-simplicity and the unique trace property of a group},
  author={Haagerup, Uffe},
  journal={arXiv preprint arXiv:1509.05880},
  year={2015}
}


\end{filecontents}

\addbibresource{bibtest.bib}


\chapterstyle{verville}

\newcommand{\restr}[2]{\ensuremath{\left.#1\right|_{#2}}}

\setlength{\parindent}{1.5em}
\setlength{\parskip}{1em}
\renewcommand{\baselinestretch}{1}

\makeatletter
\def\bign#1{\mathclose{\hbox{$\left#1\vbox to8.5\p@{}\right.\n@space$}}\mathopen{}}
\def\Bign#1{\mathclose{\hbox{$\left#1\vbox to11.5\p@{}\right.\n@space$}}\mathopen{}}
\def\biggn#1{\mathclose{\hbox{$\left#1\vbox to14.5\p@{}\right.\n@space$}}\mathopen{}}
\def\Biggn#1{\mathclose{\hbox{$\left#1\vbox to17.5\p@{}\right.\n@space$}}\mathopen{}}
\makeatother

\DeclareMathOperator{\supp}{supp}
\DeclareMathOperator{\Ext}{Ext}
\DeclareMathOperator{\Aut}{Aut}
\DeclareMathOperator{\Ran}{Ran}
\DeclareMathOperator{\Prob}{Prob}
\DeclareMathOperator{\conv}{conv}
\DeclareMathOperator{\AR}{AR}
\DeclareMathOperator{\Homeo}{Homeo}

\makepagestyle{abs}
    \makeevenhead{abs}{}{}{}
    \makeoddhead{abs}{}{}{}
    \makeevenfoot{abs}{}{\scshape I }{}
    \makeoddfoot{abs}{}{\scshape  I }{}
    %\makeheadrule{abs}{\textwidth}{\normalrulethickness}
    %\makefootrule{abs}{\textwidth}{\normalrulethickness}{\footruleskip}
\pagestyle{abs}

\makeatletter
\def\moverlay{\mathpalette\mov@rlay}
\def\mov@rlay#1#2{\leavevmode\vtop{%
   \baselineskip\z@skip \lineskiplimit-\maxdimen
   \ialign{\hfil$\m@th#1##$\hfil\cr#2\crcr}}}
\newcommand{\charfusion}[3][\mathord]{
    #1{\ifx#1\mathop\vphantom{#2}\fi
        \mathpalette\mov@rlay{#2\cr#3}
      }
    \ifx#1\mathop\expandafter\displaylimits\fi}
\makeatother

\newcommand{\cupdot}{\charfusion[\mathbin]{\cup}{\cdot}}
\newcommand{\bigcupdot}{\charfusion[\mathop]{\bigcup}{\cdot}}

\makepagestyle{cont}
    \makeevenhead{cont}{}{}{}
    \makeoddhead{cont}{}{}{}
    \makeevenfoot{cont}{}{\scshape II }{}
    \makeoddfoot{cont}{}{\scshape  II }{}
    %\makeheadrule{abs}{\textwidth}{\normalrulethickness}
    %\makefootrule{abs}{\textwidth}{\normalrulethickness}{\footruleskip}
\pagestyle{cont}

\newcommand{\lv}{\lVert}
\newcommand{\rv}{\rVert}


\renewcommand\chaptermarksn[1]{}
\nouppercaseheads
\createmark{chapter}{left}{shownumber}{}{.\space}
\makepagestyle{dut}
    \makeevenhead{dut}{\scshape\rightmark}{}{Malthe Karbo}
    \makeoddhead{dut}{\scshape\leftmark}{}{Malthe Karbo}
    \makeevenfoot{dut}{}{\scshape $-$ \thepage\ $-$}{}
    \makeoddfoot{dut}{}{\scshape $-$ \thepage\ $-$}{}
    \makeheadrule{dut}{\textwidth}{\normalrulethickness}
    \makefootrule{dut}{\textwidth}{\normalrulethickness}{\footruleskip}
\pagestyle{dut}

\makepagestyle{chap}
    \makeevenhead{chap}{}{}{}
    \makeoddhead{chap}{}{}{}
    \makeevenfoot{chap}{}{\scshape $-$ \thepage\ $-$}{}
    \makeoddfoot{chap}{}{\scshape $-$ \thepage\ $-$}{}
    \makefootrule{chap}{\textwidth}{\normalrulethickness}{\footruleskip}
\copypagestyle{plain}{chap}

\newcommand{\R}{\mathbb{R}}
\newcommand{\C}{\mathbb{C}}
\newcommand{\N}{\mathbb{N}}
\newcommand{\mbr}{(X,\mathcal{A})}
\newcommand{\Z}{\mathbb{Z}}
\newcommand{\Q}{\mathbb{Q}}
\newcommand{\F}{\mathcal{F}}
\newcommand{\A}{\mathcal{A}}
\newcommand{\cc}{C_c}
\newcommand{\PP}{\mathcal{P}}
\newcommand{\B}{\mathcal{B}}
\newcommand{\dd}{\partial}
\newcommand{\ee}{\epsilon}
\newcommand{\la}{\lambda}
\renewcommand{\H}{\mathcal{H}}
\newcommand{\pp}{\text{Prob}}
\newcommand{\U}{\mathcal{U}}

\makeatletter
\newcommand{\Spvek}[2][r]{%
  \gdef\@VORNE{1}
  \left(\hskip-\arraycolsep%
    \begin{array}{#1}\vekSp@lten{#2}\end{array}%
  \hskip-\arraycolsep\right)}

\def\vekSp@lten#1{\xvekSp@lten#1;vekL@stLine;}
\def\vekL@stLine{vekL@stLine}
\def\xvekSp@lten#1;{\def\temp{#1}%
  \ifx\temp\vekL@stLine
  \else
    \ifnum\@VORNE=1\gdef\@VORNE{0}
    \else\@arraycr\fi%
    #1%
    \expandafter\xvekSp@lten
  \fi}
\makeatother

\def\acts{\curvearrowright}

\newcommand{\K}{\mathbb{K}}

\newtheoremstyle{break}
	{\topsep}{\topsep}
	{\bfseries}{}
	{\newline}{}
\theoremstyle{break}
\newtheorem{theorem}[section]{Theorem}
\newtheorem{lemma}[section]{Lemma}
\newtheorem{proposition}[section]{Proposition}
\newtheorem{corollary}[section]{Corollary}
\newtheorem{definition}[section]{Definition}
\newtheoremstyle{Break}
	{\topsep}{\topsep}
	{}{}
	{\bfseries}{}
	{\newline}{}
\theoremstyle{Break}
\newtheorem{example}[section]{Example}
\newtheorem{remark}[section]{Remark}
\newtheorem{note}[section]{Note}
\setcounter{secnumdepth}{0}
\usepackage{xpatch}
\xpatchcmd{\proof}{\ignorespaces}{\mbox{}\\\ignorespaces}{}{}

\newcommand*{\diff}{\mathop{}\!\mathrm{d}}
